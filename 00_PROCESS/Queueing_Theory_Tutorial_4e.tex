


Jobs are sent to a supercomputer at a rate of 10 per hour and take the supercomputer on average 4 minutes to process. 
We will assume that the number of arrivals is $X_a \sim \text{Poisson}(\lambda_a)$ and the processing (i.e., service) time is $T_s \sim \text{Exponential}(\lambda_s)$. 

This leads to an $M/M/1$ system.

#### Exercises

A. Let $T$ be the total time in the system - what distribution has $T$? 
B. What is the average time spent in the system? Calculate $Sd(T)$ also.
C. How many jobs are in the system on average? (hint: Little's law)  
D. From the time the job is sent, what is the probability that it takes more than 15 minutes to complete? 
E. From the time the job enters the processor (i.e., service component), what is the probability that it takes more than 15 minutes to complete? 
F. What is the average number of jobs completed in a 3 hour period of operation? (hint: Burke's theorem)
G. What is the probability that more than 40 jobs are completed in a 3 hour period? (hint: Burke's theorem again)


Let's tackle these exercises step by step.

### A. Distribution of Total Time in the System
In an $M/M/1$ queue, the total time $T$ a job spends in the system (waiting time plus service time) follows an exponential distribution with parameter $\lambda_s - \lambda_a$, where $\lambda_s$ is the service rate and $\lambda_a$ is the arrival rate.

### Exercise B. 
Average Time Spent in the System and Standard Deviation
The average time spent in the system $E(T)$ is given by:
$$E(T) = \frac{1}{\lambda_s - \lambda_a}$$

Given:
- Arrival rate $\lambda_a = 10$ jobs per hour.
- Service rate $\lambda_s = \frac{1}{4/60} = 15$ jobs per hour.

So,
$$E(T) = \frac{1}{15 - 10} = \frac{1}{5} \text{ hours} = 12 \text{ minutes}$$

The standard deviation $Sd(T)$ is the same as the mean for an exponential distribution:
$$Sd(T) = E(T) = 12 \text{ minutes}$$

---
### C. 

**Average Number of Jobs in the System (Little's Law)**
Little's Law states:
$$L = \lambda_a \times E(T)$$

So,
$$L = 10 \times \frac{1}{5} = 2$$

Thus, on average, there are 2 jobs in the system.

---

### Exercise D. 


**Probability that it Takes More Than 15 Minutes to Complete**
The total time $T$ in the system follows an exponential distribution with mean $E(T) = 12$ minutes. The probability that $T$ exceeds 15 minutes is:
$$P(T > 15) = e^{-\frac{15}{12}} \approx e^{-1.25} \approx 0.2865$$

---

### Exercise E. 
Probability that Service Time Exceeds 15 Minutes
The service time $T_s$ follows an exponential distribution with mean $\frac{1}{\lambda_s} = \frac{1}{15} \text{ hours} = 4 \text{ minutes}$. The probability that $T_s$ exceeds 15 minutes is:
$$P(T_s > 15) = e^{-\frac{15}{4}} \approx e^{-3.75} \approx 0.0235$$

---

### Exercise F.
 Average Number of Jobs Completed in a 3-Hour Period
Using Burke's theorem, the output process of an $M/M/1$ queue is a Poisson process with the same rate as the arrival rate. So, the average number of jobs completed in 3 hours is:
$$\lambda_a \times 3 = 10 \times 3 = 30$$

---

### G. Probability that More Than 40 Jobs are Completed in a 3-Hour Period
The number of jobs completed in a 3-hour period follows a Poisson distribution with parameter $\lambda_a \times 3 = 30$. The probability of more than 40 jobs is:
$$P(X > 40) = 1 - P(X \leq 40)$$

Using the Poisson cumulative distribution function (CDF), we can find $P(X \leq 40)$ and subtract from 1. This requires numerical methods or a Poisson table.

If you have any further questions or need more details, feel free to ask!

Burke's Theorem is a principle in queueing theory named after Paul J. Burke. It relates to the behavior of queues, specifically the $ M/M/1 $ queue, which stands for:

- **M**: Poisson arrival process (Markovian)
- **M**: Exponential service time (Markovian)
- **1**: Single server

### Key Points of Burke's Theorem
- **Departure Process**: In an $ M/M/1 $ queue, the departure process of completed jobs (or customers) is also a Poisson process with the same rate as the arrival process. This means if jobs arrive at the queue at a rate $ \lambda $, they will also depart the queue at the same rate $ \lambda $.
- **Queue Length Distribution**: The number of jobs (or customers) in an $ M/M/1 $ queue in steady state is distributed according to a geometric distribution, which means that the probability of having $ n $ jobs in the system is given by:
  $$P(N = n) = (1 - \rho) \rho^n$$
  where $ \rho = \frac{\lambda}{\mu} $ is the traffic intensity (arrival rate divided by service rate).


---

BLANK
---
---

---


---

BLANK---
---

---
