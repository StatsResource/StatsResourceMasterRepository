
The \emph{average time} between customers arriving to a shop is 5 minutes. 
We will assume that the time, $T$, has an exponential distribution. 

### Exercises Calculate the following:
<p>

A.  The average arrival \emph{rate}, i.e., $\lambda$ customers per minute.  

B.  The probability that we wait more than 15 minutes for the next customer.  

C.  The probability that the next customer arrives within 1 minute.  

D.  The average \emph{number of customers} in a 1 hour period. What is the standard deviation that goes with this average?  

E.  The probability that \emph{15 or more} customers arrive in a 1 hour period.
<p>

Let's tackle each part of the exercise step by step.

### Given:
- The average time between customers arriving is 5 minutes.

For an exponential distribution, the rate parameter $\lambda$ is the reciprocal of the mean time between arrivals.

---

### A. The Average Arrival Rate ($\lambda$)

Given:
$$\text{Average Time} = 5 \text{ minutes}$$

The average arrival rate $\lambda$ is:
$$\lambda = \frac{1}{\text{Average Time}} = \frac{1}{5} = 0.2 \text{ customers per minute}$$


---

### Exercise B. 
The Probability That We Wait More Than 15 Minutes for the Next Customer

For an exponential distribution, the probability that we wait more than $ t $ minutes is given by:
$$P(T > t) = e^{-\lambda t}$$

Given $\lambda = 0.2$ and $ t = 15 $ minutes:
$$P(T > 15) = e^{-0.2 \times 15} = e^{-3} \approx 0.0498$$

So, the probability that we wait more than 15 minutes for the next customer is approximately **0.0498** (or 4.98%).

### C. The Probability That the Next Customer Arrives Within 1 Minute

The probability that the next customer arrives within $ t $ minutes is given by the cumulative distribution function (CDF):
$$P(T \leq t) = 1 - e^{-\lambda t}$$

Given $\lambda = 0.2$ and $ t = 1 $ minute:
$$P(T \leq 1) = 1 - e^{-0.2 \times 1} = 1 - e^{-0.2} \approx 1 - 0.8187 = 0.1813$$

So, the probability that the next customer arrives within 1 minute is approximately **0.1813** (or 18.13%).

### Exercise D. 
The Average Number of Customers in a 1-Hour Period and the Standard Deviation

The average number of customers in a 1-hour period is given by:
$$\lambda \times \text{Time Period} = 0.2 \times 60 = 12 \text{ customers}$$

For a Poisson distribution, the standard deviation is the square root of the mean:
$$\sigma = \sqrt{\lambda \times \text{Time Period}} = \sqrt{12} \approx 3.46$$

So, the average number of customers in a 1-hour period is **12**, and the standard deviation is approximately **3.46**.

### Exercise E. 
The Probability That 15 or More Customers Arrive in a 1-Hour Period

The number of customers arriving in a given period follows a Poisson distribution with parameter $\lambda \times \text{Time Period}$:
$$\lambda \times 60 = 12$$

We need to find $ P(X \geq 15) $. This can be calculated using the Poisson cumulative distribution function (CDF), but we subtract it from 1 for the complement rule:

$$P(X \geq 15) = 1 - P(X < 15)$$

Where:
$$P(X < 15) = \sum_{k=0}^{14} \frac{12^k e^{-12}}{k!}$$

Using a Poisson calculator or table, we get:
$$P(X \geq 15) \approx 1 - 0.8341 = 0.1659$$

So, the probability that 15 or more customers arrive in a 1-hour period is approximately **0.1659** (or 16.59%).

Feel free to ask if you have any more questions or need further assistance!