

\documentclass[a4paper12pt]{article}
%%%%%%%%%%%%%%%%%%%%%%%%%%%%%%%%%%%%%%%%%%%%%%%%%%%%%%%%%%%%%%%%%%%%%%%%%%%%%%%%%%%%%%%%%%%%%%%%%%%%%%%%%%%%%%%%%%%%%%%%%%%%%%%%%%%%%%%%%%%%%%%%%%%%%%%%%%%%%%%%%%%%%%%%%%%%%%%%%%%%%%%%%%%%%%%%%%%%%%%%%%%%%%%%%%%%%%%%%%%%%%%%%%%%%%%%%%%%%%%%%%%%%%%%%%%%
\usepackage{eurosym}
\usepackage{vmargin}
\usepackage{amsmath}
\usepackage{graphics}
\usepackage{epsfig}
\usepackage{enumerate}
\usepackage{multicol}
\usepackage{subfigure}
\usepackage{fancyhdr}
\usepackage{listings}
\usepackage{framed}
\usepackage{graphicx}
\usepackage{amsmath}
\usepackage{chngpage}
%\usepackage{bigints}

\usepackage{vmargin}
% left top textwidth textheight headheight
% headsep footheight footskip
\setmargins{2.0cm}{2.5cm}{16 cm}{22cm}{0.5cm}{0cm}{1cm}{1cm}
\renewcommand{\baselinestretch}{1.3}

\setcounter{MaxMatrixCols}{10}

\begin{document}
\large 


%%- http://webpages.iust.ac.ir/matashbar/teaching/schaum_probability.pdf

%%-- PAge 287
%%-- 9.13


A corporate computing center has two computers of the same capacity. The jobs arriving at the 
center are of two types, internal jobs and external jobs. These jobs have Poisson arrival times 
with rates 18 and 15 per hour, respectively. The service time for a job is an exponential r.v. with 
mean 3 minutes. 
\begin{enumerate}[(a)]
    \item Find the average waiting time per job when one computer is used exclusively for internal 
jobs and the other for external jobs. 
    \item Find the average waiting time per job when two computers handle both types of jobs. 
\end{enumerate}


\section*{Solution}

\subsection*{Part (a)}
\begin{itemize}
    \item (a) When the computers are used separately, we treat them as two M/M/1 queueing systems. 
    
        \item Let W,, and 
W,, be the average waiting time per internal job and per external job, respectively. 

    \item For internal jobs, 

A1 = = & and p1 = 3. Then, from Eq. (9.16), 
- 3 
10 Wq I = ---- = 27 min 4(4 - 6) 
    \item For external jobs, 1, = = $ and p2 = 5, and 
1 
;I Wq2 = ---- - 9 min +(3 - $1 
\end{itemize}

Let's break down the problem step-by-step:

### Given Data:
- **Internal Jobs Arrival Rate (\(\lambda_{\text{internal}}\))**: 18 jobs per hour
- **External Jobs Arrival Rate (\(\lambda_{\text{external}}\))**: 15 jobs per hour
- **Service Rate (\(\mu\))**: Since the mean service time is 3 minutes, the service rate is \(\mu = \frac{1}{3} \text{ jobs per minute} = 20 \text{ jobs per hour}\).

### Utilization Factor (\(\rho\)):
For each computer, we calculate the utilization factor separately.

1. **Internal Jobs**:
   $$\rho_{\text{internal}} = \frac{\lambda_{\text{internal}}}{\mu} = \frac{18}{20} = 0.9$$

2. **External Jobs**:
   $$\rho_{\text{external}} = \frac{\lambda_{\text{external}}}{\mu} = \frac{15}{20} = 0.75$$

### Average Waiting Time in the Queue (\(W_q\)):
The average waiting time in the queue for an M/M/1 queue is given by:
$$W_q = \frac{\rho}{\mu (1 - \rho)}$$

1. **Internal Jobs**:
   $$W_{q,\text{internal}} = \frac{0.9}{20 (1 - 0.9)} = \frac{0.9}{20 \times 0.1} = \frac{0.9}{2} = 0.45 \text{ hours} = 27 \text{ minutes}$$

2. **External Jobs**:
   $$W_{q,\text{external}} = \frac{0.75}{20 (1 - 0.75)} = \frac{0.75}{20 \times 0.25} = \frac{0.75}{5} = 0.15 \text{ hours} = 9 \text{ minutes}$$

### Average Time a Job Spends in the System (\(W\)):
The average time a job spends in the system is the sum of the average waiting time in the queue and the average service time:
$$W = W_q + \frac{1}{\mu}$$

1. **Internal Jobs**:
   $$W_{\text{internal}} = 27 \text{ minutes} + 3 \text{ minutes} = 30 \text{ minutes}$$

2. **External Jobs**:
   $$W_{\text{external}} = 9 \text{ minutes} + 3 \text{ minutes} = 12 \text{ minutes}$$

### Summary:
- **Average Waiting Time for Internal Jobs (\(W_{q,\text{internal}}\))**: 27 minutes
- **Average Waiting Time for External Jobs (\(W_{q,\text{external}}\))**: 9 minutes
- **Average Time in the System for Internal Jobs (\(W_{\text{internal}}\))**: 30 minutes
- **Average Time in the System for External Jobs (\(W_{\text{external}}\))**: 12 minutes

These calculations show how the system's performance varies for internal and external jobs when each computer is dedicated to one type of job¹².

Would you like to explore another scenario or delve into a different aspect of queueing theory?

Source : conversation avec Copilot, 4/10/2024
(1) CPU Scheduling: Arrival, Burst, Completion, Turnaround, Waiting, and .... https://www.baeldung.com/cs/cpu-scheduling.
(2) CPU Scheduling Criteria - GeeksforGeeks. https://www.geeksforgeeks.org/cpu-scheduling-criteria/.
(3) Lecture 11: CPU Scheduling - Yale University. https://flint.cs.yale.edu/cs422/lectureNotes/Fall18/L11.pdf.

%%%%%%%%%%%%%%%%%%%%%%%%%%%%%%%%%%%%%%%%%%%%%%%%%%%%%%%%%%%%%%%%%%%%%%%%%%%%%%%%%%%%%%%%%
\subsection*{Part (b)}
\begin{itemize}
    \item (b) When two computers handle both types of jobs, we model the computing service as an M/M/2 
queueing system with 
\item Now, substituting s = 2 in Eqs. (9.20), (9.22), (9.24), and (9.25), we get 
\item Thus, from Eq. (9.54), the average waiting time per job when both computers handle both types of jobs 
is given by 
2(%) %= mu 11 = 6.39 min - (%)21 
\item From these results, we see that it is more efficient for both computers to handle both types of jobs.
\end{itemize}


In an M/M/s queueing system, where there are \(s\) servers, the arrival process follows a Poisson distribution with rate \(\lambda\), and the service times are exponentially distributed with rate \(\mu\). Here are the key formulas to determine the average number of customers in the system (\(L\)) and the average number of customers in the queue (\(L_q\)):

### Utilization Factor (\(\rho\)):
$$\rho = \frac{\lambda}{s \mu}$$

### Average Number of Customers in the System (\(L\)):
$$L = L_q + \frac{\lambda}{\mu}$$

### Average Number of Customers in the Queue (\(L_q\)):
$$L_q = \frac{P_0 (\lambda/\mu)^s \rho}{s! (1 - \rho)^2}$$

Where \(P_0\) is the probability that there are zero customers in the system, given by:
$$P_0 = \left[ \sum_{n=0}^{s-1} \frac{(\lambda/\mu)^n}{n!} + \frac{(\lambda/\mu)^s}{s! (1 - \rho)} \right]^{-1}$$

### Example Calculation:
Let's assume:
- Arrival rate (\(\lambda\)): 30 customers per hour
- Service rate (\(\mu\)): 10 customers per hour
- Number of servers (\(s\)): 3

1. **Utilization Factor**:
   $$\rho = \frac{30}{3 \times 10} = 1$$

2. **Probability of Zero Customers (\(P_0\))**:
   $$P_0 = \left[ \sum_{n=0}^{2} \frac{(30/10)^n}{n!} + \frac{(30/10)^3}{3! (1 - 1)} \right]^{-1}$$
   Since \(\rho = 1\), the system is at full capacity, and \(P_0\) calculation needs to be adjusted for practical scenarios where \(\rho < 1\).

3. **Average Number of Customers in the Queue (\(L_q\))**:
   $$L_q = \frac{P_0 (30/10)^3 \times 1}{3! (1 - 1)^2}$$
   This formula simplifies under practical conditions where \(\rho < 1\).

4. **Average Number of Customers in the System (\(L\))**:
   $$L = L_q + \frac{30}{10}$$

These formulas help in understanding the performance of an M/M/s queueing system¹².

Would you like to see a more detailed example or explore another aspect of queueing theory?

Source : conversation avec Copilot, 4/10/2024
(1) M/M/s Queueing Model - Real Statistics Using Excel. https://real-statistics.com/probability-functions/queueing-theory/m-m-s-queueing-model/.
(2) Queuing Theory Tutorial - M/M/s Queuing System - Revoledu. https://people.revoledu.com/kardi/tutorial/Queuing/MMs-Queuing-System.html.
(3) The M/M/S/M+ Queuing System with Clients Abandonment - Springer. https://link.springer.com/content/pdf/10.1007/s41096-024-00200-0.pdf.
(4) The M/M/S/M+ Queuing System with Clients Abandonment. https://link.springer.com/article/10.1007/s41096-024-00200-0.
(5) undefined. http://people.revoledu.com/kardi/tutorial/Queuing/.








\end{document}
