
\subsection*{Question 3}
The income of a technician (in thousands) is $X_1 \sim \text{Normal}(\mu=30,\sigma=2)$. The income of an engineer is $X_2 \sim \text{Normal}(\mu=40,\sigma=3.5)$. \\[-0.2cm]

\item  Calculate the probability that an engineer earns more than a technician. 
 \item  Calculate 90\% limits for the difference in their income. 
 \item  For a group of 25 technicians, calculate the probability that the average wage is less than 30500, i.e., $\Pr(\,\overline{\!X}_1 < 30.5)$. 
 \item  In a group of 10 engineers, what is the probability that \emph{at least two} of them earn more than 45000? (hint: binomial with $p = \Pr(X_2 > 45)$) 
 {\bf(e)} For a sample of 30 technicians and 35 engineers, calculate the 80\% limits for the difference in their average wages.



\section{KB Tutorial 3}
\subsection*{Question 1}
Assume that there are three different routes to get to a particular location: $R_1$, $R_2$ and $R_3$. You take $R_1$ 75\% of the time, $R_2$ 20\% of the time and $R_3$ the rest of the time. If you take $R_1$, there is a 90\% chance that you will be on time; if you take $R_2$, there is a 50\% chance that you will be on time and, if you take $R_3$, there is a 70\% chance that you will be on time. \\[0.1cm]
Let $T$ represent on time.\\[-0.2cm]

\item  If $T$ represents ``on time'', what notation would we use for ``late''? 
 \item  What is the value of $\Pr(R_1 \cap R_2)$? 
 \item  Calculate the probability of being on time. 
 \item  \emph{Given that} you \emph{are} on time, calculate the probabilities of having used each of the routes. 
 {\bf(e)} Given that you are late, what is the probability that you used $R_1$?




