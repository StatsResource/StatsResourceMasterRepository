\documentclass[]{report}

\voffset=-1.5cm
\oddsidemargin=0.0cm
\textwidth = 480pt

\usepackage{framed}
\usepackage{subfiles}
\usepackage{graphics}
\usepackage{newlfont}
\usepackage{eurosym}
\usepackage{amsmath,amsthm,amsfonts}
\usepackage{amsmath}
\usepackage{color}
\usepackage{amssymb}
\usepackage{multicol}
\usepackage[dvipsnames]{xcolor}
\usepackage{graphicx}
\begin{document}


\section{Other useful Continuous Distributions}
\begin{description}
\item [Pareto distribution] for a single such quantity whose log is exponentially distributed; 
the prototypical power law distribution
\item [Log-normal distribution] for a single such quantity whose log is normally distributed
\item [Weibull distribution]
\end{description}




\noindent \textbf{ Continuous Distributions}
\begin{multicols}{2}
\begin{itemize}
\item[(a)] The continuous uniform distribution
\item[(b)] The exponential distribution
\item[(c)] The Weibull distribution
\item[(d)] The Pareto distribution
\end{itemize}
\end{multicols}






Gamma Distribution[edit]
Gamma
Probability density function
Probability density plots of gamma distributions
Cumulative distribution function
Cumulative distribution plots of gamma distributions
Parameters

\begin{verbatim}
\scriptstyle k \;>\; 0 shape
\scriptstyle \theta \;>\; 0\, scale
Support\scriptstyle x \;\in\; (0,\, \infty)\!
PDF\scriptstyle \frac{1}{\Gamma(k) \theta^k} x^{k \,-\, 1} e^{-\frac{x}{\theta}}\,\!
CDF\scriptstyle \frac{1}{\Gamma(k)} \gamma\left(k,\, \frac{x}{\theta}\right)\!
Mean\scriptstyle \operatorname{E}[ X] = k \theta \!
\scriptstyle \operatorname{E}[\ln X] = \psi(k) +\ln(\theta)\!
(see digamma function)
MedianNo simple closed form
Mode\scriptstyle (k \,-\, 1)\theta \text{ for } k \;>\; 1\,\!
Variance\scriptstyle\operatorname{Var}[ X] = k \theta^2\,\!
\scriptstyle\operatorname{Var}[\ln X] = \psi_1(k)\!
(see trigamma function )
Skewness\scriptstyle \frac{2}{\sqrt{k}}\,\!
Ex. kurtosis\scriptstyle \frac{6}{k}\,\!
Entropy\scriptstyle \begin{align}
\scriptstyle k &\scriptstyle \,+\, \ln\theta \,+\, \ln[\Gamma(k)]\\
\scriptstyle   &\scriptstyle \,+\, (1 \,-\, k)\psi(k)
\end{align}
\end{verbatim}




\section{Other useful Continuous Distributions}
%http://www.computing.dcu.ie/~jhorgan/chapter16slides.pdf

%Related to positive real-valued quantities that grow exponentially (e.g. prices, incomes, populations)[edit]
%---------------------------------------------------------------------------------%






\section{The General Pareto Distribution}

\begin{itemize}
\item As with many other distributions, the Pareto distribution is often generalized by adding a scale parameter. Thus, suppose that Z has the basic Pareto distribution with shape parameter a>0.
\item If b>0, the random variable X=bZ has the Pareto distribution with shape parameter a and scale parameter b. Note that X takes values in the interval $[b, \infty)$.

\item Analogies of the results given above follow easily from basic properties of the scale transformation.

\end{itemize}


%%- \subsection{The Pareto Distribution}

\begin{itemize}
\item The probability density function is
\[ f(x)=abaxa+1,b\leq x< \infty \]
\item The distribution function is
\[F(x)=1-(bx)a,b\leq x< \infty \]


\item The quantile function is
\[F-1(p)=b(1-p)1/a,0\leq p<1\]
\end{itemize}


\subsection{Cumulative distribution function}
From the definition, the cumulative distribution function of a Pareto random variable with parameters $\alpha$ and xm is
\[F_X(x) = \begin{cases}
1-\left(\frac{x_\mathrm{m}}{x}\right)^\alpha & \mbox{for } x \ge x_\mathrm{m}, \\
0 & \mbox{for }x < x_\mathrm{m}.
\end{cases}
\]


When plotted on linear axes, the distribution assumes the familiar J-shaped curve which approaches each of the orthogonal axes asymptotically. All segments of the curve are self-similar (subject to appropriate scaling factors). When plotted in a log-log plot, the distribution is represented by a straight line.
\subsection{Probability density function}
It follows (by differentiation) that the probability density function is
\[
f_X(x)= \begin{cases} \alpha\,\dfrac{x_\mathrm{m}^\alpha}{x^{\alpha+1}} & \mbox{for }x \ge x_\mathrm{m}, \\[12pt] 0 & \mbox{for } x < x_\mathrm{m}. \end{cases} 
\]



% Moments and characteristic function[edit]
The expected value of a random variable following a Pareto distribution is
\[
E(X)= \begin{cases} \infty & \text{if }\alpha\le 1, \\ \frac{\alpha x_\mathrm{m}}{\alpha-1} & \text{if }\alpha>1. \end{cases}
\]
The variance of a random variable following a Pareto distribution is
\[
\mathrm{Var}(X)= \begin{cases} \infty & \text{if }\alpha\in(1,2], \\ \left(\frac{x_\mathrm{m}}{\alpha-1}\right)^2 \frac{\alpha}{\alpha-2} & \text{if }\alpha>2. \end{cases}
(If \alpha\le 1, the variance does not exist.)
\]






The Pareto distribution is 
a continuous distribution with the probability density function (pdf):
\[
f(x; \alpha, \beta) = \alpha\beta\alpha / x\alpha+ 1
\]
For shape parameter $\alpha > 0$, and location parameter $\beta > 0$, and $\alpha > 0$.


% The following graph illustrates how the PDF varies with the location parameter \beta:

\subsection{Cumulative distribution function}

From the definition, the cumulative distribution function of a Pareto random variable with parameters $\alpha$ and $x_m$ is
\[F_X(x) = \begin{cases}
1-\left(\frac{x_\mathrm{m}}{x}\right)^\alpha & \mbox{for } x \ge x_\mathrm{m}, \\
0 & \mbox{for }x < x_\mathrm{m}.
\end{cases}
\]


{
%%- \subsection{Probability density function}
\begin{itemize}
%When plotted on linear axes, the distribution assumes the familiar J-shaped curve which approaches each of the orthogonal axes asymptotically. All segments of the curve are self-similar (subject to appropriate scaling factors). When plotted in a log-log plot, the distribution is represented by a straight line.

\item It follows (by differentiation) that the probability density function is
\[
f_X(x)= \begin{cases} \alpha\,\dfrac{x_\mathrm{m}^\alpha}{x^{\alpha+1}} & \mbox{for }x \ge x_\mathrm{m}, \\[12pt] 0 & \mbox{for } x < x_\mathrm{m}. \end{cases} 
\]

% Moments and characteristic function[edit]
\item The expected value of a random variable following a Pareto distribution is
\[
E(X)= \begin{cases} \infty & \mbox{if }\alpha\le 1, \\ \frac{\alpha x_\mathrm{m}}{\alpha-1} & \mbox{if }\alpha>1. \end{cases}
\]


\item The variance of a random variable following a Pareto distribution is
\[
\mathrm{Var}(X)= \begin{cases} \infty & \mbox{if }\alpha\in(1,2], \\ \left(\frac{x_\mathrm{m}}{\alpha-1}\right)^2 \frac{\alpha}{\alpha-2} & \mbox{if }\alpha>2. \end{cases}
(If \alpha\le 1, the variance does not exist.)
\]



\item The Pareto distribution is 
a continuous distribution with the probability density function (pdf):
\[
f(x; \alpha, \beta) = \alpha\beta\alpha / x\alpha+ 1
\]
\item For shape parameter $\alpha > 0$, and location parameter $\beta > 0$, and $\alpha > 0$.
\end{itemize}

\begin{itemize}
\item The Pareto distribution often 
describes the larger compared
to the smaller. 
\item A classic example is that 
80\% of the wealth is owned by 20\% of the population.

% The following graph illustrates how the PDF varies with the location parameter \beta:
\end{itemize}

\newpage
\section{The General Pareto Distribution}

\begin{itemize}
\item As with many other distributions, the Pareto distribution is often generalized by adding a scale parameter. Thus, suppose that Z has the basic Pareto distribution with shape parameter a>0.
\item If b>0, the random variable X=bZ has the Pareto distribution with shape parameter a and scale parameter b. Note that X takes values in the interval $[b, \infty)$.

\item Analogies of the results given above follow easily from basic properties of the scale transformation.

\end{itemize}


%%- \subsection{The Pareto Distribution}

\begin{itemize}
\item The probability density function is
\[ f(x)=abaxa+1,b\leq x< \infty \]
\item The distribution function is
\[F(x)=1-(bx)a,b\leq x< \infty \]


\item The quantile function is
\[F-1(p)=b(1-p)1/a,0\leq p<1\]
\end{itemize}


\noindent \textbf{Cumulative distribution function}
From the definition, the cumulative distribution function of a Pareto random variable with parameters $\alpha$ and xm is
\[F_X(x) = \begin{cases}
1-\left(\frac{x_\mathrm{m}}{x}\right)^\alpha & \mbox{for } x \ge x_\mathrm{m}, \\
0 & \mbox{for }x < x_\mathrm{m}.
\end{cases}
\]


When plotted on linear axes, the distribution assumes the familiar J-shaped curve which approaches each of the orthogonal axes asymptotically. All segments of the curve are self-similar (subject to appropriate scaling factors). When plotted in a log-log plot, the distribution is represented by a straight line.
\subsection{Probability density function}
It follows (by differentiation) that the probability density function is
\[
f_X(x)= \begin{cases} \alpha\,\dfrac{x_\mathrm{m}^\alpha}{x^{\alpha+1}} & \mbox{for }x \ge x_\mathrm{m}, \\[12pt] 0 & \mbox{for } x < x_\mathrm{m}. \end{cases} 
\]


\subsection{The Weibull Distribution}






\begin{itemize}
\item The two-parameter Weibull distribution is the most widely used distribution for life data analysis. Apart from the 2-parameter Weibull distribution, the 3-parameter and the 1-parameter Weibull distribution are often used for detailed analysis.
\item 
The 2-parameter Weibull cumulative distribution function (CDF), has the explicit equation:
\end{itemize}

\end{document}
%--------------------------%
%--------------------------%

\subsection{The Weibull Distribution}



\begin{itemize}
\item F(t) = Probability of failure at time t;
\item t = time, cycles, miles, or any appropriate parameter;
\item = characteristic life or scale parameter;
\item = slope or shape parameter.
\end{itemize}


Note that for no other, apart from the Exponential , such an explicit equation is available. Below the F(t) equations for the




%--------------------------%
%--------------------------%

\subsection{Weibull Distribution}

Normal and the Log-Normal distributions.

The Weibull Probability Density Function (PDF) is:



%--------------------------%
%--------------------------%

\subsection{Weibull Distribution}

Weibull Parameters


The two parameters of the 2-Parameter Weibull distribution are:
, the slope or shape parameter, and
, the characteristic life or scale parameter.

%--------------------------%
%--------------------------%

Shape parameter
BETA () shows how the failure rate develops in time. Failure-modes related with Infant Mortality, Random, or Wear-Out have significant different Beta values. is named the shape parameter as it determines which member of the Weibull family of distributions 
is most appropriate. 
Different members have 
different shaped probability 
density functions (PDF

%--------------------------%

Hypergeometric Distribution[edit]
Hypergeometric
Notationh(k) = {{{m \choose k} {{N-m} \choose {n-k}}}\over {N \choose n}} 






\end{document}
