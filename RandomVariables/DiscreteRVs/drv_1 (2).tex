


%------------------------------------------------------------%
%-----------------------------------------------------------------%
\frame{
\frametitle{Last lecture}
In the last lecture we looked at how to compute
\begin{itemize}
\item the expected value
\item the variance 
\end{itemize}
of a discrete random variable. In our example, we considered the experiment of throwing a fair die.

}

%-----------------------------------------------------------------%
\frame{
\frametitle{Crooked die}
\begin{itemize}
\item Consider the random experiment of rolling a `crooked' six-sided die, i.e. the outcomes of the throw occur with different probabilities.

\item Suppose we have a die with which an outcome `5' or `6' is twice as likely to occur compared to the other numbers.

\item What is the probability of each outcome?

\begin{itemize}  \item Remark: The ratio of outcomes is 1:1:1:1:2:2
\end{itemize}

\item The probability distributiion can be tabulated as follows

\begin{center}
\begin{tabular}{|c||c|c|c|c|c|c|}
\hline
$x_i$  & 1 & 2 & 3 & 4 & 5 & 6 \\\hline
$p(x_i)$ & 1/8 & 1/8& 1/8 & 1/8& 2/8 & 2/8\\
\hline
\end{tabular}
\end{center}

\item What is expected value and variance of the outcomes?
\end{itemize}
}

%------------------------------------------------------------------%
\frame{
\frametitle{Variance of the crooked die}

Recall the formula for computing the variance of a discrete random variable:

\[ V(x) = E(X^2) - E(X)^2 \]

We must compute $E(X^2)$

\begin{center}
\begin{tabular}{|c||c|c|c|c|c|c|}
\hline
$x_i$  & 1 & 2 & 3 & 4 & 5 & 6 \\\hline
$x^2_i$  & 1 & 4 & 9 & 16 & 25 & 36 \\\hline
$p(x_i)$ & 1/8 & 1/8& 1/8 & 1/8& 2/8 & 2/8\\
\hline
\end{tabular}
\end{center}

\[E(X) = (0 \times 1/8) + (1 \times 1/8) +  \ldots + (25 \times 2/8) + (36 \times 2/8) = {32 \over 8} = 4\]

}
%------------------------------------------------------------------%
\frame{
\frametitle{Expected value of the crooked die}

What is the variance?


}

\end{document}
%---------------------------------------------------------------------------%
\frame{
\frametitle{Today's Class}
\begin{itemize}
\item Discrete Probability Distributions
\item Probability Mass Function
\item The Binomial Distribution
\item The Poisson Probability Distribution
\end{itemize}
}

%---------------------------------------------------------------------------%
\frame{
\frametitle{Probability Mass Function}
\begin{itemize} \item a probability mass function (pmf) is a function that gives the probability that a discrete random variable is exactly equal to some value. \item The probability mass function is often the primary means of defining a discrete probability distribution \end{itemize}
}
%---------------------------------------------------------------------------%


\frame{
\frametitle{The Binomial Probability Distribution}
\begin{itemize}
\item The number of independent trials is denoted $n$.
\item The probability of a `success' is $p$
\item The expected number of `successes' from $n$ trials is $E(X) = np$
\end{itemize}
}


\frame{
\frametitle{Binomial Experiment}
A binomial experiment (also known as a Bernoulli trial) is a statistical experiment that has the following properties:
\begin{itemize}
\item The experiment consists of n repeated trials.
\item Each trial can result in just two possible outcomes. We call one of these outcomes a success and the other, a failure.
\item The probability of success, denoted by P, is the same on every trial.
\item The trials are independent; that is, the outcome on one trial does not affect the outcome on other trials.
\end{itemize}
}
%--------------------------------------------------------------------------------------%
\frame{
Consider the following statistical experiment. You flip a coin 2 times and count the number of times the coin lands on heads. This is a binomial experiment because:
\begin{itemize}
\item The experiment consists of repeated trials. We flip a coin 2 times.
\item Each trial can result in just two possible outcomes - heads or tails.
\item The probability of success is constant - 0.5 on every trial.
\item The trials are independent; that is, getting heads on one trial does not affect whether we get heads on other trials.
\end{itemize}
}

%--------------------------------------------------------------------------------------%
\frame{
\frametitle{ Binomial Example 1}

Suppose a die is tossed 5 times. What is the probability of getting exactly 2 fours?

Solution: This is a binomial experiment in which the number of trials is equal to 5, the number of successes is equal to 2, and the probability of success on a single trial is 1/6 or about 0.167. Therefore, the binomial probability is:

\[P(X=2) = ^5C_2 \times (1/6)^2 \times (5/6)^3 = 0.161\]
}

