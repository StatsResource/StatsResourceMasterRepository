

Categorical vs. Quantitative Variables
Variables can be classified as categorical (aka, qualitative) or quantitative (aka, numerical).
Categorical. Categorical variables take on values that are names or labels. The color of a ball (e.g., red, green, blue) or the breed of a dog (e.g., collie, shepherd, terrier) would be examples of categorical variables. 
Quantitative. Quantitative variables are numerical. They represent a measurable quantity. For example, when we speak of the population of a city, we are talking about the number of people in the city - a measurable attribute of the city. Therefore, population would be a quantitative variable.
In algebraic equations, quantitative variables are represented by symbols (e.g., x, y, or z).
Discrete vs. Continuous Variables
Quantitative variables can be further classified as discrete or continuous. If a variable can take on any value between its minimum value and its maximum value, it is called a continuous variable; otherwise, it is called a discrete variable.
Some examples will clarify the difference between discrete and continouous variables.
Suppose the fire department mandates that all fire fighters must weigh between 150 and 250 pounds. The weight of a fire fighter would be an example of a continuous variable; since a fire fighter's weight could take on any value between 150 and 250 pounds. 
Suppose we flip a coin and count the number of heads. The number of heads could be any integer value between 0 and plus infinity. However, it could not be any number between 0 and plus infinity. We could not, for example, get 2.3 heads. Therefore, the number of heads must be a discrete variable.
Univariate vs. Bivariate Data
Statistical data is often classified according to the number of variables being studied.
Univariate data. When we conduct a study that looks at only one variable, we say that we are working with univariate data. Suppose, for example, that we conducted a survey to estimate the average weight of high school students. Since we are only working with one variable (weight), we would be working with univariate data.
Bivariate data. When we conduct a study that examines the relationship between two variables, we are working with bivariate data. Suppose we conducted a study to see if there were a relationship between the height and weight of high school students. Since we are working with two variables (height and weight), we would be working with bivariate data.
Sample Question on introductory Statistics

Problem 1
Which of the following statements are true?
I. All variables can be classified as quantitative or categorical variables. 
II. Categorical variables can be continuous variables. 
III. Quantitative variables can be discrete variables.
(A) I only 
(B) II only 
(C) III only 
(D) I and II 
(E) I and III
Solution
The correct answer is (E). All variables can be classified as quantitative or categorical variables. Discrete variables are indeed a category of quantitative variables. Categorical variables, however, are not numeric. Therefore, they cannot be classified as continuous variables.
Introductory Probability Notes
