\documentclass[a4paper,12pt]{article}
%%%%%%%%%%%%%%%%%%%%%%%%%%%%%%%%%%%%%%%%%%%%%%%%%%%%%%%%%%%%%%%%%%%%%%%%%%%%%%%%%%%%%%%%%%%%%%%%%%%%%%%%%%%%%%%%%%%%%%%%%%%%%%%%%%%%%%%%%%%%%%%%%%%%%%%%%%%%%%%%%%%%%%%%%%%%%%%%%%%%%%%%%%%%%%%%%%%%%%%%%%%%%%%%%%%%%%%%%%%%%%%%%%%%%%%%%%%%%%%%%%%%%%%%%%%%
\usepackage{eurosym}
\usepackage{vmargin}
\usepackage{amsmath}
\usepackage{multicol}
\usepackage{graphics}
\usepackage{epsfig}
\usepackage{enumerate}
\usepackage{framed}
\usepackage{subfigure}
\usepackage{fancyhdr}

\setcounter{MaxMatrixCols}{10}
%TCIDATA{OutputFilter=LATEX.DLL}
%TCIDATA{Version=5.00.0.2570}
%TCIDATA{<META NAME="SaveForMode" CONTENT="1">}
%TCIDATA{LastRevised=Wednesday, February 23, 2011 13:24:34}
%TCIDATA{<META NAME="GraphicsSave" CONTENT="32">}
%TCIDATA{Language=American English}

\pagestyle{fancy}
\setmarginsrb{20mm}{0mm}{20mm}{25mm}{12mm}{11mm}{0mm}{11mm}
\lhead{StatsResource} \rhead{Statistics for Computing}
\chead{Information Theory}
%\input{tcilatex}

\begin{document}

\section*{Question 4. (25 marks) Two Sample Inference Procedures }

%\subsection*{Part A (10 Marks)}
%
%\begin{itemize}
%	\item In a survey of perceived health risks in Dublin, each member of a random
%	sample of 200 people was asked the question "\textit{When buying food, do you check the
%		pack for artificial additives?}". 
%	
%	\item The researchers wanted to discover whether females or
%	males were more likely to check for artificial additives when buying food.
%\end{itemize}
%
%
%\begin{center}
%	\begin{tabular}{|c|c|c|}
%		\hline 
%		&  Yes & No \\ 
%		
%		\hline 
%		Female	& 60 & 60 \\ 
%		\hline 
%		Male	& 32 & 48  \\ 
%		\hline 
%	\end{tabular} 
%\end{center}
%Using a 5\% significance level, test for a difference between the percentages of males and females responding ``Yes" to the question about checking for artificial additives.
%
%	
%\begin{itemize}
%\item Clearly state your null and alternative hypotheses.
%	\item[(ii)](4 Marks) Compute the Test Statistic.
%\item Discuss your conclusion to this test, supporting your statement with reference to appropriate values.
%\end{itemize}
%	
