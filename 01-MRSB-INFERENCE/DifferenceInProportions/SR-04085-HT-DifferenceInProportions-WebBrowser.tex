	\documentclass[a4paper,12pt]{article}
%%%%%%%%%%%%%%%%%%%%%%%%%%%%%%%%%%%%%%%%%%%%%%%%%%%%%%%%%%%%%%%%%%%%%%%%%%%%%%%%%%%%%%%%%%%%%%%%%%%%%%%%%%%%%%%%%%%%%%%%%%%%%%%%%%%%%%%%%%%%%%%%%%%%%%%%%%%%%%%%%%%%%%%%%%%%%%%%%%%%%%%%%%%%%%%%%%%%%%%%%%%%%%%%%%%%%%%%%%%%%%%%%%%%%%%%%%%%%%%%%%%%%%%%%%%%
\usepackage{eurosym}
\usepackage{vmargin}
\usepackage{framed}
\usepackage{amsmath}
\usepackage{graphics}
\usepackage{epsfig}
\usepackage{subfigure}
\usepackage{enumerate}
\usepackage{fancyhdr}

\setcounter{MaxMatrixCols}{10}
%TCIDATA{OutputFilter=LATEX.DLL}
%TCIDATA{Version=5.00.0.2570}
%TCIDATA{<META NAME="SaveForMode"CONTENT="1">}
%TCIDATA{LastRevised=Wednesday, February 23, 201113:24:34}
%TCIDATA{<META NAME="GraphicsSave" CONTENT="32">}
%TCIDATA{Language=American English}

\pagestyle{fancy}
\setmarginsrb{20mm}{0mm}{20mm}{25mm}{12mm}{11mm}{0mm}{11mm}
\lhead{MS4222} \rhead{Kevin O'Brien} \chead{Confidence Intervals} %\input{tcilatex}

\begin{document}
%------------------------------------------------------------- %


\section*{Difference of Proportions : Worked Example}
\begin{itemize} \item
	A study finds that a percentage of $40\%$ of IT users out of a random sample of 400 in a large
	community preferred one web browser to all others. \item In another large community, $30\%$ of IT users out of a random sample
	of 300 prefer the same web browser. \item Compute a 95 percent confidence interval for the difference in the proportion of IT users who prefer this particular web browser. \end{itemize}


%--------------------------------------------------------%

\subsection*{Solution}





% \subsection{Confidence Interval}
\begin{itemize}
	\item The point estimate is the difference in two proportions i.e. $\hat{p}_1 - \hat{p}_2$ = $40 \% - 30 \% = 10 \%$
	\item We have a large sample, and the confidence level is $95\%$. Therefore the quantile is 1.96.
\item Now we can compute the standard error.

\[ S.E. (\hat{p}_1 - \hat{p}_2) =
\sqrt{ \left[{\hat{p}_1 \times (100 - \hat{p}_1) \over n_1}\right] + \left[{\hat{p}_2 \times (100 - \hat{p}_2) \over n_2}\right] } [\%] \]

\[ S.E. (\hat{p}_1 - \hat{p}_2) =
\sqrt{ \left[{40 \times 60 \over 400}\right] + \left[{30 \times 70 \over 300}\right] }  = \sqrt{ \left[{2400 \over 400}\right] + \left[{2100\over 300}\right] } \]

\[ S.E. (\hat{p}_1 - \hat{p}_2)
= \sqrt{ 6 + 7 } = 3.6\% \]

	\item We can now compute the confidence interval for the difference of proportions:
	\[ 10\% \pm (1.96 \times 3.6 \%)  =\; 10\% \pm 7.05 \% = \;(2.95\%, 17.05\%) \]
	
\end{itemize}


%--------------------------------------------------------%


\begin{itemize}
\item SE = $\sqrt{ [p_1 \times (1 - p_1) / n_1] + [p_2 \times (1 - p_2) / n_2] } $
\item SE = $\sqrt{ [0.40 \times 0.60 / 400] + [0.30 \times 0.70 / 300] } $
\item SE  = $\sqrt{[ (0.24 / 400) + (0.21 / 300) ]}$ = $\sqrt{(0.0006 + 0.0007)}$ = $\sqrt{0.0013} = 0.036$
\end{itemize}




\subsubsection{Confidence Interval}
\begin{itemize}
\item The point estimate is the difference in two proportions i.e. $\hat{p}_1 - \hat{p}_2$ = $40 \% - 30 \% = 10 \%$
\item We have a large sample, and the confidence level is $95\%$. Therefore the quantile is 1.96.
\item We can now compute the confidence interval for the difference of proportions:
\[ 10\% \pm (1.96 \times 3.6 \%)  =\; 10\% \pm 7.05 \% = \;(2.95\%, 17.05\%) \]

\end{itemize}









\section{Confidence Interval}
\begin{itemize}


\item We are 95\% confident that there is a difference in usage rates between the communities.\medskip \item We are 95\% confidence that this difference is at least 2.95\%, and as much as 17\%.
\item \textbf{Remark: }\textit{ 0\% is not in that range of values of the confidence interval.}
\end{itemize}
%--------------------------------------------------------%
\end{document}
