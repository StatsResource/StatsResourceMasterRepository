
\documentclass[a4paper,12pt]{article}
%%%%%%%%%%%%%%%%%%%%%%%%%%%%%%%%%%%%%%%%%%%%%%%%%%%%%%%%%%%%%%%%%%%%%%%%%%%%%%%%%%%%%%%%%%%%%%%%%%%%%%%%%%%%%%%%%%%%%%%%%%%%%%%%%%%%%%%%%%%%%%%%%%%%%%%%%%%%%%%%%%%%%%%%%%%%%%%%%%%%%%%%%%%%%%%%%%%%%%%%%%%%%%%%%%%%%%%%%%%%%%%%%%%%%%%%%%%%%%%%%%%%%%%%%%%%
\usepackage{eurosym}
\usepackage{vmargin}
\usepackage{amsmath}
\usepackage{graphics}
\usepackage{epsfig}
\usepackage{enumerate}
\usepackage{multicol}
\usepackage{subfigure}
\usepackage{fancyhdr}
\usepackage{listings}
\usepackage{framed}
\usepackage{graphicx}
\usepackage{amsmath}
\usepackage{chngpage}
%\usepackage{bigints}

\usepackage{vmargin}
% left top textwidth textheight headheight
% headsep footheight footskip
\setmargins{2.0cm}{2.5cm}{16 cm}{22cm}{0.5cm}{0cm}{1cm}{1cm}
\renewcommand{\baselinestretch}{1.3}

\setcounter{MaxMatrixCols}{10}

\begin{document}



Question 4. 
A study was carried out to determine the proportion of students who owned
mobile devices, such as i-phones in a number of European countries.
Out of 150 Austrian students who took part in the survey, 100 stated that they owned mobile devices.
Out of 100 Irish students who took part in the survey, 70 stated that they owned mobile devices.

Answer the following questions. You may assume a significance level of 5%

 Provide an estimate for the proportion of students who owned mobile devices in both Austria and Ireland.
(1 mark) State the 95% confidence intervals for these estimates.
(1 mark) Hence determine an estimate for the difference in population proportions for both countries.
(1 mark) State the 95% confidence intervals for the estimate for difference in proportions.
(2  Marks) Formally state the null and alternative hypothesis for a test of significance for this difference in proportion.
 With reference to both the critical value and confidence interval, state your conclusions about this test of significance.  

%----------------------------------------------------------------------------------------------------%
\section{Inferences around two proportions}

{
\subsection{Assumptions}

\begin{itemize}
\item We have proportions from two independent simple random samples
\item For both samples the conditions $np \geq 5$ $ n(1-p) \geq 5$ are met.
\end{itemize}
For population $1$, let
\begin{itemize}
\item $p_1$  population proportion
\item $n_1$ sample size
\item $x_1$ number of successes in sample 1
\item $hat{p}_1$ is the sample proportion, an estimate for $p_1$.
\end{itemize}



\subsection*{Part (a)}

\begin{itemize}
\item
\item
\end{itemize}

%%%%%%%%%%%%%%%%%%%%%%%%%%%%%%%%%%%%%%


\subsection*{Part (b)}

\begin{itemize}
\item
\item
\end{itemize}

%%%%%%%%%%%%%%%%%%%%%%%%%%%%%%%%%%%%%%



\subsection*{Part (c)}

\begin{itemize}
\item
\item
\end{itemize}