
%-----------------------------------------------------------%

\begin{frame}
\frametitle{CI for Proportion: Example (1)}

\begin{itemize}
\item $\hat{p}  = 0.62$
\item Sample Size $n=250$
\item Confidence level $1-\alpha$ is $95\%$
\end{itemize}

\end{frame}
%-----------------------------------------------------------%

\begin{frame}\frametitle{CI for Proportion: Example (2)}

\begin{itemize}
\item First, lets determine the quantile.
\item The sample size is large, so we will use the Z distribution.
\item (Alternatively we can uses the $t-$ distribution with $\infty$ degrees of freedom.
\end{itemize}

\end{frame}


%------------------------------------------------------------------------------%
\begin{frame}
Although the sample mean is useful as an unbiased estimator of the population mean, there is no way of
expressing the degree of accuracy of a point estimator. In fact, mathematically speaking, the probability that the
sample mean is exactly correct as an estimator of the population mean is $P = 0$.
\end{frame}
%------------------------------------------------------------------------------%
\begin{frame}
A confidence interval for the
mean is an estimate interval constructed with respect to the sample mean by which the likelihood that the interval
includes the value of the population mean can be specified.

The \emph{level of confidence} associated with a confidence interval indicates the long-run percentage
 of such intervals which would include the parameter being estimated.
\end{frame}
%------------------------------------------------------------------------------%
\begin{frame}
\begin{itemize}
\item Confidence intervals for the mean typically are constructed with the unbiased estimator $\bar{x}$ at the midpoint
of the interval.

\item The $\pm Z \sigma_x$ or $\pm Z s_x$ frequently is called the \textbf{\emph{margin of error}} for the confidence interval.
\end{itemize}
\end{frame}
%------------------------------------------------------------------------------%
\begin{frame}
We indicated that use of the normal distribution in estimating a population mean is warranted
for any large sample ($n > 30$), \textbf{and} for a small sample ($n \leq 30$) only if the population is normally distributed
and $\sigma$ is known.
\end{frame}
%------------------------------------------------------------------------------%
\begin{frame}
\begin{itemize}
\item Now we consider the situation in which the sample is small and the population is normally distributed,
but $\sigma$ is not known.
\item The distribution is a family of distributions, with
a somewhat different distribution associated with the degrees of freedom ($df$). For a confidence interval for the
population mean based on a sample of size n, $df = n - 1$.
\end{itemize}
\end{frame}

%------------------------------------------------------------------------------%
\frame{

\frametitle{Computing the Standard Error}

\[
S.E. (\hat{p}) \;=\; \sqrt{ {\hat(p) \times (100 -\hat{p} )\over n}}
\]



}

%
------------------------------------------------------------------------------%

\frame{
\[
\hat{p} = {144/200}  \times 100\%  = 0.72 \times 100\%.  = 72%
\]

$100\% - \hat{p} = 100\% - 72\% = 28\% $

}


%------------------------------------------------------------------------------%
\frame{
\textbf{Computing the Standard Error}

\[
S.E. (\hat{p}) \;=\; \sqrt{ {72 \times 28 \over 200 }}
\]


}
%------------------------------------------------------------------------------%

\end{document}
