\documentclass[a4paper,12pt]{article}
%%%%%%%%%%%%%%%%%%%%%%%%%%%%%%%%%%%%%%%%%%%%%%%%%%%%%%%%%%%%%%%%%%%%%%%%%%%%%%%%%%%%%%%%%%%%%%%%%%%%%%%%%%%%%%%%%%%%%%%%%%%%%%%%%%%%%%%%%%%%%%%%%%%%%%%%%%%%%%%%%%%%%%%%%%%%%%%%%%%%%%%%%%%%%%%%%%%%%%%%%%%%%%%%%%%%%%%%%%%%%%%%%%%%%%%%%%%%%%%%%%%%%%%%%%%%
\usepackage{eurosym}
\usepackage{vmargin}
\usepackage{amsmath}
\usepackage{graphics}
\usepackage{framed}
\usepackage{epsfig}
\usepackage{subfigure}
\usepackage{enumerate}
\usepackage{fancyhdr}

\setcounter{MaxMatrixCols}{10}
%TCIDATA{OutputFilter=LATEX.DLL}
%TCIDATA{Version=5.00.0.2570}
%TCIDATA{<META NAME="SaveForMode"CONTENT="1">}
%TCIDATA{LastRevised=Wednesday, February 23, 201113:24:34}
%TCIDATA{<META NAME="GraphicsSave" CONTENT="32">}
%TCIDATA{Language=American English}

\pagestyle{fancy}
\setmarginsrb{20mm}{0mm}{20mm}{25mm}{12mm}{11mm}{0mm}{11mm}
\lhead{StatsResource} \rhead{Inference Procedures} \chead{Dixon Q-Test for Outliers} %\input{tcilatex}

\begin{document}

\begin{enumerate}
\item The mean and the standard deviation of the number of marks obtained in the biology leaving certificate exam by randomly selected male and female pupils are described below:\\
\begin{center}
\begin{tabular}{|c|c|c|c|}

  \hline
  % after \\: \hline or \cline{col1-col2} \cline{col3-col4} ...
&Number&Mean&Std. Dev.\\ \hline
Male&10&57&12\\
Female&12&61&11\\
  \hline
\end{tabular}
\end{center}

Calculate a 95\% confidence interval for the difference between the mean number of marks obtained by males and females in the population of school leavers as a whole.


Test the hypothesis that males and females on average obtain the same mark in the biology leaving certificate exam. Use a significance level of $5\%$. You may assume that all required assumptions have been validated.\\% State your hypotheses clearly. What is the significance level of this test?
\bigskip
\begin{itemize}
\item[i.] Formally state the null and alternative hypotheses.
\item[ii.] Compute the Test Statistic.
\item[iii.] State the appropriate Critical Value for this hypothesis test.
\item[iv.] Discuss your conclusion to this test, supporting your statement with reference to appropriate values.
\end{itemize}

