

\item The average height of a sample of 16 students was 173cm with a variance of 144cm$^2$. The average height of the Irish population is 169cm. 
\begin{enumerate}[(a)]
\item Can it be stated at a significance level of 5\% that students are on average taller than the population as a whole? 
\item What assumption is used to carry out this test? Is this assumption reasonable?
\end{enumerate}


%----------------------------------------------------------------------- %
\item 
The average mass of a sample of 64 Irish teenagers (Let say - 18 year old males) was 73.5kg with a variance of 100kg$^2$. 
The average mass of an equivalent sample of 81 Japanese teenagers was 68.5kg with a variance of 81 kg$^2$. 
\begin{enumerate}[(a)]
\item Test the hypothesis that Irish students are larger (in terms of mass) than Japanese teenagers.
\item By calculating the appropriate $p-$value, test the null hypothesis that the mean mass of all Irish students is 70kg at significance levels of 5\%. 
\item Using the appropriate confidence interval, test the hypotheses that the average mass of all Irish students is a) 70kg, b) 72kg, c) 75kg at a significance level of 5\%.
\end{enumerate}


% \textbf{\textit{Standard Error Formula}}
% \[ S.E.(\bar{x}_1 - \bar{x}_2)  = \sqrt{\frac{s_1^2}{n_1} + \frac{s_2^2}{n_2}} \]
% \[ S.E.(\bar{x}_1 - \bar{x}_2)  = \sqrt{\frac{10^2}{64} + \frac{9^2}{81}}  = \sqrt{2.56} = 1.6\]
% \newpage


\item A study was carried out in which researchers collected crime data. Of those convicted of
arson, 50 were drinkers and 43 abstained. Of those convicted of fraud, 63 were drinkers and 144
abstained. Use a 0.01 level of significance to test the claim that the proportion of drinkers among
convicted arsonists is greater than the proportion of drinkers convicted of fraud.



