\documentclass[a4paper,12pt]{article}
%%%%%%%%%%%%%%%%%%%%%%%%%%%%%%%%%%%%%%%%%%%%%%%%%%%%%%%%%%%%%%%%%%%%%%%%%%%%%%%%%%%%%%%%%%%%%%%%%%%%%%%%%%%%%%%%%%%%%%%%%%%%%%%%%%%%%%%%%%%%%%%%%%%%%%%%%%%%%%%%%%%%%%%%%%%%%%%%%%%%%%%%%%%%%%%%%%%%%%%%%%%%%%%%%%%%%%%%%%%%%%%%%%%%%%%%%%%%%%%%%%%%%%%%%%%%
\usepackage{eurosym}
\usepackage{vmargin}
\usepackage{amsmath}
\usepackage{graphics}
\usepackage{epsfig}
\usepackage{framed}
\usepackage{subfigure}
\usepackage{enumerate}
\usepackage{fancyhdr}

\setcounter{MaxMatrixCols}{10}
%TCIDATA{OutputFilter=LATEX.DLL}
%TCIDATA{Version=5.00.0.2570}
%TCIDATA{<META NAME="SaveForMode"CONTENT="1">}
%TCIDATA{LastRevised=Wednesday, February 23, 201113:24:34}
%TCIDATA{<META NAME="GraphicsSave" CONTENT="32">}
%TCIDATA{Language=American English}

\pagestyle{fancy}
\setmarginsrb{20mm}{0mm}{20mm}{25mm}{12mm}{11mm}{0mm}{11mm}
\lhead{MS4222} \rhead{Kevin O'Brien} \chead{Computing Confidence Intervals For Means} %\input{tcilatex}

\begin{document}




\section*{Computing Confidence Intervals For Means}
Confidence limits are the lower and upper boundaries / values of a confidence interval, that is, the values which define the range of a confidence interval. The general structure of a confidence interval is as follows:

\[ \mbox{Point Estimate}  \pm \left[ \mbox{Quantile} \times \mbox{Standard Error} \right] \]


\begin{itemize}
\item Point Estimate: estimate for population parameter of interest, i.e. sample mean, sample proportion.
\item Quantile: a value from a probability distribution that scales the intervals according to the specified confidence level.
\item Standard Error: measures the dispersion of the sampling distribution for a given sample size $n$.
\end{itemize}




%-----------------------------------------------------------%


\subsection*{Quantiles}

\begin{itemize}
\item The quantile is a value from a probability distribution that scales the intervals according to the specified confidence level.
\item For practical purposes, the quantile can be taken from the standard normal distribution, if the sample is larger than 30, further to the central limit theorem.
\item For a specified confidence level $1-\alpha $, the corresponding quantile is the value $a$ that satisfies the following identity (when $n > 30$):

\[ p( -a \leq Z \leq a) = 1- \alpha \]

\end{itemize}




\begin{itemize} \item When the sample size $n$ is greater than 30, we can compute the quantile using Murdoch Barnes table 3.
Remark

\item $95\%$ of Z random variables are between -1.96 ( quantile for $2.5\%$)and 1.96 ( quantile for $97.5\%$)
\end{itemize}

\begin{framed}
\begin{itemize}
\item If the confidence level is $95\%$, then the quantile is 1.96. Recall
\[ \Pr( -1.96 \leq Z \leq 1.96) = 0.95 \]

\item If the confidence level is $90\%$, then the quantile is 1.645.
\[ \Pr( -1.645 \leq Z \leq 1.645) = 0.90 \]

\item If the confidence level is $99\%$, then the quantile is 2.576.
\[ \Pr( -2.576 \leq Z \leq 2.576) = 0.99 \]

\end{itemize}

\end{framed}




%-----------------------------------------------------------%


\subsection*{Standard Error}

\begin{itemize}
\item The standard error measures the dispersion of the sampling distribution.
\item For each type of point estimate, there is a corresponding standard error.
\item A full list of standard error formulae will be attached in your examination paper.
\item The standard error for a  mean is
\[ S.E( \bar{x} )  = \frac{\sigma}{ \sqrt{n}} \]
\item However, we often do not know the value for $\sigma$. For practical purposes, we use the sample standard deviation $s$ as an estimate for $\sigma$ instead.
\[ S.E( \bar{x} )  = {s \over \sqrt{n}} \]
\end{itemize}





%------------------------------------------------------------------------------%

\subsection*{Example 1: Confidence Interval for a mean}

\begin{itemize}
\item For a given week, a random sample of 100 hourly employees selected from a very large number of
employees in a manufacturing firm has a sample mean wage of $\bar{x} = 280$ dollars, with a sample standard deviation of
$s = 40$ dollars.
\item Estimate the mean wage for all hourly employees in the firm with an interval estimate such that we can be 95
percent confident that the interval includes the value of the population mean.
\end{itemize}




\begin{itemize}
\item The point estimate in this case is the sample mean $\bar{x} = 280$ dollars.
\item We have a large sample (n=100) and the confidence level is $95\%$. Therefore the quantile  is 1.96.
\item The standard error is computed as follows:

\[ S.E( \bar{x} )  = {s \over \sqrt{n}}  =  {40 \over \sqrt{100}} = 4  \]
\item \textbf{Confidence Interval for mean}

\[
280 \pm (1.96 \times 4)  = (280 \pm 7.84) = (\;272.16\;,\;287.84\;)
\]

\end{itemize}



\subsection*{Example 2: Confidence Interval for a Mean (Small Sample)}
\begin{itemize}
\item The mean operating life for a random sample of $n = 10$ light bulbs is $\bar{x} = 4,000$ hours, with the sample
standard deviation $s = 200$ hours. \item The operating life of bulbs in general is assumed to be approximately normally distributed.

\item This is a small-sample estimation. We use the student $t-$ distribution for computing the quantile. We will assume a confidence level of $95\%$ (hence $\alpha = 0.05$).
\item The point estimate is 4,000 hours. 
\item The sample standard deviation is 200 hours, and the sample size is 10. Hence \[S.E(\bar{x} ) = { 200 \over \sqrt{10}} = 63.3\]

\item From Murdoch Barnes Table 7, the t-quantile with $df=9$ is 2.262.
\item
We estimate the mean operating life for the population of bulbs from which this sample was taken, using a 95 percent
confidence interval as follows:

\[4,000\pm(2.262)(63.3)  = (3857,4143)\]
\end{itemize}


\end{document}
