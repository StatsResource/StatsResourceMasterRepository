\documentclass[a4paper,12pt]{article}
%%%%%%%%%%%%%%%%%%%%%%%%%%%%%%%%%%%%%%%%%%%%%%%%%%%%%%%%%%%%%%%%%%%%%%%%%%%%%%%%%%%%%%%%%%%%%%%%%%%%%%%%%%%%%%%%%%%%%%%%%%%%%%%%%%%%%%%%%%%%%%%%%%%%%%%%%%%%%%%%%%%%%%%%%%%%%%%%%%%%%%%%%%%%%%%%%%%%%%%%%%%%%%%%%%%%%%%%%%%%%%%%%%%%%%%%%%%%%%%%%%%%%%%%%%%%
\usepackage{eurosym}
\usepackage{vmargin}
\usepackage{framed}
\usepackage{amsmath}
\usepackage{graphics}
\usepackage{epsfig}
\usepackage{subfigure}
\usepackage{enumerate}
\usepackage{fancyhdr}

\setcounter{MaxMatrixCols}{10}
%TCIDATA{OutputFilter=LATEX.DLL}
%TCIDATA{Version=5.00.0.2570}
%TCIDATA{<META NAME="SaveForMode"CONTENT="1">}
%TCIDATA{LastRevised=Wednesday, February 23, 201113:24:34}
%TCIDATA{<META NAME="GraphicsSave" CONTENT="32">}
%TCIDATA{Language=American English}

%\pagestyle{fancy}
\setmarginsrb{20mm}{0mm}{20mm}{25mm}{12mm}{11mm}{0mm}{11mm}
%\lhead{MS4222} \rhead{Kevin O'Brien} \chead{Confidence Intervals} %\input{tcilatex}

\begin{document}
\section{Inference}


\subsection{Independent one-sample $t$-test}
In testing the null hypothesis that the population mean is equal to a specified value $\mu_{0}$, one uses the statistic

\begin{equation}t = \frac{\overline{x} - \mu_0}{s / \sqrt{n}}\end{equation}

where $s$ is the sample standard deviation and $n$ is the sample size. The degrees of freedom used in this test is $n - 1$.



\subsection{Hypothesis testing}
The standard deviation of the life for a particular brand of
ultraviolet tube is known to be $S = 500 hr$, and the operating
life of the tubes is normally distributed. The manufacturer claims
that average tube life is at least 9,000hr. Test this claim at the
5 percent level of significance against the alternative hypothesis
that the mean life is less than 9,000 hr, and given that for a
sample of $n = 18$ tubes the mean operating life was $\bar{X}=
8,800 hr.$


%%%%%%%%%%%%%%%%%%%%%%%%%%%%%%%%%%%%%%%%%%%%%%%%%%%%%%%%%%%%%%%%%%%%%%%%%%%%%%%%%%%

%--------------------------------------------------------------------------------------%
%--------------------------------------------------------------------------------------%
\begin{frame}
\frametitle{Example 1 (a)}
\large
\begin{itemize}
\item The standard deviation of the life for a particular brand of ultraviolet tube is known to be $s = 500$ hr,
\item Also it is assumed, but not known, that the operating life of the tubes is normally distributed. \item The manufacturer claims that average tube life
is at least 9,000hr. \item Test this claim at the 5 percent level of significance against the alternative hypothesis
that the mean life is 9,000 hr, and given that for a sample of $n = 10$ tubes the mean operating
life was $\bar{x} = 8,800$ hr.
\item (Intuitively this would suggest a one-tailed test that the mean is less than 9000 hours)
\end{itemize}

%%%%%%%%%%%%%%%%%%%%%%%%%%%%%%%%%%%%%%%%%%%%%%%
\large
\begin{itemize}
\item $H_0 \mbox{ : } $ $\mu = 9000$ Average life span is 9000 hours.
\item $H_1 \mbox{ : } $ $\mu \neq 9000$ Average life span is not 9000 hours.
\end{itemize}
\bigskip
\begin{itemize}
\item The observed difference is -200 hours. (i.e. 8,800 - 9,000 hours)
\item The standard error is determined from formulae.
\[ S.E.(\bar{x}) = {s \over \sqrt{n}} = {500 \over \sqrt{10}}  = 158.1139 \]
\end{itemize}

%%%%%%%%%%%%%%%%%%%%%%%%%%%%%%%%%%%%%%%%%%%%%%%
\frametitle{Example 1 (c) : Test Statistic }
\large
\begin{itemize}
\item The test statistic is ${(8800-9000) -0 \over  158.11} = -1.265$
\item The CV is determined from Murdoch Barnes Table 7, with $\alpha = 0.05$ and $k = 2$.
\item The sample is small n= 10 $df = n-1 = 9$.Therefore $CV = 2.262$
\item (Remark: If the distribution was known to be normal, we could use $df = \infty$, i.e $CV = 1.96$).
\end{itemize}

%%%%%%%%%%%%%%%%%%%%%%%%%%%%%%%%%%%%%%%%%%%%%%%
\frametitle{Example 1 (d) }
\large
\begin{itemize}


\item \textbf{Decision:}Is $|TS| >CV$? Is $1.265 > 2.262$?
\item No. We fail to reject the null hypothesis. Not enough evidence to say that the mean lifespan is not 9000 hours.
\end{itemize}

%%%%%%%%%%%%%%%%%%%%%%%%%%%%%%%%%%%%%%%%%%%%%%%

\begin{itemize}
\item We will approach the same problem in example 1, but this time using the p-value approach.
\item The first two steps i.e. formally stating the null and alternative hypothesis, and computing the test statistic are the same, are the same as example 1.
\item The third step is to compute the p-value:  $P(Z \geq |TS|)$.
\item From Murdoch Barnes table 3: $P(Z \geq 2.65) = 0.00402$ (i.e. less than half a percent).

\end{itemize}

%%%%%%%%%%%%%%%%%%%%%%%%%%%%%%%%%%%%%%%%%%%%%%%
\frametitle{Example 3: Difference in Means (b) }
\begin{itemize}
\item The p-value is  $0.00402$
\item The critical region has size $\alpha/k = 0.05/2 = 0.0250$.
\item We reject the null hypothesis because the computed p-value is less than $0.0250$.
\item \textbf{Conclusion:} we reject the null hypothesis. There is a significant different between both drugs, in terms of recovery times.
\end{itemize}
\end{document}

