\documentclass[a4paper,12pt]{article}
%%%%%%%%%%%%%%%%%%%%%%%%%%%%%%%%%%%%%%%%%%%%%%%%%%%%%%%%%%%%%%%%%%%%%%%%%%%%%%%%%%%%%%%%%%%%%%%%%%%%%%%%%%%%%%%%%%%%%%%%%%%%%%%%%%%%%%%%%%%%%%%%%%%%%%%%%%%%%%%%%%%%%%%%%%%%%%%%%%%%%%%%%%%%%%%%%%%%%%%%%%%%%%%%%%%%%%%%%%%%%%%%%%%%%%%%%%%%%%%%%%%%%%%%%%%%
\usepackage{eurosym}
\usepackage{vmargin}
\usepackage{amsmath}
\usepackage{framed}
\usepackage{graphics}
\usepackage{epsfig}
\usepackage{subfigure}
\usepackage{enumerate}
\usepackage{fancyhdr}

\setcounter{MaxMatrixCols}{10}
%TCIDATA{OutputFilter=LATEX.DLL}
%TCIDATA{Version=5.00.0.2570}
%TCIDATA{<META NAME="SaveForMode"CONTENT="1">}
%TCIDATA{LastRevised=Wednesday, February 23, 201113:24:34}
%TCIDATA{<META NAME="GraphicsSave" CONTENT="32">}
%TCIDATA{Language=American English}

\pagestyle{fancy}
\setmarginsrb{20mm}{0mm}{20mm}{25mm}{12mm}{11mm}{0mm}{11mm}
\lhead{StatsResource} \chead{Confidence Intervals} \rhead{Statistical Inference} %\input{tcilatex}
\begin{document}

\subsection{Confidence interval of a mean (small sample)}

If the data have a normal probability distribution and the sample
standard deviation $s$ is used to estimate the population
standard deviation $\sigma$, the interval estimate is given by:
\begin{equation}
\bar{X} \pm t_{1-\alpha/2,n-1}\frac{s}{\sqrt{n}}
\end{equation}
where $t_{1-\alpha/2,n-1}$ is the value providing an area of $\alpha/2$ in the upper tail of a Student’s t distribution with n - 1 degrees of freedom.


The intelligence quotient (IQ) of 36 randomly chosen students was measured.
Their average IQ was 109.9 with a variance of 324.
The average IQ of the population as a whole is 100.
\begin{enumerate}[(a)]
\item Calculate the p-value for the test of the hypothesis that on average
students are as intelligent as the population as a whole against the alternative that on average students are more intelligent.
\item Can we conclude at a significance level of $1\%$ that students are on average more intelligent than the population as a whole?
\item Calculate a $95\%$ confidence interval for the mean IQ of all students.
\end{enumerate}

\subsection*{Solution}

\[Z_{Test} = \frac{X- \mu}{\frac{\sigma}{\sqrt{n}}} = \frac{109.9 - 100}{\frac{18}{\sqrt{36}}} = \frac{9.9}{3} = 3.3\]
\[p.value = P(Z \geq Z_{Test}) = P(Z \geq 3.3) = 0.00048\]

%--------------------------------------------------------------%

\begin{itemize}
\item $\bar{X} \pm t_{1-\alpha/2,\nu}S.E.(\bar{X})$
\item $\nu = 1.96$
\item $t_{1-\alpha/2,\nu} = 1.96$
\item $109.9 \pm (1.96 \times 3) = [104.02, 115.79]$
\end{itemize}

%--------------------------------------------------------------%

\end{document}
