\item \textbf{Worked Example  - Single Sample t-Test (Small Sample)  } \\ % 10 Marks
\begin{itemize}
\item A web-based software company claims that the average amount of time it takes for
online queries to be dealt with is less than 2 hours. 
\item Out of a sample of 15 queries, the
sample mean $\bar{x}$ = 3.5 hours and the standard deviation is 30 minutes.
\end{itemize}

\begin{itemize}
\item[a.](2 marks) Construct the null and alternative hypothesis statements.
\item[b.](2 marks) Test this claim using a significance level of 0.05.
\item[c.](2 marks) Describe the two types of errors associated with hypothesis testing and how
they relate to this question?
\end{itemize}

\item \textbf{Worked Example} \\ ABC Software has 125 programmers divided into two groups with 75 in
Group A and 50 in Group B. In order to compare the efficiencies of the
two groups, the programmers are observed for one day. \begin{itemize} \item The 75
programmers of Group A averaged 76.21 lines of code with a standard
deviation of 10.37. \item The 50 programmers of Group B averaged 72.72
lines of code with a standard deviation of 10.07. \end{itemize}
\begin{itemize}
\item[a.](10 marks) Using a significance
level of 5\%, test the hypothesis that there is no difference between the
two groups versus the alternative that there is a difference. Clearly state
your null and alternative hypotheses and your conclusion.
\end{itemize}



\subsection{Example 2}

Seven measurements of the pH of a buffer solution gave the
following results:

\begin{center}
\begin{tabular}{|c|c|c|c|c|c|c|}
\hline
5.12 & 5.20 & 5.15 & 5.17 & 5.16 & 5.19 & 5.15\\
\hline
\end{tabular}
\end{center}


Task 1: Calculate the 95\% confidence limits for the true pH
utilizing $R$.


Solution. We are using Student t distribution with six degrees of
freedom and the following code gives us the confidence interval
for this problem.
%%\begin{verbatim}
%%>x <- c(5.12, 5.20, 5.15, 5.17, 5.16, 5.19, 5.15)
%%>n =length(x)
%%>alpha =0.05
%%>stderr =sd(x)/sqrt(n)
%%>LB=mean(x)+qt(alpha/2,6)* stderr
%%>UB=mean(x)+qt(1-alpha/2,6)* stderr
%%>LB
%%#[1] 5.137975
%%>UB
%%#[1]5.187739
%%\end{verbatim}

\subsection{Example 3} Ten replicate analyses of the concentration
of mercury in a sample of commercial gas condensate gave the
following results (in ng/ml) :

\begin{center}
\begin{tabular}{|c|c|c|c|c|c|c|c|c|c|}
\hline
23.3 & 22.5 & 21.9 & 21.5 & 19.9 & 21.3 & 21.7 & 23.8 & 22.6 &
24.7\\
\hline
\end{tabular}
\end{center}

Compute 99\% confidence limits for the mean.
%http://www.stats.gla.ac.uk/steps/glossary/hypothesis_testing.html


\newpage
\section*{Question Set 2 : Confidence Intervals}
\begin{enumerate}

\item \textbf{Worked Example}
Calculate a 99\% confidence interval for the difference between the proportion of all Irish having access to the
Internet and the proportion of all Spaniards having access to the internet.  (4 marks)



\noindent \textbf{Standard Error for confidence interval}

\[\frac{p1(1 -p1)}{n1}+ \frac{p2(1 -p2)}{n2}\]
\[=\frac{0.750.25}{1000}+ \frac{0.700.30}{2000}     =  0.017103\]

\noindent \textbf{Quantile for a 99\% confidence interval}
\begin{itemize}
\item significance level  =1\%
\itemnumber of tails = 2
\itemdegrees of freedom = 
\itemquantile = 2.576 
\end{itemize}



99\% Confidence Interval for difference of two proportions

%================================================================= %


Useful pieces of information


Sample size  n=100


\item \textbf{Worked Examples} \\

The strength of concrete depends, to some extent, on the method used for drying. Two different methods showed the following results for independently tested specimens.  ( You may assume that there are equal variances).


\begin{itemize}
\item[(i)] Does Method 1 appear to produce concrete with a greater mean strength? State your conclusions clearly.
\item[(ii)] Construct a 95\% confidence interval for the difference between the two means. Interpret this interval.

\end{itemize}

\item \textbf{Independent Sample Means hypothesis Test }
%Question 3.1 (d) \\

\begin{itemize}
\item A survey was carried out to investigate absenteeism in the building industry.
\item Data on the number of sick days per year and type of job (unskilled or skilled)
were collected for a random sample of employees.
\item The mean number of sick days for 50 unskilled workers was 3.8 days with a standard deviation of 2.6 days. 
\item The mean number of sick days for 60 skilled workers was 3 days with a standard deviation of 2.2 days. 
\item (Hint: This is a test of whether that the means of the two groups are the same)

\item Clearly state your null and alternative hypotheses.
Is this a One Tail or Two Tail test?
\end{itemize}

\end{enumerate}







%------------------------------------------------------------- %

\subsection{Small Sample Test For Means - Worked Example}
%D Hypothesis Testing - Example
The catering manager in a hotel suspects that the weight of loaves of bread delivered
daily by a bakery is consistently below the nominal weight of 800g. To test this,
10 loaves chosen at random from a day’s deliveries are weighed. The mean and
standard deviation of the ten weights are 792g and 25g, respectively.

\begin{enumerate}
\item  Carry out a formal significance test.
\item List the steps involved in this test 
\item Calculate a 95\% confidence interval for the average weight of loaves produced
\item Comment on the correspondence between the interval, as calculated, and the
result of the test.
\end{enumerate}



%\noindent \textbf{Solution 2}
%
%\begin{itemize}
%\item Confidence interval width is 3, so half-width is 1.5
%
%\item Seek n such that $1.96 \times \frac{9}{\sqrt{n}} = 1.5$
%
%\item Divide both sides by $1.96 \times 9$ \\
%\[\frac{1}{\sqrt{n}} = \frac{1.5}{1.96 \times 9} =\]
%
%
%\item invert and square boths sides.
%\end{itemize}



