\documentclass[a4paper,12pt]{article}
%%%%%%%%%%%%%%%%%%%%%%%%%%%%%%%%%%%%%%%%%%%%%%%%%%%%%%%%%%%%%%%%%%%%%%%%%%%%%%%%%%%%%%%%%%%%%%%%%%%%%%%%%%%%%%%%%%%%%%%%%%%%%%%%%%%%%%%%%%%%%%%%%%%%%%%%%%%%%%%%%%%%%%%%%%%%%%%%%%%%%%%%%%%%%%%%%%%%%%%%%%%%%%%%%%%%%%%%%%%%%%%%%%%%%%%%%%%%%%%%%%%%%%%%%%%%
\usepackage{eurosym}
\usepackage{vmargin}
\usepackage{amsmath}
\usepackage{graphics}
\usepackage{epsfig}
\usepackage{subfigure}
\usepackage{enumerate}
\usepackage{fancyhdr}

\setcounter{MaxMatrixCols}{10}
%TCIDATA{OutputFilter=LATEX.DLL}
%TCIDATA{Version=5.00.0.2570}
%TCIDATA{<META NAME="SaveForMode"CONTENT="1">}
%TCIDATA{LastRevised=Wednesday, February 23, 201113:24:34}
%TCIDATA{<META NAME="GraphicsSave" CONTENT="32">}
%TCIDATA{Language=American English}

\pagestyle{fancy}
\setmarginsrb{20mm}{0mm}{20mm}{25mm}{12mm}{11mm}{0mm}{11mm}
\lhead{MS4222} \rhead{Kevin O'Brien} \chead{Paired Measurements} %\input{tcilatex}

\begin{document}

\subsection{Paired T test}
The mean and standard deviation of the sample d values are
obtained by use of the basic formulas in Chapters 3 and 4, except
that d is substituted for X.

The mean difference for a set of differences between paired
observations is $\bar{d} = \frac{\sum d_{i}}{n}$.

The deviations formula and the computational formula for the
standard deviation of the differences between paired observations
are, respectively,

\begin{eqnarray}
S_{d} = \sqrt{\frac{\sum (d_{i}-\bar{d})^2}{n-1}}\\
S_{d} = \sqrt{\frac{ \sum (d^2)- n(\bar{d}^2)}{n-1}}\\
\end{eqnarray}

The standard error of the mean difference between paired
observations is obtained for the standard error of the mean.
\subsubsection{Hypotheses}
\begin{eqnarray*}
H_{0}: \mu_{d} = 0\\
H_{1}: \mu_{d} \neq 0\\
\end{eqnarray*}


\subsection*{Question 3}
	Seven athletes were asked to run 100m without warming up prior to running. On another day they warmed up first and then ran. On both occasions they were timed and the results (in seconds) are as follows:\\[-0.3cm]
	\begin{center}
		\begin{tabular}{|c|ccccccc|}
			\hline
			&&&&&&&\\[-0.3cm]
			Individual & 1 & 2 & 3 & 4 & 5 & 6 & 7 \\[0.1cm]
			\hline
			&&&&&&&\\[-0.3cm]
			No Warm Up    & 13.6 & 12.8 & 12.3 & 11.7 & 12.0 & 13.3 & 10.5 \\[0.1cm]
			\hline
			&&&&&&&\\[-0.3cm]
			Warm Up       & 13.9 & 12.4 & 12.2 & 11.6 & 11.9 & 12.7 & 10.4 \\[0.1cm]
			\hline
		\end{tabular}
	\end{center}
	
	{\bf(a)} Calculate a 95\% confidence interval for the \emph{average difference} in times and hence comment on the usefulness of warming up (hint: the data is paired).
	
	\end{document}
	