

	\documentclass[a4paper,12pt]{article}
%%%%%%%%%%%%%%%%%%%%%%%%%%%%%%%%%%%%%%%%%%%%%%%%%%%%%%%%%%%%%%%%%%%%%%%%%%%%%%%%%%%%%%%%%%%%%%%%%%%%%%%%%%%%%%%%%%%%%%%%%%%%%%%%%%%%%%%%%%%%%%%%%%%%%%%%%%%%%%%%%%%%%%%%%%%%%%%%%%%%%%%%%%%%%%%%%%%%%%%%%%%%%%%%%%%%%%%%%%%%%%%%%%%%%%%%%%%%%%%%%%%%%%%%%%%%
\usepackage{eurosym}
\usepackage{vmargin}
\usepackage{framed}
\usepackage{amsmath}
\usepackage{graphics}
\usepackage{epsfig}
\usepackage{subfigure}
\usepackage{enumerate}
\usepackage{fancyhdr}

\setcounter{MaxMatrixCols}{10}
%TCIDATA{OutputFilter=LATEX.DLL}
%TCIDATA{Version=5.00.0.2570}
%TCIDATA{<META NAME="SaveForMode"CONTENT="1">}
%TCIDATA{LastRevised=Wednesday, February 23, 201113:24:34}
%TCIDATA{<META NAME="GraphicsSave" CONTENT="32">}
%TCIDATA{Language=American English}

%\pagestyle{fancy}
\setmarginsrb{20mm}{0mm}{20mm}{25mm}{12mm}{11mm}{0mm}{11mm}
%\lhead{MS4222} \rhead{Kevin O'Brien} \chead{Confidence Intervals} %\input{tcilatex}

\begin{document}

\subsection*{Confidence Intervals for Paired Differences  }
(Revision)
\begin{itemize}
\item The $95\%$ confidence interval is a range of values which contain the true population parameter (i.e. mean, proportion etc) with a probability of $95\%$.
\item We can expect that a $95\%$ confidence interval will not include the true parameter values $5\%$ of the time.
\item A confidence level of $95\%$ is commonly used for computing confidence interval, but we could also have confidence levels of $90\%$,$99\%$ and $99.9\%$.

\item A confidence level for an interval is denoted to $1-\alpha$ (in percentages: $100(1-\alpha)\%$) for some value $\alpha$.
\item A confidence level of $95\%$ corresponds to $\alpha = 0.05$.
\item $100(1-\alpha)\%$ = $100(1-0.05)\%$  = $100(0.95)\%$ = $95\%$
\item For a confidence level of $99\%$, $\alpha = 0.01$.
\item Knowing the correct value for $\alpha$ is important when determining quantiles.
\end{itemize}
%%%%%%%%%%%%%%%%%%%%%%%%%%%%%%%%%%%%%%%%%%%%%%%%%%%%%%%%%%
Recall: General Structure of Confidence Intervals:
\smallskip
\[ ( \bar{d} ) \pm \left[ \mbox{Quantile } \times S.E(\bar{d}) \right] \]
\[ ( \bar{d} ) \pm \left[ \mbox{Quantile } \times \frac{s_d}{\sqrt{n}} \right] \]
\begin{itemize}
\item If the combined sample size of X and Y is greater than 30, even if the individual sample sizes are less than 30, then we consider it to be a large sample.\\ \textit{(we would never do a large sample by pen-and-paper calculations)}
\item The quantile is calculated according to the procedure we met in the previous class.\\ \textit{(Murdoch Barnes Table 7 - recall \textit{df=n-1})}
\end{itemize}

%---------------------------------------------------------%

%---------------------------------------------------------------------------------------------------------------%
\newpage 
\subsection*{Example}
Ten soldiers visit the rifle range on two different weeks. The first
week their scores are:
\[67, 24, 57, 55, 63, 54, 56, 68, 33, 43\]
The second week they scores were as follows (in same order of soldier)
\[70, 38, 58, 58, 56, 67, 68, 77, 42, 38\]
Give a 95\% confidence interval for the improvement in scores from week one to
week two.


\subsection*{Answer}

\begin{itemize}
\item This is a case of paired samples, for the scores are repeated observations for each
soldier, and there is good reason to think that the soldiers will differ from each other
in their shooting skill. 
\item So we work with the individual differences between the scores.
\item We shall have to assume that the pairwise differences are a random sample from a
normal distribution.
\item 
The differences are (using an \textbf{\textit{after-before}} basis):

\[3, 14, 1, 3, -7, 13, 12, 9, 9, -5\]
\end{itemize}
\smallskip

\begin{itemize}
\item Effectively we now have a single sample of size 10, and want a 95\% confidence
interval for the mean of the population from which these differences are drawn. For
this we use a Student's $t$ interval.
\item  The sample mean of the differences is 5.2, and
$s^2$ = 54.84. 
\item So $s = 7.41$, and the 95\% $t$ interval for the difference in the means is
\begin{eqnarray*}5.2 \pm 2.26(7.41)/\sqrt{10} &=&  5.2 \pm 2.26(7.41)/\sqrt{10} \\&=& (0.1, 10.5)\end{eqnarray*}
\end{itemize}
\end{document}
