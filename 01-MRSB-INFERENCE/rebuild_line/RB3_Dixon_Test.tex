\documentclass[a4paper,12pt]{article}
%%%%%%%%%%%%%%%%%%%%%%%%%%%%%%%%%%%%%%%%%%%%%%%%%%%%%%%%%%%%%%%%%%%%%%%%%%%%%%%%%%%%%%%%%%%%%%%%%%%%%%%%%%%%%%%%%%%%%%%%%%%%%%%%%%%%%%%%%%%%%%%%%%%%%%%%%%%%%%%%%%%%%%%%%%%%%%%%%%%%%%%%%%%%%%%%%%%%%%%%%%%%%%%%%%%%%%%%%%%%%%%%%%%%%%%%%%%%%%%%%%%%%%%%%%%%
\usepackage{eurosym}
\usepackage{vmargin}
\usepackage{amsmath}
\usepackage{graphics}
\usepackage{framed}
\usepackage{epsfig}
\usepackage{subfigure}
\usepackage{enumerate}
\usepackage{fancyhdr}

\setcounter{MaxMatrixCols}{10}
%TCIDATA{OutputFilter=LATEX.DLL}
%TCIDATA{Version=5.00.0.2570}
%TCIDATA{<META NAME="SaveForMode"CONTENT="1">}
%TCIDATA{LastRevised=Wednesday, February 23, 201113:24:34}
%TCIDATA{<META NAME="GraphicsSave" CONTENT="32">}
%TCIDATA{Language=American English}

\pagestyle{fancy}
\setmarginsrb{20mm}{0mm}{20mm}{25mm}{12mm}{11mm}{0mm}{11mm}
\lhead{StatsResource} \rhead{Inference Procedures} \chead{Confidence Intervals} %\input{tcilatex}

\begin{document}
\large 

\subsection{Dixon Q Test}
\begin{itemize}
	\item The Dixon's Q test, or simply the Q test, is used for identification and rejection of outliers. 
	\item \textbf{(Important)} - This test assumes normal distribution. Also this test should be used sparingly and never more than once in a data set. 
\end{itemize}

To apply a Q test for suspicious data, arrange the data in order of increasing values and calculate Q as defined:

\[ Q = \frac{\text{gap}}{\text{range}} \]
Where gap is the absolute difference between the outlier in question and the closest number to it. 
%% - \end{frame}

\begin{itemize}
	\item 	If $Q_{Test} > Q_{CV}$ , where $Q_{CV}$ is a critical value corresponding to the sample size and confidence level, then reject the questionable data point. 
	\item Again, note that only one point may be rejected from a data set using a Q test.
\end{itemize}


%% - \end{frame}
\subsection{Dixon Q Test: Example}
Consider the data set:
\begin{framed}
	\[0.189,\ 0.167,\ 0.187,\ 0.183,\ 0.186,\]\[ 0.182,\ 0.181,\ 0.184,\ 0.181,\ 0.177 \,\]
\end{framed}
Now rearrange in increasing order:
\begin{framed}
	\[0.167,\ 0.177,\ 0.181,\ 0.181,\ 0.182,\]\[ 0.183,\ 0.184,\ 0.186,\ 0.187,\ 0.189 \, \]
\end{framed}
%% - \end{frame}

We hypothesize 0.167 is an outlier. \\ Calculate The Test Statistic $Q_{Test}$:
{
	\[ Q_{Test}=\frac{\text{gap}}{\text{range}}  \]
	\[ Q_{Test} 
	= \frac{0.177-0.167}{0.189-0.167}=0.455.\]
}
%% - \end{frame}

%==================================================================%
%% - \begin{frame}[fragile]
\begin{figure}
	\centering
	\includegraphics[width=0.5\linewidth]{images/DixonQTestTables}
	\caption{}
	\label{fig:dixonqtesttables}
\end{figure}

\textit{Here: N is the sample size.}
%% - \end{frame}

\begin{itemize}
	\item	With 10 observations and at 90\% confidence, $Q_{Test} = 0.455 > 0.412 =Q_{CV}$ , so we conclude 0.167 is an outlier.
	\item  However, at 95\% confidence, $Q_{Test} = 0.455 < 0.466$ = $Q_{CV}$ 0.167 is not considered an outlier. 
	
	\item This means that for this example we can be 90\% sure that 0.167 is an outlier, but we cannot be 95\% sure.
	\bigskip
	\item (Remark 95\% confidence is equivalent to 5\% signifificance)
\end{itemize}

\end{document}
