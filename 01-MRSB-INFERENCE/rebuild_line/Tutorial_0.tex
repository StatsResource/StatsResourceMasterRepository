\documentclass[]{report}

\voffset=-1.5cm
\oddsidemargin=0.0cm
\textwidth = 480pt

\usepackage{framed}
\usepackage{subfiles}
\usepackage{graphics}
\usepackage{newlfont}
\usepackage{eurosym}
\usepackage{amsmath,amsthm,amsfonts}
\usepackage{amsmath}
\usepackage{color}
\usepackage{amssymb}
\usepackage{multicol}
\usepackage[dvipsnames]{xcolor}
\usepackage{graphicx}
\begin{document}
\chapter{Normal Distribution - Worked Examples}
\section{Calculations}

\begin{framed}
	\begin{itemize}
		\item The Complement Rule
		\begin{equation}
		P(Z \leq z) = 1 - P(Z \geq z)
		\end{equation}
		\item The Symmetry Rule
		\begin{equation}
		P(Z \leq -A) = P(Z \geq A)
		\end{equation}
		\item The Interval Rule.
		Where $L$ and $U$ are the lower and upper bounds of an interval.
		\begin{equation}
		P(L \leq Z \leq U) = P(Z \geq L) -  P(Z \geq U)
		\end{equation}
		
	\end{itemize}
\end{framed}

\section{Summary of Normal Distribution}

\begin{enumerate}
	\item Working with Tables
	
	\[P(Z \geq Zo)\]
	
	\item The Standardisation Formula
	
	\[P(X \leq Xo) = P(Z \leq Zo)	  \]  
	
	when   \[Zo=\frac{Xo- \mu}{\sigma}\]
	
	\item Complement Rule
	
	\[P(Z\geq Z_0) = 1 - P(Z \leq Z_0)\]
	
	\item  Symmetry Rule
	
	
	\[P(Z \leq -Z_0) = P(Z \geq Z_0)\]
	
\end{enumerate}


%-------------------------------------------------%



\section{Normal - example}

In an examination the scores of students who attend schools of type A are
normally distributed about a mean of 55 with a standard deviation of 6. The
scores of students who attend type B schools are normally distributed about a
mean of 60 with a standard deviation of 5.

Which type of school would have a higher proportion of students with marks above 70?

\begin{itemize}
	\item $\mu_A$ = 55
	\item $\sigma_A$ = 8
	\item $\mu_B$ = 60
	\item $\sigma_B$ = 5
\end{itemize}

We have to fins $P(X_A \geq 70)$
and $P(X_B \geq 70)$.


using the standardisation formula
$Z_A = \frac{70 - 55}{6} = \frac{15}{6} = 2.5 $

$Z_B = \frac{70 - 60}{5} = \frac{10}{5} = 2 $



