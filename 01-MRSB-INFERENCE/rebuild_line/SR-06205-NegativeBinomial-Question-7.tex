
\documentclass[a4paper,12pt]{article}
%%%%%%%%%%%%%%%%%%%%%%%%%%%%%%%%%%%%%%%%%%%%%%%%%%%%%%%%%%%%%%%%%%%%%%%%%%%%%%%%%%%%%%%%%%%%%%%%%%%%%%%%%%%%%%%%%%%%%%%%%%%%%%%%%%%%%%%%%%%%%%%%%%%%%%%%%%%%%%%%%%%%%%%%%%%%%%%%%%%%%%%%%%%%%%%%%%%%%%%%%%%%%%%%%%%%%%%%%%%%%%%%%%%%%%%%%%%%%%%%%%%%%%%%%%%%
\usepackage{eurosym}
\usepackage{vmargin}
\usepackage{amsmath}
\usepackage{graphics}
\usepackage{epsfig}
\usepackage{enumerate}
\usepackage{multicol}
\usepackage{subfigure}
\usepackage{fancyhdr}
\usepackage{listings}
\usepackage{framed}
\usepackage{graphicx}
\usepackage{amsmath}
\usepackage{chngpage}
%\usepackage{bigints}

\usepackage{vmargin}
% left top textwidth textheight headheight
% headsep footheight footskip
\setmargins{2.0cm}{2.5cm}{16 cm}{22cm}{0.5cm}{0cm}{1cm}{1cm}
\renewcommand{\baselinestretch}{1.3}

\setcounter{MaxMatrixCols}{10}

\begin{document}
\large


A Recursive Formula

The probability functions described in (1), (2), (7), (8), (9), (10) and (11) describe clearly how the negative binomial probabilities are calculated based on the two given parameters. The probabilities can also be calculated recursively. Let $P_k=P[X=k] where k=0,1,2,\cdots$. We introduce a recursive formula that allows us to compute the value P_k if P_{k-1} is known. The following is the form of the recursive formula.
\[
\displaystyle (12) \ \ \ \ \frac{P_k}{P_{k-1}}=a+\frac{b}{k} \ \ \ \ \ \ \ \ \ \ k=1,2,3,\cdots\]
In (12), the numbers a and b are constants. Note that the formula (12) calculates probabilities P_k for all k \ge 1. It turns out that the initial probability P_0 is determined by the constants a and b. Thus the constants a and b completely determines the probability distribution represented by P_k. Any discrete probability distribution that satisfies this recursive relation is said to be a member of the (a,b,0) class of distributions.

We show that the negative binomial distribution is a member of the (a,b,0) class of distributions. 


First, assume that the negative binomial distribution conforms to the parametrization in (8) with parameters \alpha and \theta. Then let a and b be defined as follows:

\displaystyle a=\frac{\theta}{1+\theta}
\displaystyle b=\frac{(\alpha-1) \theta}{1+\theta}.

Let the initial probability be P_0=(1+\theta)^{-\alpha}. We claim that the probabilities generated by the formula (12) are identical to the ones calculated from (8). 


To see this, let’s calculate a few probabilities using the formula.

\displaystyle P_0=\biggl(\frac{1}{1+\theta} \biggr)^\alpha
\displaystyle \begin{aligned} P_1&=(a+b) P_0 \\&=\biggl(\frac{\theta}{1+\theta}+ \frac{(\alpha-1) \theta}{1+\theta} \biggr) \ \biggl(\frac{1}{1+\theta} \biggr)^\alpha \\&=\alpha \ \biggl(\frac{1}{1+\theta} \biggr)^\alpha \ \frac{\theta}{1+\theta} \\&=\binom{\alpha}{1} \ \biggl(\frac{1}{1+\theta} \biggr)^\alpha \ \frac{\theta}{1+\theta}=P[X=1]  \end{aligned}

\displaystyle \begin{aligned} P_2&=\biggl(a+\frac{b}{2} \biggr) P_1 \\&=\biggl(\frac{\theta}{1+\theta}+ \frac{(\alpha-1) \theta}{2(1+\theta)} \biggr) \ \alpha \ \biggl(\frac{1}{1+\theta} \biggr)^\alpha \ \frac{\theta}{1+\theta} \\&=\frac{(\alpha+1) \alpha}{2!} \ \biggl(\frac{1}{1+\theta} \biggr)^\alpha \ \biggl( \frac{\theta}{1+\theta} \biggr)^2 \\&=\binom{\alpha+1}{2} \ \biggl(\frac{1}{1+\theta} \biggr)^\alpha \ \biggl( \frac{\theta}{1+\theta} \biggr)^2=P[X=2]  \end{aligned}

\displaystyle \begin{aligned} P_3&=\biggl(a+\frac{b}{3} \biggr) P_2 \\&=\biggl(\frac{\theta}{1+\theta}+ \frac{(\alpha-1) \theta}{3(1+\theta)} \biggr) \ \frac{(\alpha+1) \alpha}{2!} \ \biggl(\frac{1}{1+\theta} \biggr)^\alpha \ \biggl( \frac{\theta}{1+\theta} \biggr)^2 \\&=\frac{(\alpha+2) (\alpha+1) \alpha}{3!} \ \biggl(\frac{1}{1+\theta} \biggr)^\alpha \ \biggl( \frac{\theta}{1+\theta} \biggr)^3 \\&=\binom{\alpha+2}{3} \ \biggl(\frac{1}{1+\theta} \biggr)^\alpha \ \biggl( \frac{\theta}{1+\theta} \biggr)^3=P[X=3]  \end{aligned}

%%%%%%%%%%%%%%%%%%%%%%%%%%%%%%%%%%%%%%%%%5


The above derivation demonstrates that formula (12) generates the same probabilities as (8). By adjusting the constants a and b, the recursive formula can also generate the probabilities in the other versions of the negative binomial distribution. For the negative binomial version (9) with parameters \alpha and p, the a and b should be defined as follows:

a=1-p
b=(\alpha-1) \ (1-p)

With the initial probability P_0=p^\alpha, the recursive formula (12) will generate the same probabilities as those from version (9).

\end{document}

%%-- https://actuarialmodelingpractice.wordpress.com/2017/11/09/several-versions-of-negative-binomial-distributions/