

\documentclass[a4paper,12pt]{article}
%%%%%%%%%%%%%%%%%%%%%%%%%%%%%%%%%%%%%%%%%%%%%%%%%%%%%%%%%%%%%%%%%%%%%%%%%%%%%%%%%%%%%%%%%%%%%%%%%%%%%%%%%%%%%%%%%%%%%%%%%%%%%%%%%%%%%%%%%%%%%%%%%%%%%%%%%%%%%%%%%%%%%%%%%%%%%%%%%%%%%%%%%%%%%%%%%%%%%%%%%%%%%%%%%%%%%%%%%%%%%%%%%%%%%%%%%%%%%%%%%%%%%%%%%%%%
\usepackage{eurosym}
\usepackage{vmargin}
\usepackage{amsmath}
\usepackage{graphics}
\usepackage{epsfig}
\usepackage{multicol}
\usepackage{subfigure}
\usepackage{enumerate}
\usepackage{fancyhdr}
\usepackage{framed}

\setcounter{MaxMatrixCols}{10}
%TCIDATA{OutputFilter=LATEX.DLL}
%TCIDATA{Version=5.00.0.2570}
%TCIDATA{<META NAME="SaveForMode"CONTENT="1">}
%TCIDATA{LastRevised=Wednesday, February 23, 201113:24:34}
%TCIDATA{<META NAME="GraphicsSave" CONTENT="32">}
%TCIDATA{Language=American English}

\pagestyle{fancy}
\setmarginsrb{20mm}{0mm}{20mm}{25mm}{12mm}{11mm}{0mm}{11mm}
\lhead{StatsResource} \rhead{The Pareto Distribution} \chead{Probability Distributions} %\input{tcilatex}

\begin{document}
\large 



\item[(d)] \textbf{\textit{Inference Procedures with \texttt{R} (6 Marks)}}\\
The standard deviations of data sets \texttt{X} and \texttt{Y} are 10 and 9 respectively. 
An inference procedure was carried out to assess whether or not \texttt{X} and \texttt{Y} can be assumed to have equal variance.
\begin{itemize}

\item[i.] Formally state the null and alternative hypothesis.

\item[ii.] The Test Statistic has been omitted from the computer code output. Compute the value of the Test Statistic.

\item[iii.]() What is your conclusion for this procedure? Justify your answer.

\item[iv.] Explain how a conclusion for this procedure can be based on the $95\%$ confidence interval.
\end{itemize}

%---- R code for Variance Test ----%
%---- Dummy Code Included                   ----%
\begin{framed}
\begin{verbatim}
        F test to compare two variances
data:  X and Y
F = ......, num df = 13, denom df = 11, p-value = 0.7349
alternative hypothesis: true ratio of variances is not equal to 1
95 percent confidence interval:
 0.3639938 3.9475262
sample estimates:
ratio of variances
          .......
\end{verbatim}
\end{framed}
