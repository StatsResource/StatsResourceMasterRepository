\documentclass[]{report}

\voffset=-1.5cm
\oddsidemargin=0.0cm
\textwidth = 480pt

\usepackage{framed}
\usepackage{subfiles}
\usepackage{graphics}
\usepackage{newlfont}
\usepackage{eurosym}
\usepackage{amsmath,amsthm,amsfonts}
\usepackage{amsmath}
\usepackage{color}
\usepackage{amssymb}
\usepackage{multicol}
\usepackage[dvipsnames]{xcolor}
\usepackage{graphicx}
\begin{document}

\subsection*{Dixon Q Test For Outliers}
\noindent 
The typing speeds for one group of 12 Engineering students were recorded both at the beginning 
of year 1 of their studies. 
The results (in words per minute) are given below: 

\[\{121, 146, 150, 149, 142, 170, 153, 137, 161, 156, 165,  137, 178, 159\} \]

Use the Dixon Q-test to determine if the lowest value (121) is an outlier. 
You may assume a significance level of 5\%. 
\begin{enumerate}[(a)]
\item Formally state the null hypothesis and the alternative hypothesis. \item Compute the test statistic. 
\item By comparing the test statistic to the appropriate critical value, state your conclusion for this test. 

\end{enumerate}
\subsection*{Solution}

\subsection*{Critical Values}
\end{document}
