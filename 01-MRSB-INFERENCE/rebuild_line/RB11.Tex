
\documentclass[a4paper,12pt]{article}
%%%%%%%%%%%%%%%%%%%%%%%%%%%%%%%%%%%%%%%%%%%%%%%%%%%%%%%%%%%%%%%%%%%%%%%%%%%%%%%%%%%%%%%%%%%%%%%%%%%%%%%%%%%%%%%%%%%%%%%%%%%%%%%%%%%%%%%%%%%%%%%%%%%%%%%%%%%%%%%%%%%%%%%%%%%%%%%%%%%%%%%%%%%%%%%%%%%%%%%%%%%%%%%%%%%%%%%%%%%%%%%%%%%%%%%%%%%%%%%%%%%%%%%%%%%%
\usepackage{eurosym}
\usepackage{vmargin}
\usepackage{amsmath}
\usepackage{graphics}
\usepackage{epsfig}
\usepackage{enumerate}
\usepackage{multicol}
\usepackage{subfigure}
\usepackage{fancyhdr}
\usepackage{listings}
\usepackage{framed}
\usepackage{graphicx}
\usepackage{amsmath}
\usepackage{chngpage}
%\usepackage{bigints}

\usepackage{vmargin}
% left top textwidth textheight headheight
% headsep footheight footskip
\setmargins{2.0cm}{2.5cm}{16 cm}{22cm}{0.5cm}{0cm}{1cm}{1cm}
\renewcommand{\baselinestretch}{1.3}

\setcounter{MaxMatrixCols}{10}
\begin{document}
 Suppose that the prevalence of a genetic condition is new-born children is 1 in 5,000 births. Suppose that the maternity hospitals in Munster typically will have 10,000 live births in any given year.\\ 
Use an appropriate approximation method to estimate the the probability of the following events.

%%%%%%%%%%%%%%%%%%%%%%%%%%%%%%%%%%%%%%%%%%%%%%%%%%%%%%%%%%
% Poisson Approximation (7 Marks)
\begin{itemize}
\item[(a)]  State the approximation method that you will use, and show that this approach is valid in this instance.
\item [(b)] There will be no occurrences of the genetic condition in newborn children for a given year.
\item [(c)] There will be exactly one occurrence of the genetic condition in newborn children for a given year.
\item [(d)] There will be two or more occurrences of the genetic condition in newborn children for a given year.
\end{itemize}


\end{enumerate}

%=====================================================%