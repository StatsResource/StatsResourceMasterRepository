
Two sets of patients are given courses of treatment under two different drugs. The benefits
derived from each drug can be stated numerically in terms of the recovery times; the readings are given below.

\begin{itemize}
\item Drug 1:  $n_1$ = 40 , $\bar{x}_1$ = 3.3 days and $s_1 = 1.524$
\item Drug 2:  $n_2$ = 45 , $\bar{x}_2$ = 4.3 days and $s_2 = 1.951 $
\end{itemize}


\frametitle{Example 2: Difference in Means (b) }
\begin{itemize}
\item
The first step in hypothesis testing is to specify the null hypothesis and an alternative hypothesis.
\item When testing differences between mean recovery times, the null hypothesis is that the two population means are equal.
\item That is, the null hypothesis is:\\
$H_0: \mu_1 = \mu_2$\\
$H_1: \mu_1 \neq \mu_2$\\
\end{itemize}
(Remark: Two Tailed Test k = 2, and $\alpha = 0.05$)

\frametitle{Example 2: Difference in Means (c) }
\begin{itemize}
\item The observed difference in means is 1 day.
\item The relevant formula for the standard error is \[ S.E.(x_1 - x_2) = \sqrt{{s^2_1\over n_1}+{s^2_2 \over n_2}} \]
\item  \[ S.E(x_1 - x_2) = \sqrt{{(1.524)^2 \over 40}+{(1.951)^2 \over 45}}  = 0.377\mbox{ days} \]
\end{itemize}

\begin{itemize}
\item The Test Statistic is therefore
\[ TS = {\mbox{observed} - \mbox{null} \over \mbox{Std. Error}}  = {1 - 0 \over 0.377 } = 2.65 \]
\item The critical value $CV = 1.96$.
\item Is the TS greater than the CV? Is $2.65 > 1.96$?

\item \textbf{Conclusion:} we reject the null hypothesis. There is a significant different between both drugs, in terms of recovery times.

\end{itemize}

\end{document}
