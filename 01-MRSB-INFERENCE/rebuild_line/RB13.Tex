\documentclass[a4paper,12pt]{article}
%%%%%%%%%%%%%%%%%%%%%%%%%%%%%%%%%%%%%%%%%%%%%%%%%%%%%%%%%%%%%%%%%%%%%%%%%%%%%%%%%%%%%%%%%%%%%%%%%%%%%%%%%%%%%%%%%%%%%%%%%%%%%%%%%%%%%%%%%%%%%%%%%%%%%%%%%%%%%%%%%%%%%%%%%%%%%%%%%%%%%%%%%%%%%%%%%%%%%%%%%%%%%%%%%%%%%%%%%%%%%%%%%%%%%%%%%%%%%%%%%%%%%%%%%%%%
\usepackage{eurosym}
\usepackage{vmargin}
\usepackage{framed}
\usepackage{amsmath}
\usepackage{graphics}
\usepackage{epsfig}
\usepackage{subfigure}
\usepackage{enumerate}
\usepackage{fancyhdr}

\setcounter{MaxMatrixCols}{10}
%TCIDATA{OutputFilter=LATEX.DLL}
%TCIDATA{Version=5.00.0.2570}
%TCIDATA{<META NAME="SaveForMode"CONTENT="1">}
%TCIDATA{LastRevised=Wednesday, February 23, 201113:24:34}
%TCIDATA{<META NAME="GraphicsSave" CONTENT="32">}
%TCIDATA{Language=American English}

%\pagestyle{fancy}
\setmarginsrb{20mm}{0mm}{20mm}{25mm}{12mm}{11mm}{0mm}{11mm}
%\lhead{StatsResource} \rhead{Kevin O'Brien} \chead{Confidence Intervals} %\input{tcilatex}

\begin{document}


%--------------------------------------------------------------------------------------%
\large 
\subsection*{The Exponential Distribution}
\begin{itemize}
\item The exponential distribution is a continuous probability distribution commonly used to model durations or ``lifetimes".
\item A lifetime could mean
\begin{itemize}
\large
\item the lifespan of a component
\item the time it takes to complete a task
\item the amount of time between two successive occurrences, such as withdrawals from a bank machine.
\end{itemize}
\item The average lifetime is denoted $E(X) = \mu$.
\item The variance of lifetimes is computed as $V(X) = \mu^2$
\end{itemize}


%--------------------------------------------------------------------------------------%
\begin{framed}
\noindent \textbf{Important Formulas}\\
\large
The probability that a lifetime $X$ will be less than a period of $k$ time units is given by
\[
P( X \leq k) = 1- e^{{-k \over \mu}}.
\]
Similarly, the probability that a lifetime $X$ will be greater than a period of $k$ time units is given by
\[
P( X \geq k) = e^{{-k \over \mu}}.
\]
\end{framed}
%--------------------------------------------------------------------------------------%
\newpage 
\subsection*{Sample Question}
\large
In a large company computer network, there is an average of 40 log-ons to the network per hour.
\begin{enumerate}[(a)]
\item What is the average amount of time between log-ons?
\item What is the probability that there will be no log-ons for at least 2.4 minutes
\item What is the probability that the next log-on within 1 minutes of the last?
\item What proportions of log-ons occur between 1 minutes and 2.4 minutes of the last log-on?
\end{enumerate}

\subsection*{Solution (Part a) }

\large
\begin{itemize} \item What is the average amount of time between log-ons?

\item If there is 40 log-ons in 60 minutes, it is reasonable to think that someone logs on every 1.5 minutes.
\item Therefore $\mu = 1.5$
\end{itemize}


\subsection*{Solution (Part b) }
\large

What is the probability that there will be no log-ons for at least 2.4 minutes?\\
\bigskip
From the formulae:
\[
P( X \geq k) = e^{{-k \over \mu}} .
\]
\medskip
\[
P( X \geq 2.4) = e^{{-2.4 \over 1.5}} = e^{-1.6} = 0.2018.
\]

\medskip 
\subsection*{Solution (Part c) }
\large

What is the probability that the next log-on within 1 minutes of the last?\\
i.e. $P(X \leq 1)$
\bigskip

\[
P( X \leq 1) = 1 - e^{{-1 \over 1.5}} = 1 -  e^{-0.6666}
\]

\[
P( X \leq 1) = 1 -  0.5135  = 0.4865
\]
\medskip 
\subsection*{Solution (Part d) }
\large

What proportions of log-ons occur between 1 minutes and 2.4 minutes of the last log-on?\\
\bigskip
\begin{itemize}
\item \textbf{Too Low} $P(X \leq 1) = 0.4865$\\
\item \textbf{Too High} $P(X \geq 2.4) = 0.2018$\\
\item Probability of being inside interval %$P(1 \leq X \leq 2.4) = 0.31152$.
\[P(1 \leq X \leq 2.4) = 1- ( 0.4865 + 0.2018) = 0.3117\]
\end{itemize}

\end{document}





\item What is the probability that the lifespan of the laptop will be at least
6 years?
\item What is the probability that the lifespan of the laptop will not exceed
4 years?
\item What is the probability of the lifespan being between 5 years and 6
years?


%----------------------------------------------------------------------------%
{
\subsection*{The Exponential Distribution}
A continuous random variable having p.d.f. f(x), where:
$f(x) = \lambda x e ^{-\lambda x} $
is said to have an exponential distribution, with parameter $\lambda$.
The cumulative distribution is given by:
$F(x) = 1 - e^{\lambda x}$

Expectation and Variance
$E(X) = 1 / \lambda$\\
$\operatorname{Var}(X = 1 / \lambda^2$\\


\subsection*{Example}
Suppose that the service time for a customer at a fast-food outlet
has an exponential distribution with mean 3 minutes. What is the probability that a
customer waits more than 4 minutes?

\[ P(X  \leq 4) = 1 -  e^{-4/3} \]

\[ P(X  \leq 4) = e^{-4/3} = 0.2636 \]
}



Suppose the lifetime of a PC is exponentially distributed with
mean $\mu =5$
We should be told the average lifetime $\mu$.
\[
P( X \geq x_o) = e^{{-x_o \over \mu}}
\]



\end{document}

