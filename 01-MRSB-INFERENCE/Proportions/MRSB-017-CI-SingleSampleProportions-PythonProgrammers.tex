	\documentclass[a4paper,12pt]{article}
%%%%%%%%%%%%%%%%%%%%%%%%%%%%%%%%%%%%%%%%%%%%%%%%%%%%%%%%%%%%%%%%%%%%%%%%%%%%%%%%%%%%%%%%%%%%%%%%%%%%%%%%%%%%%%%%%%%%%%%%%%%%%%%%%%%%%%%%%%%%%%%%%%%%%%%%%%%%%%%%%%%%%%%%%%%%%%%%%%%%%%%%%%%%%%%%%%%%%%%%%%%%%%%%%%%%%%%%%%%%%%%%%%%%%%%%%%%%%%%%%%%%%%%%%%%%
\usepackage{eurosym}
\usepackage{vmargin}
\usepackage{framed}
\usepackage{amsmath}
\usepackage{graphics}
\usepackage{epsfig}
\usepackage{subfigure}
\usepackage{enumerate}
\usepackage{fancyhdr}

\setcounter{MaxMatrixCols}{10}
%TCIDATA{OutputFilter=LATEX.DLL}
%TCIDATA{Version=5.00.0.2570}
%TCIDATA{<META NAME="SaveForMode"CONTENT="1">}
%TCIDATA{LastRevised=Wednesday, February 23, 201113:24:34}
%TCIDATA{<META NAME="GraphicsSave" CONTENT="32">}
%TCIDATA{Language=American English}

\pagestyle{fancy}
\setmarginsrb{20mm}{0mm}{20mm}{25mm}{12mm}{11mm}{0mm}{11mm}
\lhead{MS4222} \rhead{Kevin O'Brien} \chead{Confidence Intervals} %\input{tcilatex}

\begin{document}



% \section*le proportion test}
% \[ H_0: \pi = 60\%\]
% \[ H_1: \pi \neq 60\%\]


\section*{Point Estimates for proportions }

\begin{itemize}
\item Of a sample of 160 computer programmers, 56 reported than Python was their primary programming language.
\item Let $\pi$ be the proportion of all programmers who regard Python as their programming language. 
\item What is the point estimate for $\pi$?
\end{itemize}


% \noindent \textbf{Confidence Intervals for Sample Proportion}



\subsection*{Quantile}
\begin{itemize}
\item In the cases of large samples ($ n > 30$) , the standard normal ( `Z' ) distribution is used.

\end{itemize}


\textbf{ Sample Proportion : Example}
\[ \mbox{S.E.}(\hat{p}) =  \sqrt{\frac{0.35 \times 0.65}{160}} = \sqrt{\frac{0.2275}{160}}\]

\[ \mbox{S.E.}(\hat{p}) = \sqrt{0.001421875} = 0.03770\]

\textbf{95\% Confidence Interval}
\[0.35 \pm (1.96 \times 0.0377) = (0.2761, 0.4239)\]

%-----------------------------------------------------------%

%------------------------------------------------------------------------------%

\section{Confidence Intervals}





\begin{itemize}

\item It is often easier to work in terms of percentages, rather than proportions.
If you are working in terms of percentages, remember to use the appropriate \textbf{\textit{complement value}} in the standard error formula (i.e. $100 - \hat{p}$)

\item The standard error, in the case of sample proportions, is
\[ \mbox{S.E.}(\hat{p}) = \sqrt{\frac{\hat{p}\times (100-\hat{p})}{n}}\]
\end{itemize}


%--------------------------------------------------%

\subsection*{Confidence Intervals for Sample Proportion}
Unlike confidence intervals for sample means, there is only one type of confidence interval when dealing with sample proportions.

\textbf{Optional}
\begin{itemize}

\item  It is often easier to work in terms of percentages, rather than proportions.
\item  If you are working in terms of percentages, remember to use the appropriate \textbf{\textit{complement value}} in the standard error formula (i.e. $100 - \hat{p}$)
\end{itemize}


\subsection*{Confidence Intervals for Sample Proportion}
\begin{itemize}
\item  The standard error, in the case of sample proportions, is
\[ \mbox{S.E.}(\hat{p}) = \sqrt{\frac{\hat{p}\times (100-\hat{p})}{n}}\] \smallskip


\end{itemize}



\begin{description}
\item[Point Estimate] The sample proportion is computed as follows
\[ \hat{p} = \frac{x}{n} = \frac{56}{160} = 0.35 9i.e. 35\%\]
\item[Quantile] We are asked for a 95\% confidence interval. The quantile is therefore
\[ z_{\alpha/2} =1.96\]
\item[Standard Error] The standard error, with sample size n=120 is computed as follows
\[ \mbox{S.E.}(\hat{p}) = \sqrt{\frac{\hat{p} \times (1-\hat{p})}{n}} =  \sqrt{\frac{0.35 \times 0.65}{160}}\]
\item (Full solution to follow on whiteboard)
\end{description}
%===================================================================================%
\subsection*{Confidence Interval for a proportion}

Refer back to our earlier example of the proportion of Python programmers. Compute a $95\%$ confidence interval.\\

\begin{itemize}
    \item \textbf{Determining the Quantile}
The confidence level is $95\%$. The sample size is greater than 30. Therefore the appropriate quantile is 1.96.\\
\item 
\textbf{Computing the Standard Error}

\[
S.E. (\hat{p}) \;=\; \sqrt{ {35 \times 65 \over 160 }} =  3.77\%
\] \bigskip

\item \textbf{Confidence Interval for proportion}

\[
35 \pm (1.96 \times 3.77) \%  = (35 \pm7.4) \% = (27.6\%,42.4\%)
\]
\end{itemize}



\end{document}
