
\documentclass[]{report}

\voffset=-1.5cm
\oddsidemargin=0.0cm
\textwidth = 480pt

\usepackage{framed}
\usepackage{subfiles}
\usepackage{graphics}
\usepackage{newlfont}
\usepackage{eurosym}
\usepackage{amsmath,amsthm,amsfonts}
\usepackage{amsmath}
\usepackage{color}
\usepackage{amssymb}
\usepackage{multicol}
\usepackage[dvipsnames]{xcolor}
\usepackage{graphicx}
\begin{document}

Question 4
An insurance company wants to estimate the percentage of drivers who talk on their mobile phones while driving. 
A random sample of 850 drivers results in 544 who talk on their mobile phones while driving.
(a) Find the point estimate of the percentage of all drivers who talk on their cell phones while driving.
(b) Find a 95\% interval estimate of the percentage of all drivers who talk on their cell phones while driving. 



Q8. Of a sample of 500 data scientists, 344 cited R as their primary programming language. 
Let $\pi$ be the proportion of all computer programmers who regard R as their primary programming 
language. Estimate $\pi$ and provide a 95\% confidence interval for $\pi$.



\subsection*{Question 6}
Assume that a particular brand dominates the market. More specifically, it is well-known that at least 60\% of people use this brand (i.e., $p \ge 0.6$). However, in response to recent media claims that this brand is weakening, the company wish to test the hypothesis that $p \ge 0.6$. \\[-0.2cm]

{\bf(a)} State the null and alternative hypotheses. \quad {\bf(b)} From a sample of 1000 people, it is found that 629 use this brand; calculate the test statistic and, hence, the p-value. \quad {\bf(c)} Based on the evidence, state your conclusion.


%---------------------------------------------------------------------------------------------%
\section{Inference on Proportions}

Out of 800 randomly selected drivers, 376 admitted that, on occation, they have driven through red lights.

The claim that the majority 0f drivers have drivien through red lights.
\begin{enumerate}[(a)]
\item What is the point estimate? 
\item Compute the 95\% confidence interval. 
\item Based on the confidence interval, state your conclusions about the claim.
\end{enumerate}

\subsection*{Point Estimate}
\begin{itemize}
\item $\hat{p}$ is the sample proportion (or sample percentage).
\item $x$ is the number of successes
\item $n$ is the sample size 
\end{itemize}
\[ \hat{p} = { x \over n} \]
$\hat{p}$ = 376/800 = 0.47  (i.e. $47\%$)

\subsection*{Standard Error}
The relevant formulae for standard errors are generally included at the back of exam papers.
The standard error is computed according to the following formula.


\vspace{0.1cm}
\[
S.E. (\hat{p}) = \sqrt{ { \hat{p} \times ( 1 - \hat{p}) \over n}}
\]
However, it is easier to perform such calculation when working in terms of percentages.

\vspace{0.1cm}
\[
S.E. (\hat{p}) = \sqrt{ { p \times (100-p) \over n}}  \;[\%]
\]
\vspace{0.1cm}

\begin{itemize}
\item $n$ is the sample size. For our example $n = 800$.


\item $\hat{p}$ is the sample proportion.
(Recall $\hat{p} = 0.47$ i.e (47\%))


\item $1 - \hat{p}$ is the complement of the sample proportion \begin{itemize}
\item For our example $1- \hat{p} = 1 - 0.47 =  0.53$. (i.e. 53\%)
\end{itemize}
\end{itemize}



Using these values, we can calculate the standard error with this expression. For the sake of simplicity, we will work in terms of percentages.

\vspace{0.1cm}
\[
S.E. (\hat{p}) = \sqrt{ { 47 \times 53 \over 800}} = \frac{\sqrt{2491}}{800} = \sqrt{3.11} = 1.76
\]

\vspace{0.1cm}

\subsection*{Quantile}
For a large sample, as is the case here, the quantile for the 95\% confidence interval can simply be stated as 1.96, without any requirement to justify it.

\subsection*{The 95\% Confidence Interval}

\[ \mbox{Point Estimate} \pm \left( \mbox{Quantile} \mbox{Std. Error} \right) \]

\[\mbox{95\% Confidence Interval} = (47 - (1.96 \times 1.76), 47 + (1.96 \times 1.76) ) = (43.55,50.45) \]

\subsection*{Conclusion}

\newpage


\subsection*{Question 8}
A soft drinks company is working on a new recipe for its best-selling drink. The company intends to carry out a study where participants will taste both flavours (current and new) and then answer the question:
\begin{quotation}
	``Do you prefer the new flavour?''
\end{quotation}
It is assumed that the \emph{current} recipe is superior, i.e., that \emph{less than or equal to} 50\% of people prefer the new drink ($p \le 0.5$).\\[0.4cm]
We wish to test the hypothesis that $p \le 0.5$.\\[-0.2cm]

\begin{enumerate}[(a)]
\item State the null and alternative hypotheses. 
\item From a sample of 65 people, we find that 43 people prefer the new recipe. Calculate the test statistic and, hence, the p-value. 
\item Based on the evidence, state your conclusion.
\end{enumerate}

\end{document}
