	\documentclass[a4paper,12pt]{article}
%%%%%%%%%%%%%%%%%%%%%%%%%%%%%%%%%%%%%%%%%%%%%%%%%%%%%%%%%%%%%%%%%%%%%%%%%%%%%%%%%%%%%%%%%%%%%%%%%%%%%%%%%%%%%%%%%%%%%%%%%%%%%%%%%%%%%%%%%%%%%%%%%%%%%%%%%%%%%%%%%%%%%%%%%%%%%%%%%%%%%%%%%%%%%%%%%%%%%%%%%%%%%%%%%%%%%%%%%%%%%%%%%%%%%%%%%%%%%%%%%%%%%%%%%%%%
\usepackage{eurosym}
\usepackage{vmargin}
\usepackage{framed}
\usepackage{amsmath}
\usepackage{graphics}
\usepackage{epsfig}
\usepackage{subfigure}
\usepackage{enumerate}
\usepackage{fancyhdr}

\setcounter{MaxMatrixCols}{10}
%TCIDATA{OutputFilter=LATEX.DLL}
%TCIDATA{Version=5.00.0.2570}
%TCIDATA{<META NAME="SaveForMode"CONTENT="1">}
%TCIDATA{LastRevised=Wednesday, February 23, 201113:24:34}
%TCIDATA{<META NAME="GraphicsSave" CONTENT="32">}
%TCIDATA{Language=American English}

\pagestyle{fancy}
\setmarginsrb{20mm}{0mm}{20mm}{25mm}{12mm}{11mm}{0mm}{11mm}
\lhead{MS4222} \rhead{Kevin O'Brien} \chead{Confidence Intervals} %\input{tcilatex}

\begin{document}

\section{Worked Examples}
%% Question 2 part b

A study of 1000 randomly chosen adults indicated that 450 had been to the cinema at least once in the previous year.

A cinema wants to test the hypothesis that 50\% of all Irish adults have been to the cinema in the last year.

Calculate the p-value for such a test and draw the appropriate conclusion.

Discussion: Based on this sample, we estimate the proportion to be 0.45  (i.e. 45\%)

p= 0.45

\noindent \textbf{Step A : Formally state the null and alternative hypotheses}

\begin{itemize}
	\item p : true proportion of Irish adults who have been to the cinema in the last year.
	
	\item	Null Hypothesis               Ho:p = 0.50        True proportion is 50%
	
	\item Alternative Hypothesis      Ha:p 0.50        True proportion is not 50%.
	
	
	\item	N.B. This is a two-tailed procedure.
\end{itemize}




\noindent \textbf{Step B : Compute the test statistic.}

Remember the general structure of a test statistic

TS =Observed Value-Null ValueStd. Error 



From the formulae

We have to compute the standard error for a proportion. 

( From formulae at back of exam paper)

S.E.(p) =p(1-p)n=0.450.551000= 0.0157




\noindent \textbf{Step 3: Calculate p-value}

P-value is found from Murdoch Barnes Tables 3 ( Normal distribution)

Absolute value  |-3.18| =3.18




\[ \mbox{P-Value} = P(Z \geq 3.18) = 0.00074\]


\noindent \textbf{Step 4: Interpret the p-value to make a decision.}

The significance level is 5\%.  The procedure is a two tailed test.


[ Black Board ]





\end{document}


