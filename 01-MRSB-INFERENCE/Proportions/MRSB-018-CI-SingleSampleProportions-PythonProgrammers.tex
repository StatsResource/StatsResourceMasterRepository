	\documentclass[a4paper,12pt]{article}
%%%%%%%%%%%%%%%%%%%%%%%%%%%%%%%%%%%%%%%%%%%%%%%%%%%%%%%%%%%%%%%%%%%%%%%%%%%%%%%%%%%%%%%%%%%%%%%%%%%%%%%%%%%%%%%%%%%%%%%%%%%%%%%%%%%%%%%%%%%%%%%%%%%%%%%%%%%%%%%%%%%%%%%%%%%%%%%%%%%%%%%%%%%%%%%%%%%%%%%%%%%%%%%%%%%%%%%%%%%%%%%%%%%%%%%%%%%%%%%%%%%%%%%%%%%%
\usepackage{eurosym}
\usepackage{vmargin}
\usepackage{amsmath}
\usepackage{framed}
\usepackage{graphics}
\usepackage{epsfig}
\usepackage{subfigure}
\usepackage{enumerate}
\usepackage{fancyhdr}

\setcounter{MaxMatrixCols}{10}
%TCIDATA{OutputFilter=LATEX.DLL}
%TCIDATA{Version=5.00.0.2570}
%TCIDATA{<META NAME="SaveForMode"CONTENT="1">}
%TCIDATA{LastRevised=Wednesday, February 23, 201113:24:34}
%TCIDATA{<META NAME="GraphicsSave" CONTENT="32">}
%TCIDATA{Language=American English}

\pagestyle{fancy}
\setmarginsrb{20mm}{0mm}{20mm}{25mm}{12mm}{11mm}{0mm}{11mm}
\lhead{StatsResource} \rhead{Confidence Intervals} \chead{Inference Procedures} %\input{tcilatex}

\begin{document}


\section*{Confidence Intervals for Proportions}
Of a sample of 160 computer programmers, 56 reported than Python was their primary programming language.
\begin{enumerate}[(a)]
	\item Let $\pi$ be the proportion of all programmers who regard Python as their programming language. What is the point estimate for $\pi$ (expressed as a percentage)?
	\item Compute a $95\%$ confidence interval for $\pi$.
\end{enumerate}


\begin{framed}
The general structure of a confidence interval is as follows:

\[ \mbox{Point Estimate}  \pm \left[ \mbox{Quantile} \times \mbox{Standard Error} \right] \]


\begin{itemize}
\item Point Estimate: estimate for population parameter of interest, i.e. sample mean, sample proportion.
\item Quantile: a value from a probability distribution that scales the intervals according to the specified confidence level.
\item Standard Error: measures the dispersion of the sampling distribution for a given sample size $n$.
\end{itemize}
\end{framed}

\subsection*{Point Estimate}
\begin{itemize}
\item The point estimate is the sample proportion, denoted $\hat{p}$.  
\item The sample proportion is calculated as the number of `successes' ($x$) divided by the total number of cases, in other words, the sample size $n$.
\[  \hat{p} = \frac{x}{n}  \]
\end{itemize}

\[
\hat{p} = \frac{x}{n} \times 100\% 
\]
\[
\hat{p} = {56 \over 160} = 35\%
\]


\begin{framed}
	\textbf{Confidence Intervals for Sample Proportion}
	
	\begin{itemize}
		\item The structure of a confidence interval for sample proportion is 
		\[ \hat{p} \pm z_{(\alpha/2)} \times \mbox{S.E.}(\hat{p})\]
		
		\item The standard error, in the case of sample proportions, is
		\[ \mbox{S.E.}(\hat{p}) = \sqrt{\frac{\hat{p}\times (1-\hat{p})}{n}}\]
		
	\end{itemize}
\end{framed}
\medskip
\begin{framed}
\begin{description}
	\item[Point Estimate] The sample proportion is computed as follows
	\[ \hat{p} = \frac{x}{n} = \frac{56}{160} = 0.35 \;\; \mbox{ i.e. }35\%\]
	\item[Quantile] We are asked for a 95\% confidence interval.The sample size is greater than 30.\\ \noindent The quantile is therefore
	\[ Z_{\alpha/2} =1.96\]
	\item[Standard Error] The standard error, with sample size n=120 is computed as follows
	\[ \mbox{S.E.}(\hat{p}) = \sqrt{\frac{\hat{p} \times (1-\hat{p})}{n}} =  \sqrt{\frac{0.35 \times 0.65}{160}}\]
	\end{description}
	\end{framed}
\noindent In terms of percentages:
\[
S.E. (\hat{p}) \;=\; \sqrt{\frac{\hat{p} \times (100-\hat{p})}{n}} \;=\; \sqrt{ {35 \times 65 \over 160 }} =  3.77\%
\] 

\medskip
\noindent Piecing it together.

\[
35\% \pm (1.96 \times 3.77) \%  = (35 \pm7.4) \% = (27.6\%,42.4\%)
\]
\smallskip
\noindent The 95\% confidence interval for the true proportion is (27.6\%,42.4\%).
\end{document}
