\documentclass[]{report}

\voffset=-1.5cm
\oddsidemargin=0.0cm
\textwidth = 480pt

\usepackage{framed}
\usepackage{subfiles}
\usepackage{graphics}
\usepackage{newlfont}
\usepackage{eurosym}
\usepackage{amsmath,amsthm,amsfonts}
\usepackage{amsmath}
\usepackage{color}
\usepackage{amssymb}
\usepackage{multicol}
\usepackage[dvipsnames]{xcolor}
\usepackage{graphicx}
\begin{document}

\section{Confidence Interval examples}


%--------------------
\subsection{Example}
In an election campaign, a campaign manager requests that a sample of votes be polled to determine public support for a candidate. In a sample of 150 votes 72 expressed plans to support the candidate.



\begin{itemize}
	\item What is the point estimate of the proportion of the voters who will support the candidate in the election?
	\item Contruct and interpret a 95\% confidence interval for the proportion of votes in the population that support the candidate.
	
	\item Given the confidence interval, is the campaign manager justified in feeling confident that the candidate has at least 50\% support
	
	\[S.E. (\hat{P}) = \sqrt{{\hat{p}(1-\hat{p} \over n}}\]
\end{itemize}


\begin{framed}
The general structure of a confidence interval is as follows:

\[ \mbox{Point Estimate}  \pm \left[ \mbox{Quantile} \times \mbox{Standard Error} \right] \]


\begin{itemize}
\item Point Estimate: estimate for population parameter of interest, i.e. sample mean, sample proportion.
\item Quantile: a value from a probability distribution that scales the intervals according to the specified confidence level.
\item Standard Error: measures the dispersion of the sampling distribution for a given sample size $n$.
\end{itemize}
\end{framed}

\textbf{Computing the point estimate}

Sample percentage

\[
\hat{p} = \frac{x}{n} \times 100%
\]

\begin{itemize}
\item $\hat{p}$ - sample proportion.
\item $x$  - number of ``successes".
\item $n$  - the sample size.
\end{itemize}


\subsection*{Point Estimates for proportions }

Sample percentage

\[
\hat{p} = \frac{x}{n} \times 100\%
\]

\begin{itemize}
\item $\hat{p}$ - sample proportion.
\item $x$  - number of ``successes".
\item $n$  - the sample size.
\end{itemize}



\end{document}