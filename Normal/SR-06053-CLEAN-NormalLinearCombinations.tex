
\setcounter{MaxMatrixCols}{10}

\begin{document}
\large
\noindent 
Let \(X_i\) denote the weight of a randomly selected prepackaged one-pound bag of carrots. Of course, one-pound bags of carrots won't weigh exactly one pound. 

In fact, history suggests that \(X_i\) is normally distributed with a mean of 1.18 pounds and a standard deviation of 0.07 pound.

Now, let \(W\) denote the weight of randomly selected prepackaged three-pound bag of carrots. 

Three-pound bags of carrots won't weigh exactly three pounds either. In fact, history suggests that \(W\) is normally distributed with a mean of 3.22 pounds and a standard deviation of 0.09 pound.

Selecting bags at random, what is the probability that the sum of three one-pound bags exceeds the weight of one three-pound bag?


\section*{Solution}

Because the bags are selected at random, we can assume that \(X_1, X_2, X_3\) and \(W\) are mutually independent. 

The theorem helps us determine the distribution of \(Y\), the sum of three one-pound bags:

\[Y=(X_1+X_2+X_3) \sim N(1.18+1.18+1.18, 0.07^2+0.07^2+0.07^2)=N(3.54,0.0147)\]

That is, \(Y\) is normally distributed with a mean of 3.54 pounds and a variance of 0.0147. 

Now, \(Y-W\), the difference in the weight of three one-pound bags and one three-pound bag is normally distributed with a mean of 0.32 and a variance of 0.0228, as the following calculation suggests:

\[(Y-W) \sim N(3.54-3.22,(1)^2(0.0147)+(-1)^2(0.09^2))=N(0.32,0.0228)\)\]

\noindent Therefore, finding the probability that \(Y\) is greater than \(W\) reduces to a normal probability calculation:

\begin{eqnarray*} P(Y\geq W) &=& P(Y-W\geq 0)\\ &=&  P\left(Z\geq \dfrac{0-0.32}{\sqrt{0.0228}}\right)\\ &=&  P(Z\geq -2.12)\\ &=& P(Z\leq 2.12)\\ &=& 0.9830\\ \end{eqnarray*}

That is, the probability that the sum of three one-pound bags exceeds the weight of one three-pound bag is 0.9830. 

%%-- Hey, if you want more bang for your buck, it looks like you should buy multiple one-pound bags of carrots, as opposed to one three-pound bag!

\end{document}

      
          
                
                                      

          