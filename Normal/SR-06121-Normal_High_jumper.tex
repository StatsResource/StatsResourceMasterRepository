
%--------------------------------------------------------------%
\subsection*{Part 2B}
(b)	A student is practising for an upcoming high jump event.  The height that she will clear each time she jumps is normally distributed with a 
mean of 72 inches and a standard deviation of 4 inches.  
\begin{itemize}
\item[(i)]  	What is the probability that the jumper will clear 76 inches or higher on a single jump?
\item[(ii)]  	What is the probability that the height she jumps is between 68 and 76 inches on a single jump?
\item[(iii)]  	What is the minimum height she must jump in order for the jump to be in the highest 10%?
\end{itemize}
    

%--------------------------------------------------------------%
\subsection*{Part 2C}
(c)	Telephone calls coming in to a busy switchboard follow a Poisson distribution with 3 calls expected in a one minute period.  
The switchboard operator can answer at most 3 calls in a one minute period; the fourth and succeeding calls receive a busy signal.
\begin{itemize} 
\item[(i)]  	   Find the probability of receiving a busy signal.
\item[(ii)]  	   The switchboard operator leaves the switchboard unattended for 2 minutes.  
\item[(iii)]  What is the probability that exactly 1 call will be missed during that 2 minute period?
\end{itemize}       

(d)	In what circumstances can the Poisson distribution be used to approximate the Binomial distribution?
      
