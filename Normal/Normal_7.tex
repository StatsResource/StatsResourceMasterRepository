  \documentclass[a4]{beamer}
\usepackage{amssymb}
\usepackage{graphicx}
\usepackage{subfigure}
\usepackage{newlfont}
\usepackage{amsmath,amsthm,amsfonts}
%\usepackage{beamerthemesplit}
\usepackage{pgf,pgfarrows,pgfnodes,pgfautomata,pgfheaps,pgfshade}
\usepackage{mathptmx} % Font Family
\usepackage{helvet} % Font Family
\usepackage{color}
\mode<presentation> {
\usetheme{Frankfurt} % was Frankfurt
\useinnertheme{rounded}
\useoutertheme{infolines}
\usefonttheme{serif}
%\usecolortheme{wolverine}
% \usecolortheme{rose}
\usefonttheme{structurebold}
}
\setbeamercovered{dynamic}
\title[MathsCast]{MathsCast Presentations \\ {\normalsize MA4413 Lecture 5B}}
\author[Kevin O'Brien]{Kevin O'Brien \\ {\scriptsize kevin.obrien@ul.ie}}
\date{Summer 2011}
\institute[Maths \& Stats]{Dept. of Mathematics \& Statistics, \\ University \textit{of} Limerick}
\renewcommand{\arraystretch}{1.5}
%------------------------------------------------------------------------%
\begin{document}
\begin{frame}
\titlepage
\end{frame}

%-----------------------------------------------------%
\begin{frame}
\frametitle{Normal Distribution : Solving problems}
In today's class, we will continue with the Normal Distribution.
\begin{itemize}
\item We must know the normal mean $\mu$ and the normal standard deviation $\sigma$.
\item The normal random variable is $X \sim \mbox{N} ( \mu , \sigma^2)$.
\item (If we don't, we usually have to determine them, given the information in the question.)
\item The standard normal random variable is $Z\sim \mbox{N} ( 0 , 1^2)$.
\item The standard normal distribution is well described in Murdoch Barnes Table 3, which tabulates $P(Z \geq z_o)$ for a range of $Z$ values.
\end{itemize}
\end{frame}
%-----------------------------------------------------%
\begin{frame}
\frametitle{Normal Distribution : Solving problems}
\begin{itemize}
\item For the given value $x_o$ from the variable $X$, we compute the corresponding z-score $z_o$.
\[ z_o = { x_o - \mu \over \sigma} \]
\item When $z_o$ corresponds to $x_o$, the following identity applies:
\[  P(X \geq x_o )= P(Z \geq z_o ) \]
\item Alternatively $ P(X \leq x_o )= P(Z \leq z_o ) $
\end{itemize}
\end{frame}
%-----------------------------------------------------%
\begin{frame}
\frametitle{Normal Distribution : Solving problems}
\begin{itemize}
\item \textbf{Complement Rule}: \[ P(Z \leq k) = 1-P(Z \geq k) \] for some value $k$
\item Alternatively $ P(Z \geq k) = 1-P(Z \leq k) $ 
\item \textbf{Symmetry Rule}: \[ P(Z \leq -k) = P(Z \geq k) \] for some value $k$
\item Alternatively $ P(Z \geq -k) = P(Z \leq k) $
\end{itemize}
\end{frame}
%-----------------------------------------------------%
\begin{frame}
\frametitle{Normal Distribution : Solving problems}
\begin{itemize}

\item \textbf{Intervals}: \[ P(L \leq Z \leq U) = 1- [ P(Z \leq L) + P(Z \geq U)] \]
where $L$ and $U$ are the lower and upper bounds of an interval.
\item Probability of having a value too low for the interval : $P(Z \leq L)$
\item Probability of having a value too high for the interval : $P(Z \geq U)$
\end{itemize}
\end{frame}
%-----------------------------------------------------%



%-------------------------------------------------------------%
\frame{
\frametitle{Working Backwards}
\begin{itemize}

\item Suppose we wish to find a value (lets call it A) from the normal distribution, such that a certain proportion of values is greater than A (e.g. 10\%)
\item Find A such that $P(X \geq A) = 0.10$. (with $\mu  = 350$ and $\sigma = 17$)
\item In general, our first step is to use the standardization equation to find the corresponding Z-score $z_A$.
\item Because we don't know what value A has, we can't use this approach.
\item However, we can say the following
\[  P(X \geq A) = P(Z \geq z_A) = 0.10 \]

\item From the tables, we can approximate a value for $z_A$, by finding the closest probability value, and determining the corresponding Z-score.
\end{itemize}
}

%------------------------------------------------------------------------%
\frame{

\frametitle{Find $z_A$ such that $ P(Z \geq z_a) = 0.10$}
\begin{itemize}
\item The closest probability value in the tables is $0.1003$.\\
\item The Z-score that corresponds to $0.1003$ is 1.28.\\
\item (Row : 1.2 , Column : 0.08) 
\item Therefore $z_A  \approx 1.28$
\end{itemize}
\small
\begin{table}[ht]
%\caption{Standard Normal Distribution } % title of Table
\centering % used for centering table
\begin{tabular}{|c|| c c c c c c|} % centered columns (4 columns)
\hline %inserts double horizontal lines
& \ldots & \ldots & 0.006 &0.07&0.08&0.09 \\
%heading
\hline \hline% inserts single horizontal line
\ldots & \ldots & \ldots &\ldots& \ldots &\ldots&\dots \\ % inserting body of the table
1.0 & \ldots & \ldots &0.1446& 0.1423 &0.1401&0.1379 \\ % inserting body of the table
1.1 & \ldots & \ldots&0.1230& 0.1210 &0.1190&0.1170 \\ % inserting body of the table
1.2 & \ldots & \ldots&0.1038 & 0.1020 &\alert{0.1003}&0.0985\\
1.3 & \ldots & \ldots &0.0869& 0.0853 &0.0838&0.0823 \\ % inserting body of the table
\ldots & \ldots &\ldots&\ldots & \ldots &\ldots&\ldots\\
\hline %inserts single line
\end{tabular}
%\label{table:nonlin} % is used to refer this table in the text
\end{table}
}
%-------------------------------------------------------------%
\frame{
\frametitle{Working Backwards}
\begin{itemize}
\item We can now use the standardization formula. 
\item We have only one unknown in the formula: $A$.
\[ 1.28 = {A - 350 \over 17} \]
\item Re-arranging ( multiply both sides by 17):\\
$ 21.76 = A - 350 $
\item Re-arranging ( add 350 to both sides ):\\
$ A = 371.76 $
\item $P(X \geq 371.76) \approx 0.10$
\item (Remark: for sums of die-throws, round it to nearest value)
\end{itemize}
}
%-----------------------------------------------------%

%-------------------------------------------------------------%
\frame{
\frametitle{Working Backwards: Another Example}
\begin{itemize}

\item Find B such that $P(X \geq B) = 0.90$. (with $\mu  = 350$ and $\sigma = 17$)
\item Necessarily $P(X \leq B) = 0.10$
\item Find some value $Z_B$ such that $P(Z \leq z_B) = 0.10$
\item $z_B$ could be negative.
\item Use the symmetry rule $P(Z \leq z_B) = P(Z \geq -z_B)$
\item $-z_B$ could be positive.
\item Based on last example $-z_B = 1.28$. Therefore $z_B = -1.28$
\end{itemize}
}
%-------------------------------------------------------------%
\frame{
\frametitle{Working Backwards}
\begin{itemize}
\item Again ,we can now use the standardization formula
\item We have only one unknown in the formula: $B$.
\[ -1.28 = {B - 350 \over 17} \]
\item Re-arranging ( multiply both sides by 17):\\
$ -21.76 = B - 350 $
\item Re-arranging ( add 350 to both sides ):\\
$ x_o = 350 - 21.76 = 328.24 $
\item $P(X \leq 328.24) \approx 0.10$
\end{itemize}
}
%-----------------------------------------------------%



%---------------------------------------------%
\end{document} 
