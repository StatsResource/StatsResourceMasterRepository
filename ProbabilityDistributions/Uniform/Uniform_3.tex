
%---------------------------------------------------------------------------%
{
\textbf{Continuous Uniform Distribution}
A random variable X is called a continuous uniform random variable over the interval $(a,b)$ if it's probability density function is given by
\[ f_{X}(x) = { 1 \over b-a} \hspace{1cm} \mbox{ when } a \leq x \leq b \mbox{     (otherwise } f_X(x) = 0 ) \]
The corresponding cumulative density function is
\[ F_x(x) = { x-a \over b-a} \hspace{1cm} \mbox{ when } a \leq x \leq b\]
}
%----------------------------------------------------------------------------------------------------%

\textbf{The Continuous Uniform Distribution}



\begin{center}
\includegraphics[scale=0.35]{images/6AUniform}

\end{center}
\medskip
%---------------------------------------------------------------------------%
{
\textbf{Continuous Uniform Distribution}
\begin{itemize}
\item The continuous uniform distribution is very simple to understand and implement, and is commonly used in computer applications (e.g. computer simulation).
\item It is also known as the `Rectangle Distribution' for obvious reasons.
\item We specify the word ``continuous" so as to distinguish it from it's discrete equivalent: the discrete uniform distribution.
\item Remark; the dice distribution is a discrete uniform distribution with lower and upper limits 1 and 6 respectively.
\end{itemize}
}
%-----------------------------------------------------%
{
\textbf{Uniform Distribution Parameters}


The continuous uniform distribution is characterized by the following parameters

\begin{itemize}
\item The lower limit $a$
\item The upper limit $b$
\item We denote a uniform random variable $X$ as $X \sim U(a,b)$
\end{itemize}

It is not possible to have an outcome that is lower than $a$ or larger than $b$.

\[ P(X < a) = P(X > b) = 0\]
}

%------------------------------------------------------------------------%
{\textbf{Interval Probability}

\begin{itemize}
\item We wish to compute the probability of an outcome being within a range of values.
\item We shall call this lower bound of this range $L$ and the upper bound $ U$.
\item Necessarily $L$ and $U$ must be possible outcomes.
\item The probability of $X$ being between $L$ and $U$ is denoted $P( L \leq X \leq U )$.

\[
P( L \leq X \leq U ) = { U - L \over b - a}
\]
\item (This equation is based on a definite integral).
\end{itemize}
}

%---------------------------------------------------------------------------------------------------------%

