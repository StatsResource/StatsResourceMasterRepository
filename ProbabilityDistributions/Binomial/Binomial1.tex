	\documentclass[a4paper,12pt]{article}
%%%%%%%%%%%%%%%%%%%%%%%%%%%%%%%%%%%%%%%%%%%%%%%%%%%%%%%%%%%%%%%%%%%%%%%%%%%%%%%%%%%%%%%%%%%%%%%%%%%%%%%%%%%%%%%%%%%%%%%%%%%%%%%%%%%%%%%%%%%%%%%%%%%%%%%%%%%%%%%%%%%%%%%%%%%%%%%%%%%%%%%%%%%%%%%%%%%%%%%%%%%%%%%%%%%%%%%%%%%%%%%%%%%%%%%%%%%%%%%%%%%%%%%%%%%%
\usepackage{eurosym}
\usepackage{vmargin}
\usepackage{amsmath}
\usepackage{framed}
\usepackage{graphics}
\usepackage{epsfig}
\usepackage{subfigure}
\usepackage{enumerate}
\usepackage{fancyhdr}

\setcounter{MaxMatrixCols}{10}
%TCIDATA{OutputFilter=LATEX.DLL}
%TCIDATA{Version=5.00.0.2570}
%TCIDATA{<META NAME="SaveForMode"CONTENT="1">}	\documentclass[a4paper,12pt]{article}
%%%%%%%%%%%%%%%%%%%%%%%%%%%%%%%%%%%%%%%%%%%%%%%%%%%%%%%%%%%%%%%%%%%%%%%%%%%%%%%%%%%%%%%%%%%%%%%%%%%%%%%%%%%%%%%%%%%%%%%%%%%%%%%%%%%%%%%%%%%%%%%%%%%%%%%%%%%%%%%%%%%%%%%%%%%%%%%%%%%%%%%%%%%%%%%%%%%%%%%%%%%%%%%%%%%%%%%%%%%%%%%%%%%%%%%%%%%%%%%%%%%%%%%%%%%%
\usepackage{eurosym}
\usepackage{vmargin}
\usepackage{amsmath}
\usepackage{framed}
\usepackage{graphics}
\usepackage{epsfig}
\usepackage{subfigure}
\usepackage{enumerate}
\usepackage{fancyhdr}

\setcounter{MaxMatrixCols}{10}
%TCIDATA{OutputFilter=LATEX.DLL}
%TCIDATA{Version=5.00.0.2570}
%TCIDATA{<META NAME="SaveForMode"CONTENT="1">}
%TCIDATA{LastRevised=Wednesday, February 23, 201113:24:34}
%TCIDATA{<META NAME="GraphicsSave" CONTENT="32">}
%TCIDATA{Language=American English}

\pagestyle{fancy}
\setmarginsrb{20mm}{0mm}{20mm}{25mm}{12mm}{11mm}{0mm}{11mm}
\lhead{MS4222} \rhead{Kevin O'Brien} \chead{Binomial Distribution} %\input{tcilatex}
\begin{document}

%================================================%
\section*{Bernouilli Trial}
\begin{itemize}
\item Now consider an experiment with only two outcomes. Independent repeated trials of such an experiment are
called \textbf{\textit{Bernoulli trials}}, named after the Swiss mathematician Jacob Bernoulli (1654 to 1705). \item The term \textbf{\emph{independent
trials}} means that the outcome of any trial does not depend on the previous outcomes (such as tossing a coin).
\item We will call one of the outcomes the ``success" and the other outcome the ``failure".
\item
Let $p$ denote the probability of success in a Bernoulli trial, and so $q = 1 - p$ is the probability of failure.


\item A \textbf{\textit{Bernoulli
Trial}} is a sampling process in which
\begin{itemize}
	\item[(1)] Only two mutually exclusive possible outcomes are possible in each trial, or observation. For
	convenience these are called the success and the failure. The "success" usually refers to the outcome of interest.
	\item[(2)] The outcomes in the series of trials, or observations, constitute independent events.
	\item[(3)] The probability of success in each trial, denoted by p, remains constant from trial to trial. That is,
	the process is stationary.
	\item[(4)] The trials are independent; that is, the outcome on one trial does not affect the outcome on other trials.
\end{itemize}
\end{itemize}
\section*{Binomial Random Variables}
\begin{itemize}
\item A binomial experiment consists of a fixed number of Bernoulli trials, which we denote as $n$. 

% \item A binomial experiment with $n$ trials and
% probability $p$ of success will be denoted by
%\[B(n, p)\]


\item We would denote a binomial random variable $X$ with $n$ trials and
probability $p$ of success as follows
\[X \sim Bin(n, p)\]
\item The binomial distribution  is a discrete probability distribution that is applicable as a model for decision-making
situations in which a sampling process can be assumed to conform to a Bernoulli process. 
	\item Frequently, we are interested in the \textbf{\emph{number of successes}} in a binomial experiment, not in the order in which they occur.
	\item Furthermore, we are interested in the probability of that number of successes.
\end{itemize}

\newpage


\section*{The Binomial Probability Distribution}

\begin{itemize}
	\item The binomial probability distribution is a discrete probability distribution that has many applications.
	\item It is associated with a multiple step experiment that we call the binomial experiment.
	\item In a binomial experiment our interest is in the number of successes occurring in the n trials.
\end{itemize}

\medskip

%The binomial distribution is a particular example of a probability distribution involving a discrete random variable. 
\noindent It is important that you can identify situations which can be modelled using the binomial distribution. 
\begin{itemize}
	\item There are n independent trials 
	\item There are just two possible outcomes to each trial, success and failure, with fixed probabilities of p and q respectively, where q = 1 - p. 
\end{itemize}




%A binomial experiment (also known as a Bernoulli trial) is a statistical experiment that has the following properties:
%
%The experiment consists of n repeated trials.



%
%If the outcome of interest is something like a flat tire, using the word "success" is coutner intuituive.
%
%
%Each trial can result in just two possible outcomes. We call one of these outcomes a success and the other, a failure.
%
%












%http://www.wbs.eu.com/SharedFiles/Maths/statistics%201%20revision/introducing%20binomial.pdf

\begin{itemize}
	\item The Binomial Distribution is characterized by the following parameters
	
	
	\begin{description}
		\item[n] - the number of trials
		
		\item[p] - The probability of a "success"
	\end{description}	
	
	\item The expected number of `successes' from $n$ trials is $E(X) = np$


	\end{itemize}
%===============================================================%
\subsection*{Examples}
Consider the following statistical experiment. You flip a coin five times and count the number of times the coin lands on heads. This is a binomial experiment because:
\begin{itemize}
\item The experiment consists of repeated trials. We flip a coin five times.
\item Each trial can result in just two possible outcomes: heads or tails. We call one of these outcomes a success and the other, a failure, depending on the question asked.

\item The probability of success, denoted by $p$ is constant, e.g. 0.5 on every trial for a fair coin. The probability of success is the same on every trial.
\item The trials are independent; that is, getting heads on one trial does not affect whether we get heads on other trials.
\end{itemize}



\end{document}

%================================================%
\section*{Bernouilli Trial}
\begin{itemize}
\item Now consider an experiment with only two outcomes. Independent repeated trials of such an experiment are
called \textbf{\textit{Bernoulli trials}}, named after the Swiss mathematician Jacob Bernoulli (1654 to 1705). \item The term \textbf{\emph{independent
trials}} means that the outcome of any trial does not depend on the previous outcomes (such as tossing a coin).
\item We will call one of the outcomes the ``success" and the other outcome the ``failure".
\item
Let $p$ denote the probability of success in a Bernoulli trial, and so $q = 1 - p$ is the probability of failure.


\item A \textbf{\textit{Bernoulli
Trial}} is a sampling process in which
\begin{itemize}
	\item[(1)] Only two mutually exclusive possible outcomes are possible in each trial, or observation. For
	convenience these are called the success and the failure. The "success" usually refers to the outcome of interest.
	\item[(2)] The outcomes in the series of trials, or observations, constitute independent events.
	\item[(3)] The probability of success in each trial, denoted by p, remains constant from trial to trial. That is,
	the process is stationary.
	\item[(4)] The trials are independent; that is, the outcome on one trial does not affect the outcome on other trials.
\end{itemize}
\end{itemize}
\section*{Binomial Random Variables}
\begin{itemize}
\item A binomial experiment consists of a fixed number of Bernoulli trials, which we denote as $n$. 

% \item A binomial experiment with $n$ trials and
% probability $p$ of success will be denoted by
%\[B(n, p)\]


\item We would denote a binomial random variable $X$ with $n$ trials and
probability $p$ of success as follows
\[X \sim Bin(n, p)\]
\item The binomial distribution  is a discrete probability distribution that is applicable as a model for decision-making
situations in which a sampling process can be assumed to conform to a Bernoulli process. 
	\item Frequently, we are interested in the \textbf{\emph{number of successes}} in a binomial experiment, not in the order in which they occur.
	\item Furthermore, we are interested in the probability of that number of successes.
\end{itemize}

\newpage


\section*{The Binomial Probability Distribution}

\begin{itemize}
	\item The binomial probability distribution is a discrete probability distribution that has many applications.
	\item It is associated with a multiple step experiment that we call the binomial experiment.
	\item In a binomial experiment our interest is in the number of successes occurring in the n trials.
\end{itemize}

\medskip

%The binomial distribution is a particular example of a probability distribution involving a discrete random variable. 
\noindent It is important that you can identify situations which can be modelled using the binomial distribution. 
\begin{itemize}
	\item There are n independent trials 
	\item There are just two possible outcomes to each trial, success and failure, with fixed probabilities of p and q respectively, where q = 1 - p. 
\end{itemize}




%A binomial experiment (also known as a Bernoulli trial) is a statistical experiment that has the following properties:
%
%The experiment consists of n repeated trials.



%
%If the outcome of interest is something like a flat tire, using the word "success" is coutner intuituive.
%
%
%Each trial can result in just two possible outcomes. We call one of these outcomes a success and the other, a failure.
%
%












%http://www.wbs.eu.com/SharedFiles/Maths/statistics%201%20revision/introducing%20binomial.pdf

\begin{itemize}
	\item The Binomial Distribution is characterized by the following parameters
	
	
	\begin{description}
		\item[n] - the number of trials
		
		\item[p] - The probability of a "success"
	\end{description}	
	
	\item The expected number of `successes' from $n$ trials is $E(X) = np$


	\end{itemize}
