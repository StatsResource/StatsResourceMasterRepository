\documentclass[a4paper,12pt]{article}
%%%%%%%%%%%%%%%%%%%%%%%%%%%%%%%%%%%%%%%%%%%%%%%%%%%%%%%%%%%%%%%%%%%%%%%%%%%%%%%%%%%%%%%%%%%%%%%%%%%%%%%%%%%%%%%%%%%%%%%%%%%%%%%%%%%%%%%%%%%%%%%%%%%%%%%%%%%%%%%%%%%%%%%%%%%%%%%%%%%%%%%%%%%%%%%%%%%%%%%%%%%%%%%%%%%%%%%%%%%%%%%%%%%%%%%%%%%%%%%%%%%%%%%%%%%%
\usepackage{eurosym}
\usepackage{vmargin}
\usepackage{amsmath}
\usepackage{framed}
\usepackage{graphics}
\usepackage{epsfig}
\usepackage{subfigure}
\usepackage{enumerate}
\usepackage{fancyhdr}

\setcounter{MaxMatrixCols}{10}
%TCIDATA{OutputFilter=LATEX.DLL}
%TCIDATA{Version=5.00.0.2570}
%TCIDATA{<META NAME="SaveForMode"CONTENT="1">}
%TCIDATA{LastRevised=Wednesday, February 23, 201113:24:34}
%TCIDATA{<META NAME="GraphicsSave" CONTENT="32">}
%TCIDATA{Language=American English}

\pagestyle{fancy}
\setmarginsrb{20mm}{0mm}{20mm}{25mm}{12mm}{11mm}{0mm}{11mm}
\lhead{StatsResource} 
\chead{Information Theory} \rhead{Exercises} %\input{tcilatex}
\begin{document}

\large 
\chead{Probability Distributions}
%\input{tcilatex}

\begin{document}
\large 

%---------------------------------------------------------------------------%
{
\textbf{Binomial Probability Distribution: Example 2}
The vice-president of a computer firm has reviewed the records of the firm�s personnel and has found that 70\% of the employees read a well known industry magazine ``The IT Journal". \\ \bigskip
If the vice-president was to choose 10 employees at random, what is the probability that the number of these employees who do not read the ``IT Journal" is the following?
\normalsize
\begin{itemize}
\item [1] At least five.
\item [2] Between four and eight, inclusive.
\item [3] No more than seven.
% \item [4] What are the mean and variance of this distribution?
\end{itemize}
}

%---------------------------------------------------------------------------%

\textbf{Binomial Probability Distribution: Example 2}


\begin{itemize}
\item
Firstly, identify the probability distribution to be used?
    \begin{itemize}
    \item
    Answer: the binomial distribution
    \end{itemize}
\item We are given the number of trials ( `` choose 10 employees")

\item We are given a definition of a ``success", which is finding an employee that did NOT reads the journal.

\item We are given the probability of such a success : 30\%  or 0.30

\item So our binomial parameters are n= 10 and p = 0.30

\item Let's use the following \texttt{R} code to solve.

\end{itemize}

%---------------------------------------------------------------------------%

\textbf{Binomial Probability Distribution: Example 2}

\begin{verbatim}
> 0:10
 [1]  0  1  2  3  4  5  6  7  8  9 10
>
> dbinom(0:10,size=10,prob=0.30)
 [1] 0.0282475249 0.1210608210 0.2334744405 0.2668279320
 [5] 0.2001209490 0.1029193452 0.0367569090 0.0090016920
 [9] 0.0014467005 0.0001377810 0.0000059049
>
> pbinom(0:10,size=10,prob=0.30)
 [1] 0.02824752 0.14930835 0.38278279 0.64961072 0.84973167
 [6] 0.95265101 0.98940792 0.99840961 0.99985631 0.99999410
[11] 1.00000000
\end{verbatim}

%---------------------------------------------------------------------------%

\textbf{Binomial Probability Distribution: Example 2}
Question 1: Probability of at least five $P(X \geq 5)$.

\bigskip We have already determined the probability of the complement event $P(X \leq 4)$, which is $84.97\%$. Therefore the answer is $P(X \geq 5)$ = $15.03\%$.

\begin{verbatim}
> pbinom(0:10,size=10,prob=0.30)
 [1] 0.02824752 0.14930835 0.38278279 0.64961072 0.84973167
 [6] 0.95265101 0.98940792 0.99840961 0.99985631 0.99999410
[11] 1.00000000
\end{verbatim}


%---------------------------------------------------------------------------%

\textbf{Binomial Probability Distribution: Example 2}
Question 2: Probability of between 4 and 8 inclusive $P(4 \leq X \leq 8)$.

\bigskip Lets look at the sample space again, with the relevant sample points in bold: $S = \{0,1,2,3,\textbf{4,5,6,7,8},9,10,\}$.
\begin{verbatim}
> pbinom(0:10,size=10,prob=0.30)
 [1] 0.02824752 0.14930835 0.38278279 0.64961072 0.84973167
 [6] 0.95265101 0.98940792 0.99840961 0.99985631 0.99999410
[11] 1.00000000
\end{verbatim}
\begin{itemize}
\item $P(X \leq 8)$ is $99.98\%$, but this includes the probability of $X = \{0,1,2,3\}$
\item We can simply subtract $P(X \leq 3)$ from $P(X \leq 8)$ to get the desired value.
\item $(P 4 \leq X \geq 8)$ = $99.98\%$ - $64.96\%$ = $35.02\%$
\end{itemize}



%---------------------------------------------------------------------------%

\textbf{Binomial Probability Distribution: Example 2}
Question 3: Probability of no more than 7 $P(X \leq 7)$.
\begin{verbatim}
> pbinom(0:10,size=10,prob=0.30)
 [1] 0.02824752 0.14930835 0.38278279 0.64961072 0.84973167
 [6] 0.95265101 0.98940792 0.99840961 0.99985631 0.99999410
[11] 1.00000000
\end{verbatim}
\[ P(X \leq 7) = 99.84\% \]

%---------------------------------------------------------------------------%