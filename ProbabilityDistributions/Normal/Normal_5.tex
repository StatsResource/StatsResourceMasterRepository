
%------------------------------------------------------------%
\begin{frame}

\frametitle{Introduction to the Normal Distribution}
\begin{itemize}
\item
Recall the experiment whereby a die was rolled 100 times, and the sum of the 100 values was recorded.
\item
This experiment was repeated a very large number of times (e.g. 100,000 times ) in a simulation study.
\item
A histogram was drawn to depict the distribution of outcomes of this experiment.
\item Recall that we agreed that ``bell-shaped" was a good description of the histogram.

\end{itemize}
\end{frame}


\frame{
\frametitle{Normal Distribution}

\begin{center}
\includegraphics[scale=0.30]{images/3aDieHist3}
\end{center}

}
%----------------------------------------------------------------------%

\frame{
\frametitle{Normal Distribution}
\begin{itemize}
\item The normal distribution is perhaps the most widely used distribution for a random variable.
\item Normal distributions : the bell curve.
\item The distributions are \textbf{symmetric} with values concentrated more in the middle than in the tails.
%\item Examples of normal distributions are shown below. Notice that they differ in how spread out they are. The area under each curve is the same.
\item \alert{Important} The height of a normal distribution can be defined mathematically in terms of two fundamental parameters: the normal mean ($\mu$) and the normal
    standard deviation ($\sigma$).
\item A normally distributed random variable X is denoted $ X \sim \mbox{N} (\mu, \sigma^2)$ (note that we use the variance term here)
    \item The mean ($\mu$) and standard deviation ($\sigma$) are vital for calculating probabilities.
\end{itemize}
}


\frame{
\frametitle{Normal Distribution} 
\begin{itemize}
\item Normal distributions are a family of distributions that . 
\item They are symmetric with scores more concentrated in the middle than in the tails. Normal distributions are sometimes described as bell shaped. 
\item Examples of normal distributions are shown below. Notice that they differ in how spread out they are. The area under each curve is the same. 
\item The height of a normal distribution can be specified mathematically in terms of two parameters: the mean ($\mu$) and the standard deviation ($\sigma$). 

\end{itemize}
}

%------------------------------------------------------------------------%
\frame{
\frametitle{The Normal Distribution}
The \textbf{\emph{probability density function}} of the normal distribution is given as
\[ f(x) = \frac{1}{\sqrt{2\pi\sigma^2}} e^{ -\frac{(x-\mu)^2}{2\sigma^2} } \]

Integrating this formula would allow us to compute probabilities.
However, it is not required to use this formula.
}
%------------------------------------------------------------------%
\frame{
\frametitle{Normal Distribution}
\vspace{-0.25cm}
\begin{center}
\includegraphics[scale=0.450]{images/5ABellCurve}
\end{center}

}
%------------------------------------------------------------------%
\frame{
\frametitle{ Characteristics of the Normal probability distribution}
\begin{itemize}
\item[1] The highest point on the normal curve is at the mean, which is also the median and mode of the distribution. \smallskip
\item[2] \alert{[VERY IMPORTANT]}
The normal probability curve is bell-shaped and symmetric, with the shape of the curve to the left of the mean a mirror image of the shape of the curve to the right of the mean. (This is the basis of an important rule, called the \textbf{Symmetry Rule}, that we shall meet later.) \smallskip
\item[3] The standard deviation determines the width of the curve. Larger values of the the standard deviation result in wider flatter curves, showing more dispersion in data.
\item[4] As with all density curves, the total area under the curve for the normal probability distribution is 1.
\end{itemize}
}
%------------------------------------------------------------------%
\frame{
\frametitle{ Characteristics of the Normal probability distribution}
\textbf{Remark:} It is useful to know the following statements as rules of thumb, but we will do all relevant calculations from first principles. However, in an exam situation, these rules of thumb may be invoked, and it is required to show your workings.
\begin{itemize}
\item The interval defined by the mean $ \pm 1 \times $ standard deviation includes approximately $68\%$ of the observations, leaving $16\%$ (approx) in each tail.
\item The interval defined by the mean $ \pm 1.645 \times $ standard deviation includes approximately $90\%$ of the observations, leaving $5\%$ (approx) in each tail.
\item The interval defined by the mean $ \pm 1.96 \times $ standard deviation includes approximately $95\%$ of the observations, leaving $2.5\%$ (approx) in each tail.

\item The interval defined by the mean $ \pm 2.58 \times $ standard deviation includes approximately $99\%$ of the observations, leaving $0.5\%$ (approx) in each tail.
\end{itemize}

}


%------------------------------------------------------------------%
\frame{
\frametitle{Complement and Symmetry Rules}

Any normal distribution problem can be solved with some combination of the following rules.
\begin{itemize} \item \textbf{Complement rule} \item Common to all continuous random variables
\[P(Z \geq k) = 1 - P(Z \leq k) \]
Similarly
\[P(X \geq k) = 1 - P(X \leq k) \]
\end{itemize}

\[P(Z \leq 1.28) = 1 - P(Z \geq 1.28)  = 1-0.1003 = 0.8997\]
}

%------------------------------------------------------------------%
\frame{
\frametitle{Complement and Symmetry Rules}
\begin{itemize}
\item \textbf{Symmetry rule}
\item
This rule is based on the property of symmetry mentioned previously.
\item
Only the probabilities corresponding to values between 0 and 4 are tabulated in Murdoch Barnes.
\item
If we have a negative value of k, we can use the symmetry rule.
\end{itemize}
\[P(Z \leq -k) = P(Z \geq k) \]
by extension, we can say
\[P(Z \geq -k) = P(Z \leq k) \]
}
%------------------------------------------------------------------%
\frame{
\frametitle{Z Scores: Example 1 }
Find $P(Z \geq -1.28)$\\
\textbf{Solution}\\
\begin{itemize}
\item Using the symmetry rule
\[P(Z \geq -1.28) = P(Z \leq 1.28) \]
\item Using the complement rule
\[P(Z \geq -1.28) = 1 - P(Z \geq 1.28) \]
\[P(Z \geq -1.28) = 1 - 0.1003 = 0.8997 \]
\end{itemize}
}
%------------------------------------------------------------------%
\frame{
\frametitle{Z Scores: Example 2 }
Find the probability of a ``z" random variable being between -1.8 and 1.96?
i.e. Compute $P(-1.8 \leq Z \leq 1.96)$\\
Solution
\begin{itemize}
\item Consider the complement event of being in this interval: a combination of being too low or too high.
\item
The probability of being too low for this interval is $P(Z \leq -1.80) = 0.0359$ (check)
\item
The probability of being too high for this interval is $P(Z \geq 1.96) = 0.0250$ (check)
\item
Therefore the probability of being \textbf{outside} the interval is 0.0359 + 0.0250 = 0.0609.
\item
Therefore the probability of being \textbf{inside} the interval is 1- 0.0609 = 0.9391
$P(-1.8 \leq Z \leq 1.96) = 0.9391$
\end{itemize}
}
%-------------------------------------------------------------%
\frame{
\frametitle{Working Backwards}
\begin{itemize}

\item Suppose we wish to find a value (lets call it A) from the normal distribution, such that a certain proportion of values is greater than A (e.g. 10\%)
\item Find A such that $P(X \geq A) = 0.10$. (with $\mu  = 350$ and $\sigma = 17$)
\item In general, our first step is to use the standardization equation to find the corresponding Z-score $z_A$.
\item Because we don't know what value A has, we can't use this approach.
\item However, we can say the following
\[  P(X \geq A) = P(Z \geq z_A) = 0.10 \]

\item From the tables, we can approximate a value for $z_A$, by finding the closest probability value, and determining the corresponding Z-score.
\end{itemize}
}

%------------------------------------------------------------------------%
\frame{

\frametitle{Find $z_A$ such that $ P(Z \geq z_a) = 0.10$}
\begin{itemize}
\item The closest probability value in the tables is $0.1003$.\\
\item The Z-score that corresponds to $0.1003$ is 1.28.\\
\item (Row : 1.2 , Column : 0.08)
\item Therefore $z_A  \approx 1.28$
\end{itemize}
\small
\begin{table}[ht]
%\caption{Standard Normal Distribution } % title of Table
\centering % used for centering table
\begin{tabular}{|c|| c c c c c c|} % centered columns (4 columns)
\hline %inserts double horizontal lines
& \ldots & \ldots & 0.006 &0.07&0.08&0.09 \\
%heading
\hline \hline% inserts single horizontal line
\ldots & \ldots & \ldots &\ldots& \ldots &\ldots&\dots \\ % inserting body of the table
1.0 & \ldots & \ldots &0.1446& 0.1423 &0.1401&0.1379 \\ % inserting body of the table
1.1 & \ldots & \ldots&0.1230& 0.1210 &0.1190&0.1170 \\ % inserting body of the table
1.2 & \ldots & \ldots&0.1038 & 0.1020 &\alert{0.1003}&0.0985\\
1.3 & \ldots & \ldots &0.0869& 0.0853 &0.0838&0.0823 \\ % inserting body of the table
\ldots & \ldots &\ldots&\ldots & \ldots &\ldots&\ldots\\
\hline %inserts single line
\end{tabular}
%\label{table:nonlin} % is used to refer this table in the text
\end{table}
}
%-------------------------------------------------------------%
\frame{
\frametitle{Working Backwards}
\begin{itemize}
\item We can now use the standardization formula.
\item We have only one unknown in the formula: $A$.
\[ 1.28 = {A - 350 \over 17} \]
\item Re-arranging ( multiply both sides by 17):\\
$ 21.76 = A - 350 $
\item Re-arranging ( add 350 to both sides ):\\
$ A = 371.76 $
\item $P(X \geq 371.76) \approx 0.10$
\item (Remark: for sums of die-throws, round it to nearest value)
\end{itemize}
}
%-----------------------------------------------------%
\end{document}
%-------------------------------------------------------------%
\frame{
\frametitle{Working Backwards: Another Example}
\begin{itemize}

\item Find B such that $P(X \geq B) = 0.90$. (with $\mu  = 350$ and $\sigma = 17$)
\item Necessarily $P(X \leq B) = 0.10$
\item Find some value $Z_B$ such that $P(Z \leq z_B) = 0.10$
\item $z_B$ could be negative.
\item Use the symmetry rule $P(Z \leq z_B) = P(Z \geq -z_B)$
\item $-z_B$ could be positive.
\item Based on last example $-z_B = 1.28$. Therefore $z_B = -1.28$
\end{itemize}
}
%-------------------------------------------------------------%
\frame{
\frametitle{Working Backwards}
\begin{itemize}
\item Again ,we can now use the standardization formula
\item We have only one unknown in the formula: $B$.
\[ -1.28 = {B - 350 \over 17} \]
\item Re-arranging ( multiply both sides by 17):\\
$ -21.76 = B - 350 $
\item Re-arranging ( add 350 to both sides ):\\
$ x_o = 350 - 21.76 = 328.24 $
\item $P(X \leq 328.24) \approx 0.10$
\end{itemize}
}
