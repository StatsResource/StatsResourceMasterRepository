



%--------------------------------------------------------------
\begin{frame}
\frametitle{Application : Example }
The mean time spent waiting by customers before their queries are dealt with at an information centre is 10 minutes.\\ \smallskip
The waiting time is normally distributed with a standard deviation of 3 minutes.
\begin{itemize}
\item [i)] What percentage of customers will be waiting longer than 15 minutes

\item [ii)] $90\%$ of customers will be dealt with in at most 12 minutes. Is this statement true or false?
Justify your answer.

\item [iii)] What percentage of customers will wait between 7 and 13 minutes before their query is dealt with?
\end{itemize}
\end{frame}
%---------------------------------------------%
\begin{frame}
\frametitle{Solutions}

Let x be the normal random variable describing waiting times\\
$P(X \geq 15) =?$ \\
\bigskip
     First , we find the z-value that corresponds to x = 15  (remember $\mu=10$ and $\sigma=3$  )\\
\[ z_o = { x_o - \mu \over \sigma }  = { 15 - 10 \over 3 } = 1.666 \]
\begin{itemize}
\item We will use $z_o =1.67$
\item Therefore we can say $P(X \geq 15 ) = P(Z \geq 1.67)$
\item The Murdoch Barnes tables are tabulated to give $P(Z \geq z_o)$ for some value $ z_o$ .
\item We can evaluate $P(Z \geq 1.67)$  as 0.0475.
\item Necessarily $P(X \geq 15) = 0.0475$.
\end{itemize}
\end{frame}

%---------------------------------------------%
\begin{frame}
\frametitle{Solutions}
\begin{itemize}
\item "$90\%$ of customers will be dealt with in at most 12 minutes."
\item To answer this question, we need to know  $P(X\leq 12)$
\item First , we find the z-value that corresponds to x = 12  (remember $\mu=10$ and $\sigma=3$ )
\end{itemize}
\[ z_o = { x_o - \mu \over \sigma }  = { 12 - 10 \over 3 } = 0.666 \]

\end{frame}

%---------------------------------------------%
\begin{frame}
\frametitle{Solutions}
\begin{itemize}
\item We will use $z_o =0.67$ (although 0.66 would be fine too) 
\item Therefore we can say $P(X \geq 12 ) = P(Z \geq 0.67)  = 0.2514$
\item Necessarily  $P(X \leq 12 ) = P(Z \leq 0.67) = 0.7486$
\item $74.86\%$ of customers will be dealt with in at most 12 minutes.
\item The statement that $90\%$ will be dealt with in at most 12 minutes is false.
\end{itemize}
\end{frame}
%---------------------------------------------%
\begin{frame}
\frametitle{Solutions}
What percentage will wait between 7 and 13 minutes ?\\

\[P(7 \leq X \leq 13)   = ?\]

\textbf{Solution:}
\begin{itemize}
	\item Compute the probability of being too low, and the probability of being too high for the interval.\item The probability of being inside the interval is the complement of the combination of these events.
\end{itemize}

\end{frame}
%---------------------------------------------%
\begin{frame}
\frametitle{Solutions}
\textbf{Too high:}\\
$P(X \geq 13) = ?$
\[ z_o  = {13 - 10  \over 3} = 1\]
\smallskip
From tables, $P(Z \geq 1) = 0.1587$. Therefore $P(X \geq 13) = 0.1587$\\ \bigskip
\textbf{Too low:}\\
$P(X \leq 7) = ?$
\[ z_o  = {7 - 10  \over 3} = -1\]
By symmetry, and using tables, $P(X \leq 7) = P(Z \leq -1)= 0.1587$\\ \bigskip
\end{frame}
%---------------------------------------------%
\begin{frame}
\frametitle{Solutions}

\[P(7 \leq X \leq 13)  = 1 - [ P(X \leq 7)  + P(X \geq 13) ] \]

\[P(7 \leq X \leq 13)  =  1 - [0.1587+0.1587] = 0.6826\]

\end{frame}

%-----------------------------------------------------%
\begin{frame}
\frametitle{Normal Distribution : Solving problems}
Recap:
\begin{itemize}
\item We must know the normal mean $\mu$ and the normal standard deviation $\sigma$.
\item The normal random variable is $X \sim \mbox{N} ( \mu , \sigma^2)$.\smallskip
\item (If we don't, we usually have to determine them, given the information in the question.)\smallskip
\item The standard normal random variable is $Z\sim \mbox{N} ( 0 , 1^2)$.\smallskip
\item The standard normal distribution is well described in Murdoch Barnes Table 3, which tabulates $P(Z \geq z_o)$ for a range of $Z$ values.
\end{itemize}
\end{frame}
%-----------------------------------------------------%
\begin{frame}
\frametitle{Normal Distribution : Solving problems}
\begin{itemize}
\item For the given value $x_o$ from the variable $X$, we compute the corresponding z-score $z_o$.
\[ z_o = { x_o - \mu \over \sigma} \]
\item When $z_o$ corresponds to $x_o$, the following identity applies:
\[  P(X \geq x_o )= P(Z \geq z_o ) \]
\item Alternatively $ P(X \leq x_o )= P(Z \leq z_o ) $
\end{itemize}
\end{frame}
%-----------------------------------------------------%
\begin{frame}
\frametitle{Normal Distribution : Solving problems}
\begin{itemize}
\item \textbf{Complement Rule}: \[ P(Z \leq k) = 1-P(Z \geq k) \] for some value $k$
\item Alternatively $ P(Z \geq k) = 1-P(Z \leq k) $
\item \textbf{Symmetry Rule}: \[ P(Z \leq -k) = P(Z \geq k) \] for some value $k$
\item Alternatively $ P(Z \geq -k) = P(Z \leq k) $
\end{itemize}
\end{frame}
%-----------------------------------------------------%
\begin{frame}
\frametitle{Normal Distribution : Solving problems}
\begin{itemize}

\item \textbf{Intervals}: \[ P(L \leq Z \leq U) = 1- [ P(Z \leq L) + P(Z \geq U)] \]
where $L$ and $U$ are the lower and upper bounds of an interval.
\item Probability of having a value too low for the interval : $P(Z \leq L)$
\item Probability of having a value too high for the interval : $P(Z \geq U)$
\end{itemize}
\end{frame}
%-----------------------------------------------------%

%------------------------------------------------------------%
\begin{frame}

\frametitle{Normal Distribution: Simulation Study}
\begin{itemize}
\item
Recall the experiment whereby a die was rolled 100 times, and the sum of the 100 values was recorded.\smallskip
\item
This experiment was repeated a very large number of times (e.g. 100,000 times ) in a simulation study.\smallskip
\item
A histogram was drawn to depict the distribution of outcomes of this experiment.


\end{itemize}
\end{frame}


\frame{
\frametitle{Normal Distribution: Simulation Study}

\begin{center}
\includegraphics[scale=0.30]{images/3aDieHist3}
\end{center}

}

\frame{
\frametitle{Normal Distribution: Simulation Study}
Recall some observations made about the results of the simulation study, made in a previous lecture.
\begin{itemize}
% \item Approximately 76\% of the values are between 330 and 370.
\item Approximately 68.7\% of the values in the simulation study are between 332 and 367.
\item Approximately 95\% of the values are between 316 and 383.
\item $2.5\%$ of the values output are less than 316.
\item $2.5\%$ of the values study output are greater than 383.
\item 175 values are greater than or equal to 400, whereas 198 values are less than or equal to 300.
\item Results such as these are unusual, but they are not impossible.
\end{itemize}
}
%---------------------------------------------------------------%
\frame{
\frametitle{Normal Distribution: Simulation Study}
\begin{itemize}
\item Suppose we can \textbf{\emph{approximate}} the summation of the die-throws using the normal distribution. \smallskip
\item The normal mean is necessarily $\mu = 350$. \smallskip
\item The normal standard deviation is approximately 17. (68\% of values between $350 \pm 17$).\smallskip
\item Using the normal distribution, lets estimate the proportion of values greater than 383.
\end{itemize}
}

%---------------------------------------------------------------%
\frame{
\frametitle{Normal Distribution: Simulation Study}

\begin{center}
\includegraphics[scale=0.40]{images/5BNormalA}
\end{center}
}

%-------------------------------------------------------------%
\frame{
\frametitle{Normal Distribution: Simulation Study}
\begin{itemize}
\item X is the normal random variable that approximates the sum of values from 100 throws of a die.\smallskip
\item Find $P(X \leq 383)$\smallskip
\item First use the standardization formula to find the Z-score.
\[ z_o = {383 - 350 \over 17} = {33 \over 17} = 1.94 \]
\item Use the tables to compute $P(Z \geq 1.94)$ (\alert{Answer: 0.0262} )\smallskip
\item Because $P(Z \geq 1.94)  = 0.0262$, we can say $P(X \geq 383)  = 0.0262$
\item This is close to the proportion of observed values, which was 2.5\%.\smallskip
\item Remark : The standard deviation of 17 was an estimate. The actual standard deviation should 17.12.
\end{itemize}
}

\frame{
\frametitle{Normal Distribution: Simulation Study}

\begin{center}
\includegraphics[scale=0.40]{images/5BNormalB}
\end{center}

}
%-------------------------------------------------------------%
\frame{
\frametitle{Working Backwards}
\begin{itemize}

\item Suppose we wish to find a value (lets call it A) from the normal distribution, such that a certain proportion of values is greater than A (e.g. 10\%)
\item Find A such that $P(X \geq A) = 0.10$. (with $\mu  = 350$ and $\sigma = 17$)
\item In general, our first step is to use the standardization equation to find the corresponding Z-score $z_A$.
\item Because we don't know what value A has, we can't use this approach.
\item However, we can say the following
\[  P(X \geq A) = P(Z \geq z_A) = 0.10 \]

\item From the tables, we can approximate a value for $z_A$, by finding the closest probability value, and determining the corresponding Z-score.
\end{itemize}
}

%------------------------------------------------------------------------%
\frame{

\frametitle{Find $z_A$ such that $ P(Z \geq z_a) = 0.10$}
\begin{itemize}
\item The closest probability value in the tables is $0.1003$.\\
\item The Z-score that corresponds to $0.1003$ is 1.28.\\
\item (Row : 1.2 , Column : 0.08)
\item Therefore $z_A  \approx 1.28$
\end{itemize}
\small
\begin{table}[ht]
%\caption{Standard Normal Distribution } % title of Table
\centering % used for centering table
\begin{tabular}{|c|| c c c c c c|} % centered columns (4 columns)
\hline %inserts double horizontal lines
& \ldots & \ldots & 0.006 &0.07&0.08&0.09 \\
%heading
\hline \hline% inserts single horizontal line
\ldots & \ldots & \ldots &\ldots& \ldots &\ldots&\dots \\ % inserting body of the table
1.0 & \ldots & \ldots &0.1446& 0.1423 &0.1401&0.1379 \\ % inserting body of the table
1.1 & \ldots & \ldots&0.1230& 0.1210 &0.1190&0.1170 \\ % inserting body of the table
1.2 & \ldots & \ldots&0.1038 & 0.1020 &\alert{0.1003}&0.0985\\
1.3 & \ldots & \ldots &0.0869& 0.0853 &0.0838&0.0823 \\ % inserting body of the table
\ldots & \ldots &\ldots&\ldots & \ldots &\ldots&\ldots\\
\hline %inserts single line
\end{tabular}
%\label{table:nonlin} % is used to refer this table in the text
\end{table}
}
%-------------------------------------------------------------%
\frame{
\frametitle{Working Backwards}
\begin{itemize}
\item We can now use the standardization formula.
\item We have only one unknown in the formula: $A$.
\[ 1.28 = {A - 350 \over 17} \]
\item Re-arranging ( multiply both sides by 17):\\
$ 21.76 = A - 350 $
\item Re-arranging ( add 350 to both sides ):\\
$ A = 371.76 $
\item $P(X \geq 371.76) \approx 0.10$
\item (Remark: for sums of die-throws, round it to nearest value)
\end{itemize}
}
%-----------------------------------------------------%
\end{document}
%-------------------------------------------------------------%
\frame{
\frametitle{Working Backwards: Another Example}
\begin{itemize}

\item Find B such that $P(X \geq B) = 0.90$. (with $\mu  = 350$ and $\sigma = 17$)
\item Necessarily $P(X \leq B) = 0.10$
\item Find some value $Z_B$ such that $P(Z \leq z_B) = 0.10$
\item $z_B$ could be negative.
\item Use the symmetry rule $P(Z \leq z_B) = P(Z \geq -z_B)$
\item $-z_B$ could be positive.
\item Based on last example $-z_B = 1.28$. Therefore $z_B = -1.28$
\end{itemize}
}
%-------------------------------------------------------------%
\frame{
\frametitle{Working Backwards}
\begin{itemize}
\item Again ,we can now use the standardization formula
\item We have only one unknown in the formula: $B$.
\[ -1.28 = {B - 350 \over 17} \]
\item Re-arranging ( multiply both sides by 17):\\
$ -21.76 = B - 350 $
\item Re-arranging ( add 350 to both sides ):\\
$ x_o = 350 - 21.76 = 328.24 $
\item $P(X \leq 328.24) \approx 0.10$
\end{itemize}
}
