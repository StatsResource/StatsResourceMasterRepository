

\documentclass[a4paper12pt]{article}
%%%%%%%%%%%%%%%%%%%%%%%%%%%%%%%%%%%%%%%%%%%%%%%%%%%%%%%%%%%%%%%%%%%%%%%%%%%%%%%%%%%%%%%%%%%%%%%%%%%%%%%%%%%%%%%%%%%%%%%%%%%%%%%%%%%%%%%%%%%%%%%%%%%%%%%%%%%%%%%%%%%%%%%%%%%%%%%%%%%%%%%%%%%%%%%%%%%%%%%%%%%%%%%%%%%%%%%%%%%%%%%%%%%%%%%%%%%%%%%%%%%%%%%%%%%%
\usepackage{eurosym}
\usepackage{vmargin}
\usepackage{amsmath}
\usepackage{graphics}
\usepackage{epsfig}
\usepackage{enumerate}
\usepackage{multicol}
\usepackage{subfigure}
\usepackage{fancyhdr}
\usepackage{listings}
\usepackage{framed}
\usepackage{graphicx}
\usepackage{amsmath}
\usepackage{chngpage}
%\usepackage{bigints}

\usepackage{vmargin}
% left top textwidth textheight headheight
% headsep footheight footskip
\setmargins{2.0cm}{2.5cm}{16 cm}{22cm}{0.5cm}{0cm}{1cm}{1cm}
\renewcommand{\baselinestretch}{1.3}

\setcounter{MaxMatrixCols}{10}

\begin{document}
\large 

%%- http://webpages.iust.ac.ir/matashbar/teaching/schaum_probability.pdf

%%-- PAge 287
%%-- 9.5


9.5. Derive Eqs. (9.1 6) to (9.1 8). 
Since 1, = A, by Eqs. (9.2) and (9.15), we get 

which is Eq. (9.1 6).

\begin{framed}
\begin{itemize}
\item The average number of customers is denoted $L$
\item The average amount of time that a customer spends waiting for service is denoted $W_q$
    \item  The ratio ${\rho = \frac{\lambda}{\mu}}$ is the traffic intensity of the system.
\end{itemize}
\end{framed}
\section*{Solution}
\begin{itemize}
    \item 
\end{itemize}
\[W =  \frac{L}{\lambda} = \frac{1}{\mu - \lambda } = \frac{1}{\mu(1-\rho)}\]
\begin{itemize}
    \item Next, by definition, 
\[W_q = W - W_S,\] 
where $\displaystyle {W_S = 1/\mu}$, that is, the average service time. 
\end{itemize}
\[{ \displaystyle W_q =   \frac{1}{\mu - \lambda } - \frac{1}{\mu }   = \frac{\lambda}{\mu(\mu - \lambda) } = \frac{\rho}{\mu(1- \rho) }
}\]

Thus, 
\[{ \displaystyle L_q =  \lambda W_q =   \frac{\lambda^2}{\mu(\mu - \lambda) } =\frac{\rho^2}{1- \rho }
}\]


which is Eq. (9.1 7). Finally, by Eq. (9.31, 
which is Eq. (9.1 8). 


\end{document}