

\documentclass[a4paper12pt]{article}
%%%%%%%%%%%%%%%%%%%%%%%%%%%%%%%%%%%%%%%%%%%%%%%%%%%%%%%%%%%%%%%%%%%%%%%%%%%%%%%%%%%%%%%%%%%%%%%%%%%%%%%%%%%%%%%%%%%%%%%%%%%%%%%%%%%%%%%%%%%%%%%%%%%%%%%%%%%%%%%%%%%%%%%%%%%%%%%%%%%%%%%%%%%%%%%%%%%%%%%%%%%%%%%%%%%%%%%%%%%%%%%%%%%%%%%%%%%%%%%%%%%%%%%%%%%%
\usepackage{eurosym}
\usepackage{vmargin}
\usepackage{amsmath}
\usepackage{graphics}
\usepackage{epsfig}
\usepackage{enumerate}
\usepackage{multicol}
\usepackage{subfigure}
\usepackage{fancyhdr}
\usepackage{listings}
\usepackage{framed}
\usepackage{graphicx}
\usepackage{amsmath}
\usepackage{chngpage}
%\usepackage{bigints}

\usepackage{vmargin}
% left top textwidth textheight headheight
% headsep footheight footskip
\setmargins{2.0cm}{2.5cm}{16 cm}{22cm}{0.5cm}{0cm}{1cm}{1cm}
\renewcommand{\baselinestretch}{1.3}

\setcounter{MaxMatrixCols}{10}

\begin{document}
\large 


%%- http://webpages.iust.ac.ir/matashbar/teaching/schaum_probability.pdf

%%-- PAge 287
%%-- 9.4



Derive Eq. (9.15). 

%%%%%%%%%%%%%%%%%%%%%%%%%%%%%%%%%%%%%
 The ratio ${\rho = \frac{\lambda}{\mu}}$ is sometimes referred to as the trafJic intensity of the 

system. The traffic intensity of the system is defined as 
mean service time mean arrival rate Traffic intensity = - mean interarrival time mean service rate 
The average number of customers in the system is given by (Prob. 9.4) 
The

\newpage 

%%%%%%%%%%%%%%%%%%%%%%%%%%%%%%%%%%%%%%
Since $p_n$ is the steady-state probability that the system contains exactly $n$ customers, using Eq. (9.14), 
the average number of customers in the M/M/1 queueing system is given by \[ L = \sum^{\infty}_{n=0}p_{n} =\sum^{\infty}_{n=0} n(1-\rho) \rho^{n} = (1-\rho) p_{0}\left(\sum^{\infty}_{n=0}np^{n}\right)  \]

where ${\rho = \frac{\lambda}{\mu} < 1}$ . Using the algebraic identity 
\[\sum^{\infty}_{n=0}nX^{n} = \frac{x}{(1-x)^3} \mbox{ where } \mid x \mid < 1 \]

we obtain 

\[L =  \frac{\rho}{(1-\rho)} = \frac{\frac{\lambda}{\mu}}{1- \frac{\lambda}{\mu} } = \frac{\lambda}{\mu-\lambda}\]
\end{document}


It seems like you have a LaTeX document here to derive Equation (9.15) and explain the traffic intensity of a queueing system. Your document looks well-structured with various LaTeX packages for formatting, math, and graphics. Here’s a quick breakdown of the derivation you’ve started:

1. **Traffic Intensity (\(\rho\))**: 
   \[
   \rho = \frac{\lambda}{\mu}
   \]
   where \(\lambda\) is the mean arrival rate, and \(\mu\) is the mean service rate.

2. **Average Number of Customers (L)**: 
   \[
   L = \sum^{\infty}_{n=0}p_{n} =\sum^{\infty}_{n=0} n(1-\rho) \rho^{n}
   \]
   using the fact that \(p_n\) is the steady-state probability.

3. **Using the Series Sum Identity**: 
   \[
   \sum^{\infty}_{n=0} nX^{n} = \frac{x}{(1-x)^2} \quad \text{for} \quad |x| < 1
   \]

4. **Final Derivation**:
   \[
   L =  \frac{\rho}{(1-\rho)} = \frac{\frac{\lambda}{\mu}}{1- \frac{\lambda}{\mu} } = \frac{\lambda}{\mu-\lambda}
   \]

You are using the relationship between \(\rho\), \(\lambda\), and \(\mu\) to express the average number of customers in the system.

If you need further assistance with any specific part or more detailed explanations, feel free to let me know. I’m here to help!