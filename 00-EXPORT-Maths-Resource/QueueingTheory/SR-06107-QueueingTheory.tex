
%%-- https://www.win.tue.nl/~resing/SOR/eng/college10_09_eng.pdf

%%%%%%%%%%%%%%%%%%%%%%%%%%%%%%%%%%%%%%%


\subsection*{The M/M/s queue}
\begin{itemize}
\item  Customers arrive according to a Poisson process with rate $\lambda$ .
\item  The service times of customers are exponentially distributed with parameter $\mu$ .
\item  There are $s$ servers, serving customers in order of arrival.
\end{itemize}

\noindent \textbf{Stability condition:}
\[\lambda  < s\cdot \mu\]  or alternatively written, 
\[\rho = \lambda s\cdot \mu < 1.\]

The process $\{X(t), t \geq  0\}$, the number of customers in the system at time
$t$, is again a continuous-time Markov chain with infinite state space.

%%%%%%%%%%%%%%%%%%%%%%%%%%%%%%%%%
\subsection*{Performance measures in the M/M/s queue:}
\begin{itemize}
\item probability that a customer has to wait
\begin{eqnarray*}
\Pi_{W} 
&=&\sum^{\infty}_{k=0} p_{s+k} \\
&=&\sum^{\infty}_{k=0} \left(\lambda s\mu \right)^kp_{s}\\
&=& \frac{p_s}{1-\rho}\\
\end{eqnarray*}
\item expected number of busy servers
\begin{eqnarray*}
B &=&\sum^{\infty}_{k=0} min(i, s) pi \\
&=&\sum^{\infty}_{k=0} \frac{\lambda}{\mu}  p_{i - 1} \\
&=&\frac{\lambda}{\mu}\\
\end{eqnarray*}
\item expected number of waiting customers
\begin{eqnarray*}
L_{q} &=& \sum^{\infty}_{k=0}k p_{s+k} \\
&=& \sum^{\infty}_{k=0} k\left(\lambda s\mu \right)^kp_{s}\\
&=& pp_{s}\frac{\rho}{(1  -  \rho )^2}\\
\end{eqnarray*}
${ \displaystyle W_{q} = \frac{L_{q}}{\lambda}}$ ,\qquad ${ \displaystyle L = L_{q} + B}$,\qquad ${ \displaystyle W = \frac{L}{\lambda } = W_{q} + \frac{1}{\mu} }$

%%%%%%%%%%%%%%%%%%%%%%%%%%%%%%%%%
\newpage 

The M/M/s/K queue
\begin{itemize}
\item  Customers arrive according to a Poisson process with rate $\lambda$ .
\item  The service times of customers are exponentially distributed with parameter $\mu$ .
\item  There are s servers, serving customers in order of arrival.
\item  Customers who see at arrival K ($K \geq  s$) other customers in the system
are lost.
\end{itemize}
The process ${X(t), t \geq  0}$, the number of customers in the system at time
$t$, is again a continuous-time Markov chain with state space ${0, 1, . . . , K}$.

The M/M/1/K Queueing System

%%- http://webpages.iust.ac.ir/matashbar/teaching/schaum_probability.pdf

%%-- PAge 287
%%-- 9.14

\[ \sum^{K}_{n=0} x^{n} = \frac{1-x^{K+1}}{1-x} \] 


\begin{framed}
\begin{itemize}
\item The average number of customers is denoted $L$
\item The average amount of time that a customer spends waiting for service is denoted $W_q$
    \item  The ratio ${\rho = \frac{\lambda}{\mu}}$ is the traffic intensity of the system.
\end{itemize}
\end{framed}

\[\sum^{K}_{n=0}P_{n} = 
\sum^{K}_{n=0}\rho_{n} =1 \]
%%%%%%%%%%%%%%%%%%%%%%%%%%%%%%%%
\section*{Solution}
\subsection*{Part (a)}
\begin{itemize}
    \item The watch repair shop service can be modeled as an M/M/1 queueing system with 1 = &, p = 4. 
    \item Thus, 
from Eqs. (9.1 5), (9.1 6), and (9.43), we have 
1 1 w=-=-- - 40 minutes 
-A Q-iij 
\[W, = W - W, = 40 - 8 = 32 minutes\]
\end{itemize}

%%%%%%%%%%%%%%%%%%%%%%%%%%%%%%%%
\subsection*{Part (b)}
(b) Now 1 = 4, p = g. Then 
1 w=- - 1-= 72 minutes 
p-a +-g 
W, = W - W, = 72 - 8 = 64 minutes 
\begin{itemize}
    \item It can be seen that an increase of 10 percent in the customer arrival rate doubles the average number 
of customers in the system.
\item The average time a customer spends in queue is also doubled. 
\end{itemize}

\end{document}