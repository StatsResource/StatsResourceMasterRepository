

\documentclass[a4paper12pt]{article}
%%%%%%%%%%%%%%%%%%%%%%%%%%%%%%%%%%%%%%%%%%%%%%%%%%%%%%%%%%%%%%%%%%%%%%%%%%%%%%%%%%%%%%%%%%%%%%%%%%%%%%%%%%%%%%%%%%%%%%%%%%%%%%%%%%%%%%%%%%%%%%%%%%%%%%%%%%%%%%%%%%%%%%%%%%%%%%%%%%%%%%%%%%%%%%%%%%%%%%%%%%%%%%%%%%%%%%%%%%%%%%%%%%%%%%%%%%%%%%%%%%%%%%%%%%%%
\usepackage{eurosym}
\usepackage{vmargin}
\usepackage{amsmath}
\usepackage{graphics}
\usepackage{epsfig}
\usepackage{enumerate}
\usepackage{multicol}
\usepackage{subfigure}
\usepackage{fancyhdr}
\usepackage{listings}
\usepackage{framed}
\usepackage{graphicx}
\usepackage{amsmath}
\usepackage{chngpage}
%\usepackage{bigints}

\usepackage{vmargin}
% left top textwidth textheight headheight
% headsep footheight footskip
\setmargins{2.0cm}{2.5cm}{16 cm}{22cm}{0.5cm}{0cm}{1cm}{1cm}
\renewcommand{\baselinestretch}{1.3}

\setcounter{MaxMatrixCols}{10}

\begin{document}
\large 


%%- http://webpages.iust.ac.ir/matashbar/teaching/schaum_probability.pdf

%%-- PAge 287
%%-- 9.7

Customers arrive at a watch repair shop according to a Poisson process at a rate of one per 
every 10 minutes, and the service time is an exponential r.v. with mean 8 minutes. 
\begin{enumerate}[(a)]
    \item Find the average number of customers $L$, the average time a customer spends in the shop 
$W_q$ and the average time a customer spends in waiting for service $W$,. 
\item  Suppose that the arrival rate of the customers increases 10 percent. Find the corresponding 
changes in $L$, $W$, and $W_q$. 
\end{enumerate}



\begin{framed}
\begin{itemize}
\item The average number of customers is denoted $L$
\item The average amount of time that a customer spends waiting for service is denoted $W_q$
    \item  The ratio ${\rho = \frac{\lambda}{\mu}}$ is the traffic intensity of the system.
\end{itemize}
\end{framed}
%%%%%%%%%%%%%%%%%%%%%%%%%%%%%%%%
\section*{Solution}
\subsection*{Part (a)}
\begin{itemize}
\item 
(a) The watch repair shop service can be modeled as an M/M/1 queueing system with 1 = &, p = 4. 
\item Thus, 
from Eqs. (9.1 5), (9.1 6), and (9.43), we have 
1 1 w=-=-- - 40 minutes 
-A Q-iij 
W, = W - W, = 40 - 8 = 32 minutes 
\end{itemize}


%%%%%%%%%%%%%%%%%%%%%%%%%%%%%%%%%%%%%%%


Let's break down the problem step-by-step:

1. **Arrival Rate (\(\lambda\))**:
   - Customers arrive at a rate of one every 10 minutes.
   - Convert this to a per-minute rate: \(\lambda = \frac{1}{10} \text{ customers per minute} = 0.1 \text{ customers per minute}\).

2. **Service Rate (\(\mu\))**:
   - The mean service time is 8 minutes.
   - Convert this to a per-minute rate: \(\mu = \frac{1}{8} \text{ services per minute} = 0.125 \text{ services per minute}\).

3. **Utilization Factor (\(\rho\))**:
   $$\rho = \frac{\lambda}{\mu} = \frac{0.1}{0.125} = 0.8$$

4. **Average Number of Customers in the System (L)**:
   $$L = \frac{\rho}{1 - \rho} = \frac{0.8}{1 - 0.8} = \frac{0.8}{0.2} = 4 \text{ customers}$$

5. **Average Time a Customer Spends in the System (W)**:
   $$W = \frac{1}{\mu - \lambda} = \frac{1}{0.125 - 0.1} = \frac{1}{0.025} = 40 \text{ minutes}$$

6. **Average Time a Customer Spends Waiting in the Queue (W_q)**:
   $$W_q = \frac{\rho}{\mu - \lambda} = \frac{0.8}{0.125 - 0.1} = \frac{0.8}{0.025} = 32 \text{ minutes}$$

### Summary:
- **Average Number of Customers in the System (L)**: 4 customers
- **Average Time a Customer Spends in the System (W)**: 40 minutes
- **Average Time a Customer Spends Waiting in the Queue (W_q)**: 32 minutes

These calculations help in understanding the performance and efficiency of the queueing system²³.

%%%%%%%%%%%%%%%%%%%%%%%%%%%%%%%%%%%%%%%%%%%%%%%%%%%%%%%%%%%%%%%%%%%%%%%%%%%%%%%%%%%%%%%%%%%%%%%%%%%%%%%%%%%%%%%%%%%%

\medskip 
\subsection*{Part (b)}
(b) Now 1 = 4, p = g. Then 
1 w=- - 1-= 72 minutes 
p-a +-g 
W, = W - W, = 72 - 8 = 64 minutes 

\begin{itemize}
\item It can be seen that an increase of 10 percent in the customer arrival rate doubles the average number 
of customers in the system. 
\item The average time a customer spends in queue is also doubled.
\end{itemize}

Let's calculate the changes in \(L\), \(W\), and \(W_q\) when the arrival rate increases by 10%.

### Original Parameters:
- Arrival rate (\(\lambda\)): 0.1 customers per minute
- Service rate (\(\mu\)): 0.125 services per minute
- Utilization factor (\(\rho\)): 0.8

### New Arrival Rate:
- New \(\lambda\): \(0.1 \times 1.1 = 0.11\) customers per minute

### New Utilization Factor:
- New \(\rho\): \(\frac{0.11}{0.125} = 0.88\)

### New Average Number of Customers in the System (\(L\)):
$$L = \frac{\rho}{1 - \rho} = \frac{0.88}{1 - 0.88} = \frac{0.88}{0.12} \approx 7.33 \text{ customers}$$

### New Average Time a Customer Spends in the System (\(W\)):
$$W = \frac{1}{\mu - \lambda} = \frac{1}{0.125 - 0.11} = \frac{1}{0.015} \approx 66.67 \text{ minutes}$$

### New Average Time a Customer Spends Waiting in the Queue (\(W_q\)):
$$W_q = \frac{\rho}{\mu - \lambda} = \frac{0.88}{0.125 - 0.11} = \frac{0.88}{0.015} \approx 58.67 \text{ minutes}$$

### Summary of Changes:
- **Average Number of Customers in the System (\(L\))**: Increased from 4 to approximately 7.33 customers
- **Average Time a Customer Spends in the System (\(W\))**: Increased from 40 minutes to approximately 66.67 minutes
- **Average Time a Customer Spends Waiting in the Queue (\(W_q\))**: Increased from 32 minutes to approximately 58.67 minutes

These calculations show how a 10% increase in the arrival rate significantly impacts the system's performance³.

Would you like to explore more scenarios or delve into another aspect of queueing theory?

Source : conversation avec Copilot, 4/10/2024
(1) Chapter 8 Queueing Models - University of Chicago. https://galton.uchicago.edu/~yibi/teaching/stat317/2021/Lectures/Lecture19.pdf.
(2) What is Little’s Law? Overview with formula and examples. https://blog.logrocket.com/product-management/littles-law-overview-formula-examples/.
(3) Little's Law - Abstract Algorithms. https://abstractalgorithms.dev/posts/little-s-law/.
\end{document}
