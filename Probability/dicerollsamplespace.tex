\documentclass[a4paper,12pt]{article}

\usepackage{amsmath}
\usepackage{graphicx}
\usepackage{amssymb}
\usepackage{framed}
%\usepackage{multicol}
%\usepackage[paperwidth=21cm, paperheight=29.8cm]{geometry}
%\usepackage[angle=0,scale=1,color=black,hshift=-0.4cm,vshift=15cm]{background}
%\usepackage{multirow}
\usepackage{enumerate}

\usepackage{amsfonts,amssymb}
\usepackage{color}
\usepackage{multirow}
\usepackage{eurosym}
\usepackage{framed}
\usepackage{fancyhdr}
\usepackage{listings}
\usepackage{eurosym}
\usepackage{vmargin}
%\usepackage{amsmath}
\usepackage{fancyhdr}
\usepackage{listings}
\usepackage{multicol}
\usepackage{framed}
\usepackage{graphics}
\usepackage{epsfig}
\usepackage{subfigure}
\usepackage{fancyhdr}

%\input def.tex
%\input dsdef.tex
%\input rgb.tex

%\newcommand \la{\lambda}
%\newcommand \al{a}
%\newcommand \be{b}
\newcommand \x{\overline{x}}
\newcommand \y{\overline{y}}

\pagestyle{fancy}
\setmarginsrb{20mm}{0mm}{20mm}{25mm}{12mm}{11mm}{0mm}{11mm}
\lhead{Statistics and Probability} \rhead{Introduction to Probability}
\chead{Tutorial Sheet}
%\input{tcilatex}

\begin{document}
\LARGE
\noindent \textbf{Probability}


\begin{center}
	\begin{tabular}{|c||c|c|c|c|c|c|}
		\hline
		\phantom{space}	& \phantom{sp} \textbf{1}\phantom{sp}	&\phantom{sp} \textbf{2}\phantom{sp}&\phantom{sp} \textbf{3}\phantom{sp}	&\phantom{sp} \textbf{4}	\phantom{sp}&\phantom{sp} \textbf{5} \phantom{sp}&\phantom{sp}\textbf{6}	\phantom{sp}\\ \hline	\hline
		\textbf{1}	&	2	&	3	&	4	&	5	&	6	&	7	 \\ \hline	
		\textbf{2}	&	3	&	4	&	5	&	6	&	7	&	8	 \\ \hline	
		\textbf{3}	&	4	&	5	&	6	&	7	&	8	&	9	 \\ \hline	
		\textbf{4}	&	5	&	6	&	7	&	8	&	9	&	10	 \\ \hline	
		\textbf{5}	&	6	&	7	&	8	&	9	&	10	&	11	 \\ \hline	
		\textbf{6}	&	7	&	8	&	9	&	10	&	11	&	12	 \\ \hline	
	\end{tabular}
\end{center}


%----------------------------------------%
%----------------------------------------%
\end{document}
