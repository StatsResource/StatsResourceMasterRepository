\item
A driver passes through 3 traffic lights. The chance he/she will stop at the first is 1/2 , at the second 1/3 and at the third ¼ independently of what happens at any of the other lights. What is the probability that

\begin{enumerate}
\item    the driver makes the whole journey without being stopped at any of the lights

\item   the driver is only stopped at the first and third lights

\item  the driver is stopped at just one set of lights.
\end{enumerate}

\begin{framed}
\begin{multicols}{3}
\begin{itemize}
\item $P[F] = 0.5 $  
\item $P[F^c] = 0.5 $           
\item $P[S] = 0.333 $       
\item $P[S^c] = 0.666$
\item $P[T] = 0.25  $      
\item $P[T^c] = 0.75$
\end{itemize}
\end{multicols}


\begin{itemize}
\item Probability of not getting stopped at all three lights


\[P[0] =P[Fc]P[Sc]P[Tc] = 0.5 \times 0.666 \times 0.75 = 0.25\]


\item Probability of only getting stopped at first  lights


\[P[F only] = P[F]P[Sc]P[Tc] = 0.5\times 0.666\times 0.75 = 0.25\]


\item Probability of only getting stopped at second lights


\[P[S only] =P[Fc]P[S]P[T^c] = 0.5\times 0.333\times 0.75 = 0.125\]
\item Probability of only getting stopped at third  lights

\[P[T only] =P[F^c]P[S^c]P[T] = 0.5\times 0.666\times 0.25 = 0.083\]


\item Probability of getting stopped at one lights only 


\[P[1 only] =P[F only]+P[S only]+ P[T only]\]


\[P[1 only] = 0.125 + 0.25 + 0.083 = 0.458\]

\end{itemize}
\end{framed}

%%%%%%%%%%%%%%%%%%%%%%%%%%%%%%%%%%

\item One in 10 000 people suffer from a particular disease. Given a person has the disease, a test for the disease is always positive (indicates that the person has the disease). Given a person does not have the disease, a test for the disease is positive with probability 0.01.
\begin{enumerate}[(i)]
\item calculate the probability that when a randomly chosen person is tested, the result is positive. 
\item calculate the probability that an individual has the disease, given that the test result was positive.
\end{enumerate}

\item \textbf{Example 1:} What is the probability of rolling two consecutive fives on a six-sided die?
\begin{itemize}
\item You know that the probability of rolling one five is 1/6, and the probability of rolling another five with the same die is also 1/6.
\item These are independent events, because what you roll the first time does not affect what happens the second time; you can roll a 3, and then roll a 3 again.
\end{itemize}

\item \textbf{Example 2:} Two cards are drawn randomly from a deck of cards. What is the likelihood that both cards are clubs?
\begin{itemize}
\item The likelihood that the first card is a club is 13/52, or 1/4. (There are 13 clubs in every deck of cards.) Now, the likelihood that the second card is a club is 12/51.
\item You are measuring the probability of dependent events. This is because what you do the first time affects the second; if you draw a 3 of clubs and don't put it back, there will be one less fewer club and one less card in the deck (51 instead of 52).
\end{itemize}
\item A jar contains 4 blue marbles, 5 red marbles and 11 white marbles. If three marbles are drawn from the jar at random, what is the probability that the first marble is red, the second marble is blue, and the third is white?
\begin{itemize}
\item The probability that the first marble is red is 5/20, or 1/4. 
\item The probability of the second marble being blue is 4/19, since we have one fewer marble, but not one fewer blue marble. 
\item And the probability that the third marble is white is 11/18, because we've already chosen two marbles. This is another measure of a dependent event.
\end{itemize}
