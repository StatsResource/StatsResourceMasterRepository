
\subsection*{Question 4}

You flip a coin 10 times - let $X =$ ``the number of heads''. Using the binomial probability function, calculate the following:\\[-0.2cm]

\item  $\Pr(X = 2)$. 
 \item  $\Pr(X = 0)$. 
 \item   $\Pr(X > 2)$. 
 \item  $\Pr(X \le 3)$. 
 {\bf(e)} $\Pr(5 \le X \le 7)$.  
 {\bf(f)} $E(X)$ and $Sd(X)$. 
 {\bf(g)} Using the binomial tables, calculate $\Pr(X \le10)$ in the case where the coin is flipped 20 times. 
 {\bf(h)} If the coin is flipped 50 times, what is $E(X)$?

\subsection*{Question 5}

Repeat Question 4 (a) - (e) but now using the binomial tables.




\subsection*{Question 7}
We follow on from Question 6 but now consider the case where, to reduce the probability of error, each bit is sent \emph{three} times and then a ``majority vote'' approach is used to determine the value of each received bit. The following example explains the situation:\\[-0.5cm]
\begin{center}
\begin{tabular}{ccccc}
\hline
&&&&\\[-0.3cm]
\multirow{2}{*}{Sent} & $0$ & $1$ & $1$ & $0$ \\
& $\overbrace{000}$ & $\overbrace{111}$ & $\overbrace{111}$ & $\overbrace{000}$ \\[0.2cm]
\hline
&&&&\\[-0.3cm]
\multirow{2}{*}{Received} & $\underbrace{001}$ & $\underbrace{111}$ & $\underbrace{010}$ & $\underbrace{000}$ \\
& $0$ & $1$ & $0$ & $0$ \\[0.2cm]
\hline
%\multicolumn{5}{c}{}
\end{tabular}
\end{center}
$\Rightarrow$ there is one error in decoding the first $000$, but since the majority result is taken, this bit is correctly identified as a $0$. There are two errors in decoding the second $111$, so this bit is misread as a $0$. It is clear that a character is misread if the decoder makes \emph{two or three errors} in these blocks of three replicates.\\[-0.2cm]

\item  Show that sending each bit 3 times reduces the error probability from 10\% to 2.8\%. 
\\ \item  Using this reduced value, $p=0.028$, calculate the probability that there are no errors in a 20-bit string. Compare this result to Q6(a). 
 \item  Now assume that each bit is sent 5 times and, again, the majority vote approach is used. Calculate the probability that there are no errors in a 20-bit string in this case. %
 \item  Recalculate the two probabilities from part (c) using the Poisson approximation.
