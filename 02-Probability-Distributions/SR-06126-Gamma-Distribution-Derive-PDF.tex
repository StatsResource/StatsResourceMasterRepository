


%%- https://vrcacademy.com/tutorials/gamma-distribution/



\documentclass[a4paper,12pt]{article}
%%%%%%%%%%%%%%%%%%%%%%%%%%%%%%%%%%%%%%%%%%%%%%%%%%%%%%%%%%%%%%%%%%%%%%%%%%%%%%%%%%%%%%%%%%%%%%%%%%%%%%%%%%%%%%%%%%%%%%%%%%%%%

\usepackage{eurosym}
\usepackage{vmargin}
\usepackage{amsmath}
\usepackage{graphics}
\usepackage{epsfig}
\usepackage{enumerate}
\usepackage{multicol}
\usepackage{subfigure}
\usepackage{fancyhdr}
\usepackage{listings}
\usepackage{framed}
\usepackage{graphicx}
\usepackage{amsmath}
\usepackage{chngpage}
%\usepackage{bigints}

\usepackage{vmargin}
% left top textwidth textheight headheight
% headsep footheight footskip
\setmargins{2.0cm}{2.5cm}{16 cm}{22cm}{0.5cm}{0cm}{1cm}{1cm}
\renewcommand{\baselinestretch}{1.3}

\setcounter{MaxMatrixCols}{10}

\begin{document}
\large
%%%%%%%%%%%%%%%%%%%%%%%%%%%%%%%%%%%%%%%%%%%%%%%%%%%%%%%%%%%%%%%%%%%%%%%%%%%%%%%%%%%%%%%%%%%%%%%%%%%%%%%%%%%%%%%%%%%%%%%%%%%%%
\section*{Gamma Distribution}
Gamma distribution is used to model a continuous random variable which takes positive values. Gamma distribution is widely used in science and engineering to model a skewed distribution.

\subsection*{Gamma Distribution Definition}

\noindent \textbf{Notation} \[X\sim G(\alpha, \beta).\]The parameter $\alpha$ is called the shape parameter and $\beta$ is called the scale parameter of gamma distribution.

\noindent \textbf{Probability Density Function}\\
A continuous random variable $X$ is said to have an gamma distribution with parameters $\alpha$ and $\beta$ if its p.d.f. is given by\[ \begin{align*} f(x)&= \begin{cases} \frac{1}{\beta^\alpha\Gamma(\alpha)}x^{\alpha -1}e^{-x/\beta}, & x > 0;\alpha, \beta > 0; \\ 0, & \mbox{Otherwise.} \end{cases} \end{align*} \]where for $\alpha>0$, $\Gamma(\alpha)=\int_0^\infty x^{\alpha-1}e^{-x}; dx$ is called a gamma function.



%%%<h2 id=graph-of-gamma-distribution>Graph of Gamma Distribution</h2><p>Following is the graph of probability density function (pdf) of gamma distribution with parameter $\alpha=1$ and $\beta=1,2,4$.<p><img src="data:image/svg+xml,%3Csvg xmlns=%22http://www.w3.org/2000/svg%22 width=%22480%22 height=%22384%22%3E%3C/svg%3E" alt ezimgfmt="rs rscb8 src ng ngcb8" class=ezlazyload data-ezsrc=/images/gamma01.png><h2 id=another-form-of-gamma-distribution>

\newpage
%%%%%%%%%%%%%%%%%%%%%%%%%%%%%%%%%%%%%%


Gamma distribution. Let us take two parameters $\alpha > 0$ and $\beta > 0$. 

The Gamma function
$\Gamma(\alpha)$ is defined by
\[\Gamma(\alpha) = \int^{\infty}_{0}x^{\alpha−1}e^{−x}dx.\]


If we divide both sides by $\Gamma(\alpha)$ we get
\[ 1 = \int^{\infty}_{0} \frac{1}{\Gamma(\alpha) }x^{\alpha−1}e^{−x}dx.\]  we made a change of variables $x = \beta y$. 


\begin{eqnarray*}
\int^{\infty}_{0} \frac{1}{\Gamma(\alpha) }x^{\alpha−1}e^{−x}dx &=&  \int^{\infty}_{0} \frac{1}{\Gamma(\alpha) }(\beta y)^{\alpha−1}\;e^{−\beta y}dx \\
\int^{\infty}_{0} \frac{1}{\Gamma(\alpha) }(\beta)^{\alpha−1} (y)^{\alpha−1}e\;^{−\beta y}dx\\
\int^{\infty}_{0} \frac{(beta^{\alpha−1}}{\Gamma(\alpha) } \cdot y^{\alpha−1}\;e^{−\beta y}dx\\
\end{eqnarray*}

%%--https://web.williams.edu/Mathematics/sjmiller/public_html/372Fa15/handouts/GammaFnChapter_Miller.pdf
%%-- https://onlinehw.math.ksu.edu/math340book/chap3/gamma.php###
