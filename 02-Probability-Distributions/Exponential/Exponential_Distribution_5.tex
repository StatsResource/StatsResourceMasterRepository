  \documentclass[a4]{beamer}
\usepackage{amssymb}
\usepackage{graphicx}
\usepackage{subfigure}
\usepackage{newlfont}
\usepackage{amsmath,amsthm,amsfonts}
%\usepackage{beamerthemesplit}
\usepackage{pgf,pgfarrows,pgfnodes,pgfautomata,pgfheaps,pgfshade}
\usepackage{mathptmx} % Font Family
\usepackage{helvet} % Font Family
\usepackage{color}
\mode<presentation> {
\usetheme{Frankfurt} % was Frankfurt
\useinnertheme{rounded}
\useoutertheme{infolines}
\usefonttheme{serif}
%\usecolortheme{wolverine}
% \usecolortheme{rose}
\usefonttheme{structurebold}
}
\setbeamercovered{dynamic}
\title[MA4413]{Statistics for Computing \\ {\normalsize MA4413 Lecture 5A}}
\author[Kevin O'Brien]{Kevin O'Brien \\ {\scriptsize kevin.obrien@ul.ie}}
\date{Autumn 2011}
\institute[Maths \& Stats]{Dept. of Mathematics \& Statistics, \\ University \textit{of} Limerick}
\renewcommand{\arraystretch}{1.5}
%------------------------------------------------------------------------%
\begin{document}
\begin{frame}
\titlepage
\end{frame}
\begin{frame}[fragile]
\frametitle{Exponential Distribution}
\begin{itemize}
\item The Exponential Distribution
\item The Normal Distribution
\item Applied Normal Distribution
\end{itemize}
\end{frame}

\begin{frame}[fragile]
\frametitle{Exponential Distribution}
The Exponential Distribution may be used to answer the following questions:
\begin{itemize}
\item How much time will elapse before an earthquake occurs in a given region?
\item How long do we need to wait before a customer enters our shop?
\item How long will it take before a call center receives the next phone call?
\item How long will a piece of machinery work without breaking down?
\end{itemize}
\end{frame}  

\begin{frame}[fragile]
\frametitle{Exponential Distribution}

\begin{itemize}
\item All these questions concern the time we need to wait before a given event occurs. If this waiting time is unknown, it is often appropriate to think of it as a random variable having an exponential distribution.
\item Roughly speaking, the time $X$ we need to wait before an event occurs has an exponential distribution if the probability that the event occurs during a certain time interval is proportional to the length of that time interval.

\end{itemize}
\end{frame}

%------------------------------------------------------------------------%
\begin{frame}[fragile]
\frametitle{Probability density function}
The probability density function (PDF) of an exponential distribution is

\[
f(x;\lambda) = \begin{cases}
\lambda e^{-\lambda x}, & x \ge 0, \\
0, & x < 0.
\end{cases}\]
The parameter $\lambda$  is called \textbf{\emph{rate}} parameter.
\end{frame}
%------------------------------------------------------------------------%
\begin{frame}[fragile]
	\frametitle{Probability density function}
	The probability density function (PDF) of an exponential distribution is
	
	\[
	f(x;\lambda) = \begin{cases}
	\lambda e^{-\lambda x}, & x \ge 0, \\
	0, & x < 0.
	\end{cases}\]
	The parameter $\lambda$  is called \textbf{\emph{rate}} parameter. It is the inverse of the expected duration ($\mu$).\\ \bigskip
	
	(If the expected duration is 5 ( e.g. five minutes) then the rate parameter value is 0.2.)
\end{frame}

%------------------------------------------------------------------------%
\begin{frame}[fragile]
\frametitle{Cumulative density function}
The cumulative distribution function (CDF) of an exponential distribution is

\[
F(x;\lambda) = \begin{cases}
1-e^{-\lambda x}, & x \ge 0, \\
0, & x < 0.
\end{cases}\]

\end{frame}

%------------------------------------------------------------------------%
\begin{frame}[fragile]
\frametitle{Expected Value and Variance}
The expected value of an exponential random variable $X$ is:

\[
E[X] = \frac{1}{\lambda}\]
The variance of an exponential random variable $X$ is:

\[
V[X] = \frac{1}{\lambda^2}\]

\end{frame}

%------------------------------------------------------------------------%
\begin{frame}[fragile]
\frametitle{Exponential Distribution: Example}
Assume that the length of a phone call in minutes is an exponential random variable $X$ with parameter
$\lambda = 1/10$. If someone arrives at a phone booth just before you arrive, find the probability that you
will have to wait \begin{itemize}
\item[(a)] less than 5 minutes,  
\item[(b)] between 5 and 10 minutes.
\end{itemize}
Use the \texttt{R} code on the following slide to help answer these questions.
\end{frame}



%------------------------------------------------------------------------%
\begin{frame}[fragile]
\frametitle{Exponential Distribution: Example}
\begin{verbatim}
> dexp(0:10,rate=0.10)
 [1] 0.10000000 0.09048374 0.08187308 0.07408182 0.06703200 0.06065307
 [7] 0.05488116 0.04965853 0.04493290 0.04065697 0.03678794
>
> pexp(0:10,rate=0.10)
 [1] 0.00000000 0.09516258 0.18126925 0.25918178 0.32967995 0.39346934
 [7] 0.45118836 0.50341470 0.55067104 0.59343034 0.63212056
\end{verbatim}
\end{frame}

%------------------------------------------------------------------------%
\begin{frame}[fragile]
\frametitle{Exponential Distribution: Example}

As it is CDF values that we are interested in, we use the output from the \texttt{pexp()} commands.

\begin{itemize}
\item[(a)] $P(X \leq 5)$ = 0.39346934 
\item[(b)] $P(5 \leq X \leq 10)$ \\ = $P( X \leq 10) - P( X \leq 5)$ \\ = 0.63212056- 0.39346934 \\ = 0.2386512 \\= 23.84 $\%$
\end{itemize}

\end{frame}



%------------------------------------------------------------------------%
\begin{frame}[fragile]
\frametitle{Exponential Distribution}
\begin{itemize}
\item The Exponential Rate 
\item Related to the Poisson mean (m)
\item If we expect 12 occurrences per hour - what is the rate?
\item We would expected to wait 5 minutes between occurrences.
\item 
\end{itemize}
\end{frame}
%------------------------------------------------------------------------%
\begin{frame}[fragile]
\frametitle{Exponential Distribution}
\begin{verbatim}
>
> pexp(0:9, rate = 0.25)
 [1] 0.0000000 0.2211992 0.3934693 0.5276334 0.6321206
 [6] 0.7134952 0.7768698 0.8262261 0.8646647 0.8946008
>
> pexp(0:9, rate = 0.20)
 [1] 0.0000000 0.1812692 0.3296800 0.4511884 0.5506710
 [6] 0.6321206 0.6988058 0.7534030 0.7981035 0.8347011
>
> pexp(0:9, rate = 0.50)
 [1] 0.0000000 0.3934693 0.6321206 0.7768698 0.8646647
 [6] 0.9179150 0.9502129 0.9698026 0.9816844 0.9888910
> 
\end{verbatim}
\end{frame}


%------------------------------------------------------------------------%
\begin{frame}[fragile]
	\frametitle{Exponential Distribution: Relationship to Poisson Mean}
	\begin{itemize}
		\item The Exponential Rate parameter ($\lambda$) is related to the Poisson mean (m)
		\item If we expect 12 occurrences per hour - what is the rate of occurrences?
		\item We would expected to wait 1/12 of an hour (i.e. 5 minutes) between occurrences.
		\item Be mindful to keep your time units consistent, if working with both Poisson and Exponential.
		\item If working in minutes, our rate parameter values is $\lambda$ = 0.20 (i.e. 1/5).
		\item (This could be the basis of an exam question).
	\end{itemize}
\end{frame}



%------------------------------------------------------------%
\begin{frame}
\frametitle{Continuous Random variables}
\begin{itemize}
\item Previously we have been studying discrete random variables, such as the Binomial and the Poisson random variables.
\item Now we turn our attention to continuous random variables.
\item Recall that a continuous random variable is one which takes an infinite number of possible values, rather than just a countable number of distinct values.
\item Continuous random variables are usually measurements.
\item Examples include height, weight, the amount of sugar in an orange, the time required to run a mile.
\end{itemize}

\end{frame}

\end{document}                             
