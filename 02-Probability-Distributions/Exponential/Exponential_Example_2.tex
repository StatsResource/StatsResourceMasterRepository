\documentclass[a4]{beamer}
\usepackage{amssymb}
\usepackage{graphicx}
\usepackage{subfigure}
\usepackage{newlfont}
\usepackage{amsmath,amsthm,amsfonts}
%\usepackage{beamerthemesplit}
\usepackage{pgf,pgfarrows,pgfnodes,pgfautomata,pgfheaps,pgfshade}
\usepackage{mathptmx}  % Font Family
\usepackage{helvet}   % Font Family
\usepackage{color}

\mode<presentation> {
 \usetheme{Frankfurt} % was Frankfurt
 \useinnertheme{rounded}
 \useoutertheme{infolines}
 \usefonttheme{serif}
 %\usecolortheme{wolverine}
% \usecolortheme{rose}
\usefonttheme{structurebold}
}

\setbeamercovered{dynamic}

\title[MA4413]{MA4413 Statistics for Computing \\ {\normalsize MA4413 Lecture 6A : Continuous Distributions}}
\author[Kevin O'Brien]{Kevin O'Brien \\ {\scriptsize kevin.obrien@ul.ie}}
\date{Autumn 2011}
\institute[Maths \& Stats]{Dept. of Mathematics \& Statistics, \\ University \textit{of} Limerick}

\renewcommand{\arraystretch}{1.5}


%------------------------------------------------------------------------%
\begin{document}

%--------------------------------------------------------------------------------------%
\frame{
\frametitle{Exponential Distribution: Sample Question}
\Large
In a large company computer network, there is an average of 40 log-ons to the network per hour.
\begin{enumerate}
\item[1] What is the average amount of time between log-ons?
\item[2] What is the probability that there will be no log-ons for at least 2.4 minutes
\item[3] What is the probability that the next log-on within 1 minutes of the last?
\item[4] What proportions of log-ons occur between 1 minutes and 2.4 minutes of the last log-on?
\end{enumerate}
}
%--------------------------------------------------------------%
\frame{
\frametitle{Solution (Part 1) }

\Large
\begin{itemize} \item What is the average amount of time between log-ons?

\item If there is 40 log-ons in 60 minutes, it is reasonable to think that someone logs on every 1.5 minutes.
\item Therefore $\mu = 1.5$
\item Therefore the poisson rate is ${1\over1.5}$ = 0.666
\end{itemize}

}


%--------------------------------------------------------------------------------------%
\begin{frame}[fragile]
\frametitle{Solution (Part 2) }
\Large

What is the probability that there will be no log-ons for at least 2.4 minutes?\\
\bigskip
From the formulae:
\[
P( X \geq k) = e^{{-k \over \mu}} .
\]
From the formulae:
\[
P( X \geq 2.4) = e^{{-2.4 \over 1.5}} = e^{-1.6} = 0.2018.
\]

\texttt{R} code (use complement rule)
\begin{verbatim}
> pexp(2.4,rate=(2/3))
[1] 0.7981035
\end{verbatim}
\end{frame}

%--------------------------------------------------------------------------------------%
\begin{frame}[fragile]
\frametitle{Solution (Part 3) }
\Large

What is the probability that the next log-on within 1 minutes of the last?\\
i.e. $P(X \leq 1)$
\bigskip
From the formulae:
\[
P( X \leq 1) = 1 - e^{{-1 \over 1.5}} = 1 -  e^{-0.6666}
\]

\[
P( X \leq 1) = 1 -  0.5135  = 0.4865
\]
\texttt{R} code 
\begin{verbatim}
> pexp(1,rate=(2/3))
[1] 0.4865829
\end{verbatim}
\end{frame}

%--------------------------------------------------------------------------------------%
\frame{
\frametitle{Solution (Part 4) }
\Large

What proportions of log-ons occur between 1 minutes and 2.4 minutes of the last log-on?\\
\bigskip
\begin{itemize}
\item \textbf{Too Low} $P(X \leq 1) = 0.4865$\\
\item \textbf{Too High} $P(X \geq 2.4) = 0.2018$\\
\item Probability of being inside interval $P(1 \leq X \leq 2.4) = 0.31152$.
\item $P(1 \leq X \leq 2.4) = 1- ( 0.4865 + 0.2018) = 0.3117$
\end{itemize}
}

%--------------------------------------------------------------------------------------%
\begin{frame}[fragile]
\Large
\frametitle{Another Example}
Suppose that the service time for a customer at a IT helplien
has an exponential distribution with mean 3 minutes. What is the probability that a
customer waits more than 4 minutes?

\[ P(X  \leq 4) = 1 -  e^{-4/3} \]

\[ P(X  \leq 4) = e^{-4/3} = 0.2636 \]


\texttt{R} code (use complement rule)
\begin{verbatim}
> pexp(4,rate=(1/3))
[1] 0.7364029
\end{verbatim}
\end{frame}




\end{document}





\item What is the probability that the lifespan of the laptop will be at least
6 years?
\item What is the probability that the lifespan of the laptop will not exceed
4 years?
\item What is the probability of the lifespan being between 5 years and 6
years?


%----------------------------------------------------------------------------%
\frame{
\frametitle{The Exponential Distribution}
A continuous random variable having p.d.f. f(x), where:
$f(x) = \lambda x e ^{-\lambda x} $
is said to have an exponential distribution, with parameter $\lambda$.
The cumulative distribution is given by:
$F(x) = 1 - e^{\lambda x}$

Expectation and Variance
$E(X) = 1 / \lambda$\\
$V(X) = 1 / \lambda^2$\\
}



%---------------------------------------------------------------------------------%
\begin{frame}
\frametitle{Exponential Distribution Lifetimes}
The average lifespan of a laptop is 5 years. You may assume that
the lifespan of computers follows an exponential probability
distribution. \begin{itemize}\item (3 marks) What is the
probability that the lifespan of the laptop will be at least 6
years? \item
What is the probability that the lifespan of the laptop will not
exceed 4 years? \item What is the probability of the
lifespan being between 5 years and 6 years?
\end{itemize}
Suppose the lifetime of a PC is exponentially distributed with
mean $\mu =5$
We should be told the average lifetime $\mu$.
\[
P( X \geq x_o) = e^{{-x_o \over \mu}}
\]
\end{frame}


\end{document}








