

%------------------------------------------------------------------------%
\section{The Continuous Uniform Distribution}
%------------------------------------------------------------------------%
\frame{
\frametitle{Parameters}
\Large

The continuous uniform distribution is characterised by the following parameters

\begin{itemize}
\item The lower limit $a$
\item The upper limit $b$
\end{itemize}

It is not possible to have an outcome that is lower than $a$ or larger than $b$.

\[ P(X < a) = P(X > b) = 0\]
}

%------------------------------------------------------------------------%
\frame{
\Large
\begin{itemize}
\item The only possible outcomes are between $a$ and $b$. Suppose $a = 3$ and $b = 6$.\bigskip
\item The following values are possible outcomes: $3.14, \;3.78,\; 4.66,\; 5.8,\;5.9999.$ \bigskip
\item The probability of being exactly equal to $3$ or $6$ can be assumed to be zero. \bigskip
\item The following outcomes are not possible, either because they are too high or too low.
$1.67,\;2,\;67,\;7.14,\; 8.78.$
\end{itemize}
}

%---------------------------------------------------------------------------%
\frame{
	\frametitle{Continuous Uniform Distribution}
	A random variable X is called a continuous uniform random variable over the interval $(a,b)$ if it's probability density function is given by
	
	\[ f_{X}(x)  =  { 1 \over b-a}   \hspace{2cm}  \mbox{ when } a \leq x \leq b\]
	
	The corresponding cumulative density function is
	
	\[ F_x(x) = { x-a \over b-a}   \hspace{2cm}  \mbox{ when } a \leq x \leq b\]
	
}

%-----------------------------------------------------------------------------%

\frame{
	
	The mean of the continuous uniform distribution is
	
	\[ E(X) = {a+b \over 2}\]
	
	\[ V(X) = {(b-a)^2\over12}\]
}


%---------------------------------------------------------------------------%
\frame{
\frametitle{Continuous Random Variables}
\begin{itemize}
\item Probability Density Function
\item Cumulative Density Function
\end{itemize}
If X is a continuous random variable then we can say that the probability of obtaining a \textbf{precise} value $x$ is infinitely small, i.e. close to zero.
\[P(X=x) \approx 0 \]
Consequently, for continuous random variables (only), $P(X \leq x)$ and $P(X < x)$ can be used interchangeably.
\[P(X \leq x) \approx P(X < x) \]
}
%---------------------------------------------------------------------------------------------------------%
\section{Continuous Uniform distribution}
\frame{
\begin{itemize}
\item $L$ :lower bound of an interval \item $U$: upper bound of an
interval
\end{itemize}
Probability of an outcome being between lower bound L and upper
bound U \[P( L \leq X \leq U) = { U - L \over b - a }\]
\textbf{Reminder}
"$\leq$" is less than or equal to.\\
"$\geq$" is greater than or equal to.\\
$L \leq X \leq U$ xan be verbalized as X between L and U. simply
states that X is between L and U inclusively.
("inclusively" mean that X could be exactly L or U also, although
the probability of this is extremely low)\\
}


%------------------------------------------------%
\frame{
\frametitle{Continuous Uniform Distribution} 
\begin{itemize}
\item 
The Uniform distributions model (some) continuous random variables and (some) discrete random variables. 
\item
The values of a uniform random variable are uniformly distributed over an interval. 
\item
For example, if buses arrive at a given bus stop every 15 minutes, and you arrive at the bus stop at a random time, the time you wait for the 
next bus to arrive could be described by a uniform distribution over the interval from 0 to 15.
\end{itemize}
 
}
%---------------------------------------------------------------------------%
\frame{
\frametitle{Continuous Uniform Distribution}
A random variable X is called a continuous uniform random variable over the interval $(a,b)$ if it's probability density function is given by
\[ f_{X}(x) = { 1 \over b-a} \hspace{2cm} \mbox{ when } a \leq x \leq b\]
The corresponding cumulative density function is
\[ F_x(x) = { x-a \over b-a} \hspace{2cm} \mbox{ when } a \leq x \leq b\]
}
%-----------------------------------------------------------------------------%
\frame{
\frametitle{Continuous Uniform Distribution}
The mean of the continuous uniform distribution is
\[ E(X) = {a+b \over 2}\]
\[ V(X) = {(b-a)^2\over12}\]
}
%------------------------------------------------------------------------%
\frame{\frametitle{Uniform Distribution: Variance}
\Large
The variance  of the continuous uniform distribution, denoted $\textrm{Var}[X]$,  is  computed using the following formula
\vspace{0.1cm}
\[
\textrm{Var}[X] = {(b - a)^2 \over 12}
\]
\vspace{0.1cm}
For our previous example this is
\[
\textrm{Var}[X] = \alert{{(3 - 0)^2 \over 12} =  {3^2 \over 12} = {9 \over 12} = 0.75}
\]
}
\begin{frame}[fragile]
	\frametitle{The Uniform Distribution}
	In the last class, we had a look at the continuous uniform distribution. It is very useful in constructing simulations. Briefly we will look at some relevant \texttt{R} function.
	The distribution has two parameters: i.e \texttt{min} and \texttt{max}. (Here chosen as 5 and 10 respectively)
	\begin{verbatim}
	># Generate Four Random Number
	> runif(4, min=5,max=10)
	[1] 9.709372 7.884805 5.571331 5.017549
	>
	># Compute Density
	> dunif(4:11,min=5,max=10)
	[1] 0.0 0.2 0.2 0.2 0.2 0.2 0.2 0.0
	>
	> #Compute distribution of
	> punif(4:11,min=5,max=10)
	[1] 0.0 0.0 0.2 0.4 0.6 0.8 1.0 1.0
	\end{verbatim}
	
\end{frame}

\end{document}
