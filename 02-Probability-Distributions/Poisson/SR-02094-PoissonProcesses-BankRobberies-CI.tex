\documentclass[a4paper,12pt]{article}

%%%%%%%%%%%%%%%%%%%%%%%%%%%%%%%%%%%%%%%%%%%%%%%%%%%%%%%%%%%%%%%%%%%%%%%%%%%%%%%%%%%%%%%%%%%%%%%%%%%%%%%%%%%%%%%%%%%%%%%%%%%%%%%%%%%%%%%%%%%%%%%%%%%%%%%%%%%%%%%%%%%%%%%%%%%%%%%%%%%%%%%%%%%%%%%%%%%%%%%%%%%%%%%%%%%%%%%%%%%%%%%%%%%%%%%%%%%%%%%%%%%%%%%%%%%%

\usepackage{eurosym}
\usepackage{vmargin}
\usepackage{amsmath}
\usepackage{graphics}
\usepackage{epsfig}
\usepackage{enumerate}
\usepackage{multicol}
\usepackage{subfigure}
\usepackage{fancyhdr}
\usepackage{listings}
\usepackage{framed}
\usepackage{graphicx}
\usepackage{amsmath}
\usepackage{chngpage}

%\usepackage{bigints}
\usepackage{vmargin}

% left top textwidth textheight headheight

% headsep footheight footskip

\setmargins{2.0cm}{2.5cm}{16 cm}{22cm}{0.5cm}{0cm}{1cm}{1cm}

\renewcommand{\baselinestretch}{1.3}

\setcounter{MaxMatrixCols}{10}

\begin{document}

\large
\noindent Bank robberies in various countries are assumed to occur according to Poisson
processes with rates that vary from year to year.
\begin{itemize}
    \item It was reported that the number of robberies in a particular country in a specific year was 123. 
    \item The number of robberies
in a different country in the same year was 111. 
\item It can be assumed that each robbery is an independent event and that robberies occur independently in the two countries.
\end{itemize}
Determine an approximate 90\% confidence interval for the difference between the true yearly robbery rates in the two countries.

%%%%%%%%%%%%%%%%%%5
\newpage

%%%%%%%%%%%%%%%%%%%%%%%%%%%%%%%%%%%%%%%%%%%%%%%%%%%%%%%%%%%%%%%%%%%%%%%%%%%%%%%%%%%%%%
%%- Solution to Question 5
\begin{itemize}
\item Under given assumptions $X_1 \sim \mbox{Poisson}(\lambda_1 )$, $X_2 \sim \mbox{Poisson}(\lambda_2 )$ and approximately
\[X_1 \sim  N(\lambda_1 , \lambda_1 ),\] and \[ X_{2} \sim  N(\lambda_2 , \lambda_2 )\]
giving $X_1 \;-\; X_2 \sim  N(\lambda_1 \;-\; \lambda_2 \;,\; \lambda_1 + \lambda_2 )$
\item Equivalently
\[
\frac{X_{1} \;-\; X_{2} \;-\; ( \lambda_1 \;-\; \lambda_2 )}{\sqrt{\lambda_1 + \lambda_2 }}
\sim N (0,1)
\]
\item Approximate 90\% interval given as

\begin{eqnarray*}
(X_{1} \;-\; X_{2}) \pm Z_{0.05} \sqrt{\hat{\lambda}_1 + \hat{\lambda}_2 } &=& (X_{1} \;-\; X_{2}) \pm Z_{0.05} \sqrt{X_{1} + X_{2}}\\
&=& 12 \pm 1.6449 \times \sqrt{(234)} \\ &=& 12 \pm 25.162 \\ &=&  (\;-\;13.162, 37.162)\\
\end{eqnarray*}

% \item A common error here involved the normal approximation of the difference of the two variables – especially its variance.
\end{itemize}
\end{document}
