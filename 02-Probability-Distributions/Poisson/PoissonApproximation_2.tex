\documentclass[a4paper,12pt]{article}
%%%%%%%%%%%%%%%%%%%%%%%%%%%%%%%%%%%%%%%%%%%%%%%%%%%%%%%%%%%%%%%%%%%%%%%%%%%%%%%%%%%%%%%%%%%%%%%%%%%%%%%%%%%%%%%%%%%%%%%%%%%%%%%%%%%%%%%%%%%%%%%%%%%%%%%%%%%%%%%%%%%%%%%%%%%%%%%%%%%%%%%%%%%%%%%%%%%%%%%%%%%%%%%%%%%%%%%%%%%%%%%%%%%%%%%%%%%%%%%%%%%%%%%%%%%%
\usepackage{eurosym}
\usepackage{vmargin}
\usepackage{amsmath}
\usepackage{graphics}
\usepackage{epsfig}
\usepackage{subfigure}
\usepackage{enumerate}
\usepackage{fancyhdr}

\setcounter{MaxMatrixCols}{10}
%TCIDATA{OutputFilter=LATEX.DLL}
%TCIDATA{Version=5.00.0.2570}
%TCIDATA{<META NAME="SaveForMode"CONTENT="1">}
%TCIDATA{LastRevised=Wednesday, February 23, 201113:24:34}
%TCIDATA{<META NAME="GraphicsSave" CONTENT="32">}
%TCIDATA{Language=American English}

\pagestyle{fancy}
\setmarginsrb{20mm}{0mm}{20mm}{25mm}{12mm}{11mm}{0mm}{11mm}
\lhead{StatsResource} \rhead{Kevin O'Brien} \chead{Poisson Distribution} %\input{tcilatex}

\begin{document}
	
		\titlepage
	\medskip

%---------------------------------------------------------------%
{
	\textbf{Poisson Approximation of the Binomial}
	\begin{itemize}
		\item The Poisson distribution can sometimes be used to approximate the binomial distribution
		\item When the number of observations n is large, and the success probability p is small, the $\mbox{Bin}(n,p)$ distribution approaches the Poisson distribution with the parameter given by $m = np$.
		\item This is useful since the computations involved in calculating binomial probabilities are greatly reduced.
		\item As a rule of thumb, n should be greater than 50 with p very small, such that $np$ should be less than 5.
		\item If the value of $p$ is very high, the definition of what constitutes a ``success" or ``failure" can be switched.
	\end{itemize}
}
%---------------------------------------------------------------%
{
	\textbf{Poisson Approximation: Example}
	
	Suppose we sample 1000 items from a production line that is producing, on average, 0.1\% defective components.\\
	
	
	\bigskip
	
	Using the binomial distribution, the probability of exactly 3 defective items in our sample is
	
	\[P(X=3) = ^{1000}C_3 \times (0.001)^3 \times 0.999^{997} \]
	
}
%---------------------------------------------------------------%
{
	\textbf{Poisson Approximation: Example}
	Lets compute each of the component terms individually.
	
	
	\begin{itemize}
		\item $^{1000}C_3$
		
		\[ ^{1000}C_3 = \frac{1000 \times 999 \times 998}{3 \times 2 \times 1} =
		166,167,000 \]
		
		\item $0.001^3$
		
		\[0.001^3 = 0.000000001 \]
		
		
		\item $0.999^{997}$
		
		\[0.999^{997} = 0.36880 \]
		
	\end{itemize}
	Multiply these three values to compute the binomial probability \[P(X=3) = 0.06128 \]
	
}
%---------------------------------------------------------------%
%---------------------------------------------------------------%
{
	\textbf{Poisson Approximation: Example}
	\begin{itemize}
		\item Lets use the Poisson distribution to approximate a solution.
		
		\item First check that $n \geq 50$ and $np <5$ (Yes to both).
		
		\item We choose as our parameter value $m = np = 0.001 \times 1000  = 1$
		
		\[P(X=3) = e^{-1}\frac{1^3}{3!} = \frac{e^{-1}}{6} = \frac{0.36787}{6} =  0.06131\]
		\item Compare this answer with the Binomial probability \\ $P(X=3) = 0.06128$.
		\item Very good approximation, with much less computation effort.
	\end{itemize}
}

[fragile]
	\textbf{Implementation using \texttt{R}}
	
	
	\begin{verbatim}
	> # Poisson Mean m = 1000 * 0.001 = 1
	> dbinom(3,size=1000,prob=0.001)
	[1] 0.06128251
	>
	> dpois(3,lambda=1)
	[1] 0.06131324
	\end{verbatim}
	
\medskip

%---------------------------------------------------------------%
{
\textbf{Poisson Approximation of the Binomial}
\begin{itemize}
\item The Poisson distribution can sometimes be used to approximate the binomial distribution
\item When the number of observations n is large, and the success probability p is small, the $\mbox{Bin}(n,p)$ distribution approaches the Poisson distribution with the parameter given by $m = np$.
\item This is useful since the computations involved in calculating binomial probabilities are greatly reduced.
\item As a rule of thumb, n should be greater than 50 with p very small, such that $np$ should be less than 5.
\item If the value of $p$ is very high, the definition of what constitutes a ``success" or ``failure" can be switched.
\end{itemize}
}


%---------------------------------------------------------------------%

\textbf{Poisson Approximation: Example}

\begin{itemize}
\item Suppose we sample 1000 items from a production line that is producing, on
average, $0.1\%$ defective components.
\item Using the binomial distribution, the probability of exactly 3 defective items in
our sample is
\[P(X = 3) = ^{1000}C_{3} \times 0.001^{3} \times 0.999^{997}\]
\end{itemize}
\medskip
%---------------------------------------------------------------%
{
\textbf{Poisson Approximation: Example}

Suppose we sample 1000 items from a production line that is producing, on average, 0.1\% defective components.\\


\bigskip

Using the binomial distribution, the probability of exactly 3 defective items in our sample is

\[P(X=3) = ^{1000}C_3 \times (0.001)^3 \times 0.999^{997} \]

}
%---------------------------------------------------------------%
{
\textbf{Poisson Approximation: Example}
Lets compute each of the component terms individually.


\begin{itemize}
\item $^{1000}C_3$

\[ ^{1000}C_3 = \frac{1000 \times 999 \times 998}{3 \times 2 \times 1} =
166,167,000 \]

\item $0.001^3$

\[0.001^3 = 0.000000001 \]


\item $0.999^{997}$

\[0.999^{997} = 0.36880 \]

\end{itemize}
Multiply these three values to compute the binomial probability \[P(X=3) = 0.06128 \]

}
%---------------------------------------------------------------%
%---------------------------------------------------------------%
{
\textbf{Poisson Approximation: Example}
\begin{itemize}
\item Lets use the Poisson distribution to approximate a solution.

\item First check that $n \geq 50$ and $np <5$ (Yes to both).

\item We choose as our parameter value $m = np = 0.001 \times 1000  = 1$

\[P(X=3) = e^{-1}\frac{1^3}{3!} = \frac{e^{-1}}{6} = \frac{0.36787}{6} =  0.06131\]
\item Compare this answer with the Binomial probability \\ $P(X=3) = 0.06128$.
\item Very good approximation, with much less computation effort.
\end{itemize}
}


%---------------------------------------------------------------------%

\textbf{Poisson Approximation: Example}
Lets compute each of the component terms individually.

\begin{itemize}
\item $^{1000}C_{3}$
\[^{1000}C_{3} = \frac{1000 \times 999 \times 998}{3 \times 2 \times 1} = 166,167,000\]
\item $0.001^3$
\[0.001^3 = 0.000000001\]
\item $0.999^{997}$
\[0.999^{997} = 0.36880\]
\end{itemize}


Multiply these three values to compute the binomial probability
$P(X = 3) = 0.06128$
\medskip


\textbf{Poisson Approximation: Example}
\begin{itemize}
\item Lets use the Poisson distribution to approximate a solution.
\item First check that $n \geq 50$ and $np < 5$ (Yes to both).
\item We choose as our parameter value $m = np = 1000 \times 0.001 = 1$
\end{itemize}
\[P(X = 3) = \frac{e^{-1} \times 1^3}{3!} = \frac{e^{-1}}{6} = \frac{0.36787}{6} = 0.06131 \]
Compare this answer with the Binomial probability
P(X = 3) = 0.06128.
Very good approximation, with much less computation effort.
\medskip
%---------------------------------------------------------------------%
\end{document}
[fragile]
\textbf{Implementation using \texttt{R}}


\begin{verbatim}
> # Poisson Mean m = 1000 * 0.001 = 1
> dbinom(3,size=1000,prob=0.001)
[1] 0.06128251
>
> dpois(3,lambda=1)
[1] 0.06131324
\end{verbatim}

\medskip

\end{document}