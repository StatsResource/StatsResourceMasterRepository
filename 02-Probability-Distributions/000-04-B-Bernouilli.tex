
\section{Bernouilli Trial}
\begin{itemize}
	\item Now consider an experiment with only two outcomes. Independent repeated trials of such an experiment are
	called \textbf{\textit{Bernoulli trials}}, named after the Swiss mathematician Jacob Bernoulli (1654–1705). \item The term \textbf{\emph{independent
			trials}} means that the outcome of any trial does not depend on the previous outcomes (such as tossing a coin).
	\item We will call one of the outcomes the ``success" and the other outcome the ``failure".
	
	\item
	Let $p$ denote the probability of success in a Bernoulli trial, and so $q = 1 - p$ is the probability of failure.
	A binomial experiment consists of a fixed number of Bernoulli trials. \item A binomial experiment with $n$ trials and
	probability $p$ of success will be denoted by
	\[B(n, p)\]
	\item a probability mass function (pmf) is a function that gives the probability that a 
	discrete random variable is exactly equal to some value. 
	\item The probability mass function is often the primary means of defining a discrete probability distribution 
\end{itemize}


\section*{Bernouilli Trial}
\begin{itemize}


\item We would denote a binomial random variable $X$ with $n$ trials and
probability $p$ of success as follows
\[X \sim Bin(n, p)\]

\end{itemize}



%===============================================================%
\subsection*{Examples}
Consider the following statistical experiment. You flip a coin five times and count the number of times the coin lands on heads. This is a binomial experiment because:
\begin{itemize}
\item The experiment consists of repeated trials. We flip a coin five times.
\item Each trial can result in just two possible outcomes: heads or tails. We call one of these outcomes a success and the other, a failure, depending on the question asked.

\item The probability of success, denoted by $p$ is constant : 0.5 on every trial for a fair coin.
\item The probability of success, denoted by P, is the same on every trial.
\item The trials are independent; that is, getting heads on one trial does not affect whether we get heads on other trials.
\end{itemize}




\section*{What is a Success}
\begin{itemize}
\item
The word ``success" implies that the outcome is the outcome of interest.
\item If the outcome of interest is something like a flat tire, using the word ``Success" is counter intuituive.
\item Typically the success event is the less likely of the two events.
\end{itemize}

%============================================================================================%
\section*{Binomial Experiments}
\begin{itemize}
\item A binomial experiment with n trials and
probability $p$ of success will be denoted by
\[B(n, p)\]
\item Frequently, we are interested in the \textbf{\emph{number of successes}} in a binomial experiment, not in the order in which they occur.
\item Furthermore, we are interested in the probability of that number of successes.

\item The number of successes X in n trials of a binomial experiment is called a binomial random variable.
\item The number of independent trials is denoted $n$.
\item The probability of a `success' is $p$
\item The expected number of `successes' from $n$ trials is $E(X) = np$
\end{itemize}

%---------------------------------------------------------------------------%

\section*{Bernoulli Distribution: The coin toss}

There is no more basic random event than the flipping of a coin. Heads or tails. It's as simple as you can get! The "Bernoulli Trial" refers to a single event which can have one of two possible outcomes with a fixed probability of each occurring. You can describe these events as "yes or no" questions. For example:

%============================================================ %
\begin{itemize}
	\item Will the coin land heads?
	\item Will the newborn child be a girl?
	\item Are a random person's eyes green?
	\item Will a mosquito die after the area was sprayed with insecticide?
	\item Will a potential customer decide to buy my product?
	\item Will a citizen vote for a specific candidate?
	\item Is an employee going to vote pro-union?
\end{itemize}

Will this person be abducted by aliens in their lifetime?
The Bernoulli Distribution has one controlling parameter: the probability of success. A "fair coin" or an experiment where success and failure are equally likely will have a probability of 0.5 (50%). Typically the variable p is used to represent this parameter.

If a random variable X is distributed with a Bernoulli Distribution with a parameter p we write its probability mass function as:

%================================================================================================ %
\[f(x) = \begin{cases}p, & \mbox{if } x = 1\\1-p, & \mbox{if } x = 0\end{cases}\quad 0\leq p \leq 1\]
Where the event X=1 represents the "yes."

This distribution may seem trivial, but it is still a very important building block in probability. The Binomial distribution extends the Bernoulli distribution to encompass multiple "yes" or "no" cases with a fixed probability. Take a close look at the examples cited above. Some similar questions will be presented in the next section which might give an understanding of how these distributions are related.




\subsection*{Mean}
The mean (E[X]) can be derived:
\[ \operatorname{E}[X] = \sum_i f(x_i) \cdot x_i\]
\[ \operatorname{E}[X]  = p \cdot 1 + (1-p) \cdot 0\]
\[ \operatorname{E}[X]= p \,\]
Variance
\[\operatorname{Var}(X) = \operatorname{E}[(X-\operatorname{E}[X])^2] = \sum_i f(x_i)  \cdot (x_i - \operatorname{E}[X])^2
\operatorname{Var}(X)= p \cdot (1-p)^2 + (1-p) \cdot (0-p)^2 
\]
\[\operatorname{Var}(X)= [p(1-p) + p^2](1-p) \,
\operatorname{Var}(X)= p(1-p) \,
\]



\begin{verbatim}
Bernoulli
Parameters	0<p<1, p\in\R
Support	k=\{0,1\}\,
PMF	
\begin{cases}
	q=(1-p) & \text{for }k=0 \\ p & \text{for }k=1
\end{cases}

CDF	
\begin{cases}
	0 & \text{for }k<0 \\ q & \text{for }0\leq k<1 \\ 1 & \text{for }k\geq 1
\end{cases}

Mean	p\,
Median	\begin{cases}
	0 & \text{if } q > p\\
	0.5 & \text{if } q=p\\
	1 & \text{if } q<p
\end{cases}
Mode	\begin{cases}
	0 & \text{if } q > p\\
	0, 1 & \text{if } q=p\\
	1 & \text{if } q < p
\end{cases}
Variance	p(1-p)\,
Skewness	\frac{q-p}{\sqrt{pq}}
Ex. kurtosis	\frac{1-6pq}{pq}
Entropy	-q\ln(q)-p\ln(p)\,
MGF	q+pe^t\,
CF	q+pe^{it}\,
PGF	q+pz\,
Fisher information	 \frac{1}{p(1-p)} 
\end{verbatim}

\begin{verbatim}
Bernoulli
Parameters	0<p<1, p\in\R
Support	k=\{0,1\}\,
PMF	
\begin{cases}
q=(1-p) & \text{for }k=0 \\ p & \text{for }k=1
\end{cases}
\end{verbatim}

CDF	
\[\begin{cases}
0 & \text{for }k<0 \\ q & \text{for }0\leq k<1 \\ 1 & \text{for }k\geq 1
\end{cases}\]


\begin{itemize}
	\item Mean	\[E(X) = p \]
	\item Median \[	\begin{cases}
	0 & \text{if } q > p\\
	0.5 & \text{if } q=p\\
	1 & \text{if } q<p \end{cases}	\]
	
	\item Mode	\[\begin{cases}
	0 & \text{if } q > p\\
	0, 1 & \text{if } q=p\\
	1 & \text{if } q < p
	\end{cases}\]
\end{itemize}
Variance	$p(1-p)\,$
Skewness	$\frac{q-p}{\sqrt{pq}}$
Ex. kurtosis	$\frac{1-6pq}{pq}$
%\begin{verbatim}
%Entropy	-q\ln(q)-p\ln(p)\,
%MGF	q+pe^t\,
%CF	q+pe^{it}\,
%PGF	q+pz\,
%Fisher information	 \frac{1}{p(1-p)} 
%\end{verbatim}

			\noindent \textbf{Binomial Experiment}
			A binomial experiment (also known as a Bernoulli trial) is a statistical experiment that has the following properties:
			\begin{itemize}
				\item The experiment consists of n repeated trials.
				\item Each trial can result in just two possible outcomes. We call one of these outcomes a success and the other, a failure.
				\item The probability of success, denoted by P, is the same on every trial.
				\item The trials are independent; that is, the outcome on one trial does not affect the outcome on other trials.
			\end{itemize}
			
Consider the following statistical experiment. You flip a coin 2 times and count the number of times the coin lands on heads. This is a binomial experiment because:
						\begin{itemize}
							\item The experiment consists of repeated trials. We flip a coin 2 times.
							\item Each trial can result in just two possible outcomes - heads or tails.
							\item The probability of success is constant - 0.5 on every trial.
							\item The trials are independent; that is, getting heads on one trial does not affect whether we get heads on other trials.
						\end{itemize}




\end{document}
