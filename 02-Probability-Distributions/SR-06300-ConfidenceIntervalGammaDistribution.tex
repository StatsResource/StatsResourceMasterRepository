



\begin{framed}
Scaling
If

\[{\displaystyle X\sim \mathrm {Gamma} (k,\theta ),}\]
then, for any $c > 0$,

${\displaystyle cX\sim \mathrm {Gamma} (k,c\,\theta ),}$ by moment generating functions,
or equivalently, if

${\displaystyle X\sim \mathrm {Gamma} \left(\alpha ,\lambda \right)}$ 
(shape-rate parameterization)
\[{\displaystyle cX\sim \mathrm {Gamma} \left(\alpha ,\frac{ \lambda  }{c}\right),}\]

\end{framed}



\begin{framed}
(2) Note that the distribution in (1) does depend on $\theta$ and the form of the statistic doesn't. You need to modify the statistic ($Q=f(T,\theta)$) in such a way that both of those change. (This part is trivial!)

Let $Q=T/\theta$. Then $Q\sim gamma(12,1)$.

Q satisfies the conditions required for a pivotal quantity.

(3) Once you have a pivotal quantity (i.e. Q), write down an interval for the pivotal quantity (in the form of a pair of inequalities, $a<Q<b$) with the given coverage. Since the distribution doesn't depend on the parameter, this interval is always the same (at a given sample size) no matter what the value of $\theta$.

One such interval is (a,b), where P(a<Q<b)=0.95, when a is the 0.025 point of the gamma(12,1) distribution and b is the 0.975 point.

(4) Now write the interval involving the pivotal quantity back in terms of the data and $\theta$. Back out an interval for the parameter, for which the corresponding probability statement must still hold (keeping in mind that the random quantity is not $\theta$ but the interval).

$P(a<T/\theta<b)=0.95$ implies $P(1/b<\theta/T<1/a)=0.95$, so $P(T/b<\theta<T/a)=0.95$. Therefore $(T/b,T/a)$ is a 95\% interval for $\theta$.

Our observed total, t=4.91. The 0.025 point of a gamma(12,1) is 6.2006 and the 0.975 point is 19.682. Hence a 95\% interval for $\theta$ is (4.91/19.682,4.91/6.200)
= (0.249,0.792).

\end{framed}

