\documentclass[a4paper,12pt]{article}
%%%%%%%%%%%%%%%%%%%%%%%%%%%%%%%%%%%%%%%%%%%%%%%%%%%%%%%%%%%%%%%%%%%%%%%%%%%%%%%%%%%%%%%%%%%%%%%%%%%%%%%%%%%%%%%%%%%%%%%%%%%%%%%%%%%%%%%%%%%%%%%%%%%%%%%%%%%%%%%%%%%%%%%%%%%%%%%%%%%%%%%%%%%%%%%%%%%%%%%%%%%%%%%%%%%%%%%%%%%%%%%%%%%%%%%%%%%%%%%%%%%%%%%%%%%%
\usepackage{eurosym}
\usepackage{vmargin}
\usepackage{amsmath}
\usepackage{framed}
\usepackage{graphics}
\usepackage{epsfig}
\usepackage{subfigure}
\usepackage{enumerate}
\usepackage{fancyhdr}

\setcounter{MaxMatrixCols}{10}
%TCIDATA{OutputFilter=LATEX.DLL}
%TCIDATA{Version=5.00.0.2570}
%TCIDATA{<META NAME="SaveForMode"CONTENT="1">}
%TCIDATA{LastRevised=Wednesday, February 23, 201113:24:34}
%TCIDATA{<META NAME="GraphicsSave" CONTENT="32">}
%TCIDATA{Language=American English}

\pagestyle{fancy}
\setmarginsrb{20mm}{0mm}{20mm}{25mm}{12mm}{11mm}{0mm}{11mm}
\lhead{StatsResource} 
\chead{Information Theory} \rhead{Exercises} %\input{tcilatex}
\begin{document}

\large 
\chead{Probability Distributions}
%\input{tcilatex}

\begin{document}
\large 
%------------------------------------------------------------------------------------%
\subsection{Binomial Distribution}
According to a recent poll, approximately seventy percent of U.S. adults drink alcohol.
Suppose 5 U.S. adults are randomly selected. Let represent the number of adults in the sample who drink 
alcohol. Use the binomial probability formula, the binomial probability table, or your calculator to find the 
following probabilities.
%%%%%%%%%%%%%%%%%%%%%%%%%%%%%%%%%%%%%%%%%%%%%%%%%%%%%%%%%%%
\begin{itemize}
\item a. That exactly 2 adults in the sample drink alcohol.
= 0.1323
\item b. That at least three adults in the sample drink alcohol.
= P(3) + P(4) + P(5) = 0.3087 + 0.36015 + 0.16807 = 0.83692
\item Alternatively, you can use binomcdf on the calculator:
P(at least 3) = 1 – P(2 or fewer) = 1 – binomcdf(5, 0.70, 2) = 0.83692
\item c. That everyone in the sample drinks alcohol.
= 0.16807
\end{itemize}


\end{document}