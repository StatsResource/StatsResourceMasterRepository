\documentclass[a4]{beamer}
\usepackage{amssymb}
\usepackage{graphicx}
\usepackage{subfigure}
\usepackage{newlfont}
\usepackage{amsmath,amsthm,amsfonts}
%\usepackage{beamerthemesplit}
\usepackage{pgf,pgfarrows,pgfnodes,pgfautomata,pgfheaps,pgfshade}
\usepackage{mathptmx}  % Font Family
\usepackage{helvet}   % Font Family
\usepackage{color}


\setbeamercovered{dynamic}

\title[MA4704]{Technology Mathematics 4 (Statistics) \\ {\normalsize MA4704 Lecture 3C}}
\author[Kevin O'Brien]{Kevin O'Brien \\ {\scriptsize Kevin.obrien@ul.ie}}
\date{Spring Semester 2013}
\institute[Maths \& Stats]{Dept. of Mathematics \& Statistics, \\ University \textit{of} Limerick}

\renewcommand{\arraystretch}{1.5}

\begin{document}

\begin{frame}
\titlepage
\end{frame}

% Cumulative Distribution Function - Definition (probability mass function)
% Changing the unit space.
% Poisson Example (MA4102)
% Key Rules
% Distinguishing between binomial and poisson.
% Poisson Approximation of Binomial
%

% Last class : Cumulative Tables tables
% Formal Definition
% Sample Space and Partitioning




%---------------------------------------------------------------------%
\begin{frame}
\frametitle{Today's Class}
\begin{itemize}
\item Definition of Cumulative Distribution Function.
\item Binomial Example
\item Using cumulative tables.
\item Poisson distribution - example
%\item Poisson approximation of the binomial Distribution
%\item Poisson approximation - example

\end{itemize}
\end{frame}



%---------------------------------------------------------------------%
\begin{frame}
\frametitle{The Cumulative Distribution Function}
\begin{itemize}
\item The Cumulative Distribution Function, denoted $F(x)$, is a common way that the probabilities
of a random variable (both discrete and continuous) can be summarized.
\item The Cumulative Distribution Function, which also can be
described by a formula or summarized in a table, is defined as:
\[F(x) = P(X \leq x) \]
\item The notation for a cumulative distribution function, F(x), entails using a capital
"F".  (The notation for a probability mass or density function, f(x), i.e. using a lowercase "f". The notation is not interchangeable.
\end{itemize}
\end{frame}

%---------------------------------------------------------------------%
\begin{frame}
\frametitle{Useful Results}
(Demonstration on the blackboard re: partitioning of the sample space, using examples on next slide)
\begin{itemize}
\item $P(X \leq 1) = P(X=0) + P(X=1)$
\item $P(X \leq r) = P(X=0)+ P(X=1) + \ldots P(X= r)$
\item $P(X \leq 0) = P(X=0)$
\item $P(X = r) = P(X \geq r ) - P(X \geq r + 1)$
\item \textbf{Complement Rule}: $P(X \leq r-1) = P(X < r) = 1 - P(X \geq r)$
\item \textbf{Interval Rule}:$ P(a \leq X \leq  b)= P(X \geq a) - P(X \geq b + 1).$
\end{itemize}
For the binomial distribution, if the probability of success is greater than 0.5, instead of
considering the number of successes, to use the table we consider
the number of failures.
\end{frame}
%---------------------------------------------------------------------%


%---------------------------------------------------------------------%
\begin{frame}
\frametitle{Binomial Example 1}
Suppose a signal of 100 bits is transmitted and the probability of
sending a bit correctly is 0.9. What is the probability of
\begin{enumerate}
\item at least 10 errors
\item exactly 7 errors
\item Between 5 and 15 errors (inclusively).
\end{enumerate}
\end{frame}
%---------------------------------------------------------------------%
\begin{frame}
\frametitle{Binomial Example 1}
\begin{itemize}
\item Since the probability of success is 0.9. We consider the distribution
of the number of failures (errors).
\item We reverse the definition of `success' and `failure'. Success is now defined as an error.
\item The probability that a bit is sent incorrectly is 0.1.
\item Let X be the total number of errors. $X \sim B(100, 0.1)$.
\item Answer : $P(X \geq 10) = 0.5487$.
\item $P(X = 7)=P(X \geq 7) - P(X \geq 8) =0.8828 - 0.7939 = 0.0889$.
\item $P(5 \leq X  \leq 15) = P(X \geq 5) - P(X \geq 16) =0.9763 - 0.0399 = 0.9364$
\end{itemize}
\end{frame}

%---------------------------------------------------------------------%
\end{document}
