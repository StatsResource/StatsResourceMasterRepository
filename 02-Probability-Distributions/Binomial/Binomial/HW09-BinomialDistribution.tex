\documentclass[12pt, a4paper]{report}
\usepackage{epsfig}
\usepackage{subfigure}
%\usepackage{amscd}
\usepackage{amssymb}
\usepackage{graphicx}
%\usepackage{amscd}

\usepackage{subfiles}
\usepackage{framed}
\usepackage{subfiles}
\usepackage{amsthm, amsmath}
\usepackage{amsbsy}
\usepackage{framed}
\usepackage[usenames]{color}
\usepackage{listings}
\lstset{% general command to set parameter(s)
	basicstyle=\small, % print whole listing small
	keywordstyle=\color{red}\itshape,
	% underlined bold black keywords
	commentstyle=\color{blue}, % white comments
	stringstyle=\ttfamily, % typewriter type for strings
	showstringspaces=false,
	numbers=left, numberstyle=\tiny, stepnumber=1, numbersep=5pt, %
	frame=shadowbox,
	rulesepcolor=\color{black},
	columns=fullflexible
} %
%\usepackage[dvips]{graphicx}
\usepackage{natbib}
\bibliographystyle{chicago}
\usepackage{vmargin}
% left top textwidth textheight headheight
% headsep footheight footskip
\setmargins{3.0cm}{2.5cm}{15.5 cm}{22cm}{0.5cm}{0cm}{1cm}{1cm}
\renewcommand{\baselinestretch}{1.5}
\pagenumbering{arabic}
%\theoremstyle{plain}
\newtheorem{theorem}{Theorem}[section]
\newtheorem{corollary}[theorem]{Corollary}
\newtheorem{ill}[theorem]{Example}
\newtheorem{lemma}[theorem]{Lemma}
\newtheorem{proposition}[theorem]{Proposition}
\newtheorem{conjecture}[theorem]{Conjecture}
\newtheorem{axiom}{Axiom}
\theoremstyle{definition}
\newtheorem{definition}{Definition}[section]
\newtheorem{notation}{Notation}
\theoremstyle{remark}
\newtheorem{remark}{Remark}[section]
\newtheorem{example}{Example}[section]
\renewcommand{\thenotation}{}
%\renewcommand{\thetable}{\thesection.\arabic{table}}
%\renewcommand{\thefigure}{\thesection.\arabic{figure}}

\author{ } \date{ }

\begin{document}
	Binomal Distribution
	Bernoiulli Trials
	
	Page 44
	Section 4.2.1 
	
	\begin{itemize}
		\item The binomial probability distribution is a discrete probability distribution that has many applications.
		\item It is associated with a multiple step experiment that we call the binomial experiment.
	\end{itemize}
	
	A binomial experiment has the following properties.
	
	\begin{enumerate}
		\item The experiment consists of a sequence of n identical trials.
		
		\item Two outcomes are possible at each trial. We refer to one as a success and the other as a failure.
		
		\item The probability of a success, denoted by p, does not change from trial to trial. Similarly the probability of a failure, denoted by 1-p, also does not change from trial to trial.
		
		\item The trials are independent.
	\end{enumerate}
	

	\textbf{Binomial Distribution: Expected Value and Variance}
	
	
	If the random variable X has a binomial distribution with parameters n
	and p, we write
	\[ X \sim B(n,p) \]
	
	Expectation and Variance
	If $X \sim B(n,p)$, then:
	
	\begin{itemize}
		\item Expected Value of X : $E(X) = np$
		\item Variance of X : $Var(X) = np(1-p)$
	\end{itemize}
	
	Suppose n=3 and p=0.5 
	Then $E(X) = 1.5$ and $V(X) = 0.75$.
	
	Remark: Referring to the expected value and variance may be used to validate
	the assumption of a binomial distribution.


	%=================================================%
	
	%% \frametitle{The Binomial Distribution}
	
	
	In a binomial experiment our interest is in the number of successes occurring in the n trials.
	Binomial Probability Function
	
	In general, if the random variable X follows the binomial distribution with parameters n ∈ ℕ and p ∈ [0,1], we write X ~ B(n, p). The probability of getting exactly k successes in n trials is given by the probability mass function:
	
\[	f(k;n,p)=\Pr(X=k)={\binom {n}{k}}p^{k}(1-p)^{n-k}\]
	for k = 0, 1, 2, ..., n, where
	
\[	{\binom {n}{k}}={\frac {n!}{k!(n-k)!}}\]
	is the binomial coefficient, hence the name of the distribution. 
	
	[Remark ; Provided in exam formulae. Please see pg 142]
	
	
	
	where
	
	= the probability of   successes in   trials
	
	= the number of trials
	
	\begin{itemize}
		\item The formula can be understood as follows: we want exactly k successes ($p^k$) and n − k failures $(1 − p)^{n − k}$.
		\item However, the k successes can occur anywhere among the n trials, and there are ${n \choose k}$ different ways of distributing k successes in a sequence of n trials.
	\end{itemize}
	
Binomial probability
	\[y \;=\; \frac{{n!}}{{k!\left( {n - k} \right)!}}p^k q^{n - k} \; = \; \left( {\begin{array}{*{20}c} n \\ k \\ \end{array}} \right)p^k q^{n -
		k}\]
	
\subsection{Mean and variance of Binomial distribution}
	
	$M_{bin} = np$  and $\sigma ^2 _{bin} = np(1-p)$
	
	For example, if the sample size is 12 and the probability of
	success is 0.25, the mean is $12 \times 0.25 = 3$ and the variance
	is $\sigma ^2 _{bin} = 12 \times 0.25 \times 0.75 = 2.25$.
	
\end{document}	
