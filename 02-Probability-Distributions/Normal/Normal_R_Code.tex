
%------------------------------------------%

\begin{frame}[fragile]
\frametitle{Probability Density Function}
Recall: As the Normal distribution is a continuous distribution, the PDF for a particular observed value will not give us an intuitive
result (as far as this module is concerned). It is, in fact, the height of the density curve at a particular point.

Nonetheless, the relevant \texttt{R} code may be included in exam questions, so as to add complexity to questions.

\begin{verbatim}
> dnorm(0.7)
[1] 0.3122539
>
> dnorm(1.7)
[1] 0.09404908

\end{verbatim}
\end{frame}
%------------------------------------------%
\begin{frame}[fragile]
\frametitle{Sample Question}
Suppose $X$ is a normally distributed random variable with mean $\mu = 2000$ and standard deviation $\sigma=200$.
Compute the probability of X being less than (or equal to) 2340.

\[P(X \leq 2340)\]

As always, we compute the z-score that corresponds to 2340.
\[ z_o = \frac{x_o - \mu}{\sigma}  = \frac{2340-2000}{200} = 1.7\]
\end{frame}
%------------------------------------------%

\begin{frame}[fragile]
\frametitle{\texttt{R} Implementation}


Using the following \texttt{R} code, we can determine $P(Z \leq 1.7)$.
\begin{verbatim}

> pnorm(1.7)
[1] 0.9554345

\end{verbatim}
\end{frame}

%------------------------------------------%

\begin{frame}[fragile]
\frametitle{Direct \texttt{R} Implementation}

This can easily be implemented directly - without using the standardization formula, by specifying the normal mean and normal standard deviation directly. However, we will not be using this approach in this module.
\begin{verbatim}

> pnorm(2340,mean=2000,sd=200)
[1] 0.9554345


\end{verbatim}
\end{frame}
\end{document}
