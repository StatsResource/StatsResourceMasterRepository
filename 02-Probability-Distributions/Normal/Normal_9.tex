
%------------------------------------------------------------------%
\frame{
\frametitle{ Characteristics of the Normal probability distribution}
\begin{itemize}
\item[1]  the mean, which is also the median and mode of the distribution.
\item[2] \alert{[VERY IMPORTANT]}
The normal probability curve is bell-shaped and symmetric, with the shape of the curve to the left of the mean a mirror image of the shape of the curve to the right of the mean.
\item[3] The standard deviation determines the width of the curve. Larger values of the the standard deviation result in wider flatter curves, showing more dispersion in data.
\item[4] The total area under the curve for the normal probability distribution is 1.
\end{itemize}
}

%------------------------------------------------%
\frame{
\frametitle{Normal Distribution}
The standard normal distribution is a normal distribution with a mean of 0 and a standard deviation of 1. 
Normal distributions can be transformed to standard normal distributions by the formula:
\[ Z = {X - \mu \over \sigma} \]
where X is a score from the original normal distribution, $\mu$ is the mean of the original normal distribution, and $\sigma$ is the standard deviation 
of original normal distribution. The standard normal distribution is sometimes called the Z distribution. 
A z score always reflects the number of standard deviations above or below the mean a particular score is. 
For instance, if a person scored a 68 on a test with a mean of 50 and a standard deviation of 9, then they scored 2 standard deviations above the mean. 
Converting the test scores to z scores, an X of 70 would be:
\[ Z = {68 - 50 \over 9} \]
So, a Z score of 2 means the original score was 2 standard deviations above the mean. Note that the z distribution will only be a normal distribution if the original distribution (X) is normal. 
 
}



