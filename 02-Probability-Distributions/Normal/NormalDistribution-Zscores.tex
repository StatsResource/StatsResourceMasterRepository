\documentclass[a4paper,12pt]{article}
%%%%%%%%%%%%%%%%%%%%%%%%%%%%%%%%%%%%%%%%%%%%%%%%%%%%%%%%%%%%%%%%%%%%%%%%%%%%%%%%%%%%%%%%%%%%%%%%%%%%%%%%%%%%%%%%%%%%%%%%%%%%%%%%%%%%%%%%%%%%%%%%%%%%%%%%%%%%%%%%%%%%%%%%%%%%%%%%%%%%%%%%%%%%%%%%%%%%%%%%%%%%%%%%%%%%%%%%%%%%%%%%%%%%%%%%%%%%%%%%%%%%%%%%%%%%
\usepackage{eurosym}
\usepackage{vmargin}
\usepackage{amsmath}
\usepackage{graphics}
\usepackage{epsfig}
\usepackage{subfigure}
\usepackage{framed}
\usepackage{enumerate}
\usepackage{fancyhdr}

\setcounter{MaxMatrixCols}{10}
%TCIDATA{OutputFilter=LATEX.DLL}
%TCIDATA{Version=5.00.0.2570}
%TCIDATA{<META NAME="SaveForMode"CONTENT="1">}
%TCIDATA{LastRevised=Wednesday, February 23, 201113:24:34}
%TCIDATA{<META NAME="GraphicsSave" CONTENT="32">}
%TCIDATA{Language=American English}

\pagestyle{fancy}
\setmarginsrb{20mm}{0mm}{20mm}{25mm}{12mm}{11mm}{0mm}{11mm}
\lhead{MS4222} \rhead{Kevin O'Brien} \chead{Normal Distribution} %\input{tcilatex}

\begin{document}


	\section*{Normal Distribution : Simple Example}
	
	
	Consider the normally distributed random variable $X$
	
	\[ X \sim \mathcal{N}(\mu=1000,\sigma^2 = 2500) \]
	
\noindent \textbf{Parameters:}
	\begin{itemize}
		\item Normal Mean $\mu =1000$
		\item Normal Standard Deviation $\sigma =50$
	\end{itemize}
\subsection*{Questions}
	
	\begin{enumerate}[(A)]
\item $P(X \geq 1025$
\item $P(X \geq 1075$
\item $P(X \leq 975)$
\item $P(X \leq 925)$
	\end{enumerate}
	%---------------------------------------------------%
	
\subsection*{Solution to Part A}
	
\noindent \textbf{Z-Score}	
\[ Z_{1025} = \frac{1025-1000}{50} = 0.50\]
Therefore we can say
\[ P(X \geq 1025)  = P(Z \geq 0.50) \]
	
\noindent From Murdoch Barnes Tables 3
	\[  P(Z \geq 0.5) = 0.3085
	\]
	
\noindent 	Therefore we can say 
	\[ P(X \geq 1025)  =  0.3085 \]
		%---------------------------------------------------%
	
\subsection*{Solution to Part B}
\noindent \textbf{Z-Score}	
\[ Z_{1075} = \frac{1075-1000}{50} = 1.50\]
Therefore we can say
\[ P(X \geq 1075)  = P(Z \geq 1.50) \]
	
\noindent From Murdoch Barnes Tables 3
	\[  P(Z \geq 1.5) =  0.0668
	\]
	
\noindent Therefore we can say 
	\[ P(X \geq 1075) = 0.0668 \]
	%---------------------------------------------------%
	
\subsection*{Solution to Part C }
\noindent \textbf{Z-Score}	
\[ Z_{975} = \frac{975-1000}{50} = -0.50\]	
Therefore we can say
	\[ P(X \leq 975)  = P(Z \leq -0.5) \]
	
\noindent Applying the Symmetry Rule
	\[ P(Z \leq -0.5) = P(Z \geq 0.5) = 0.3085\]
	
\noindent Therefore we can say 
	\[ P(X \leq 975) = 0.3085 \]
	
	%---------------------------------------------------%
\subsection*{Solution to Part D}
\[ Z_{925} = \frac{925-1000}{50} = -1.50\]
Therefore we can say
\[ P(X \leq 925)  = P(Z \leq -1.50) \]
	
\noindent Applying the Symmetry Rule
	\[ P(Z \leq -1.5) = P(Z \geq 1.5) = 0.0668\]
	
\noindent Therefore we can say 
	\[ P(X \leq 925) = 0.0668 \]
	
	%---------------------------------------------------%
\end{document}	
