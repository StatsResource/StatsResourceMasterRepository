\documentclass[12pt, a4paper]{report}
\usepackage{epsfig}
\usepackage{subfigure}
%\usepackage{amscd}
\usepackage{amssymb}
\usepackage{graphicx}
%\usepackage{amscd}

\usepackage{subfiles}
\usepackage{framed}
\usepackage{subfiles}
\usepackage{amsthm, amsmath}
\usepackage{amsbsy}
\usepackage{framed}
\usepackage[usenames]{color}
\usepackage{listings}
\lstset{% general command to set parameter(s)
	basicstyle=\small, % print whole listing small
	keywordstyle=\color{red}\itshape,
	% underlined bold black keywords
	commentstyle=\color{blue}, % white comments
	stringstyle=\ttfamily, % typewriter type for strings
	showstringspaces=false,
	numbers=left, numberstyle=\tiny, stepnumber=1, numbersep=5pt, %
	frame=shadowbox,
	rulesepcolor=\color{black},
	columns=fullflexible
} %
%\usepackage[dvips]{graphicx}
\usepackage{natbib}
\bibliographystyle{chicago}
\usepackage{vmargin}
% left top textwidth textheight headheight
% headsep footheight footskip
\setmargins{3.0cm}{2.5cm}{15.5 cm}{22cm}{0.5cm}{0cm}{1cm}{1cm}
\renewcommand{\baselinestretch}{1.5}
\pagenumbering{arabic}
%\theoremstyle{plain}
\newtheorem{theorem}{Theorem}[section]
\newtheorem{corollary}[theorem]{Corollary}
\newtheorem{ill}[theorem]{Example}
\newtheorem{lemma}[theorem]{Lemma}
\newtheorem{proposition}[theorem]{Proposition}
\newtheorem{conjecture}[theorem]{Conjecture}
\newtheorem{axiom}{Axiom}
\theoremstyle{definition}
\newtheorem{definition}{Definition}[section]
\newtheorem{notation}{Notation}
\theoremstyle{remark}
\newtheorem{remark}{Remark}[section]
\newtheorem{example}{Example}[section]
\renewcommand{\thenotation}{}
%\renewcommand{\thetable}{\thesection.\arabic{table}}
%\renewcommand{\thefigure}{\thesection.\arabic{figure}}

\author{ } \date{ }

\begin{document}

Normal Distribution 
Calculating Z scores
Three Rules 


%---------------------------------------------------------%
%- Section 4.3.1  
\section{The normal probability distribution}
The form, or shape, of the normal distribution is the bell shaped curve we met in section 2.
The probability density function that defines the bell-shaped curve if the normal distribution is given by

 
where 
\[ Formula\]
\begin{itemize}
\item $\mu$  = expected value, or mean, of the random variable x.
\item $\sigma$  = standard deviation of the random variable x.
\end{itemize}

[Remark - We will never be using this formula]

%---------------------------------------------------------%
 
\subsection{Characteristics of the Normal Probability Distribution}

\begin{enumerate}
\item The highest point on the normal curve is at the mean, which is also the median and mode of the distribution.


\item  [VERY IMPORTANT]
The normal probability curve is bell-shaped and symmetric, with the shape of the curve to the left of the mean a mirror image of the shape of the curve to the right of the mean.

\item  The standard deviation determines the width of the curve. Larger values of the the standard deviation result in wider flatter curves, showing more dispersion in data.


\item  The total area under the curve for the normal probability distribution is 1.

\item Useful Rules of Thumb

\begin{itemize}
\item	The mean  $pm$ 1 standard deviation includes 68% of the observations ,leaving 16\% (approx) in each tail.

\item	The mean  $pm$ 1.96 standard deviation includes 95% of the observations ,leaving 2.5\% (approx) in each tail.
   
\item	The mean  $pm$ 2.58 standard deviation includes 99% of the observations ,leaving 0.5\% (approx) in each tail.
\end{itemize}
\end{enumerate}
%---------------------------------------------------------%
Problem 1
Molly earned a score of 940 on a national achievement test. 
The mean test score was 850 with a standard deviation of 100. 
What proportion of students had a higher score than Molly? (Assume that test scores are normally distributed.)
(A) 0.10 
(B) 0.18 
(C) 0.50 
(D) 0.82 
(E) 0.90
Solution
The correct answer is B. As part of the solution to this problem, we assume that test scores are normally distributed. In this way, we use the normal distribution as a model for measurement. Given an assumption of normality, the solution involves three steps.
First, we transform Molly's test score into a z-score, using the z-score transformation equation. 

z = (X - μ) / σ = (940 - 850) / 100 = 0.90
Then, using an online calculator (e.g., Stat Trek's free normal distribution calculator), a handheld graphing calculator, or the standard normal distribution table, we find the cumulative probability associated with the z-score. In this case, we find P(Z < 0.90) = 0.8159.
Therefore, the P(Z > 0.90) = 1 - P(Z < 0.90) = 1 - 0.8159 = 0.1841.
Thus, we estimate that 18.41 percent of the students tested had a higher score than Molly.



Normal Distribution Example 2
Suppose scores on an IQ test are normally distributed. If the test has a mean of 100 and a standard deviation of 10, what is the probability that a person who takes the test will score between 90 and 110?

Solution: Here, we want to know the probability that the test score falls between 90 and 110. The "trick" to solving this problem is to realize the following:
P( 90 < X < 110 ) = P( X < 110 ) - P( X < 90 )
We use the Normal Distribution Calculator to compute both probabilities on the right side of the above equation.
To compute P( X < 110 ), we enter the following inputs into the calculator: The value of the normal random variable is 110, the mean is 100, and the standard deviation is 10. We find that P( X < 110 ) is 0.84.
To compute P( X < 90 ), we enter the following inputs into the calculator: The value of the normal random variable is 90, the mean is 100, and the standard deviation is 10. We find that P( X < 90 ) is 0.16.

We use these findings to compute our final answer as follows:
P( 90 < X < 110 ) = P( X < 110 ) - P( X < 90 )
P( 90 < X < 110 ) = 0.84 - 0.16
P( 90 < X < 110 ) = 0.68
Thus, about 68% of the test scores will fall between 90 and 110.

 
\end{document}