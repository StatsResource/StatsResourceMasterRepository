\documentclass[a4paper,12pt]{article}
%%%%%%%%%%%%%%%%%%%%%%%%%%%%%%%%%%%%%%%%%%%%%%%%%%%%%%%%%%%%%%%%%%%%%%%%%%%%%%%%%%%%%%%%%%%%%%%%%%%%%%%%%%%%%%%%%%%%%%%%%%%%%%%%%%%%%%%%%%%%%%%%%%%%%%%%%%%%%%%%%%%%%%%%%%%%%%%%%%%%%%%%%%%%%%%%%%%%%%%%%%%%%%%%%%%%%%%%%%%%%%%%%%%%%%%%%%%%%%%%%%%%%%%%%%%%
\usepackage{eurosym}
\usepackage{vmargin}
\usepackage{amsmath}
\usepackage{graphics}
\usepackage{epsfig}
\usepackage{subfigure}
\usepackage{enumerate}
\usepackage{fancyhdr}

\setcounter{MaxMatrixCols}{10}
%TCIDATA{OutputFilter=LATEX.DLL}
%TCIDATA{Version=5.00.0.2570}
%TCIDATA{<META NAME="SaveForMode"CONTENT="1">}
%TCIDATA{LastRevised=Wednesday, February 23, 201113:24:34}
%TCIDATA{<META NAME="GraphicsSave" CONTENT="32">}
%TCIDATA{Language=American English}

\pagestyle{fancy}
\setmarginsrb{20mm}{0mm}{20mm}{25mm}{12mm}{11mm}{0mm}{11mm}
\lhead{MS4222} \rhead{Kevin O'Brien} \chead{Normal Distribution} %\input{tcilatex}

\begin{document}

\section*{The Standard Normal ($Z$) Distribution}

\begin{itemize}
\item Since every different combination of $\mu$ and $\sigma$ would generate a different normal
probability distribution, tables of normal probabilities are based on one
particular distribution: the standard normal distribution.  
	\item The standard normal distribution is a special case of the normal distribution with a mean $\mu= 0$ and a standard deviation $\sigma =1$.
	\item We denote the standard normal random variable as $Z$ rather than $X$. We denote it mathematically
	\[ Z \sim N(\mu=0,\sigma^2 = 1)\]
	
	\item The distribution is well described in statistical tables (i.e. Murdoch Barnes Table 3).
	\item Rather than computing probabilities from first principles, which is very difficult, probabilities from distributions other than the Z distribution, e.g. X $\sim N(\mu=100, \sigma^2 =15^2$), can be computed using the Z distribution, a much easier approach. (We shall demonstrate how shortly.)
\end{itemize}



%===================================================================================%
\subsection*{Standardization Formula}
\begin{itemize}
\item All normally distributed random variables have corresponding $Z$ values, called \textbf{\emph{Z-scores}}.
\item 
For normally distributed random variables, the Z-score can be found using the \textbf{\emph{standardization formula}};
\[
z_{0} = {x_0 - \mu \over \sigma}
\]
where $x_o$ is a score from the underlying normal (``X") distribution, $\mu$ is the mean of the original normal distribution, and $\sigma$ is the standard deviation of original normal distribution.
\item 
Therefore $z_o$ is the z-score that corresponds to $x_o$.
% \item 
% Any value X from a normally distributed population % can be converted into the equivalent standard 
% normal value Z (i.e. a `Z value') by the formula
% \[ Z = \frac{X - \mu}{\sigma}\]
\end{itemize}
%==================================================================================%
\subsection*{An Important Identity}
\begin{itemize}
\item If two values $z_o$ and $x_o$ are related in the following way, for some values $\mu$ and $\sigma$,
\[
z_{0} = {x_0 - \mu \over \sigma}
\]
\item Then we can can say

\[ P(X \geq x_o) = P(Z \geq z_o) \]

or alternatively

\[ P(X \leq x_o) = P(Z \leq z_o) \]

\item This is fundamental to solving problems involving normal distributions.
\end{itemize}


\noindent \textbf{Remarks}
\begin{itemize}
\item For extra clarity : Terms with subscripts mean particular values, and are not variable names. Variable names are denoted with capital letters.
\item A computed Z-score is a normally distributed random variable only if the underlying distribution (X) is normally distributed. If the underlying distribution is not normal, then using Z-scores is not a valid approach.
\end{itemize}
%=========================================================================================== %
\subsubsection*{Example 1}
\begin{itemize}
\item Suppose that mean $\mu = 105 $ and that standard deviation $\sigma = 8$.
\item What is the Z-score for $x_o = 117$?
\[
z_{117} = {x_o - \mu \over \sigma} = {117 - 105 \over 8} = {12 \over 8} = 1.5
\]
\item Therefore $z_{117} = 1.50$
\item Remark: $P(X \geq 117) = P(Z \geq 1.50)$.
\end{itemize}

\subsubsection*{Example 2}

\begin{itemize}
\item Suppose that mean $\mu = 80 $ and that standard deviation $\sigma = 8$.
\item What is the Z-score for $x_o = 100$?
\[
z_{100} = {x_0 - \mu \over \sigma} = {100 - 80 \over 8} = {20 \over 8} = 2.50
\]
\item Therefore the Z score is : $z_{100} = 2.50$
\end{itemize}
\end{document}
