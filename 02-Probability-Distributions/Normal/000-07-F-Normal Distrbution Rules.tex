\documentclass[]{report}

\voffset=-1.5cm
\oddsidemargin=0.0cm
\textwidth = 480pt

\usepackage{framed}
\usepackage{subfiles}
\usepackage{graphics}
\usepackage{newlfont}
\usepackage{eurosym}
\usepackage{amsmath,amsthm,amsfonts}
\usepackage{amsmath}
\usepackage{color}
\usepackage{amssymb}
\usepackage{multicol}
\usepackage[dvipsnames]{xcolor}
\usepackage{graphicx}
\begin{document}



\section{Important rules for Normal distribution}



\subsection{Example }
Find the probability of a "z" random variable greater than (or equal to) -1.8?
Find  $P(Z \geq -1.8)$
\noindent \t{Solution }

(From a previous question, $P(Z \leq -1.8) = 0.0359$)

\[P(Z \geq -1.8) = 1 - P(Z \leq -1.8) \]
\[P(Z \geq -1.8)= 1 - 0.0359 = 0.9641\]


\section{Complement and Symmetry Rules}

For a normally distributed random variable with mean $\mu = 1000$ and standard deviation $\sigma = 100$, compute $P(X \geq 873)$.

\begin{itemize} \item First, find the Z-value using the standardization formula.
\[
z_{873} = {x_o - \mu \over \sigma} = {873 - 1000 \over 100} = {-127 \over 100} = -1.27
\]
\item We can say $P(X \geq 873) = P(Z \geq -1.27)$.
\item Use complement rule and symmetry rule to evaluate  $P(Z \geq -1.27)$.
\item $ P(Z \geq -1.27) = P(Z \leq 1.27) = 1 - P(Z \geq 1.27) $  = 1 - 0.1020 = {0.8980}.
\end{itemize}


%------------------------------------------------------------------%


\section{Summary}

\begin{itemize}

\item \textbf{Complement Rule}\\
For some value $A$, and for any continuous distribution $X$ (including any normal distribution and the Z distribution) we can say.
\[ P(X \leq a) = 1 - P(X \geq A) \]

\item \textbf{Symmetry Rule}\\
For the standard normal ($Z$) distribution only, we can say
\[ P(Z \leq -A) = P(Z \geq A) \]

or conversely

\[ P(Z \geq -A) = P(Z \leq A) \]

\end{itemize}




%===============================================%
\section{Normal Distributions}
\begin{equation}
Z_{o}= \frac{ X_{o}-\mu }{\sigma}
\end{equation}

\begin{equation}
P(Z \geq Z_{o}) = P(X \geq X_{o})
\end{equation}

%------------------------------------------------------------------------%
{
\subsection{Using Murdoch Barnes tables 3}
\begin{itemize}
\item $ P(Z \geq 1.64) = 0.505$
\item $ P(Z \geq 1.65) = 0.495$ \bigskip
\item $ P(Z \geq 1.645)$ is approximately the average value of $ P(Z \geq 1.64)$ and $ P(Z \geq 1.65)$.
\item $ P(Z \geq 1.645)$ = (0.0495 + 0.0505)/2 = 0.0500. ( i.e. $5\%$ )
\end{itemize}
}






%============================================================% 

\subsection{Solving using the Z distribution}
When we have a normal distribution with any mean $\mu$ and any standard deviation $\sigma$ , we answer probability questions about the distribution by first converting all values to corresponding values of the standard normal ("z") distribution.
The formula used to convert any random variable "X" ( with mean $\mu$ and standard deviation $\sigma$ specified) to the standard normal ("z") distribution is given as follows.
\[ Z_o = {X_o - \mu \over \sigma} \]
$Z$ is the standard normal random variable with a mean of zero and a standard deviation of 1.
It can be thought of as a measure of how many standard deviations that a value "x" is from mean $\mu$ .


\noindent \textbf{Remarks}

\begin{itemize}
\item A value of x equal to mean $\mu$  results in a z -value of 0

\[ z = \frac{\mu - \mu}{\sigma} = \frac{0}{\sigma} = 0\]


\item Thus we can see that a value of "x" corresponding to its mean $\mu$ corresponds to a z-value at its mean , which is 0.

\item A value of "x" that is one standard deviation above its mean (i.e. $x=\mu +\sigma$  ), we see that the corresponding z value is 1.


\item Thus a value of x that is one standard deviation away from it's mean yields a z-value of 1.
\end{itemize}







%=======================================================================%

Solution 



Example 
Find the probability of a "z" random variable greater than (or equal to) -1.8?
Find  

Solution





\section{Normal Distribution : Solving problems}
Recap:
\begin{itemize}
\item We must know the normal mean $\mu$ and the normal standard deviation $\sigma$.
\item The normal random variable is $X \sim \mbox{N} ( \mu , \sigma^2)$.\smallskip
\item (If we don't, we usually have to determine them, given the information in the question.)\smallskip
\item The standard normal random variable is $Z\sim \mbox{N} ( 0 , 1^2)$.\smallskip
\item The standard normal distribution is well described in Murdoch Barnes Table 3, which tabulates $P(Z \geq z_o)$ for a range of $Z$ values.
\end{itemize}

%-----------------------------------------------------%


\begin{itemize}
\item For the given value $x_o$ from the variable $X$, we compute the corresponding z-score $z_o$.
\[ z_o = { x_o - \mu \over \sigma} \]
\item When $z_o$ corresponds to $x_o$, the following identity applies:
\[  P(X \geq x_o )= P(Z \geq z_o ) \]
\item Alternatively $ P(X \leq x_o )= P(Z \leq z_o ) $
\end{itemize}




\end{document}



