\documentclass[a4paper,12pt]{article}
%%%%%%%%%%%%%%%%%%%%%%%%%%%%%%%%%%%%%%%%%%%%%%%%%%%%%%%%%%%%%%%%%%%%%%%%%%%%%%%%%%%%%%%%%%%%%%%%%%%%%%%%%%%%%%%%%%%%%%%%%%%%%%%%%%%%%%%%%%%%%%%%%%%%%%%%%%%%%%%%%%%%%%%%%%%%%%%%%%%%%%%%%%%%%%%%%%%%%%%%%%%%%%%%%%%%%%%%%%%%%%%%%%%%%%%%%%%%%%%%%%%%%%%%%%%%
\usepackage{eurosym}
\usepackage{vmargin}
\usepackage{amsmath}
\usepackage{graphics}
\usepackage{framed}
\usepackage{epsfig}
\usepackage{enumerate}
\usepackage{subfigure}
\usepackage{fancyhdr}

\setcounter{MaxMatrixCols}{10}
%TCIDATA{OutputFilter=LATEX.DLL}
%TCIDATA{Version=5.00.0.2570}
%TCIDATA{<META NAME="SaveForMode"CONTENT="1">}
%TCIDATA{LastRevised=Wednesday, February 23, 201113:24:34}
%TCIDATA{<META NAME="GraphicsSave" CONTENT="32">}
%TCIDATA{Language=American English}

\pagestyle{fancy}
\setmarginsrb{20mm}{0mm}{20mm}{25mm}{12mm}{11mm}{0mm}{11mm}
\lhead{StatsResource} \rhead{Dixon Q Test for Outliers} \chead{Hypothesis Testing} %\input{tcilatex}

\begin{document}
\large 

\noindent Use the Dixon Q-test to determine if there is an outlier present in this sample data. You may assume
a significance level of 5\%.
\[ 131, 139, 107, 117, 123, 127, 122, 132, 135\]
\begin{enumerate}[(a)]
\item  State the null and alternative hypotheses for this test.
\item  Compute the test statistic?
\item  State the appropriate critical value.
\item  What is your conclusion to this procedure? (Use a 5\% significance level)
\end{enumerate}


%%%%%%%%%%%%%%%%%

\subsection*{Solution}
Arrange the data set into \textbf{\textit{ascending}} order. Determine which value is the potential outlier.

\[\LARGE \{ 107, 117, 122, 123, 127, 131, 132, 135, 139\} \]

\noindent Here the potential outlier is the lowest value, i.e. 107

\subsection*{Hypotheses}

We can formally state the null and alternative hypothesis as follows

\begin{description}
\item[H$_0$] There are no outliers present in the data.
\item[H$_1$] There is one outlier present (\textit{i.e. the lowest value 107})
\end{description}

\subsection*{Test Statistic}

The test statistic for this procedure is as follows:

\[ Q_{TS} =  \frac{\mbox{Gap}}{\mbox{Range}} \]

\noindent The gap is the difference of the outlier from the next value. In this case , the next value is 117, so the gap is 
\[ \mbox{Gap} = 117 - 107 = 10\]
The range is simply the difference between the maximum and minimum value.
\[ \mbox{Range} =  139-107 =32\]

\[ Q_{TS} =  \frac{\mbox{Gap}}{\mbox{Range}} = \frac{10}{32} = 0.3125 \]

\subsection*{Critical Values}

\begin{itemize}
    \item Before we look at the critical value, we confirm the size of the data set is $n=9$.

\item The critical value can be determined from the following table. 

\item  The column to chose is the significance level (here 5\% or 0.05 ). The row to use is $n$, the number of items in the data set.

\item The Critical Value $Q_{CV} = 0.493$
\end{itemize}


%%-----------%%
\begin{center}
\begin{tabular}{|c|c|c|c|}
\hline
$n$	&	$\alpha=0.10$	&	$\alpha=0.05$	&	$\alpha=0.01$	\\ \hline
3	&	0.941	&	0.970	&	0.994	\\ \hline
4	&	0.765	&	0.829	&	0.926	\\ \hline
5	&	0.642	&	0.710	&	0.821	\\ \hline
6	&	0.560	&	0.625	&	0.740	\\ \hline
7	&	0.507	&	0.568	&	0.680	\\ \hline
8	&	0.468	&	0.526	&	0.634	\\ \hline
9	&	0.437	&	0.493	&	0.598	\\ \hline
10	&	0.412	&	0.466	&	0.568	\\ \hline
11	&	0.392	&	0.444	&	0.542	\\ \hline
12	&	0.376	&	0.426	&	0.522	\\ \hline
13	&	0.361	&	0.410	&	0.503	\\ \hline
14	&	0.349	&	0.396	&	0.488	\\ \hline
15	&	0.338	&	0.384	&	0.475	\\ \hline

\end{tabular} 
\end{center}

%%%%%%%%%%%%%%%%%%%%%%%
\subsection*{Decision Rule}


\begin{itemize}
    \item 
 If the Test Statistic is greater than the Critical value, reject the null hypothesis
\[ Q_{TS} > Q_{CV}\]

\item  Otherwise we fail to reject the null hypothesis.
\end{itemize}
\subsection*{Conclusion}

\begin{itemize}
\item Here the Test Statistic ($0.3125$) is not greater than the Critical Value. Therefore we fail to reject the Null Hypothesis. 

\item At a 5\% significance level, there is no justification to treat the minimum value as an outlier.
\end{itemize}


\end{document}
