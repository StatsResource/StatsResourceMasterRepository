# Continuous Uniform Distribution: Subway Example
Suppose there is a platform in a subway station in a large large city.  
Subway trains arrive **every three minutes** at this platform.

1. What is the shortest possible time a passenger would have to wait for a train?
2. What is the longest possible time a passenger will have to wait?
3. What is the probability that they will have to wait longer than 2 minutes?
4. What is the probability that they will have to wait less than 45 seconds (i.e. 0.75 minutes)?
5. Compute the expected value and variance of the waiting time $X$.

## Solution to Part A

- **What is the shortest possible time a passenger would have to wait for a train?**
    - If the passenger arrives just before the doors close, then the waiting time is zero.
    \[ a = 0 \text{ minutes } \]

## Solution to Part B
- **What is the longest possible time a passenger will have to wait?**
    - If the passenger arrives just after the doors close, and misses the train, then they will have to wait the full three minutes for the next one.
    \[ b = 3 \text{ minutes }  = 180 \text{ seconds}  \]

## Solution to Part C
- **What is the probability that they will have to wait longer than 2 minutes?**
    \[ P(X \geq 2)  = \frac{3-2}{3-0} = \frac{1}{3} = 0.3333 \]

```{r, echo=FALSE}
knitr::include_graphics("images/6AUniform4.png")
```

## Solution to Part D
- **What is the probability that they will have to wait less than 45 seconds (i.e. 0.75 minutes)?**
    \[ P(X \leq 0.75)  = \frac{0.75 - 0}{3-0} = \frac{0.75}{3} = 0.250 \]

```{r, echo=FALSE}
knitr::include_graphics("images/6AUniform3.png")
```

## Solution to Part E
We are told that, for waiting times, the lower limit $a$ is 0, and the upper limit $b$ is 3 minutes. 

**The expected waiting time** $\textrm{E}[X]$ is computed as follows:
    \[
    \textrm{E}[X] = \frac{b + a}{2} = \frac{3 + 0}{2} = 1.5 \text{ minutes }
    \]

**The variance of the continuous uniform distribution**, denoted $\textrm{Var}(X)$, is computed using the following formula:
    \[
    \textrm{Var}(X) = \frac{(b - a)^2}{12} = \frac{(3 - 0)^2}{12} = \frac{3^2}{12} = \frac{9}{12} = 0.75
    \]
```

# Define the parameters
a <- 0
b <- 3

# Create a sequence of values from a to b
x <- seq(a, b, length.out = 1000)

# Define the density function for the uniform distribution
density <- dunif(x, min = a, max = b)

# Plot the density function
plot(x, density, type = "l", main = "Density Function of Uniform Distribution (a = 0, b = 3)",
     xlab = "x", ylab = "Density", col = "blue", lwd = 2)

# Add shading for the segment between 0 and 2
polygon(c(0, seq(0, 2, length.out = 100), 2), c(0, dunif(seq(0, 2, length.out = 100), min = a, max = b), 0), 
        col = rgb(0, 0, 1, 0.5))

