

\documentclass[a4paper,12pt]{article}
%%%%%%%%%%%%%%%%%%%%%%%%%%%%%%%%%%%%%%%%%%%%%%%%%%%%%%%%%%%%%%%%%%%%%%%%%%%%%%%%%%%%%%%%%%%%%%%%%%%%%%%%%%%%%%%%%%%%%%%%%%%%%%%%%%%%%%%%%%%%%%%%%%%%%%%%%%%%%%%%%%%%%%%%%%%%%%%%%%%%%%%%%%%%%%%%%%%%%%%%%%%%%%%%%%%%%%%%%%%%%%%%%%%%%%%%%%%%%%%%%%%%%%%%%%%%
\usepackage{eurosym}
\usepackage{vmargin}
\usepackage{amsmath}
\usepackage{framed}
\usepackage{graphics}
\usepackage{epsfig}
\usepackage{subfigure}
\usepackage{enumerate}
\usepackage{fancyhdr}

\setcounter{MaxMatrixCols}{10}
%TCIDATA{OutputFilter=LATEX.DLL}
%TCIDATA{Version=5.00.0.2570}
%TCIDATA{<META NAME="SaveForMode"CONTENT="1">}
%TCIDATA{LastRevised=Wednesday, February 23, 201113:24:34}
%TCIDATA{<META NAME="GraphicsSave" CONTENT="32">}
%TCIDATA{Language=American English}

\pagestyle{fancy}
\setmarginsrb{20mm}{0mm}{20mm}{25mm}{12mm}{11mm}{0mm}{11mm}
\lhead{StatsResource} \chead{Hypothesis Testing} \rhead{Statistical Inference} %\input{tcilatex}
\begin{document}

%--------------------------------------------------------------------------------------%
\large 
\subsection*{Worked Example}
Two sets of patients are given courses of treatment under two different drugs. The benefits
derived from each drug can be stated numerically in terms of the recovery times. There are 40 patients in treatment group 1 (i.e. Drug 1), and 45 patients in treatment group 2 (i.e. Drug 2). \\
\medskip
The mean recovery time and standard deviations are as follows:
\begin{itemize}
\item Drug 1:  $n_1$ = 40 , $\bar{x}_1$ = 3.3 days and $s_1 = 1.524$
\item Drug 2:  $n_2$ = 45 , $\bar{x}_2$ = 4.3 days and $s_2 = 1.951 $
\end{itemize}
\medskip

%-------------------------------------------------------------------------------------------%

\subsection*{Writing Hypotheses }
\begin{itemize}
\item The first step in hypothesis testing is to specify the null hypothesis and an alternative hypothesis.
\item When testing differences between mean recovery times, the null hypothesis is that the two population means are equal.
\item That is, the null hypothesis and alternative hypothesis are:\\
$H_0: \mu_1 = \mu_2$ \qquad (\textit{The population means are equal})\\
$H_1: \mu_1 \neq \mu_2$  \qquad (\textit{The population means are different})
\item We can re-express these hypotheses in terms of a numeric difference\\
$H_0: \mu_1 - \mu_2 = 0$ \qquad (\textit{The difference in population means is zero})\\
$H_1: \mu_1  - \mu_2  \neq = 0 $  \qquad (\textit{The difference in population means is zero})
\end{itemize}



(Remark: Two Tailed Test, therefore $k = 2$, and $\alpha = 0.05$)
\medskip

%-------------------------------------------------------------------------------------------%

\subsection*{Standard Error}
The relevant formula for the standard error is
\begin{eqnarray*}
S.E(\bar{x}_1 - \bar{x}_2) &=& \sqrt{{s^2_1\over n_1}+{s^2_2 \over n_2}} \\
& & \\
&=& \sqrt{{(1.524)^2 \over 40}+{(1.951)^2 \over 45}}   \\
& & \\
S.E(\bar{x}_1 - \bar{x}_2)
&=& 0.377\mbox{ days}\\
\end{eqnarray*}
\medskip
%-------------------------------------------------------------------------------------------%
\subsection*{Test Statistic}
\begin{itemize}
\item The observed difference in means is 1 day.
\item The Test statistic (the TS) is therefore
\[ TS = {\mbox{observed} - \mbox{null} \over \mbox{Std. Error}}  = {1 - 0 \over 0.377 } = 2.65 \]

\end{itemize}

\medskip 

\subsection*{Critical}
\begin{itemize}
\item Two Tailed Test, therefore $k = 2$, and $\alpha = 0.05$. Also two large samples. 
\item The Critical Value (the CV )is 1.96.
\end{itemize}
\medskip


\subsection*{Decision Rule}

\begin{framed}
\[ |TS| > CV ?  \]
\begin{itemize}
\item If Yes: Reject the null Hypothesis
\item If No : Fail to reject the Null Hypothesis
\end{itemize}
\end{framed}

%-------------------------------------------------------------------------------------------%
\medskip 
\begin{itemize}

\item Is the TS greater than the CV? Is $2.65 > 1.96$?

\item \textbf{Conclusion:} we reject the null hypothesis. There is a significant different between both drugs, in terms of recovery times.

\end{itemize}
\medskip 

\newpage 
%-------------------------------------------------------------------------------------------%
\medskip 
\begin{itemize}
\item We will approach the same problem, but this time using the p-value approach.
\item The first two steps i.e. formally stating the null and alternative hypothesis, and computing the test statistic are the same, are the same as example 1.
\item The third step is to compute the p-value:  $P(Z \geq |TS|)$.
\item From Murdoch Barnes table 3: $P(Z \geq 2.65) = 0.00402$ (i.e. less than half a percent).

\end{itemize}
\medskip 

%-------------------------------------------------------------------------------------------%
\medskip 
\begin{itemize}
\item The p-value is  $0.00402$
\item The critical region has size $\alpha/k = 0.05/2 = 0.0250$.
\item We reject the null hypothesis because the computed p-value is less than $0.0250$.
\item \textbf{Conclusion:} we reject the null hypothesis. There is a significant different between both drugs, in terms of recovery times.
\end{itemize}
\end{document}


