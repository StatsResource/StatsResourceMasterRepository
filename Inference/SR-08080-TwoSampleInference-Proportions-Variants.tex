

\documentclass[a4paper,12pt]{article}
%%%%%%%%%%%%%%%%%%%%%%%%%%%%%%%%%%%%%%%%%%%%%%%%%%%%%%%%%%%%%%%%%%%%%%%%%%%%%%%%%%%%%%%%%%%%%%%%%%%%%%%%%%%%%%%%%%%%%%%%%%%%%%%%%%%%%%%%%%%%%%%%%%%%%%%%%%%%%%%%%%%%%%%%%%%%%%%%%%%%%%%%%%%%%%%%%%%%%%%%%%%%%%%%%%%%%%%%%%%%%%%%%%%%%%%%%%%%%%%%%%%%%%%%%%%%
\usepackage{eurosym}
\usepackage{vmargin}
\usepackage{amsmath}
\usepackage{framed}
\usepackage{graphics}
\usepackage{epsfig}
\usepackage{subfigure}
\usepackage{enumerate}
\usepackage{fancyhdr}

\setcounter{MaxMatrixCols}{10}
%TCIDATA{OutputFilter=LATEX.DLL}
%TCIDATA{Version=5.00.0.2570}
%TCIDATA{<META NAME="SaveForMode"CONTENT="1">}
%TCIDATA{LastRevised=Wednesday, February 23, 201113:24:34}
%TCIDATA{<META NAME="GraphicsSave" CONTENT="32">}
%TCIDATA{Language=American English}

\pagestyle{fancy}
\setmarginsrb{20mm}{0mm}{20mm}{25mm}{12mm}{11mm}{0mm}{11mm}
\lhead{StatsResource} \chead{Probability Distributions} \rhead{Joint Distribution} %\input{tcilatex}
\begin{document}

%----------------------------------------------------------------------------------------------------%
\section*{Inferences around two proportions}


\subsection*{Assumptions}

\begin{itemize}
\item We have proportions from two independent simple random samples
\item For both samples the conditions $np \geq 5$ $ n(1-p) \geq 5$ are met.
\end{itemize}
For population $1$, let
\begin{itemize}
\item $p_1$  population proportion
\item $n_1$ sample size
\item $x_1$ number of successes in sample 1
\item $hat{p}_1$ is the sample proportion, an estimate for $p_1$.
\end{itemize}

\bigskip
%-------------------------------------------------------------------------------------------%


\section*{Worked Example}
\begin{itemize}
\item An experiment is conducted investigating the long-term effects of early childhood intervention programs (such as head start).
\item In one experiment, the high-school drop out rate of the experimental group (which attended the early childhood program)
 and the control group (which did not) were compared.
\item In the experimental group, 73 of 85 students graduated from high school. \item In the control group, only 43 of 82 students graduated.
Is this difference statistically significant? (Assume that the 0.05 level is chosen.) \end{itemize}


%-------------------------------------------------------------------------------------------%

\subsection*{Hypotheses}
\begin{itemize}
\item
The first step in hypothesis testing is to specify the null hypothesis and an alternative hypothesis.
\item When testing differences between proportions, the null hypothesis is that the two population proportions are equal.
\item That is, the null hypothesis is:\\
$H_0: \pi_1 = \pi_2$\\
$H_1: \pi_1 \neq \pi_2$\\
\end{itemize}
(Remark: Two Tailed Test k = 2, and $\alpha = 0.05$)

%-------------------------------------------------------------------------------------------%

\subsection*{Point Estimate}
\begin{itemize}
\item The next step is to compute the difference between the sample proportions.
\item In this example, $\hat{p}_1 - \hat{p}_2$ = $73/85 - 43/82$ = $0.8588 - 0.5244$.
\item $\hat{p}_1 - \hat{p}_2$ = $0.8588 - 0.5244$ = 0.3344.
\item Difference is $33.44\%$
\end{itemize}




%-------------------------------------------------------------------------------------------%

\subsection*{Standard Error}
The formula for the estimated standard error is:

\[ S.E (\hat{p}_1 - \hat{p}_2)  = \sqrt{\bar{p}(100- \bar{p} \left( {1 \over n_1} + {1 \over n_2}  \right)} \]


where $\bar{p}$ is a aggregate proportion ( proportion of successes from overall sample, regardless of which group they are in).


%-------------------------------------------------------------------------------------------%





\subsection*{Aggregate Proportion}:\\
\[ \bar{p}  = {x_1  + x_2 \over n_1 + n_2} \times 100\% = {73+43 \over 85 + 82} \times 100\% = { 116 \over 167}\times 100\% = 69.5\% \]
\textbf{Standard Error}:\\
\[ S.E (\hat{p}_1 - \hat{p}_2)  =  \sqrt{69.5 \times 30.5 \left( {1 \over 85} + {1 \over 82}  \right)}  = 7.13\% \]




%-------------------------------------------------------------------------------------------%

\subsection*{Test Statistic}:
\begin{itemize} \item Observed difference :
85.88\% - 52.44\%  = 33.44\% \item [ i.e (73/85) - (43 /82) ]
\item Under the null hypothesis, the expected difference is zero.
\item Test Statistic is therefore \[T.S. = {33.44\% \over 7.13\%} = 4.69\]
\end{itemize}


%-------------------------------------------------------------------------------------------%

\subsection*{Critical Value}
\begin{itemize}
\item The Critical value is 1.96 (Large sample , $\alpha = 0.05$, k=2).

\item The test statistic $TS = 4.69$, is greater than the critical value $CV = 1.96$, so we reject the null hypothesis.
\item The conclusion is that the probability of graduating from high school is greater for students who have participated in the early childhood intervention program than for students who have not.
\end{itemize}








\end{document}

