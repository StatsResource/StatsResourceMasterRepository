

\documentclass[a4paper,12pt]{article}
%%%%%%%%%%%%%%%%%%%%%%%%%%%%%%%%%%%%%%%%%%%%%%%%%%%%%%%%%%%%%%%%%%%%%%%%%%%%%%%%%%%%%%%%%%%%%%%%%%%%%%%%%%%%%%%%%%%%%%%%%%%%%%%%%%%%%%%%%%%%%%%%%%%%%%%%%%%%%%%%%%%%%%%%%%%%%%%%%%%%%%%%%%%%%%%%%%%%%%%%%%%%%%%%%%%%%%%%%%%%%%%%%%%%%%%%%%%%%%%%%%%%%%%%%%%%
\usepackage{eurosym}
\usepackage{vmargin}
\usepackage{amsmath}
\usepackage{framed}
\usepackage{graphics}
\usepackage{epsfig}
\usepackage{subfigure}
\usepackage{enumerate}
\usepackage{fancyhdr}

\setcounter{MaxMatrixCols}{10}
%TCIDATA{OutputFilter=LATEX.DLL}
%TCIDATA{Version=5.00.0.2570}
%TCIDATA{<META NAME="SaveForMode"CONTENT="1">}
%TCIDATA{LastRevised=Wednesday, February 23, 201113:24:34}
%TCIDATA{<META NAME="GraphicsSave" CONTENT="32">}
%TCIDATA{Language=American English}

\pagestyle{fancy}
\setmarginsrb{20mm}{0mm}{20mm}{25mm}{12mm}{11mm}{0mm}{11mm}
\lhead{MathsResource} \chead{Probability Distributions} \rhead{Joint Distribution} %\input{tcilatex}
\begin{document}

%----------------------------------------------------------------------% 
% 2000 - Q6
 
The connectors for mobile phones must have a standard deviation of 2mms or less.  
A major mobile company takes a random sample from one of its suppliers as follows.

\[ 
34.2, 33.7, 31.9, 34.3, 31.6, 32.7, 34.1, 35.2, 31.6, 32.9, 33.0, 32.4.
\] 
Based on the data from the sample you are required to 
 

\begin{itemize} 
\item[(i)]   Construct a 95\% confidence interval for the population variance.
\item[(ii)]  At the 5\% level of significance is there evidence to support the supplier’s claim that its products are within specification.
\item[(iii)] Using the sample standard deviation as an estimate of the population standard deviation determine the sample size necessary to estimate the mean length of connectors, assuming that a 99\% confidence is required with a margin of error of 0.5 mm
\end{itemize}


\subsection*{Computing Confidence Intervals}
Confidence limits are the lower and upper boundaries / values of a confidence interval, that is, the values which define the range of a confidence interval. The general structure of a confidence interval is as follows:

\[ \mbox{Point Estimate}  \pm \left[ \mbox{Quantile} \times \mbox{Standard Error} \right] \]


\begin{itemize}
\item Point Estimate: estimate for population parameter of interest, i.e. sample mean, sample proportion.
\item Quantile: a value from a probability distribution that scales the intervals according to the specified confidence level.
\item Standard Error: measures the dispersion of the sampling distribution for a given sample size $n$.
\end{itemize}
\end{document}
