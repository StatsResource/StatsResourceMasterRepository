	\documentclass[a4paper,12pt]{article}
%%%%%%%%%%%%%%%%%%%%%%%%%%%%%%%%%%%%%%%%%%%%%%%%%%%%%%%%%%%%%%%%%%%%%%%%%%%%%%%%%%%%%%%%%%%%%%%%%%%%%%%%%%%%%%%%%%%%%%%%%%%%%%%%%%%%%%%%%%%%%%%%%%%%%%%%%%%%%%%%%%%%%%%%%%%%%%%%%%%%%%%%%%%%%%%%%%%%%%%%%%%%%%%%%%%%%%%%%%%%%%%%%%%%%%%%%%%%%%%%%%%%%%%%%%%%
\usepackage{eurosym}
\usepackage{vmargin}
\usepackage{framed}
\usepackage{amsmath}
\usepackage{graphics}
\usepackage{epsfig}
\usepackage{subfigure}
\usepackage{enumerate}
\usepackage{fancyhdr}

\setcounter{MaxMatrixCols}{10}
%TCIDATA{OutputFilter=LATEX.DLL}
%TCIDATA{Version=5.00.0.2570}
%TCIDATA{<META NAME="SaveForMode"CONTENT="1">}
%TCIDATA{LastRevised=Wednesday, February 23, 201113:24:34}
%TCIDATA{<META NAME="GraphicsSave" CONTENT="32">}
%TCIDATA{Language=American English}

\pagestyle{fancy}
\setmarginsrb{20mm}{0mm}{20mm}{25mm}{12mm}{11mm}{0mm}{11mm}
\lhead{StatsResource} \rhead{Inference Procedures} \chead{Confidence Intervals} %\input{tcilatex}

\begin{document}
%---------------------------------------------------------%

% HT - Difference in Means	
	
%%%%%%%%%%%%%%%%%%%%%%%%%%%%%%%%%%%%%%%%%%%%%%%%%%%%	



In a computer hardware manufacturing plant, machine X and machine Y produce identical components. The management investigate whether or not there is a difference in the mean diameter of the components from both machines.

\begin{itemize}
\item[$\bullet$] A random sample of 144 components from machine X had a mean of 36.38 mm and a standard deviation of 3.0 mm.
\item[$\bullet$] A random sample of 225 components from machine Y had a mean of 36.88 mm and a standard deviation of 2.8 mm.
\end{itemize}
A hypothesis test was used to determine whether or not the means are significantly different.
A $5\%$ significance level was used.

\begin{itemize}
\item   What is the null and alternative hypothesis?
\item   Compute the test statistic.
\item   What is your conclusion for this procedure? Justify your answer.
\end{itemize}





%%%%%%%%%%%%%%%%%%%%%%%%%%%%%%%%%%%%%%%%%

\noindent \textbf{The Test Statistic}
\begin{itemize}
\item The Test statistic is therefore
\[ TS = {\mbox{observed} - \mbox{null} \over \mbox{Std. Error}}  \]
\item Two Tailed Test, therefore $k = 2$, and $\alpha = 0.05$. Also two large samples. The CV is 1.96.
\end{itemize}

%%%%%%%%%%%%%%%%%%%%%%%%%%%%%%%%%%%%%%%%%

\noindent \textbf{Decison Rule}
\textbf{Decision Rule}
\[ |TS| > CV ?  \]
\begin{itemize}
\item If Yes: Reject the null Hypothesis
\item If No : Fail to reject the Null Hypothesis
\end{itemize}


\newpage 
\subsection*{Confidence Intervals}
{\bf One sample}
\begin{eqnarray*} S.E.(\bar{X})&=&\frac{\sigma}{\sqrt{n}}.\\\\
	S.E.(\hat{P})&=&\sqrt{\frac{\hat{p}\times(1-\hat{p})}{n}}.\\
\end{eqnarray*}
{\bf Two samples}
\begin{eqnarray*}
	S.E.(\bar{X}_1-\bar{X}_2)&=&\sqrt{\frac{\sigma^2_1}{n_1}+\frac{\sigma_2^2}{n_2}}.\\\\
	S.E.(\hat{P_1}-\hat{P_2})&=&\sqrt{\frac{\hat{p}_1\times(1-\hat{p}_1)}{n_1}+\frac{\hat{p}_2\times(100-\hat{p}_2)}{n_2}}.\\\\
\end{eqnarray*}
\subsection*{Hypothesis tests}
{\bf One sample}
\begin{eqnarray*}
	S.E.(\bar{X})&=&\frac{\sigma}{\sqrt{n}}.\\\\
	S.E.(p)&=&\sqrt{\frac{p \times(1-p)}{n}}
\end{eqnarray*}
{\bf Two large independent samples}
\begin{eqnarray*}
	S.E.(\bar{X}_1-\bar{X}_2)&=&\sqrt{\frac{\sigma^2_1}{n_1}+\frac{\sigma_2^2}{n_2}}.\\\\
	S.E.(\hat{P_1}-\hat{P_2})&=&\sqrt{\left(\bar{p}\times(1-\bar{p})\right)\left(\frac{1}{n_1}+\frac{1}{n_2}\right)}.\\
\end{eqnarray*}
{\bf Two small independent samples}
\begin{eqnarray*}
	S.E.(\bar{X}_1-\bar{X}_2)&=&\sqrt{s_p^2\left(\frac{1}{n_1}+\frac{1}{n_2}\right)}.\\\\
	s_p^2&=&\frac{s_1^2(n_1-1)+s_2^2(n_2-1)}{n_1+n_2-2}.\\
\end{eqnarray*}
{\bf Paired sample}
\begin{eqnarray*}
	S.E.(\bar{d})&=&\frac{s_d}{\sqrt{n}}.\\\\
\end{eqnarray*}
{\bf Standard Deviation of case-wise differences (computational formula)}
\begin{eqnarray*}
	s_d = \sqrt{ {\sum d_i^2 - n\bar{d}^2 \over n-1}}.\\\\
\end{eqnarray*}

