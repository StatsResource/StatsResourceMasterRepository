

\documentclass[a4paper,12pt]{article}
%%%%%%%%%%%%%%%%%%%%%%%%%%%%%%%%%%%%%%%%%%%%%%%%%%%%%%%%%%%%%%%%%%%%%%%%%%%%%%%%%%%%%%%%%%%%%%%%%%%%%%%%%%%%%%%%%%%%%%%%%%%%%%%%%%%%%%%%%%%%%%%%%%%%%%%%%%%%%%%%%%%%%%%%%%%%%%%%%%%%%%%%%%%%%%%%%%%%%%%%%%%%%%%%%%%%%%%%%%%%%%%%%%%%%%%%%%%%%%%%%%%%%%%%%%%%
\usepackage{eurosym}
\usepackage{vmargin}
\usepackage{amsmath}
\usepackage{framed}
\usepackage{graphics}
\usepackage{epsfig}
\usepackage{subfigure}
\usepackage{enumerate}
\usepackage{fancyhdr}

\setcounter{MaxMatrixCols}{10}
%TCIDATA{OutputFilter=LATEX.DLL}
%TCIDATA{Version=5.00.0.2570}
%TCIDATA{<META NAME="SaveForMode"CONTENT="1">}
%TCIDATA{LastRevised=Wednesday, February 23, 201113:24:34}
%TCIDATA{<META NAME="GraphicsSave" CONTENT="32">}
%TCIDATA{Language=American English}

\pagestyle{fancy}
\setmarginsrb{20mm}{0mm}{20mm}{25mm}{12mm}{11mm}{0mm}{11mm}
\lhead{MathsResource} \chead{Probability Distributions} \rhead{Joint Distribution} %\input{tcilatex}
\begin{document}



\subsection*{Part A - Two Sample Test for Means (10 Marks)}
An exercise physiologist wants to determine if several short bouts of exercise provide the same benefit for cardiovascular fitness as one long bout of exercise. \\ \smallskip

\noindent 50 volunteers are randomly assigned to group 1 and do standardised aerobic exercise on a stationary bicycle for 30 minutes once a day, 5 days a week. 40 volunteers are randomly assigned to group 2 and do the same exercise for 10 minutes, 3 times a day, 5 days a week. Cardiovascular fitness was measured by VO2 max (maximum oxygen consumption while exercising). 

\begin{description}
	\item[Group 1] The mean change in VO2 after 12 weeks of exercise was 2.1 for group 1 with a standard deviation of 1.7.
	\item[Group 2] The mean change in VO2 after 12 weeks of exercise was 0.7 for group 2 with a standard deviation of 1. 
\end{description}

\noindent Test the hypothesis that there is no significant difference between two groups are the same.
	
\begin{itemize}
\item Formally state your null and alternative hypotheses.
	\item[(ii)] Compute the test statistic.
\item Discuss your conclusion to this test, supporting your statement with reference to appropriate values.
\end{itemize}
