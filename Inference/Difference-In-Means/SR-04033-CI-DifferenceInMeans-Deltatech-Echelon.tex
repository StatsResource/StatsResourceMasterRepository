\documentclass[a4paper,12pt]{article}
%%%%%%%%%%%%%%%%%%%%%%%%%%%%%%%%%%%%%%%%%%%%%%%%%%%%%%%%%%%%%%%%%%%%%%%%%%%%%%%%%%%%%%%%%%%%%%%%%%%%%%%%%%%%%%%%%%%%%%%%%%%%%%%%%%%%%%%%%%%%%%%%%%%%%%%%%%%%%%%%%%%%%%%%%%%%%%%%%%%%%%%%%%%%%%%%%%%%%%%%%%%%%%%%%%%%%%%%%%%%%%%%%%%%%%%%%%%%%%%%%%%%%%%%%%%%
\usepackage{eurosym}
\usepackage{vmargin}
\usepackage{amsmath}
\usepackage{framed}
\usepackage{graphics}
\usepackage{epsfig}
\usepackage{subfigure}
\usepackage{enumerate}
\usepackage{fancyhdr}

\setcounter{MaxMatrixCols}{10}
%TCIDATA{OutputFilter=LATEX.DLL}
%TCIDATA{Version=5.00.0.2570}
%TCIDATA{<META NAME="SaveForMode"CONTENT="1">}
%TCIDATA{LastRevised=Wednesday, February 23, 201113:24:34}
%TCIDATA{<META NAME="GraphicsSave" CONTENT="32">}
%TCIDATA{Language=American English}

\pagestyle{fancy}
\setmarginsrb{20mm}{0mm}{20mm}{25mm}{12mm}{11mm}{0mm}{11mm}
\lhead{StatsResource} \chead{Confidence Intervals} \rhead{Statistical Inference} %\input{tcilatex}
\begin{document}
The quality control manager at the Telektronic Company considers the production of telephone answering machines to be ’out of control’ when the overall rate of defects exceeds 4\%. 
 	Testing of a random sample of 150 machines revealed that 9 are defective. The production manager claims that production is not out of control and no corrective action is necessary. Use a 0.05 significance level to test the production manager’s claim.
 	
 	\item In a study of company salaries, salaries paid by 2 different IT companies were randomly selected.
 	
 	\begin{itemize}
 		\item For 40 Deltatech employees the mean is €23,870 and the standard deviation is €2960. 
 		\item For 35 Echelon employees , the mean is €22,025 and the standard deviation is €3065. 
 	\end{itemize}
 	
 	At the 0.05 level of significance, test the claim that Deltatech employees earn the same as their  Echelon counterparts.
