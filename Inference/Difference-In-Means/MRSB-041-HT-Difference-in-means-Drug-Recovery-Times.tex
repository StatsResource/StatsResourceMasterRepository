

\documentclass[a4paper,12pt]{article}
%%%%%%%%%%%%%%%%%%%%%%%%%%%%%%%%%%%%%%%%%%%%%%%%%%%%%%%%%%%%%%%%%%%%%%%%%%%%%%%%%%%%%%%%%%%%%%%%%%%%%%%%%%%%%%%%%%%%%%%%%%%%%%%%%%%%%%%%%%%%%%%%%%%%%%%%%%%%%%%%%%%%%%%%%%%%%%%%%%%%%%%%%%%%%%%%%%%%%%%%%%%%%%%%%%%%%%%%%%%%%%%%%%%%%%%%%%%%%%%%%%%%%%%%%%%%
\usepackage{eurosym}
\usepackage{vmargin}
\usepackage{amsmath}
\usepackage{framed}
\usepackage{graphics}
\usepackage{epsfig}
\usepackage{subfigure}
\usepackage{enumerate}
\usepackage{fancyhdr}

\setcounter{MaxMatrixCols}{10}
%TCIDATA{OutputFilter=LATEX.DLL}
%TCIDATA{Version=5.00.0.2570}
%TCIDATA{<META NAME="SaveForMode"CONTENT="1">}
%TCIDATA{LastRevised=Wednesday, February 23, 201113:24:34}
%TCIDATA{<META NAME="GraphicsSave" CONTENT="32">}
%TCIDATA{Language=American English}

\pagestyle{fancy}
\setmarginsrb{20mm}{0mm}{20mm}{25mm}{12mm}{11mm}{0mm}{11mm}
\lhead{MathsResource} \chead{Probability Distributions} \rhead{Joint Distribution} %\input{tcilatex}
\begin{document}



\subsection*{Example 3: Difference in Means (a) }
Two sets of patients are given courses of treatment under two different drugs. The benefits
derived from each drug can be stated numerically in terms of the recovery times; the readings are given below.

\begin{itemize}
\item Drug 1:  $n_1$ = 40 , $\bar{x}_1$ = 3.3 days and $s_1 = 1.524$
\item Drug 2:  $n_2$ = 45 , $\bar{x}_2$ = 4.3 days and $s_2 = 1.951 $
\end{itemize}


%-------------------------------------------------------------------------------------------%

\subsection*{Example 3: Difference in Means (b) }
\begin{itemize}
\item
The first step in hypothesis testing is to specify the null hypothesis and an alternative hypothesis.
\item When testing differences between mean recovery times, the null hypothesis is that the two population means are equal.
\item That is, the null hypothesis is:\\
$H_0: \mu_1 = \mu_2$ ( The population means are equal)\\
$H_1: \mu_1 \neq \mu_2$ (The population means are different)\\
\end{itemize}
(Remark: Two Tailed Test, therefore $k = 2$, and $\alpha = 0.05$)


%-------------------------------------------------------------------------------------------%

\subsection*{Example 3: Difference in Means (c) }
\begin{itemize}
\item The observed difference in means is 1 day.
\item The relevant formula for the standard error is
\[ S.E(\bar{x}_1 - \bar{x}_2) = \sqrt{{s^2_1\over n_1}+{s^2_2 \over n_2}} \]
\item  \[ S.E(\bar{x}_1 - \bar{x}_2) = \sqrt{{(1.524)^2 \over 40}+{(1.951)^2 \over 45}}   \]
\item  \[ S.E(\bar{x}_1 - \bar{x}_2) = 0.377\mbox{ days}\]
\end{itemize}


%-------------------------------------------------------------------------------------------%
[fragile]
\subsection*{Example 2: Difference in Means (d) }
\begin{itemize}
\item The Test statistic is therefore
\[ TS = {\mbox{observed} - \mbox{null} \over \mbox{Std. Error}}  = {1 - 0 \over 0.377 } = 2.65 \]
\item Lets compute the p-value of this : \\
p-value = $P(z \geq 2.65) = 0.0040$
\begin{verbatim}
> 1-pnorm(2.65)
[1] 0.004024589
\end{verbatim}

\item What is this value smaller than threshold $\alpha / k$? \\
\item $\alpha / k$ = $0.05/2$ = 0.0250? Yes the p-value is smaller than this.
\item \textbf{Conclusion:} we reject the null hypothesis. There is a significant different between both drugs, in terms of recovery times.

\end{itemize}



%-------------------------------------------------------------------------------------------%

\noindent \textbf{Example 1: Difference in Means (d) }
\begin{itemize}
\item The Test statistic is therefore
\[ TS = {\mbox{observed} - \mbox{null} \over \mbox{Std. Error}}  = {1 - 0 \over 0.377 } = 2.65 \]
\item Two Tailed Test, therefore $k = 2$, and $\alpha = 0.05$. Also two large samples. The CV is 1.96.
\end{itemize}



%-------------------------------------------------------------------------------------------%

\noindent \textbf{Example 1: Difference in Means (e) }
\textbf{Decision Rule}
\[ |TS| > CV ?  \]
\begin{itemize}
\item If Yes: Reject the null Hypothesis
\item If No : Fail to reject the Null Hypothesis
\end{itemize}
$|2.65|$ is greater than 1.96. We reject the null hypothesis. There is a significant difference in the effectiveness of the two drug treatments.




%%%%%%%%%%%%%%%%%%%%%%%%%%%%%%%%%%%%%%%

\subsection*{Example 3: Difference in Means (a) }
\begin{itemize}
\item We will approach the same problem in example 2 , but this time using the CV approach.
\item The first two steps i.e. formally stating the null and alternative hypothesis, and computing the test statistic are the same, are the same as example 1.
\item As the sample was large, we could use $CV = 1.96$ (as always, two tailed procedure, with $\alpha=0.05$).
\begin{verbatim}
> qnorm(0.975)
[1] 1.959964
\end{verbatim}
\item Is the $TS > CV$ ?   Is $2.65 > 1.96$ ? - Yes , we reject the null hypothesis.
\end{itemize}


%-------------------------------------------------------------------------------------------%

\noindent \textbf{Writing the Hypotheses}
\begin{itemize}
\item The first step in hypothesis testing is to specify the null hypothesis and an alternative hypothesis.
\item When testing differences between mean recovery times, the null hypothesis is that the two population means are equal.
\item That is, the null hypothesis is:\\
$H_0: \mu_1 = \mu_2$ ( The population means are equal)\\
$H_1: \mu_1 \neq \mu_2$ (The population means are different)
\end{itemize}



(Remark: Two Tailed Test, therefore $k = 2$, and $\alpha = 0.05$)


%-------------------------------------------------------------------------------------------%

\noindent \textbf{Standard Error}
\begin{itemize}
\item The observed difference in means is 1 day.
\item The relevant formula for the standard error is
\[ S.E(\bar{x}_1 - \bar{x}_2) = \sqrt{{s^2_1\over n_1}+{s^2_2 \over n_2}} \]
\item  \[ S.E(\bar{x}_1 - \bar{x}_2) = \sqrt{{(1.524)^2 \over 40}+{(1.951)^2 \over 45}}   \]
\item  \[ S.E(\bar{x}_1 - \bar{x}_2) = 0.377\mbox{ days}\]
\end{itemize}

%-------------------------------------------------------------------------------------------%


\noindent \textbf{The Test Statistic}
\begin{itemize}
\item The Test statistic is therefore
\[ TS = {\mbox{observed} - \mbox{null} \over \mbox{Std. Error}}  = {1 - 0 \over 0.377 } = 2.65 \]
\item Two Tailed Test, therefore $k = 2$, and $\alpha = 0.05$. Also two large samples. The CV is 1.96.
\end{itemize}



%-------------------------------------------------------------------------------------------%

\noindent \textbf{Decison Rule}
\textbf{Decision Rule}
\[ |TS| > CV ?  \]
\begin{itemize}
\item If Yes: Reject the null Hypothesis
\item If No : Fail to reject the Null Hypothesis
\end{itemize}
$|2.65|$ is greater than 1.96. We reject the null hypothesis. There is a significant difference in the effectiveness of the two drug treatments.




%--------------------------------------------------------------------------%
\end{document}
