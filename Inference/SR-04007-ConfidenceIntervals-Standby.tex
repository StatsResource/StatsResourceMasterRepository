

\documentclass[a4paper,12pt]{article}
%%%%%%%%%%%%%%%%%%%%%%%%%%%%%%%%%%%%%%%%%%%%%%%%%%%%%%%%%%%%%%%%%%%%%%%%%%%%%%%%%%%%%%%%%%%%%%%%%%%%%%%%%%%%%%%%%%%%%%%%%%%%%%%%%%%%%%%%%%%%%%%%%%%%%%%%%%%%%%%%%%%%%%%%%%%%%%%%%%%%%%%%%%%%%%%%%%%%%%%%%%%%%%%%%%%%%%%%%%%%%%%%%%%%%%%%%%%%%%%%%%%%%%%%%%%%
\usepackage{eurosym}
\usepackage{vmargin}
\usepackage{amsmath}
\usepackage{framed}
\usepackage{graphics}
\usepackage{epsfig}
\usepackage{subfigure}
\usepackage{enumerate}
\usepackage{fancyhdr}

\setcounter{MaxMatrixCols}{10}
%TCIDATA{OutputFilter=LATEX.DLL}
%TCIDATA{Version=5.00.0.2570}
%TCIDATA{<META NAME="SaveForMode"CONTENT="1">}
%TCIDATA{LastRevised=Wednesday, February 23, 201113:24:34}
%TCIDATA{<META NAME="GraphicsSave" CONTENT="32">}
%TCIDATA{Language=American English}

\pagestyle{fancy}
\setmarginsrb{20mm}{0mm}{20mm}{25mm}{12mm}{11mm}{0mm}{11mm}
\lhead{MathsResource} \chead{Probability Distributions} \rhead{Joint Distribution} %\input{tcilatex}
\begin{document}
\section{Confidence Intervals}

\begin{itemize}
	\item A sample of 300 households in a large town revealed that 183 have home computers.
	\item Construct a 95\% confidence interval for the proportion of households with home computers in the whole town.
\end{itemize}



\subsection*{Computing Confidence Intervals}
Confidence limits are the lower and upper boundaries / values of a confidence interval, that is, the values which define the range of a confidence interval. The general structure of a confidence interval is as follows:

\[ \mbox{Point Estimate}  \pm \left[ \mbox{Quantile} \times \mbox{Standard Error} \right] \]


\begin{itemize}
\item Point Estimate: estimate for population parameter of interest, i.e. sample mean, sample proportion.
\item Quantile: a value from a probability distribution that scales the intervals according to the specified confidence level.
\item Standard Error: measures the dispersion of the sampling distribution for a given sample size $n$.
\end{itemize}

 
\textbf{Sample size:}       
\begin{itemize}
	\item n =300 :  ( N.B. Large sample)
\end{itemize}  
 
Estimate:             Sample proportion   
 
    (N.B. Percentages are easier to work with.)
 
\textbf{t-value:}           

\begin{itemize}
\item 95\% confidence , therefore  = 0.05
\item two tailed procedure, therefore k = 2
\item therefore column to use = 0.025
\end{itemize}
 
                      Large sample:  degrees of freedom =  i.e. bottom row
 
                      therefore t-value is 1.96                   
 
 
Standard Error:  from Formulae   
\[  \frac{p(1-p)}{n}      = \frac{61 \times 39}{300}        = 2.81\%\]
 
 
Confidence Interval : 
\[\mbox{estimate} \pm (\mbox{t-value} \times \mbox{std. error})\]
 
\[  61\% \pm (1.96 \times 2.81\%) \]
 


\newpage
\end{document}