\documentclass[a4paper,12pt]{article}

%%%%%%%%%%%%%%%%%%%%%%%%%%%%%%%%%%%%%%%%%%%%%%%%%%%%%%%%%%%%%%%%%%%%%%%%%%%%%%%%%%%%%%%%%%%%%%%%%%%%%%%%%%%%%%%%%%%%%%%%%%%%%%%%%%%%%%%%%%%%%%%%%%%%%%%%%%%%%%%%%%%%%%%%%%%%%%%%%%%%%%%%%%%%%%%%%%%%%%%%%%%%%%%%%%%%%%%%%%%%%%%%%%%%%%%%%%%%%%%%%%%%%%%%%%%%
  \usepackage{eurosym}
\usepackage{vmargin}
\usepackage{amsmath}
\usepackage{graphics}
\usepackage{epsfig}
\usepackage{enumerate}
\usepackage{multicol}
\usepackage{subfigure}
\usepackage{fancyhdr}
\usepackage{listings}
\usepackage{framed}
\usepackage{graphicx}
\usepackage{amsmath}
\usepackage{chngpage}
%\usepackage{bigints}
\usepackage{vmargin}

% left top textwidth textheight headheight

% headsep footheight footskip

\setmargins{2.0cm}{2.5cm}{16 cm}{22cm}{0.5cm}{0cm}{1cm}{1cm}

\renewcommand{\baselinestretch}{1.3}

\setcounter{MaxMatrixCols}{10}

\begin{document}

%%-- CT3 2005_April_F_Q6_CIforProps.tex
%%-- CT3 2005_Sept_D_Q4_ConfInts.tex
%%-- CT3 2007_Sept_C_Q4_Prop_CI.tex
%%-- CT3 2013_April_E_Q6_CI_props
%%-- CT3 2011_Sept_C_Q3_ConfInts.tex

%-- CT3 2010_April_G_Q7_CI_Props.tex

%%%%%%%%%%%%%%%%%%%%%%%%%%%
\begin{enumerate}

%-- 2005_April_F_Q6_CIforProps.tex
\item In a survey conducted by a mail order company a random sample of 200 customers yielded 172 who indicated that they were highly satisfied with the delivery time of their orders.
Calculate an approximate 95\% confidence interval for the proportion of the
company's customers who are highly satisfied with delivery times.



%%%%%%%%%%%%%%%%%%%%%%%%%%%

%-- 2005_Sept_D_Q4_ConfInts
\item  Calculate a 99\% confidence interval for the percentage of claims for household accidental damage which are fully settled within six months of being submitted, given
that in a random sample of 100 submitted claims of this type, exactly 83 were fully settled within six months of being submitted.


%%%%%%%%%%%%%%%%%%%%%%%%%%%

%-- 2007_Sept_C_Q4_Prop_CI.tex

\item  In a random sample of 200 policies from a company’s private motor business, there are 68 female policyholders and 132 male policyholders.
Calculate an approximate 99\% confidence interval for the proportion of policyholders who are female in the corresponding population of all policyholders.

%%%%%%%%%%%%%%%%%%%%%%%%%%%

%--2013_April_E_Q6_CI_props

\item A survey is undertaken to investigate the proportion $p$ of an adult population that
support a certain government policy. A random sample of 100 adults is taken and
contains 30 who support the policy.
\begin{enumerate}[(a)]
\item  Calculate an approximate 95\% confidence interval for $p$. 
\item  Comment on the validity of the interval obtained in part (a). 
\item A different sample of 1,000 adults is taken and it contains 300 who support the policy.
Explain how the width of a 95\% confidence interval for $p$ in this case will
compare to the width of the interval in part (a), without performing any
calculations.
\end{enumerate}

%%%%%%%%%%%%%%%%%%%%%%%%%%%

%--2011_Sept_C_Q3_ConfInts.tex

\item A random sample of 60 adult men who live in Leeds includes 21 who have visited Majorca. An independent random sample of 70 adult women who live in Leeds includes 28 who have visited Majorca.
Calculate a 98\% confidence interval for the proportion of adults who live in Leeds
who have visited Majorca.


%%%%%%%%%%%%%%%%%%%%%%%%%%
\end{enumerate}

\newpage

%%- Sample Size Estimation

\begin{enumerate}
%-- 2010_April_G_Q7_CI_Props.tex

\item
An employment survey is carried out in order to determine the percentage, $p$, of
	unemployed people in a certain population in a way such that the estimation has a margin of error less than 0.5\% with probability at least 0.95. \\ \\
	\noindent In a similar study conducted a year ago it was found that the percentage of unemployed people in the
	population was 6\%.\\\\
	 Estimate the sample size, $n$, that is required to achieve this margin of error, by
	constructing an appropriate confidence interval (or otherwise).
\end{enumerate}
\end{document}