



\noindent \textbf{Step B : Compute the test statistic.}

Remember the general structure of a test statistic

TS =Observed Value-Null ValueStd. Error 



From the formulae

We have to compute the standard error for a proportion. 

( From formulae at back of exam paper)

S.E.(p) =p(1-p)n=0.450.551000= 0.0157




\noindent \textbf{Step 3: Calculate p-value}

P-value is found from Murdoch Barnes Tables 3 ( Normal distribution)

Absolute value  |-3.18| =3.18




\[ \mbox{P-Value} = P(Z \geq 3.18) = 0.00074\]


\noindent \textbf{Step 4: Interpret the p-value to make a decision.}

The significance level is 5\%.  The procedure is a two tailed test.


[ Black Board ]

\newpage
Q5. The standard deviation of test scores obtained for a certain exam is 18 points. 
A random sample of 81 students has a sample mean of 70 points.

(a) State the point estimate for the mean score for all the students.
(b) Find the 95\% confidence interval for the average score for all students.
(c) Find the 99\% confidence interval for the average score for all students.

Q6. The amount spent (€’s) by customers in a shop are normally distributed. 
A random sample of 16 customers have these values:
\[19 21 35 29 12 35 7 18 21 14 29 20 12 24 32 23\]
(Sample  mean of €21.94 and a sample standard deviation of €8.40) 
Estimate a 95% confidence interval for the population mean.



%%%%%%%%%%%%%%%%%%%%%%%%%%%%%%%%%%%%%%%%%%%%%%%%%%%%%%%

\newpage


Question 2.
A claim has been made that the mean body temperature of healthy adults is equal to 98.6 degrees. 
Test this hypothesis using a 0.05 level significance, given the following information.
A sample of 121 people has produced a mean body temperature of 98.2 degrees and a standard deviation of 6.6 degrees.  

Question 3.
The quality control manager at the Telektronic Company considers the production of telephone answering machines to be ’out of control’ when the overall rate of defects exceeds 6%. 
Testing of a random sample of 150 machines revealed that 12 are defective. 
The production manager claims that production is not out of control and no corrective action is necessary.
i.Compute a 95% confidence interval for the rate of defective components
ii.Use a 0.05 significance level to test the production manager’s claim.
(N.B. The Standard Error used in hypothesis testing is different to the one used for confidence intervals)



Question 5.
In a study of store checkout scanners, 240 items were checked and 6 of those items were found to be “overcharges”.
Use a 0.05 significance level to test the claim that with these scanners, 1.5 % of sales transactions are overcharges.
(N.B. The Standard Error used in hypothesis testing is different to the one used for confidence intervals)

\item \textbf{Worked Example 3} \\ Ten replicate analyses of the concentration
of mercury in a sample of commercial gas condensate gave the
following results (in ng/ml) :

\begin{tabular}{|c|c|c|c|c|c|c|c|c|c|}
\hline
23.3 & 22.5 & 21.9 & 21.5 & 19.9 & 21.3 & 21.7 & 23.8 & 22.6 &
24.7\\
\hline
\end{tabular}



\end{document}