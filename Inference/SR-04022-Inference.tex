\documentclass[]{report}

\voffset=-1.5cm
\oddsidemargin=0.0cm
\textwidth = 480pt

\usepackage{framed}
\usepackage{subfiles}
\usepackage{graphics}
\usepackage{newlfont}
\usepackage{eurosym}
\usepackage{amsmath,amsthm,amsfonts}
\usepackage{amsmath}
\usepackage{color}
\usepackage{amssymb}
\usepackage{multicol}
\usepackage[dvipsnames]{xcolor}
\usepackage{graphicx}
\begin{document}
\textbf{Hypothesis Testing}
	A manufacturer of a common cold cure claims that the product provides
	relief for 70\% of people who use it. ln a test of 400 people, it was found
	that 300 people said the treatment provided relief.
	
	\begin{itemize}
		\item[a.](4 marks) Calculate a 95\% confidence interval for the true proportion of
		people who would get relief from the product.
		
		\item[b.](4 marks) Suppose the manufacturer wishes to be 95\% confident that the
		prediction is correct to within 2\% of the true proportion. What
		sample size is needed?
		
		\item[c.](7 marks) Using a significance level of 5\%, test the hypothesis that more than
		70\% of people who use the product find relief Clearly state your
		null and alternative hypotheses and your conclusion.
	\end{itemize}
	
	\newpage
\large  % 10 Marks
\noindent A study was carried out to compare two treatments for the flu. A total of 500
	newly diagnosed flu patients were randomly assigned to one of the two treatments.
	\begin{itemize}
		\item Of the 280 assigned to the first treatment, 168 still had the flu after 2 days after
		diagnosis. \item Of the 220 assigned to the second treatment, 176 still had the flu after 2
		days after diagnosis. \end{itemize} 
		
Let $p_l$ denote the probability that a flu patient assigned to the
	first treatment will still have the flu after 2 days after diagnosis; let $p_2$ denote the
	corresponding probability for the second treatment.
	
	
	
	\end{document}