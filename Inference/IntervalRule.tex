

	\documentclass[a4paper,12pt]{article}
%%%%%%%%%%%%%%%%%%%%%%%%%%%%%%%%%%%%%%%%%%%%%%%%%%%%%%%%%%%%%%%%%%%%%%%%%%%%%%%%%%%%%%%%%%%%%%%%%%%%%%%%%%%%%%%%%%%%%%%%%%%%%%%%%%%%%%%%%%%%%%%%%%%%%%%%%%%%%%%%%%%%%%%%%%%%%%%%%%%%%%%%%%%%%%%%%%%%%%%%%%%%%%%%%%%%%%%%%%%%%%%%%%%%%%%%%%%%%%%%%%%%%%%%%%%%
\usepackage{eurosym}
\usepackage{vmargin}
\usepackage{framed}
\usepackage{amsmath}
\usepackage{graphics}
\usepackage{epsfig}
\usepackage{subfigure}
\usepackage{enumerate}
\usepackage{fancyhdr}

\setcounter{MaxMatrixCols}{10}
%TCIDATA{OutputFilter=LATEX.DLL}
%TCIDATA{Version=5.00.0.2570}
%TCIDATA{<META NAME="SaveForMode"CONTENT="1">}
%TCIDATA{LastRevised=Wednesday, February 23, 201113:24:34}
%TCIDATA{<META NAME="GraphicsSave" CONTENT="32">}
%TCIDATA{Language=American English}

\pagestyle{fancy}
\setmarginsrb{20mm}{0mm}{20mm}{25mm}{12mm}{11mm}{0mm}{11mm}
\lhead{MS4222} \rhead{Kevin O'Brien} \chead{Continuous Probability Distribution} %\input{tcilatex}

\begin{document}
\section*{The Interval Rule}


\begin{itemize}
\item We are often interested in the probability of being inside an interval, with lower bound $L$ and upper bound $U$, where $L$ and $U$ are outcomes of some continuous random variable $X$.
\item It is often easier to compute the probability of the complement event, being outside the interval.
\[ P( \mbox{Inside} ) = 1 - P( \mbox{Outside} )  \]
\item The probability of being inside this interval is the \textbf{complement} of being outside the interval. The event of being outside the interval is the union of two disjoint events, i.e. the conjunction of being too low and too high.
\[ P( \mbox{Outside} ) = P( \mbox{Too Low} ) +  P( \mbox{Too High} ) \]
\item Therefore we can say
\[ P( \mbox{Inside} ) = 1- [P( \mbox{Too Low} ) +  P( \mbox{Too High} )] \]

\end{itemize}

\begin{itemize}
\item[$\bullet$] The probability of $X$ being too low for the interval (i.e. less than the interval minimum L)
\[ P( \mbox{Too Low} ) = P(X \leq L) \]
\item[$\bullet$] The probability of $X$ being too high for the interval (i.e. greater than the interval maximum U)
\[ P( \mbox{Too High} ) =P(X \geq U ) \]
\end{itemize}



% \noindent Suppose we have an interval for 
% the random variable $X$ % defined by the 
% \begin{itemize}
% \item the lower bound L
% \item the upper bound U
% \end{itemize}
% 
% \[ L \leq X \leq U \]




\[ P(L \leq X \leq U) = 1 - ( P(X \leq L) +  P(X \geq U ) ) \]

% \begin{framed}
% The Interval rule is defined.
% \begin{equation*}\tag {1.5c}
% P(L \leq Z \leq U) = P(Z \geq L) - P(Z \geq U)
% \end{equation*}
% 
% \indent where $L$ and $U$  are the lower and upper bounds of the
% interval
% \end{framed}


\subsection*{Example}

Find the probability of a Z random variable being between -1.8 and 1.96?\\
i.e. Compute $P(-1.8 \leq Z \leq 1.96)$
\subsection*{Solution}
\begin{itemize}
%\item Consider the complement event of being in this interval: a combination of being too low or too high.
\item
The probability of being too low for this interval is $P(Z \leq -1.80) = 0.0359$ (check on Murdoch Barnes Tables)
\item
The probability of being too high for this interval is $P(Z \geq 1.96) = 0.0250$ (check on Murdoch Barnes Tables)
\item
Therefore the probability of being \textbf{outside} the interval is 0.0359 + 0.0250 = 0.0609.
\item
Therefore the probability of being \textbf{inside} the interval is $1- (0.0609) = 0.9391$.
\[P(-1.8 \leq Z \leq 1.96) = 0.9391\]
\end{itemize}





\end{document}
