	\documentclass[a4paper,12pt]{article}
%%%%%%%%%%%%%%%%%%%%%%%%%%%%%%%%%%%%%%%%%%%%%%%%%%%%%%%%%%%%%%%%%%%%%%%%%%%%%%%%%%%%%%%%%%%%%%%%%%%%%%%%%%%%%%%%%%%%%%%%%%%%%%%%%%%%%%%%%%%%%%%%%%%%%%%%%%%%%%%%%%%%%%%%%%%%%%%%%%%%%%%%%%%%%%%%%%%%%%%%%%%%%%%%%%%%%%%%%%%%%%%%%%%%%%%%%%%%%%%%%%%%%%%%%%%%
\usepackage{eurosym}
\usepackage{vmargin}
\usepackage{framed}
\usepackage{amsmath}
\usepackage{graphics}
\usepackage{epsfig}
\usepackage{subfigure}
\usepackage{enumerate}
\usepackage{fancyhdr}

\setcounter{MaxMatrixCols}{10}
%TCIDATA{OutputFilter=LATEX.DLL}
%TCIDATA{Version=5.00.0.2570}
%TCIDATA{<META NAME="SaveForMode"CONTENT="1">}
%TCIDATA{LastRevised=Wednesday, February 23, 201113:24:34}
%TCIDATA{<META NAME="GraphicsSave" CONTENT="32">}
%TCIDATA{Language=American English}

\pagestyle{fancy}
\setmarginsrb{20mm}{0mm}{20mm}{25mm}{12mm}{11mm}{0mm}{11mm}
\lhead{MS4222} \rhead{Kevin O'Brien} \chead{Confidence Intervals} %\input{tcilatex}

\begin{document}

\subsection*{Q4. Confidence Interval for a Proportion (5 Marks)}
The strength of dosage of a plant growth enhancement chemical is often measured by the proportion of plants that grow faster. A particular dosage of the chemical is fed to 115 plants of these plants, 94 actually show faster growth.

\begin{itemize}
	\item[i.] (1 Mark) Calculate a point estimate $\hat{p}$ for the proportion of plants that grow faster due to the dosage. 									 
	\item[ii.] (2 Marks)  What is the standard error of the estimate? 			
	\item[iii.] (2 Marks) Find a 95\% confidence interval for the proportion. 					
\end{itemize}
\newpage
(Write your answers here)
\newpage


\begin{framed}
The general structure of a confidence interval is as follows:

\[ \mbox{Point Estimate}  \pm \left[ \mbox{Quantile} \times \mbox{Standard Error} \right] \]


\begin{itemize}
\item Point Estimate: estimate for population parameter of interest, i.e. sample mean, sample proportion.
\item Quantile: a value from a probability distribution that scales the intervals according to the specified confidence level.
\item Standard Error: measures the dispersion of the sampling distribution for a given sample size $n$.
\end{itemize}
\end{framed}

%%%%%%%%%%%%%%5
