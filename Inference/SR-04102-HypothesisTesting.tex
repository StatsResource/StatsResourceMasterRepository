\documentclass[a4paper,12pt]{article}
\usepackage{eurosym}
\usepackage{vmargin}
\usepackage{amsmath}
\usepackage{graphics}
\usepackage{epsfig}
\usepackage{enumerate}
\usepackage{multicol}
\usepackage{subfigure}
\usepackage{fancyhdr}
\usepackage{listings}
\usepackage{framed}
\usepackage{graphicx}
\usepackage{amsmath}
\usepackage{chngpage}
%\usepackage{bigints}

\usepackage{vmargin}
% left top textwidth textheight headheight
% headsep footheight footskip
\setmargins{2.0cm}{2.5cm}{16 cm}{22cm}{0.5cm}{0cm}{1cm}{1cm}
\renewcommand{\baselinestretch}{1.3}

\setcounter{MaxMatrixCols}{10}

\begin{document}
\large

 \textbf{\textit{Inference Procedures (9 Marks)}}\\A study finds that $45\%$ of IT users out of a random sample of 500 in a large
community preferred one web browser to all others. In another large community, $30\%$ of IT users out of a random sample of 390 prefer the same web browser.

\begin{itemize}
\item[(i)] (2 Marks) Compute the point estimate for the difference in proportions of IT users who prefer this particular web browser.
\item[(ii)] (4 Marks) Compute a 95\% confidence interval for this difference in proportions.
\item[(iii)] (3 Marks) Based on this confidence interval, test the hypothesis that the proportion of IT users using this web browser is the same for both communities. State your null and alternative hypotheses clearly.
\end{itemize}

\end{itemize}