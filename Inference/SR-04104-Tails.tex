	\documentclass[a4paper,12pt]{article}
%%%%%%%%%%%%%%%%%%%%%%%%%%%%%%%%%%%%%%%%%%%%%%%%%%%%%%%%%%%%%%%%%%%%%%%%%%%%%%%%%%%%%%%%%%%%%%%%%%%%%%%%%%%%%%%%%%%%%%%%%%%%%%%%%%%%%%%%%%%%%%%%%%%%%%%%%%%%%%%%%%%%%%%%%%%%%%%%%%%%%%%%%%%%%%%%%%%%%%%%%%%%%%%%%%%%%%%%%%%%%%%%%%%%%%%%%%%%%%%%%%%%%%%%%%%%
\usepackage{eurosym}
\usepackage{vmargin}
\usepackage{framed}
\usepackage{amsmath}
\usepackage{graphics}
\usepackage{epsfig}
\usepackage{subfigure}
\usepackage{enumerate}
\usepackage{fancyhdr}

\setcounter{MaxMatrixCols}{10}
%TCIDATA{OutputFilter=LATEX.DLL}
%TCIDATA{Version=5.00.0.2570}
%TCIDATA{<META NAME="SaveForMode"CONTENT="1">}
%TCIDATA{LastRevised=Wednesday, February 23, 201113:24:34}
%TCIDATA{<META NAME="GraphicsSave" CONTENT="32">}
%TCIDATA{Language=American English}

\pagestyle{fancy}
\setmarginsrb{20mm}{0mm}{20mm}{25mm}{12mm}{11mm}{0mm}{11mm}
\lhead{MS4222} \rhead{Kevin O'Brien} \chead{Hypothesis Testing} %\input{tcilatex}

\begin{document}


	\section*{Number of Tails  }
	
	\begin{itemize}
		\item The alternative hypothesis indicates the number of tails.
		\item A rule of thumb is to consider how many alternative to the $H_0$ is offered by $H_1$.
		\item When $H_1$ includes either of these relational operators;`$>$' ,`$<$' , only one alternative is offered.
		\item When $H_1$ includes the $\neq$ relational operators, two alternatives are offered (i.e.`$>$' or `$<$').
	\end{itemize}

\subsection*{One Tailed Inference Procedures}
\begin{itemize}
	\item We will briefly look at one-tailed hypothesis.
	\item One tailed confidence intervals do exist but are rarely used in practice.
	\item Some procedures will apply corrections factors to their estimate, and are not symmetric. Although they are more or less two-tailed.
\end{itemize}



\subsection*{One Tailed Hypothesis test}
\begin{itemize}


\item Equivalently One-tailed tests are useful when determining if the population mean  (or proportion) for one group is greater than that of another group.

\item In other words, the critical region for a one-sided test is the set of values less than the critical value of the test, or the set of values greater than the critical value of the test.



	
	\item A one-sided test is a statistical hypothesis test in which the values for which we can reject the null hypothesis, $H_0$ are located entirely in one tail of the probability distribution (either the upper tail, or the lower tail, but not both).

\item One tailed procedures are a more intuitive approach when determining if a certain values exceeds a certain threshold. (Recall the election question used in the midterm)

	
	\item In other words, the critical region for a one-sided test is the set of values less than the critical value of the test, or the set of values greater than the critical value of the test.
	
	\item A one-sided test is also referred to as a one-tailed test of significance.
	
	\item A rule of thumb is to consider the alternative hypothesis.  If only one alternative is offered by $H_1$ (i.e. a $`<'$ or a $`>'$ is present, then it is a one tailed test.)
	\item (When computing quantiles from Murdoch Barnes table 7, we set $k=1$)

\item  For the sake of brevity, we will focus on two-tailed procedures in this module, as those procedures are far more commonly used. Awareness of one-tailed procedures is encouraged.
\end{itemize}

%---------------------------------------------------------------------------------------------%

\textbf{Two Tailed Hypothesis test}
\begin{itemize}
	\item
	A two-sided test is a statistical hypothesis test in which the values for which we can reject the null hypothesis, $H_0$ are located in both tails of the probability distribution, on an equal basis.
%	\item In other words, the critical region for a two-sided test is the set of values less than a first critical value of the test and the set of values greater than a second critical value of the test.
	\item A two-sided test is also referred to as a two-tailed test of significance.
	\item A rule of thumb is to consider the alternative hypothesis.  If only one alternative is offered by $H_1$ (i.e. a $`\neq'$ is present, then it is a two tailed test.)
	\item (When computing quantiles from Murdoch Barnes table 7, we set $k=2$)
	
\end{itemize}



\end{document}