
\documentclass[a4paper,12pt]{article}
\usepackage{eurosym}
\usepackage{vmargin}
\usepackage{amsmath}
\usepackage{graphics}
\usepackage{epsfig}
\usepackage{enumerate}
\usepackage{multicol}
\usepackage{subfigure}
\usepackage{fancyhdr}
\usepackage{listings}
\usepackage{framed}
\usepackage{graphicx}
\usepackage{amsmath}
\usepackage{chngpage}
%\usepackage{bigints}

\usepackage{vmargin}
% left top textwidth textheight headheight
% headsep footheight footskip
\setmargins{2.0cm}{2.5cm}{16 cm}{22cm}{0.5cm}{0cm}{1cm}{1cm}
\renewcommand{\baselinestretch}{1.3}

\setcounter{MaxMatrixCols}{10}

\begin{document}
\large

\section{Confidence Intervals: Example}

\begin{itemize}
\item The length of life of a type of battery is estimated from a sample of 100 test items taken from a large population. 
\item Sample results show that the mean length of life is 57.4 hours with a standard deviation of 15.1 hours. \\ \bigskip
\item  Construct a 
95\% confidence interval for the mean length of life of all of these batteries.
\end{itemize}






With a sample standard deviation of 15.1 and a sample size of 100, the standard error is as follows:


\[ S.E. (\bar{x}) = \frac{15.1}{\sqrt{100}} \]



With a sample standard deviation of 15.1 and a sample size of 100, the standard error is as follows:


\[ 57.4 \pm \left( 1.96 \times 1.51 \right) \]

\end{document}




\section*{Question 6}
\large 
\noindent A manufacturer of computer monitors has, for many years, used a process giving a mean working life of 4700 hours for components.
A new process is tried to see if it will increase the life significantly. A sample of 100 monitors gave a mean life of 5000 hours, with a standard deviation of 1400 hours.

Compute a 95\% confidence interval for the mean life of components built using the new process.
\begin{framed}

\[
\scriptstyle \mbox{ Point Estimate } \pm  \left[ \operatorname{ Quantile } \times \operatorname{Standard Error } 
\right] 
\]

\end{framed}

\end{document}