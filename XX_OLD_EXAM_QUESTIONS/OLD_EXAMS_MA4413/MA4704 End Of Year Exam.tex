
\subsection*{Q.2 Probability (20 Marks)}
%Question 2
%A) Expected Values
%b) Probability
%C) Bayes Theorem
%D) Testing Normality +  p.value questions
\subsubsection*{Part A}
 The probability distribution of discrete random variable $X$ is tabulated below. There are 5 possible outcomes of $X$, i.e. 1, 2, 3, 5 ,10 and 20.
\begin{center}
\begin{tabular}{|c||c|c|c|c|c|}
\hline
$x_i$  & 1 & 2 & 5 & 10 & 20 \\\hline
$P(x_i)$ &  0.10 & 0.25 & 0.30& 0.20 &0.15\\

\hline
\end{tabular}
\end{center}

\begin{itemize}
\item[i.] (3 Marks) Determine the expected value $E(X)$.
\item[ii.] (3 Marks) Evaluate $E(X^2)$.
\item[iii.] (2 Marks) Compute the variance of random variable $X$.
\end{itemize}

\subsubsection*{Part B} % 6 Marks
A doctor treating a patient issues a prescription for antibiotics and provides for two repeat prescriptions. The probability that the infection will be cleared by the first prescription is $p_1$ =0.6.
The probability that successive treatments are successful, given that previous prescriptions were not successful are $p_2$ = 0.5, $p_3$ = 0.4. Calculate the probability that:

\begin{itemize}
\item[i.](2 Marks) a patient will require the third prescription,
\item[ii.](2 Marks) the patient is still infected after the third prescription,
\item[iii.](2 Marks) the patient is cured by the second prescription, given that the patient is eventually cured.
\end{itemize}
\subsubsection*{Part C}
An electronics assembly subcontractor receives its entire supply of resistors from two suppliers. Company A provides 70\% of the subcontractor's resistors ,while company B supplies the remainder. The additional information has also been made available.
\begin{itemize}
\item 2\% of the resistors provided by company A failed the final test,
\item 3\% of company B's resistors also fail final test.
\end{itemize}
\noindent Answer the following questions:
\begin{itemize}
\item[i.](3 Marks) What is the probability that a resistor fails the final test?
\item[ii.](3 Marks)  What is the probability that a resistor fails the final test given that the resistor in question came from company A?
%\item[iii.](2 marks) What is the probability that a resistor that fails final test was supplied by company A?
\end{itemize}

%------------------------------------------------------------------------------------------------ %
\newpage
%Question 3
%A) Normal Distribution 8 Marks
%B) Binomial Distribution - Not Actioned
%C) Poisson 6 Marks
%D) Exponential 6 Marks

% Almost Ready

\subsection*{Q.3 Probability Distributions (20 Marks)}

\subsubsection*{Part A} % Exponential %6 MARKS
A power supply unit for a computer component is assumed to follow an exponential distribution with a mean life of 1,200 hours.  What is the probability that the component will:
\begin{itemize}
\item [i.](2 Marks)	fail in the first 300 hours?
\item [ii.](2 Marks)	survive more than 1,500 hours?
\item [iii.](2 Marks) last between 1,200 hours and 1,500 hours?
\end{itemize}

\subsubsection*{Part B}% Poisson %6 MARKS
Telephone calls arrive at a switchboard at a rate of 30 per hour.  Assume that the switchboard operators take 2 minutes to deal with a customer query. Calculate the following:

\begin{itemize}
\item [i.](2 Marks)	The probability of 2 or more calls arriving in any 4 minute period,
\item [ii.](2 Marks) The probability of no phone calls arriving in a 4 minute period,
\item [iii.](2 Marks) The probability of exactly three phone call arriving in a 10 minute period.
%\item [iv.]	The average and standard deviation of the number of phone calls arriving in a 2 minute period.

\end{itemize}
\subsubsection*{Part C} %NORMAL %8 MARKS
Assume that the length of injected moulded plastic components are normally distributed with a mean of 12.5mm and a standard deviation of 2.5mm.  Calculate the corresponding probability for the following measurements occurring on an individual component.

\begin{itemize}
\item [i.](2 Marks)	Between 12.5 and 15mms,
\item [ii.](2 Marks) Less than 10 mms,
\item [iii.](2 Marks) Between 12 and 15 mms,
\item [iv.](2 Marks) Less than 10.3 mms.
\end{itemize}
\noindent Illustrate each of your answers with a sketch.
%------------------------------------------------------------------------------------------------ %
\newpage

% Question 4  Hypothesis Testing  + Confidence Intervals
% Good Shape

\subsection*{Q4. Inference Procedures (20 Marks)}

\subsubsection*{Part A} %4 Marks
\begin{itemize}
\item[i.](2 Marks) In the context of hypothesis testing, explain what a p-value is, and how it is used. Support your answer with a simple example.
\item[ii.](2 Marks) What is meant by Type I error and Type II error?
\end{itemize}
\subsubsection*{Part B} %3 Marks
A well-known polling company estimates that $57\%$ of Irish voters support a new constitutional amendment. 800 people were randomly surveyed and asked about their voting preferences. 482 of the 800 people responded positively to the amendment. You are required to:

\begin{itemize}
\item [i.](1 Mark) Obtain a point estimate of the proportion of people supporting the constitutional amendment.
\item [ii.](2 Marks) Construct a 95\% confidence interval for the proportion of people in favour of the constitutional amendment.
\end{itemize}

\subsubsection*{Part C} %4 Marks
The standard deviations of data sets \texttt{X} and \texttt{Y} are 10 and 9 respectively. An inference procedure was carried out to assess whether or not \texttt{X} and \texttt{Y} can be assumed to have equal variance.
\begin{itemize}
\item[i.](1 Mark) Formally state the null and alternative hypothesis.
\item[ii.](1 Mark) The Test Statistic has been omitted from the computer code output. Compute the value of the Test Statistic.
\item[iii.](2 Marks) What is your conclusion for this procedure? Justify your answer.
%\item[iv.] (1 Marks) Explain how a conclusion for this procedure can be based on the $95\%$ confidence interval.
\end{itemize}

%---- R code for Variance Test ----%
%---- Dummy Code Included                   ----%
\begin{framed}
\begin{verbatim}
        F test to compare two variances

data:  X and Y
F = ......, num df = 13, denom df = 11, p-value = 0.7349
alternative hypothesis: true ratio of variances is not equal to 1
95 percent confidence interval:
 0.3639938 3.9475262
sample estimates:
ratio of variances
          .......
\end{verbatim}
\end{framed}

\subsubsection*{Part D} %9 Marks
Two samples of students are randomly selected from two IT training companies; Echelon and Deltatech. The mean and the standard deviation of the number of marks obtained in a well known IT competency exam by both sets of students are described below:\\

\begin{center}
\begin{tabular}{|c|c|c|c|}

  \hline
  % after \\: \hline or \cline{col1-col2} \cline{col3-col4} ...
	&Number&	Mean&	Std. Dev.\\ \hline
DeltaTech	&14	&24	&10\\
Echelon	&12	&22.5	&9\\
  \hline
\end{tabular}
\end{center}

%Calculate a 95\% confidence interval for the difference between the mean number of marks obtained by males and females in the population of school leavers as a whole.
%(7 marks)

Test the hypothesis that Echelon students and DeltaTech students, on average, obtain the same mark in the IT certification exam. Use a significance level of $5\%$. You may assume that any required assumptions have been validated.
% State your hypotheses clearly. What is the significance level of this test?
\bigskip

\begin{itemize}
\item[i.](2 Marks) Formally state the null and alternative hypotheses.
\item[ii.](3 Marks) Compute the Test Statistic.
\item[iii.](2 Marks) State the appropriate Critical Value for this hypothesis test.
\item[iv.](2 Marks) Discuss your conclusion to this test, supporting your statement with reference to appropriate values.
\end{itemize}

%        1-sample proportions test with continuity correction
%
% data:  482 out of 800, null probability 0.57
% X-squared = 3.3162, df = 1, p-value = 0.0686
% alternative hypothesis: true p is not equal to 0.57
% 95 percent confidence interval:
% 0.5675450 0.6364573
% sample estimates:
%     p
% 0.6025

%-------------------------------------------------------------------------------------------------- %
\newpage
\subsection*{Q5. Correlation and Linear Regression (20 Marks)} %NOT READY
% Correlation and Simple Linear Regression
% Non Parametric Procedures

A wood scientist wanted to establish if there was a relationship between the adhesive strength of laminated wood and the dwell time in press machine. A random sample of 9 different times and their corresponding adhesive strengths in pounds per square inch (PSI) were recorded as follows:

\begin{center}
\begin{tabular}{|c|c|c|}

  \hline
Sample &Time (Mins) & Pull Strength (PSI) \\
 & (X)  &  (Y)\\ \hline
1& 5.0& 3.5 \\
2& 4.8& 3.3\\
3& 5.6& 3.9\\
4& 4.3& 2.7\\
5& 4.2& 3.2\\
6& 5.4& 4.1\\
7& 5.5& 4.3\\
8& 4.0& 2.8\\
9& 4.7& 3.7\\
  \hline
\end{tabular}
\bigskip

\begin{tabular}{lll}
  $\sum X = 43.5$ & $\sum Y = 31.5$ & $\sum XY = 154.61$ \\
  $\sum X^2 = 213.03$ & $\sum Y^2 = 112.71$ &  \\
 \end{tabular}
 \end{center}
%> sum(X)
%[1] 43.5
%> sum(Y)
%[1] 31.5
%> sum(X*Y)
%[1] 154.61
%> sum(X^2)
%[1] 213.03
%> sum(Y^2)
%[1] 112.71

\begin{itemize}
\item[i.](5 Marks) Draw a scatter-plot and comment on its features.
\item[ii.](5 Marks) Calculate the correlation coefficient. Interpret your answer.
\item[iii.](5 Marks) Calculate the equation of the least squares regression line and interpret the value of the slope.
\item[iv.](3 Marks) Using this regression model, estimate the adhesive strength for a piece of wood that has been in the press machine for 7 minutes.
\item[v.](2 Marks) Is such an estimate reliable? Briefly explain why.
\end{itemize}



%-------------------------------------------------------------------------------------------------- % \newpage
\newpage
\section*{Formulae}
%-------------------------------------------------%
\subsection*{Descriptive Statistics}
\begin{itemize}
\item Sample Variance
\begin{equation*}
s^2 = \frac{\sum (x-\bar{x})^2}{n-1}
\end{equation*}
\end{itemize}
%-------------------------------------------------%
\subsection*{Probability}
\begin{itemize}

\item Conditional probability:
\begin{equation*}
P(B|A)=\frac{P\left( A\text{ and }B\right) }{P\left( A\right) }.
\end{equation*}


\item Bayes' Theorem:
\begin{equation*}
P(B|A)=\frac{P\left(A|B\right) \times P(B) }{P\left( A\right) }.
\end{equation*}





\item Binomial probability distribution:
\begin{equation*}
P(X = k) = ^{n}C_{k} \times p^{k} \times \left( 1-p\right) ^{n-k}\qquad \left( \text{where  }
^{n}C_{k} =\frac{n!}{k!\left(n-k\right) !}. \right)
\end{equation*}

\item Poisson probability distribution:
\begin{equation*}
P(X = k) =\frac{m^{k}\mathrm{e}^{-m}}{k!}.
\end{equation*}

\item Exponential probability distribution:
\begin{equation*}
P(X \leq k) = \begin{cases}
1-e^{- k/\mu}, & k \ge 0, \\
0, & k < 0.
\end{cases}\qquad \left( \text{where  }
\mu = {1\over \lambda}\right)
\end{equation*}
\end{itemize}



\subsection*{Confidence Intervals}
{\bf One sample}
\begin{eqnarray*} S.E.(\bar{X})&=&\frac{\sigma}{\sqrt{n}}.\\\\
S.E.(\hat{P})&=&\sqrt{\frac{\hat{p}\times(100-\hat{p})}{n}}.\\
\end{eqnarray*}
{\bf Two samples}
\begin{eqnarray*}
S.E.(\bar{X}_1-\bar{X}_2)&=&\sqrt{\frac{\sigma^2_1}{n_1}+\frac{\sigma_2^2}{n_2}}.\\\\
S.E.(\hat{P_1}-\hat{P_2})&=&\sqrt{\frac{\hat{p}_1\times(100-\hat{p}_1)}{n_1}+\frac{\hat{p}_2\times(100-\hat{p}_2)}{n_2}}.\\\\
\end{eqnarray*}
\subsection*{Hypothesis tests}
{\bf One sample}
\begin{eqnarray*}
S.E.(\bar{X})&=&\frac{\sigma}{\sqrt{n}}.\\\\
S.E.(\pi)&=&\sqrt{\frac{\pi\times(100-\pi)}{n}}
\end{eqnarray*}
{\bf Two large independent samples}
\begin{eqnarray*}
S.E.(\bar{X}_1-\bar{X}_2)&=&\sqrt{\frac{\sigma^2_1}{n_1}+\frac{\sigma_2^2}{n_2}}.\\\\
S.E.(\hat{P_1}-\hat{P_2})&=&\sqrt{\left(\bar{p}\times(100-\bar{p})\right)\left(\frac{1}{n_1}+\frac{1}{n_2}\right)}.\\
\end{eqnarray*}
{\bf Two small independent samples}
\begin{eqnarray*}
S.E.(\bar{X}_1-\bar{X}_2)&=&\sqrt{s_p^2\left(\frac{1}{n_1}+\frac{1}{n_2}\right)}.\\\\
s_p^2&=&\frac{s_1^2(n_1-1)+s_2^2(n_2-1)}{n_1+n_2-2}.\\
\end{eqnarray*}
{\bf Paired sample}
\begin{eqnarray*}
S.E.(\bar{d})&=&\frac{s_d}{\sqrt{n}}.\\\\
\end{eqnarray*}
{\bf Standard Deviation of case-wise differences (computational formula)}
\begin{eqnarray*}
s_d = \sqrt{ {\sum d_i^2 - n\bar{d}^2 \over n-1}}.\\\\
\end{eqnarray*}


\noindent{\bf Regression estimates}

\begin{eqnarray*}
S_{XY} &=&
\sum x_iy_i - \frac{\sum x_i\sum y_i}{n}\\
S_{XX} &=&
\sum x_i^2 - \frac{(\sum x_i)^2}{n}\\
S_{YY} &=&
\sum y_i^2 - \frac{(\sum y_i)^2}{n}\\
\end{eqnarray*}
{\bf Slope Estimate}
\begin{eqnarray*}
b_1 = \frac{S_{XY}}{S_{XX}}
\end{eqnarray*}
{\bf Intercept Estimate}
\begin{eqnarray*}
 b_0 = \bar{y} -b_1\bar{x}
\end{eqnarray*}
{\bf Pearson's correlation coefficient}

\begin{eqnarray*}
r = \frac{S_{XY}}{\sqrt{S_{XX} \times S_{YY}}}
\end{eqnarray*}
{\bf Standard error of the Slope}
\begin{eqnarray*}
S.E.(b1) = \sqrt{\frac{s^2}{S_{XX}}}
\end{eqnarray*}

where $s^2 = \frac{SSE}{n-2}$
and SSE $= S_{YY} - b_1S_{XY}$
\end{document} 