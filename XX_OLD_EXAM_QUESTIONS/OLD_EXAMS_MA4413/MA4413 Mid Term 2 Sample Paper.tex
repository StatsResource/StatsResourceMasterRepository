\documentclass[]{article}

%opening
\title{}
\author{}

\begin{document}





\section*{MA4413 Mid Term 2 Sample Paper}

\subsection*{Q1. Theory for Inference Procedures (4 Marks)}
Answer the four short questions. Each correct answer will be awarded 1 mark.
\begin{itemize}
\item[(i)] (1 Mark) What is a $p-$value?
\item[(ii)] (1 Mark) Briefly describe how $p-$value is used in hypothesis testing
\item[(iii)] (1 Mark) What is meant by a Type I error?
\item[(iv)] (1 Mark) What is meant by a Type II error?
\end{itemize}

\subsection*{Q2. Normal Distribution (5 Marks)}
An analogue signal received at a detector (measured in microvolts), is normally distributed with a mean of 100 microvolts 
and a standard deviation of 25 microvolts.

\begin{itemize}
\item[(i)] (1 Mark) What is the Z-score for 137.5 microvolts? 
\item[(ii)] (1 Mark) What is the Z-score for 80 microvolts?
\item[(iii)] (1 Mark) What is the probability that the signal will exceed 137.5 microvolts? 
\item[(iv)](1 Mark) What is the probability that the signal will be less than 80 microvolts?
\item[(v)] (1 Mark) What is the probability that the signal will be between 80 and 125 microvolts?
\end{itemize}

%\item[(iii)] What is the micro-voltage below which 25% of the signals will be? \item[(4 marks)]

\newpage
\subsection*{Q3. Inference Procedures (6 Marks)}


\textbf{Part A:} A sample of 200 voters was taken by a political pollster to estimate the proportion of first preference votes a 
particular candidate will obtain in a forthcoming election. 
It was found that 110 out of these 200 voters would give the candidate their first preference.


\begin{itemize}
\item[(i)] (1 Mark) State the point estimate that would be used in an inference procedure.
\item[(ii)] (1 Mark) Compute the standard error that would correspond to the point estimate you have computed.
\item[(iii)] (1 Mark) Determine the 95\% confidence interval for your point estimate.
\end{itemize}

\noindent \textbf{Part B:} Using a significance level of 5\%, test the hypothesis that the percentage of voters who will give this 
particular candidate their first preference in the election is 60\%.{\tiny } 

\begin{itemize}
\item[(i)] (1 Mark) Formally state the null and alternative hypotheses.
\item[(ii)] (1 Mark) Compute the Test Statistic for this hypothesis test.
\item[(iii)] (1 Mark) Given that the critical value is 1.96, state your conclusion for this test.
\end{itemize}

\end{document}
