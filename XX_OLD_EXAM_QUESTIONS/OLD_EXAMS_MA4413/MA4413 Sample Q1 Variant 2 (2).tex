
\documentclass[]{article}
\usepackage{framed}
\usepackage{amsmath}
\usepackage{amssymb}
\usepackage{multicol}
%opening

\begin{document}
%------------------------------------------------------------------------------------------------ %
\newpage
\section*{Question 1 (Sample Variant 1)[25 marks]}
\begin{itemize}
\item[(a)] \textbf{\textit{Discrete Random Variables (4 Marks)}}\\ 
The probability distribute of discrete random variable $X$ is tabulated below. There are 5 possible outcome of $X$, i.e. 1, 2, 4, 6 and 8.
\begin{center}
\begin{tabular}{|c||c|c|c|c|c|}
\hline
$x_i$  & 1 & 2 & 4 & 6 & 8  \\\hline
$p(x_i)$ & 0.50 & 0.15 & 0.20 & 0.05 & 0.10 \\
\hline
\end{tabular}
\end{center}

\begin{itemize}
%\item[a.] (1 Mark) Compute the value of $k$.
\item[i.] (2 Mark) What is the expected value of X?
%\item[c.] (1 Mark) Compute the value of $E(X^2)$
\item[ii.] (2 Mark) Given that $E(X^2) = 12.5$, compute the variance of $X$.
\end{itemize}

%------------------------------------------------------------------------- %
\item[(b)] \textbf{\textit{Probability (7 Marks)}}\\ 
On completion of a programming project, four programmers from a team submit a collection of subroutines to an acceptance group. The following table shows the percentage of subroutines each programmer submitted and the probability that a subroutine submitted by each programmer will pass the certification test based on historical data.
\begin{center}
\begin{tabular}{|c||c|c|c|c|}
\hline
Programmer	&A	&B	&C	&D\\
Proportion of subroutines submitted &	0.10&	0.20&	0.30&	0.40\\
Probability of acceptance	&0.55	&0.60	&0.95&	0.75\\
\hline
\end{tabular}
\end{center}
\begin{itemize}
\item[i.](4 Marks) What is the proportion of subroutines that pass the acceptance test?
\item[ii.](3 Marks) After the acceptance tests are completed, one of the subroutines is selected at random and found to have passed the test. What is the probability that it was written by Programmer A?
\end{itemize}
%------------------------------------------------------------------------- %
\item[(c)] \textbf{\textit{Descriptive Statistics (7 Marks)}}\\ 
Consider the following data set of seven numbers:

\begin{center}
\textbf{\texttt{29 14 17 30 19 25 13}}
\end{center}
% 4 Marks

\noindent For this sample, compute the following descriptive statistics:
\begin{itemize}
%\item[a.] (1 Mark) The median,
\item[i.] (1 Mark) The mean,
\item[ii.] (2 Mark) The variance,
\item[iii.] (1 Mark) The standard deviation.
\end{itemize}
%------------------------------------------------------------------------- %
\newpage
\item[(d)] \textbf{\textit{Probability (6 Marks)}}\\ 
A doctor treating a patient issues a prescription for antibiotics and provides for two repeat prescriptions. The probability that the infection will be cleared by the first prescription is $p_1$ =0.6.
The probability that successive treatments are successful, given that previous prescriptions were not successful are $p_2$ = 0.5, $p_3$ = 0.4. Calculate the probability that:

\begin{itemize}
\item[i.](2 Marks) a patient will require the third prescription,
\item[ii.](2 Marks) the patient is still infected after the third prescription,
\item[iii.](2 Marks) the patient is cured by the second prescription, given that the patient is eventually cured.
\end{itemize}

%------------------------------------------------------------------------- %
\end{itemize}
\end{document}