\documentclass[]{report}

\voffset=-1.5cm
\oddsidemargin=0.0cm
\textwidth = 480pt

\usepackage{framed}
\usepackage{subfiles}
\usepackage{graphics}
\usepackage{newlfont}
\usepackage{eurosym}
\usepackage{amsmath,amsthm,amsfonts}
\usepackage{amsmath}
\usepackage{color}
\usepackage{amssymb}
\usepackage{multicol}
\usepackage[dvipsnames]{xcolor}
\usepackage{graphicx}
\begin{document}

%--------------------------------------------------------------------------------------%

\section*{Hypothesis Test of Sample Mean: Small Sample Example}

\begin{itemize}
\item A quality control laboratory are testing the life spans of a brand of ultraviolet tubes.
\item The life spans of a sample of ten ultraviolet tubes were measured. The average life span of this sample was found to be 8,800 hours.
\[\bar{x} = 8,800 \mbox{hrs} \]
\item The standard deviation of the life for this sample was found to $s = 500$ hrs.
\item Also it is assumed that the operating life of the tubes is normally distributed. 


\item The manufacturer claims that average tube life for this brand
is 9,000hr. \item Test this claim at the 5 percent level of significance against the alternative hypothesis
that the mean life is 9,000 hr.
%\item (Intuitively this would suggest a one-tailed test that the mean is less than 9000 hours)
\end{itemize}
\section*{Solution}
\noindent \textbf{Step 1:} \\ Formally write our the Null and Alternative Hypotheses.\\
We are testing a hypothesis about the population mean $\mu$.

\begin{itemize}
\item $H_0 \mbox{ : } $ $\mu = 9000$ \\ (Average life span is 9000 hours.)\bigskip
\item $H_1 \mbox{ : } $ $\mu \neq 9000$\\ (Average life span is not 9000 hours.)
\end{itemize}
\bigskip
(Remark : this is a two-tailed procedure.
)



%--------------------------------------------------------------------------------------%


\noindent  \textbf{Step 2:} \\ Compute the Test Statistic (TS)
{

\[TS = \frac{\bar{x} - \mu}{{\sigma \over \sqrt{n}}}\]
}
\begin{itemize}
\item The difference between point estimate ($\bar{x}$) and expected value supposed by the null hypothesis ($\mu$) is -200 hours. \\(i.e. 8,800 - 9,000 hours)
\end{itemize}

%--------------------------------------------------------------------------------------%

\begin{itemize}
\item The standard error is determined from formulas.
\item \textbf{Important:} We use the sample standard deviation $s$ as an estimate for the population standard deviation $\sigma$, which is unknown.
\[ S.E. (\bar{x}) = {s \over \sqrt{n}} = {500 \over \sqrt{10}}  = 158.11 \]

\[ TS\; =\; \frac{-200-0}{158.11} =\; -1.265\]
\item The test statistic is $-1.265$
\end{itemize}



\noindent \textbf{Step 3:} \\ Determine the Critical Value (CV)
\begin{itemize}

\item The Critical Value is determined with $\alpha = 0.05$ and $k = 2$ .
\item The sample is small (i.e.n = 10) \\ Therefore degrees of freedom $df = n-1 = 9$.
\item From Murdoch Barnes statistical tables, we would find the critical value to be: 
\[CV = 2.262\].

\item (\textit{Remark: If the sample was large, we could use $CV = 1.96$}).

\end{itemize}


\noindent  \textbf{Step 4:} \\ Decision Rule
\begin{itemize}

\item \textbf{Decision:}Is the absolute value of the test statistic greater than the critical value?
IS $|TS| > CV$? \\ Is $1.265 > 2.262$?
\item No. We fail to reject the null hypothesis. \item There is not enough evidence to say that the mean lifespan is not 9000 hours.
\end{itemize}

\end{document}

