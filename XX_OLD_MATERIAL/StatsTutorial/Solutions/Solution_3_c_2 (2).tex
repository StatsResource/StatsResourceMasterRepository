\documentclass[]{report}

\voffset=-1.5cm
\oddsidemargin=0.0cm
\textwidth = 480pt

\usepackage{framed}
\usepackage{subfiles}
\usepackage{graphics}
\usepackage{newlfont}
\usepackage{eurosym}
\usepackage{amsmath,amsthm,amsfonts}
\usepackage{enumerate}
\usepackage{amsmath}
\usepackage{color}
\usepackage{amssymb}
\usepackage{multicol}
\usepackage[dvipsnames]{xcolor}
\usepackage{graphicx}
\begin{document}


\subsection*{Question 5 - Independent Events}
The probability that A hits a target is 1/3 and the probability that B hits a target is 1/5.  They both fire at the target. Find the probability that:

Remark: Use a Probability Tree for this exercise/
\begin{itemize}
\item[(a)] A does not hit the target, 
\[\Pr(A^c) = 1 - 1/3 = 2/3 \]
\item[(b)] both hit the target, 
\[\Pr(A \cap B)  = P(A) \times P(B) = 1/3 \times 1/5 = 1/15\] 
\item[(c)] only one of them hits the target,\\

Combination of [``A hits + B misses"] + [``A misses + B hits"]

\[P(\mbox{only one hits}) =  [P(A)\times P(B^c) ]+ [P(A^c)\times P(B) ]  \]
\[= [1/3 \times 4/5] + [2/3 \times 1/5 ]\]
\[= 4/15 + 2/15 = \boldsymbol{6/15}\]

\item[(e)] Neither hits the target.
\\ Remark: skip the ordering deliberately, Explain that this is the complement of what is asked in Part d
\[\Pr(A^c \cap B^c)  = P(A^c) \times P(B^c) = 2/3 \times 4/5 = 12/15\] 
\item[(d)] at least one hits the target
\[1- \Pr(A^c \cap B^c) = 3/15 =1/5\]
\end{itemize}
%==================================================================%
\end{document}
