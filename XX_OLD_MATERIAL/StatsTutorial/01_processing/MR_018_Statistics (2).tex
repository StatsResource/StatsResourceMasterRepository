
\documentclass[a4paper,12pt]{article}
%%%%%%%%%%%%%%%%%%%%%%%%%%%%%%%%%%%%%%%%%%%%%%%%%%%%%%%%%%%%%%%%%%%%%%%%%%%%%%%%%%%%%%%%%%%%%%%%%%%%%%%%%%%%%%%%%%%%%%%%%%%%%%%%%%%%%%%%%%%%%%%%%%%%%%%%%%%%%%%%%%%%%%%%%%%%%%%%%%%%%%%%%%%%%%%%%%%%%%%%%%%%%%%%%%%%%%%%%%%%%%%%%%%%%%%%%%%%%%%%%%%%%%%%%%%%
\usepackage{eurosym}
\usepackage{vmargin}
\usepackage{amsmath}
\usepackage{framed}
\usepackage{graphics}
\usepackage{epsfig}
\usepackage{amsmath}
\usepackage{amssymb}
\usepackage{subfigure}
\usepackage{multicol}
\usepackage{enumerate}
\usepackage{fancyhdr}

\setcounter{MaxMatrixCols}{10}
%TCIDATA{OutputFilter=LATEX.DLL}
%TCIDATA{Version=5.00.0.2570}
%TCIDATA{<META NAME="SaveForMode"CONTENT="1">}
%TCIDATA{LastRevised=Wednesday, February 23, 201113:24:34}
%TCIDATA{<META NAME="GraphicsSave" CONTENT="32">}
%TCIDATA{Language=American English}

\pagestyle{fancy}
\setmarginsrb{20mm}{0mm}{20mm}{25mm}{12mm}{11mm}{0mm}{11mm}
\lhead{MathsResource} \chead{Statistics} \rhead{Descriptive Statistics} %\input{tcilatex}
\begin{document}

\section*{Descriptive Statistics Tutorial Sheet}
\begin{enumerate}
	\item
\begin{enumerate}
\item The exam results for a class of 60 students are tabulated below.
\begin{table}[ht]
	\centering
	\begin{tabular}{|rrrrrrrrrr|}
		\hline
		
		19 &  25 &  30 &  35 &  35 &  36 &  36 &  37 &  37 &  38 \\ 
		38 &  38 &  39 &  39 &  40 &  43 &  43 &  43 &  44 &  45 \\ 
		46 &  47 &  47 &  47 &  47 &  47 &  48 &  48 &  49 &  49 \\ 
		50 &  51 &  52 &  53 &  53 &  53 &  54 &  56 &  57 &  57 \\ 
		59 &  60 &  60 &  60 &  61 &  62 &  63 &  63 &  64 &  64 \\ 
		65 &  66 &  69 &  72 &  78 &  85 &  88 &  89 &  93 &  99 \\ 
		\hline
	\end{tabular}
\end{table}

\begin{enumerate}[(i)]
	\item  Summarize the data in the above table using a relative frequency table and a cumulative relative frequency table. Use 8 class intervals, with 11 as the lower limit of the first interval.
	\item  Draw a histogram for the above data. Comment on the shape of the histogram. Based on the shape of the histogram, what is the best measure of centrality and variability?
	\item  Construct a box plot for the above data. Clearly demonstrate how all of the necessary values were computed.
\end{enumerate}


\item Data on the durations (measured in months) were collected for a random sample of product development projects.
The durations for these development projects were collected and tabulated as follows:

\begin{table}[ht]
	\centering
	\begin{tabular}{|rrrrrrrr|}
		\hline
		
		16 &  20 &  14 &  29 &  30 &  22 &  21 &  28 \\ 
		\hline
	\end{tabular}
\end{table}	


\begin{enumerate}[(i)]
	\item  Calculate the mean, median, variance, and sample standard deviation for the project duration times.
\end{enumerate}

 \hspace{\fill}%\textbf{[3 marks x 3]} 
\item 
Use the Dixon Q Test to determine if the lowest value is an outlier. You may assume a significance level of 5\%.
\[ 131, 136, 101, 126, 123, 120, 132, 137\]
\begin{itemize}
	\item[(i)]	State the null and alternative hypotheses for this test.
	\item[(ii)] Compute the test statistic
	\item[(iii)] State the appropriate critical value.
	\item[(iv)] What is your conclusion to this procedure.
\end{itemize}	

\end{enumerate}




\item 
\begin{enumerate}

\medskip
\noindent Use statistical tables to determine the probabilities for the above exercises. You are required to show all of your workings.
\item[] 

\item A new test has been developed to diagnose a particular disease. If a person has the disease, the test has a 95\% chance of identifying them as having the disease. 
If a person does not have the disease, the test has a 1\% chance of identifying them as having the disease. Suppose that 5\% of the population have this disease. Suppose we select a person at random from the population.


\begin{enumerate}[(i)]
	\item  What is the probability that the test will identify them as having the disease?
	
	\item  What is the probability that the person has the disease given that the test identifies them as having the disease?
	\item  What is the probability that the person has the disease given that the test identifies them as \textbf{not} having the disease?
\end{enumerate}


\item Consider the results of a statistical analysis carried on both of the sample data set $Y$. These results are presented as output from a statistical computer program.

\begin{itemize}
	\item[(i)]  What sort of analysis are we carrying out? 
	\item[(ii)]  What is the relevance of this analysis as part of an overall statistical study.
	\item[(iii)]  What is the conclusion of this analysis for the Variable $Y$? Justify your answer with reference to 3 separate indications.
\end{itemize}
%\bigskip
%\textit{\textbf{Important:} Question 3 comprises a third part: Part C. This part is presented in subsequent pages.}

\begin{figure}[h!]
	\centering
	\includegraphics[width=0.90\linewidth]{images/NormalityTesting3}
\end{figure}

\begin{figure}[h!]
	\centering
	\includegraphics[width=0.90\linewidth]{images/NormalityTesting4}
\end{figure}
\end{enumerate}

\item
\begin{enumerate}
	\item Mean blood iron concentration for children with adequate nutrition is taken to be 110mg/dl. 25 randomly selected children from a disadvantaged urban area were given blood tests. The mean concentration of iron from this sample was 98 mg/dl with a standard deviation of 25.5 mg/dl. \smallskip
	\begin{itemize}
		\item[(i)]  Calculate a 95\% confidence interval for the mean concentration of iron for children in this area. 
		\item[(ii)] Interpret this confidence interval.  Do these results provide evidence that children in this area suffer from iron deficiency? 
	\end{itemize}
	\medskip
	Test this hypothesis using a 5\% level of significance. 
	
	\begin{itemize}
		\item[(iii)] Formally state your null and alternative hypotheses.
		\item[(iv)] Compute the test statistic.
		\item[(v)] Discuss your conclusion to this test, supporting your statement with reference to appropriate values.
	\end{itemize}
	
	\item An environmental group states that fewer than 60\% of industrial plants comply with air pollution standards? An independent researcher takes a sample of 400 plants and finds that 270 are complying with air pollution standards. Carry out a hypothesis test to investigate the claim made by the environmental group. Clearly state your null and alternative hypotheses and your conclusion.
	\begin{itemize}
		\item[(i)] Compute the 95\% confidence interval.
		%	\item[(ii)] Compute the Test Statistic.
		\item[(ii)] By interpreting this confidence interval, state your conclusion about the environmental group's claim? Explain how you made this decision.
	\end{itemize}
\end{enumerate}

\item
\begin{enumerate}
	\item An exercise physiologist wants to determine if several short bouts of exercise provide the same benefit for cardiovascular fitness as one long bout of exercise. \\ \smallskip
	
	\noindent 50 volunteers are randomly assigned to group 1 and do standardised aerobic exercise on a stationary bicycle for 30 minutes once a day, 5 days a week. 40 volunteers are randomly assigned to group 2 and do the same exercise for 10 minutes, 3 times a day, 5 days a week. Cardiovascular fitness was measured by VO2 max (maximum oxygen consumption while exercising). 
	
	\begin{description}
		\item[Group 1] The mean change in VO2 after 12 weeks of exercise was 2.1 for group 1 with a standard deviation of 1.7.
		\item[Group 2] The mean change in VO2 after 12 weeks of exercise was 0.7 for group 2 with a standard deviation of 1. 
	\end{description}
	
	\noindent Test the hypothesis that there is no significant difference between two groups are the same. You may assume a 5\% level of significance.
	
	\begin{itemize}
		\item[(i)] Formally state your null and alternative hypotheses.
		\item[(ii)] Compute the test statistic.
		\item[(iii)] Discuss your conclusion to this test, supporting your statement with reference to appropriate values. You may assume a 5\% level of significance.
	\end{itemize}
	\item A microbiologist measures the total growth in 24 hours of two strains of a germ culture  in the same petri dish. Nine identical specimens are prepared. The growth rate for both each specimen, with the growth rate for both specimens, is tabulated below.
	
	\begin{center}
		\begin{tabular}{|c|c|c|} \hline 
			Specimen &	Strain 1	&	Strain 2	\\ \hline \hline
			1 & 212 & 224 \\ \hline
			2 & 234 & 231 \\ \hline
			3 & 214 & 209 \\ \hline
			4 & 236 & 243 \\ \hline
			5 & 221 & 231 \\ \hline 
			6 & 212 & 216 \\ \hline
			7 & 202 & 213 \\ \hline 
			8 & 210 & 216 \\ \hline
			9 & 248 & 242 \\ \hline
		\end{tabular} 
	\end{center}
	\noindent At a significance level of 5\%, is there sufficient evidence to state that there is any difference in growth rates between the two strains.
	
	
	% State your hypotheses clearly. What is the significance level of this test?
	\bigskip
	
	\begin{itemize}
		\item[(i)] Formally state the null and alternative hypotheses.
		\item[(ii)]  Compute the mean and standard deviation of the case-wise differences.
		\item[(iii)] Compute the test statistic.
		\item[(iv)] State the appropriate critical value for this hypothesis test. 
		\item[(v)] Discuss your conclusion to this test, supporting your statement with reference to appropriate values.
	\end{itemize}
\end{enumerate}

%==============================================================================%
\item 
\begin{enumerate}
\item A market research survey was carried out to assess preferences for three brands of chocolate bar, A, B, and C. 
The study group was categorised by gender to determine any difference in preferences.


{
	\begin{center}
		\begin{tabular}{|c||c|c|c||c|} \hline
			&	Brand A	&	Brand B	&	Brand C	&	Total	\\ \hline		\hline
			Children	&	65	&	55	&	30	&	150	\\ \hline	
			Teenages	&	35	&	75	&	40	&	150	\\ \hline	
			Adults	&	50	&	20	&	30	&	100	\\ \hline	\hline
			&	150	&	150	&	100	&	400	\\ \hline	 
		\end{tabular} 
	\end{center}
}
\begin{enumerate}[(i)]
	\item  Formally state the null and alternative hypotheses.
	\item  Compute each of the cell values expected under the null hypothesis. 
	\item  Compute the test statistic.
	\item  State the appropriate vritical value for this hypothesis test.
	\item  Discuss your conclusion to this test, supporting your statement with reference to appropriate values. You may assume a 5\% level of significance.
\end{enumerate}

% \phantom{0} \hspace{\fill}\textbf{[7 marks]} 

\end{enumerate}



\item An experiment was conducted to study the relationship between baking temperature x (in units of 10 degrees Farenheit) and yield y (as percentage) of popular cake mix. Fourteen observations were made giving the following results.

%Temp 10 10.5 11 11.5 12 15 17 19 20 21 23 25 27 30

%Yield 21.2 19.9 22.5 23.7 25 30.3 36.1 38.6 41.5 42.7 45 50 53.9 62.1



\begin{center}
	\begin{tabular}{|c||c|c|c|c|c|c|c|}
		\hline
		Specimen & 1 & 2 & 3 & 4 & 5 & 6 & 7 \\ \hline
		\hline
		Temp &  10.00 & 10.50 & 11.00 & 11.50 & 12.00 & 15.00 & 17.00 \\ \hline 
		Yield &  29.830&  26.370 & 30.325 &36.100 &33.410 &36.335 & 40.655 \\ \hline 
		\hline\hline
		Specimen & 8 & 9 & 10 & 11 & 12 & 13 & 14 \\  \hline
		Temp &  19.00 & 20.00 & 21.00 & 23.00 & 25.00 & 27.00 & 30.00 \\ \hline
		Yield &  39.475& 44.555&  42.015 & 47.795 & 45.560&  51.045 & 47.900 \\ \hline 
		\hline
	\end{tabular}
\end{center}

\begin{multicols}{3}
	\begin{itemize}
		\item $n=14$
		\item $S_{XX} = 570.5$
		\item $S_{YY} =  751.1525$
		\item $S_{XY} = 613.27$
		\item $\bar{X} = 18$
		\item $\bar{Y} = 39.383$
	\end{itemize}
\end{multicols}


\begin{figure}
	\centering
	\includegraphics[width=0.99\linewidth]{images/MA4505RegressionPlot}
\end{figure}
\medskip
\begin{enumerate}[(i)]
	\item  Using the scatter plot, describe the relationship between the yield (Y) and the baking temperature (X).
	\item  Calculate the correlation coefficient. Interpret your answer.
	\item  Calculate the equation of the least squares regression line and interpret the value of the slope.
	\item  Using this regression model, estimate the yield when the baking temperature is 16 degrees.
	\item  How much of the variation in yield is explained by fitting the regression line?
\end{enumerate}



%
%\begin{figure}
%\centering
%\includegraphics[width=0.99\linewidth]{images/MA4505RegressionResiduals}
%\caption{}
%\label{fig:MA4505RegressionResiduals}
%\end{figure}



\end{enumerate}




\end{document}



