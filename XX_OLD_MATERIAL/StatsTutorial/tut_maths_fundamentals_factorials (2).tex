\documentclass[]{report}

\voffset=-1.5cm
\oddsidemargin=0.0cm
\textwidth = 480pt

\usepackage{framed}
\usepackage{subfiles}
\usepackage{graphics}
\usepackage{newlfont}
\usepackage{eurosym}
\usepackage{amsmath,amsthm,amsfonts}
\usepackage{amsmath}
\usepackage{color}
\usepackage{amssymb}
\usepackage{multicol}
\usepackage[dvipsnames]{xcolor}
\usepackage{graphicx}
\begin{document}

\section*{Example: Factorials }


\begin{framed}
\noindent Equation of Factorials 
		\[ n! = (n)\times (n-1)\times(n-2) \times \ldots \times 1 \]
\end{framed}
\begin{itemize}
	\item $1! = 1$
	\item $0! = 1 $
\end{itemize}

\section*{Example}
	
\[\]

%%- \frametitle{Factorials Numbers}


\begin{itemize}
	\item $3!  = 3 \times 2  \times 1 = 6 $
		\item $4! = 4 \times 3 \times 2 \times 1 = 24$
		\item $5! = 5 \times 4 \times 3 \times 2 \times 1 = 120 $

		\item $7! = 7 \times 6 \times 5 \times 4 \times 3 \times 2 \times 1 = 5,040$
		\item Zero factorial : Remark $0! = 1$ not $0$.

	\[ 0! =  1 \]


	
	
\end{itemize}	

\begin{framed}
Importantly 
\begin{eqnarray}
n! &=& n \times (n-1)! \\  &=& n \times (n-1) \times (n-2)!
\\  &=& \ldots
\end{eqnarray}
\end{framed}
\begin{itemize}
	\item $6! = 6 \times 5!  = 6 \times 5 \times 4!$ 
\end{itemize}
\subsection*{Binomial Coefficients}

\[ \binom 5 2  = \frac{5!}{2!\;(5-2)!} = \frac{5.4.3!}{2! .3!} = \frac{5.4}{2.1} = 10\]


\[ \binom 5 0   = \frac{5!}{0!\;(5-0)!} = \frac{5!}{0! .5!} = \frac{5!}{2!} = 1\]
Recall $0! =1$



\end{document}