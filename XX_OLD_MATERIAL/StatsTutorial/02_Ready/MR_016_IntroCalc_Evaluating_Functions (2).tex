\documentclass[a4paper,12pt]{article}
%%%%%%%%%%%%%%%%%%%%%%%%%%%%%%%%%%%%%%%%%%%%%%%%%%%%%%%%%%%%%%%%%%%%%%%%%%%%%%%%%%%%%%%%%%%%%%%%%%%%%%%%%%%%%%%%%%%%%%%%%%%%%%%%%%%%%%%%%%%%%%%%%%%%%%%%%%%%%%%%%%%%%%%%%%%%%%%%%%%%%%%%%%%%%%%%%%%%%%%%%%%%%%%%%%%%%%%%%%%%%%%%%%%%%%%%%%%%%%%%%%%%%%%%%%%%
\usepackage{eurosym}
\usepackage{multicol}
\usepackage{vmargin}
\usepackage{amsmath}
\usepackage{amssymb}
\usepackage{framed}
\usepackage{graphics}
\usepackage{epsfig}
\usepackage{subfigure}
\usepackage{enumerate}
\usepackage{fancyhdr}

\setcounter{MaxMatrixCols}{10}
%TCIDATA{OutputFilter=LATEX.DLL}
%TCIDATA{Version=5.00.0.2570}
%TCIDATA{<META NAME="SaveForMode"CONTENT="1">}
%TCIDATA{LastRevised=Wednesday, February 23, 201113:24:34}
%TCIDATA{<META NAME="GraphicsSave" CONTENT="32">}
%TCIDATA{Language=American English}

\pagestyle{fancy}
\setmarginsrb{20mm}{0mm}{20mm}{25mm}{12mm}{11mm}{0mm}{11mm}
\lhead{MathsResource} \chead{Introduction to Calculus} \rhead{Tutorial Sheets} %\input{tcilatex}
\begin{document}
	

\begin{enumerate}

\item 
	Solve the following equations for A and B where $A,B \in \mathbb{R}$
	\begin{multicols}{2}
	\[ \mbox{(a)     } \frac{11}{x^2 - 4x - 12} = \frac{A}{x-6} + \frac{B}{x+2}\]
	
	\[ \mbox{(b)    }\frac{2x + 5}{x^2 - 4x - 12} = \frac{A}{x-6} + \frac{B}{x+2}\]
	
	\[ \mbox{(c)     }  \frac{1}{(n)(n+1)} = \frac{A}{n} + \frac{B}{n+1}\]
	
	\[ \mbox{(d)    }\frac{2}{(n+1)(n+3)} = \frac{A}{n+1} + \frac{B}{n+3}\]
	\end{multicols}
%---------------------------------------------------------- %

\item 	Complete the following table.
	\begin{center}
		{ 
			\begin{tabular}{|c|c|c|c|} \hline
				Value x & Floor $\lfloor x\rfloor$ & Ceiling  $\lceil x\rceil$ & Fractional $ \{ x \} $\\ \hline
				-1.4  &	-2	&-1&	 \\ \hline
				2.3&	&		& \\ \hline
				7/9&		&		& \\ \hline 
				-16/3&		&		& \\ \hline
				0 & 	&		& 0 \\ \hline
				1 & 	&	1	& \\
				\hline
			\end{tabular} 
		}
	\end{center}
\item 
	Provide some values for $x$ and $y$ that \textbf{contradict} the following statement.
	\[ \lfloor x + y \rfloor  = \lfloor x\rfloor + \lfloor y \rfloor\]
	
	\noindent If the values of x and y were integers, would the equation be true for all values of $x$ and $y$?
	
\item Express the following numbers as fractions. For example $ 0.77777... = \frac{7}{9}$

\begin{multicols}{2}
	\begin{itemize}
		\item[(i)] $0.29292929....$
		\item[(ii)] $0.475475475....$
		
		\item[(iii)] $0.4545454545.....$
		\item[(iv)] $0.473473473.........$
	\end{itemize} 
\end{multicols}


		\item Evaluate the following function for x = -1,0,1 and 2 respectively.
		{

			\[ f(x) = \frac{e^x - {e^{-x}}}{2} \]
		}
		\item Evaluate the function for each of the following values : $0.5,\;1,\;1.25,\;2$.
		{
			\Large
			\[f(x) =  \sqrt{1+e^{x}}  \]
		}

\item  Evaluate the following function for x = -1,0,1 and 2 respectively.
		{

			\[ f(x) = \frac{e^x - {e^{-x}}}{2} \]
		}
		\item Evaluate the function for each of the following values : $0.5,\;1,\;1.25,\;2$.
		{

			\[f(x) =  \sqrt{1+e^{x}}  \]
		}

\item Provide some values for $x$ and $y$ that \textbf{contradict} the following statement.
	\[ \lfloor x + y \rfloor  = \lfloor x\rfloor + \lfloor y \rfloor\]
	
	\noindent If the values of x and y were integers, would the equation be true for all values of $x$ and $y$?


\item  Use the Laws of Logarithms to evaluate the following expressions:

	\begin{multicols}{2}
		\begin{itemize}
		\item[(i)] $\mbox{log}_2(8)$
		\item[(ii)] $\mbox{log}_2(\sqrt{128})$
		\item[(iii)] $\mbox{log}_2(64)$
		\item[(iv)] $\mbox{log}_5(125)$ +   $\mbox{log}_3(729)$
		\item[(v)] $\mbox{log}_2(64/4)$
		\item[(vi)] $\mbox{log}_3(\frac{1}{81})$
				\end{itemize}
		\end{multicols}


	\item Determine the values of A and B from the following expression
	\[  \frac{7}{x^2-x-12} = \frac{A}{x+3} + \frac{B}{x-4}\]
	\item  Determine whether or not the function \[f(x) = x cos(x)\] is odd, even or neither.

\item Consider the functions $f(x) = \sqrt{2x-6}$ and  $g(x) = \log_e(2x + 1)$

\begin{itemize}
	\item[(a)]Find $f(4 - 2x^2)$ and simplify answer.
	\item[(b)] Write down the domain and range of f(x).
	\item[(c)] Determine $g^{-1}(x)$, the inverse of g(x).
\end{itemize}

	\item  Evaluate the function for the values of  $ x = \{0.25, 0.5 , 0.75 \}$
	
	\[  f(x) = \sqrt{1+x^2} \]
	

	
	\item Find the value of $x$ in each of the following equations.

\begin{multicols}{2}
	\begin{enumerate}[(a)]
\item $\log_3(x + 1) + \log_3(5) = 5$
\item $e^{2x-5} = 3. $
\item $ln(e^x+2) = 4$
	\item $\log_3(2x - 1) + \log_3(5) = 3$
\end{enumerate}
\end{multicols}	

\item  Determine if the function $f(x) = x^4 + x^2$ is an even function, an odd function or neither. Justify your answer.



	

\end{enumerate}




\end{document}

