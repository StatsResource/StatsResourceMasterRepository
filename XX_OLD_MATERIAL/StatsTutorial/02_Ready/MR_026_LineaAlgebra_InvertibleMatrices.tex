\documentclass[a4paper,12pt]{article}
%%%%%%%%%%%%%%%%%%%%%%%%%%%%%%%%%%%%%%%%%%%%%%%%%%%%%%%%%%%%%%%%%%%%%%%%%%%%%%%%%%%%%%%%%%%%%%%%%%%%%%%%%%%%%%%%%%%%%%%%%%%%%%%%%%%%%%%%%%%%%%%%%%%%%%%%%%%%%%%%%%%%%%%%%%%%%%%%%%%%%%%%%%%%%%%%%%%%%%%%%%%%%%%%%%%%%%%%%%%%%%%%%%%%%%%%%%%%%%%%%%%%%%%%%%%%
\usepackage{eurosym}
\usepackage{vmargin}
\usepackage{amsmath}
\usepackage{framed}
\usepackage{graphics}
\usepackage{epsfig}
\usepackage{subfigure}
\usepackage{enumerate}
\usepackage{fancyhdr}
\usepackage{multicol}

\setcounter{MaxMatrixCols}{10}
%TCIDATA{OutputFilter=LATEX.DLL}
%TCIDATA{Version=5.00.0.2570}
%TCIDATA{<META NAME="SaveForMode"CONTENT="1">}
%TCIDATA{LastRevised=Wednesday, February 23, 201113:24:34}
%TCIDATA{<META NAME="GraphicsSave" CONTENT="32">}
%TCIDATA{Language=American English}

\pagestyle{fancy}
\setmarginsrb{20mm}{0mm}{20mm}{25mm}{12mm}{11mm}{0mm}{11mm}
\lhead{MathsResource} \chead{Linear Algebra} \rhead{Invertible Matrices} %\input{tcilatex}
\begin{document}
\section*{Linear Algebra Tutorial Sheet : Invertible Matrices}
\begin{enumerate}
\item The Fundamental Theorem of Invertible Matrices states that a set of mathematical expressions concerning an $n\times n$ matrix $A$ are each equivalent to one another.

\begin{itemize}
\item[(i)] 
State any four of these expressions.
\item[(ii)]  What is the trace of a square matrix
%\item[(iii)]  What is mean by the the rank of a matrix.
\end{itemize}

\medskip
\item 
In this question, you are required to find the inverse of the following matrix using elementary row operations.

\begin{equation*}
A=\left( \begin{array}{rrr}
-2  &  -2  &  -2\\
2  &  3  &  2\\
3  & -2  &  5\\
\end{array} \right)
\end{equation*}


\begin{itemize}
\item[(i)]  Write down the augmented matrix of this system. %\marks{4}

%\item[(ii)]  What can you say about the solution set of the system? Justify your answer. %\marks{4}

\item[(ii)] Find the inverse of the matrix, using elementary row operations. Show your workings for each stage of the calculation.
\end{itemize}
\item 
In this question, you are required to find the inverse of the following matrix using elementary row operations.

\begin{equation*}
A=\left( \begin{array}{rrr}
  1 &  -4   & -3 \\
  1 &   3   & 5 \\
  -2 &   0   & -4 \\
\end{array} \right)
\end{equation*}


\begin{itemize}
\item[(i)]  Write down the augmented matrix of this system. %\marks{4}

%\item[(ii)]  What can you say about the solution set of the system? Justify your answer. %\marks{4}

\item[(ii)]  Find the inverse of the matrix, using elementary row operations. Show your workings for each stage of the calculation.
\end{itemize}

\item 
Suppose that the inverse of the following matrix $M$ is given as $M^{1}$: 
\[M = \left(\begin{array}{rrr}
2    & 2  &  2 \\
4    & 0 &  -2\\
-6   & -2 &   2\\
\end{array}\right)\;\;\;\;M^{-1} = \left(\begin{array}{rrr}
0.25 & 0.5  & 0.25\\
-0.25 & -1.0& -0.75\\
 0.50 & 0.5 & 0.50\\
\end{array}\right)\]

\begin{itemize}
\item[(i)]  State the inverse of the matrix $N$ where $N = 2M$.
\end{itemize}
\[
N = 2M = \left(\begin{array}{rrr}
4 &   4  &  4 \\
8 &   0  & -4\\
-12 &  -4  &  4\\
\end{array}\right)
\]

%%%%%%%%%%%%%%%%%%%%%%%%%%%%%%%%%%%%%%%%%%%%%%%%%%


\item Consider the matrix $B$ specified as
\begin{equation*}
B=\left( \begin{array}{rrr}
-4  &  3  & -2 \\
-2 &   3 &   4\\
-1 &   1  &  0\\
\end{array} \right).
\end{equation*}
\begin{itemize}
\item[(i)]  For each element of $B$, calculate the corresponding minor. Show your workings for each calculation. 
State the matrix of minors.
\item[(ii)]  Hence or otherwise, compute the determinant of $B$ i.e. $\det(B)$.
\item[(iii)]  Compute the cofactor matrix for $B$ i.e. $\operatorname{cof}(B)$.
\item[(iv)]  State the inverse matrix of $B$, given by
\[ B^{-1}=\frac{1}{\det(B)}  \operatorname{cof}(B)^T. \]
\end{itemize}
\item Show that if $A$ is an $n\times n$ invertible matrix that satisfies 
$$
9A^3+A^2-3A=0
$$
where $A^n=\underbrace{A\ldots A}_{\textrm{$n$ times}}$, %_{\underbrace\textrm$n $ times$, 
$I$ is the $n\times n$  identity matrix and $0$ is the $n\times n$  zero matrix,
then the inverse of $A$ is given by  %\marks{4}
$$
A^{-1}=\frac13I+3A.
$$
\item 
Consider the following diagonal matrix D. Provide answers for the following questions in terms of the values $a$, $b$ and $c$.


\[D = \left(\begin{array}{ccc}
a & 0 & 0 \\ 
0 & b & 0 \\ 
0 & 0 & c
\end{array} \right)\]
\begin{itemize}
\item[(i)] (1 Mark) Write an expression for the trace of the matrix D.
\item[(ii)] (1 Mark) State the inverse of $D$, i.e. $D^{-1}$.
\item[(iii)] (1 Mark) State the matrix $D^3$.
\end{itemize}
\smallskip
\end{enumerate}
\end{document}
