\documentclass[a4paper,12pt]{article}
%%%%%%%%%%%%%%%%%%%%%%%%%%%%%%%%%%%%%%%%%%%%%%%%%%%%%%%%%%%%%%%%%%%%%%%%%%%%%%%%%%%%%%%%%%%%%%%%%%%%%%%%%%%%%%%%%%%%%%%%%%%%%%%%%%%%%%%%%%%%%%%%%%%%%%%%%%%%%%%%%%%%%%%%%%%%%%%%%%%%%%%%%%%%%%%%%%%%%%%%%%%%%%%%%%%%%%%%%%%%%%%%%%%%%%%%%%%%%%%%%%%%%%%%%%%%
\usepackage{eurosym}
\usepackage{vmargin}
\usepackage{amsmath}
\usepackage{framed}
\usepackage{graphics}
\usepackage{epsfig}
\usepackage{subfigure}
\usepackage{enumerate}
\usepackage{fancyhdr}

\setcounter{MaxMatrixCols}{10}
%TCIDATA{OutputFilter=LATEX.DLL}
%TCIDATA{Version=5.00.0.2570}
%TCIDATA{<META NAME="SaveForMode"CONTENT="1">}
%TCIDATA{LastRevised=Wednesday, February 23, 201113:24:34}
%TCIDATA{<META NAME="GraphicsSave" CONTENT="32">}
%TCIDATA{Language=American English}

\pagestyle{fancy}
\setmarginsrb{20mm}{0mm}{20mm}{25mm}{12mm}{11mm}{0mm}{11mm}
\lhead{MathsResource} \chead{Probability} \rhead{Tutorial Sheets} %\input{tcilatex}
\begin{document}
%%%%%%%%%%%%%%%%%%%%%%%%%%%%%%%%%%%%%%%%%%%%%%%%%%%%%%%%%%%%%%%%%%%%


\begin{enumerate}
%=================================================================================%
%% RSS HC2 2014 Question 2. 
 
\item The continuous random variable X has probability density function (pdf) $f(x)$ given by

\[ f(x) = \frac{k}{x^4},  \mbox{ where} x= \geq \theta\]
where $\theta$ is a positive constant.
\begin{enumerate}[(a)]
\item  Find k in terms of $\theta$. For the case $\theta$ = 1, sketch the graph of $f(x)$, marking the value of $f(1)$ on your graph. (6)
\item  Show that $E(X) = \frac{3}{2}\theta$ and $Var(X) = \frac{3}{4} \theta^2$.
 
\item Let X denote the mean of n independent random variables, $X_1, X_2,\ldots X_n$, each of which has the pdf $f(x)$.
\begin{enumerate}[(i)]
\item Write down an approximation to the distribution of X based on the central limit theorem. How would you expect the success of the approximation to vary with $n$? 
\item Use this distribution to show that $$P\left( |X - E(X)| \; \leq \; 1.96 \theta \sqrt{\frac{3}{4n}} \right)$$ approximately. Hence find an approximation to the least value of n such that $$P\left( |X - E(X)| \; \leq \; 0.1 \theta  \right) \geq 0.95 . $$
\end{enumerate}
\end{enumerate}

%%%%%%%%%%%%%%%%%%%%%%%%%%%%%%%%%%%%%%%
%% RSS HC2 2014 Question 2.
    \item 

 The random variable Y has probability density function
\[f ( y) = k( y - y^3 ) \qquad  \mbox{ where } \;\;0 \leq y \leq 2, \]
and zero otherwise, where k is a positive constant.
\begin{enumerate}[(a)]
\item Show that $k = 1/6$.
\item Show that the cumulative distribution function is
  \[
    f(y) = \begin{cases}
        0, & \text{for } y \leq 0\\
        \frac{y^2}{12}\left( \frac{y^2+2}{2} \right)  & \text{for } 0 < y < 2\\
        1, & \text{for }  y\geq 2
        \end{cases} 
  \]
\item 
Hence find $P( 1/2 < Y < 3/2)$?
\item  Find the variance of $Y$.
\end{enumerate}

%%%%%%%%%%%%%%%%%%%%%%%%%%%%%%%%%%%%%%

%%- RSS - HC2 - 2008 Question 2 
\item  The continuous random variable $X$ has probability density function given by

$$f_X(x) = c(1-x^2), \mbox{ where } –1\leq x \leq 1,$$

where $$c$$ is a suitable constant.

\begin{enumerate}[(a)]

\item Show that $c = 3/4$ and plot the graph of $f_X(x)$ against $x$.

\item Show that the cumulative distribution function of X is given by

 \[
    f(x) = \begin{cases}
        0, & \text{for } x \leq -1\\
        \frac{2 + 3x - x^2}{4}  & \text{for } -1 < x < 1\\
        1, & \text{for }  x\geq 1
        \end{cases} 
  \]


Also find $P(-1/2 \leq X \leq 1/2 )$.

\item  Obtain the standard deviation of $X$, giving your answer correct to 3 significant figures.

\end{enumerate}


%%%%%%%%%%%%%%%%%%%%%%%%%%%%%%%%%%%%%%%%%%%%%%%%%%%%%%%%%%%%%%%%%%%%%% RSS HC 2016 Q4 A
\item The continuous random variable X has probability density function
\[f ( x) = \alpha (1-x)^{\alpha-1} , \;\;\; 0 < x <1, \alpha > 0 .\]
\begin{enumerate}[(a)]
\item  Find the cumulative distribution function, $F(x)$, of X.
\item  Find $P(0.25 < X < 0.75)$.
\item  Use $F(x)$ to obtain the median of $X$.
\end{enumerate}
%%%%%%%%%%%%%%%%%%%%%%%%%%%%%%%%%%
%% RSS HC 2016 Q4 B
\item The continuous random variable Y has probability density function given by
\[f ( y)  = 9 ye^{-3y} , \;\; y \geq 0 .\]
\begin{enumerate}[(a)]
\item Obtain $E(Y)$ .
\item What is $P(Y < 3)$?
\end{enumerate}
%%%%%%%%%%%%%%%%%%%%%%%%%%%%%%%%%%%%%
\end{enumerate}
\end{document}
