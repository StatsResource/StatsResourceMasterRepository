\documentclass[a4paper,12pt]{article}
%%%%%%%%%%%%%%%%%%%%%%%%%%%%%%%%%%%%%%%%%%%%%%%%%%%%%%%%%%%%%%%%%%%%%%%%%%%%%%%%%%%%%%%%%%%%%%%%%%%%%%%%%%%%%%%%%%%%%%%%%%%%%%%%%%%%%%%%%%%%%%%%%%%%%%%%%%%%%%%%%%%%%%%%%%%%%%%%%%%%%%%%%%%%%%%%%%%%%%%%%%%%%%%%%%%%%%%%%%%%%%%%%%%%%%%%%%%%%%%%%%%%%%%%%%%%
\usepackage{eurosym}
\usepackage{vmargin}
\usepackage{amsmath}
\usepackage{framed}
\usepackage{graphics}
\usepackage{epsfig}
\usepackage{subfigure}
\usepackage{enumerate}
\usepackage{fancyhdr}
\usepackage{multicol}

\setcounter{MaxMatrixCols}{10}
%TCIDATA{OutputFilter=LATEX.DLL}
%TCIDATA{Version=5.00.0.2570}
%TCIDATA{<META NAME="SaveForMode"CONTENT="1">}
%TCIDATA{LastRevised=Wednesday, February 23, 201113:24:34}
%TCIDATA{<META NAME="GraphicsSave" CONTENT="32">}
%TCIDATA{Language=American English}

\pagestyle{fancy}
\setmarginsrb{20mm}{0mm}{20mm}{25mm}{12mm}{11mm}{0mm}{11mm}
\lhead{MathsResource} \chead{Probability} \rhead{Tutorial Sheets} %\input{tcilatex}
\begin{document}
%================================================================================%

\begin{enumerate}
% RSS  - HC2 - 2007 - Question 1 
\item 100 men are surveyed as to whether they play cricket, tennis or golf. It is found that

\begin{multicols}{2}
\begin{itemize}
\item  10 play none of these sports

\item  5 play all three of these sports

\item  88 play cricket or tennis or both

\item  78 play cricket or golf or both

\item  30 play golf and tennis but not cricket

\item  38 play golf

\item  74 play tennis.
\end{itemize}

\end{multicols}
Using a Venn diagram, or otherwise, find the following.

\begin{enumerate}[(a)]

\item The number of the men who play at least one of these sports. 
\item The number of the men who play exactly one of these sports. 
\item The number of the men who play exactly two of these sports. 
\item Of those who do not play golf, the proportion who play cricket. 

\item The mean number of sports played by these men. 

\end{enumerate}

%====================================================================================%

% -  2008

% - http://www.hkss.org.hk/images/exam/papers/Past/2008/2008_HC_2_HKSS.pdf

\item A Personal Identification Number (PIN) consists of four digits in order, each of which may be any one of $0, 1, 2, \ldots, 9$.

\begin{enumerate}
\item Find the number of PINs satisfying each of the following requirements.

\begin{enumerate}[(a)]
\item All four digits are different.
\item There are exactly three different digits.
\item There are two different digits, each of which occurs twice.
\item There are exactly three digits the same.
\end{enumerate}
\item Two PINs are chosen independently and at random, and you are given that each PIN consists of four different digits. Let X be the random variable denoting the number of digits that the two PINs have in common.

\begin{enumerate}[(a)]
\item Explain clearly why \[P(X = k) = \frac{{4 \choose k} {6 \choose 4-k}}{{10 \choose k}}  ,\] for k = 0, 1, 2, 3, 4.




\item Hence write down the values of the probability mass function of X, and find its mean.

\end{enumerate}
\end{enumerate}
%%%%%%%%%%%%%%%%%%%%%%%
% -  2008

% - http://www.hkss.org.hk/images/exam/papers/Past/2008/2008_HC_2_HKSS.pdf

%%%%%%%%%%%%%%%%%%%%%%%%
\item A combination lock consists of four rings each labelled with the digits 1, 2, 3, 4, 5, 6. The rings may be rotated individually and independently, so that all 4-digit combinations of the digits $1, \ldots, 6$ (with repetition) can be shown. A customer buys such a lock. The instructions that come with the lock give the correct combination for opening the lock and state that this combination has been chosen at random from all possible combinations.
\begin{enumerate}[(a)]
\item Evaluate k, the total number of combinations that can be shown. 
\item Find the probability that the purchased lock has a combination
\begin{enumerate}[(i)]
\item with all digits equal, 
\item with all digits different, 
\item with a pair of digits equal, the other two digits being different from each other and from the pair, (6)
\item with exactly three digits equal, 
\item with two pairs of equal digits (but not all four digits the same).
\end{enumerate}
\end{enumerate}
\end{enumerate}
\end{document}
