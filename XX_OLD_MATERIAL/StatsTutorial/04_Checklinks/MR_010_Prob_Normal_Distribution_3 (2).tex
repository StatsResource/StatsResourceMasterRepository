\documentclass[a4paper,12pt]{article}
%%%%%%%%%%%%%%%%%%%%%%%%%%%%%%%%%%%%%%%%%%%%%%%%%%%%%%%%%%%%%%%%%%%%%%%%%%%%%%%%%%%%%%%%%%%%%%%%%%%%%%%%%%%%%%%%%%%%%%%%%%%%%%%%%%%%%%%%%%%%%%%%%%%%%%%%%%%%%%%%%%%%%%%%%%%%%%%%%%%%%%%%%%%%%%%%%%%%%%%%%%%%%%%%%%%%%%%%%%%%%%%%%%%%%%%%%%%%%%%%%%%%%%%%%%%%
\usepackage{eurosym}
\usepackage{vmargin}
\usepackage{amsmath}
\usepackage{framed}
\usepackage{graphics}
\usepackage{epsfig}
\usepackage{subfigure}
\usepackage{enumerate}
\usepackage{fancyhdr}

\setcounter{MaxMatrixCols}{10}
%TCIDATA{OutputFilter=LATEX.DLL}
%TCIDATA{Version=5.00.0.2570}
%TCIDATA{<META NAME="SaveForMode"CONTENT="1">}
%TCIDATA{LastRevised=Wednesday, February 23, 201113:24:34}
%TCIDATA{<META NAME="GraphicsSave" CONTENT="32">}
%TCIDATA{Language=American English}

\pagestyle{fancy}
\setmarginsrb{20mm}{0mm}{20mm}{25mm}{12mm}{11mm}{0mm}{11mm}
\lhead{MathsResource} \chead{Probability} \rhead{Tutorial Sheets} %\input{tcilatex}
\begin{document}

%%%%%%%%%%%%%%%%%%%%%%%%%%%%%%%%%%%%%%%%%%%%%%%%%%%%%%%%%%%%%%%%%%

\begin{enumerate}
%==================================================================================================%

\item 
%%- RSS HC2 2012. 

\begin{enumerate}
\item The random variable X has a Normal distribution with mean 3 and variance 16.
Find $P(X > –2)$.
\item The random variable Y has a Normal distribution with mean 2 and variance 1. X and Y are independent. Find the distribution of $W = X – 3Y$, and find $P(W > 0)$.
\item The independent random variables $Y_1$, $Y_2$ and $Y_3$ have the same distribution
as $Y$, and the random variable V is defined as $X – Y_1 – Y_2 – Y_3$. Find $P(V > 0)$.
\item The independent random variables $X_1, X_2,\ldots, X_n$ have the same distribution
as $X$. 
Write down in terms of n the distribution of the mean
$$ \bar{X} = \frac{X_1 + X_2 + X_3 + \ldots + X_n}{n} $$
and find the least value of n such that $P(\bar{X} > 0) > 0.9995$ .
\end{enumerate}    
    
%%%%%%%%%%%%%%%%%%%%%%%%%%%%%%%%%%

%%- RSS - HC2 - 2008 - Question 3

\item 
Let $X$ and $Y$ be independent standard Normal random variables and let $\Theta(.)$ denote the cumulative distribution function of a standard Normal random variable.
\begin{enumerate}[(a)]
    \item Find $P(3X > 4Y + 2)$, and write down $P(X \leq x, Y \leq x)$ in terms of $\Theta(x)$.
\item  Let $W = max(X, Y)$.

\begin{enumerate}[(i)]
    \item Explain why the cumulative distribution function of W is given by
$$ F_{W}(w) = \left[  \Theta(x) \right]^{2} $$ where $-infty \leq w \leq infty$.

\item  Find Q1 and Q3, the lower and upper quartiles of W.
\end{enumerate}

\item A random sample of 100 observations of W is taken. Write down the distribution of the number N of observations in the sample which lie outside the interval $(Q1, Q3)$. Use a suitable approximation to calculate $P(N \geq 58)$.


\end{enumerate}

%%%%%%%%%%%%%%%%%%%%%%%%%%%%%%%%%%%%%%%    
\item Let $X$ and $Y$ be independent standard Normal random variables and let $\Theta(.)$ denote the cumulative distribution function of the standard Normal random variable.
\begin{enumerate}[(a)]
\item Write down the distribution of $4X – 3Y$ and hence find $P(4X > 3Y + 2)$.

\item  Let $W = max(X, Y)$.

\begin{enumerate}[(a)]
\item Write down $P(X \leq x, Y \leq x)$ in terms of $\Theta(x)$ and hence explain why the cumulative distribution function of W is given by
$$ F_{W}(w) = \left[  \Theta(x) \right]^{2} $$ where $-infty \leq w \leq infty$.

\item  Find Q1 and Q3, the lower and upper quartiles of W.
\end{enumerate}
\item  A random sample of 400 observations of $W$ is taken. Write down the distribution of the number $K$ of observations in the sample that lie within the interval $(Q1, Q3)$. Use a suitable approximation to calculate $P(K \leq 210)$.
\end{enumerate}

%%%%%%%%%%%%%%%%%%%%%%%%%%%

\end{enumerate}
\end{document}
