\documentclass[a4paper,12pt]{article}
%%%%%%%%%%%%%%%%%%%%%%%%%%%%%%%%%%%%%%%%%%%%%%%%%%%%%%%%%%%%%%%%%%%%%%%%%%%%%%%%%%%%%%%%%%%%%%%%%%%%%%%%%%%%%%%%%%%%%%%%%%%%%%%%%%%%%%%%%%%%%%%%%%%%%%%%%%%%%%%%%%%%%%%%%%%%%%%%%%%%%%%%%%%%%%%%%%%%%%%%%%%%%%%%%%%%%%%%%%%%%%%%%%%%%%%%%%%%%%%%%%%%%%%%%%%%
\usepackage{eurosym}
\usepackage{vmargin}
\usepackage{amsmath}
\usepackage{framed}
\usepackage{graphics}
\usepackage{epsfig}
\usepackage{subfigure}
\usepackage{enumerate}
\usepackage{fancyhdr}
\usepackage{multicol}

\setcounter{MaxMatrixCols}{10}
%TCIDATA{OutputFilter=LATEX.DLL}
%TCIDATA{Version=5.00.0.2570}
%TCIDATA{<META NAME="SaveForMode"CONTENT="1">}
%TCIDATA{LastRevised=Wednesday, February 23, 201113:24:34}
%TCIDATA{<META NAME="GraphicsSave" CONTENT="32">}
%TCIDATA{Language=American English}

\pagestyle{fancy}
\setmarginsrb{20mm}{0mm}{20mm}{25mm}{12mm}{11mm}{0mm}{11mm}
\lhead{MathsResource} \chead{Probability} \rhead{Tutorial Sheets} %\input{tcilatex}
\begin{document}
%================================================================================%

\begin{enumerate}
%=================================================================================%
%%- RSS HC2 - 2014 Question 3
\item The probability that a given character is miscopied when I send an email is 0.001, independently of all other characters.
\begin{enumerate}[(a)]
\item If I send an email of 2000 characters, state
\begin{enumerate}[(i)]
\item  the exact distribution,
\item a suitable approximate distribution,
\end{enumerate}
for the number, X, of miscopied characters. Use the approximate distribution to find $P(X = 0)$ and $P(X > 2)$.

\item If I send a second email, consisting of 3000 characters and independent of the first, state corresponding approximate distributions
\begin{enumerate}[(i)]
\item for the number, Y, of miscopied characters in the second email,
\item for the total number, Z, of miscopied characters in the two emails combined.
\end{enumerate}
Use this distribution of Z to find $P(Z = 4)$ and then use the approximate distributions of X, Y and Z to find the conditional probability $P(X = 2 | Z = 4)$.

\item In the course of a week I send 50 emails, all independent and consisting of 100000 characters in total. State
\begin{enumerate}[(i)]
\item the exact distribution,
\item a suitable approximation,
\end{enumerate}
for the total number, W, of miscopied characters. Use the approximate distribution to find $P(W > 115)$.

\end{enumerate}
%====================================================================%
%%-- RSS HC2 -- 2013 - Question 3
\item The random variable X has the binomial distribution with probability mass function

\[ p(x) = \frac{6!}{x!(6-x)!}  \left( \frac{2}{5}\right) ^{x}  \left( \frac{3}{5}\right) ^{6-x}, \]

where $x =\{0,1,2,3\ldots,6\}$.

\begin{enumerate}
\item Derive E(X) and Var(X).
\item Find $P(X \leq 1)$
\begin{itemize}
\item[(a)] exactly,
\item[(b)] by using a Normal approximation with a suitable continuity correction.
\end{itemize}

\item State circumstances under which a Normal approximation to the binomial distribution might be useful, and comment on your results.
\item Let $\bar{X}$ be the mean of a random sample of size 400 taken from the distribution of X. Calculate $Var(X)$, and use a Normal approximation to the distribution of $\bar{X}$ to find $P(2.35< \bar{X} \leq 2.45)$. State with a reason whether or not you would expect your answer to be a good approximation to the exact probability.
\end{enumerate}

%%%%%%%%%%%%%%%%%%%%%%%%%%%%%%%%%%%%%%%%%%

%%- RSS -  HC2 - Question 1
\item  The random variable X has the binomial distribution with probability mass function
$$ {\displaystyle \Pr(X=x)={2 \choose x}p^{x}(1-p)^{2-x}} $$ where $x = \{0, 1, 2\}$ and $0 < p < 1$.


\begin{enumerate}[(a)]
\item Write down $E(X)$, $Var(X)$ and $P(X = 2)$ in terms of the parameter $p$. Also find $P(X = 0 | X < 2)$ and $P(X = 1 | X < 2)$, simplifying your answers as far as possible.

\item Let $Y = X_1 + X_2 + \ldots + X_{100}$ be the sum of 100 independent random variables, each distributed as X.
\begin{enumerate}[(i)]
\item Explain why Y has the $B(200, p)$ distribution.

\item Use a suitable approximation to find $P(Y > 140)$ when $p = 2/3$.
\item Use a suitable approximation to find $P(Y > 2)$ when $p = 0.02$.

\item Use a suitable approximation to find $P(Y \leq 197)$ when $p = 0.98$.
\end{enumerate}
\end{enumerate}

%===================================================================%
%%- RSS - HC2 - 2010 - Queston 2
\item XYZ airline operates a baggage weight allowance of 25 kg per passenger. Check-in records show that the actual weight, W kg, of a randomly chosen passenger's baggage can reasonably be assumed to be Normally distributed with mean 24 and variance 1.
\begin{enumerate}[(a)]
\item Find $P(W > 25)$.

\item  A passenger with baggage weighing more than 25 kg is charged £5 for each kg by which the weight of his or her baggage exceeds 25 kg, all fractions of a kg being rounded up to the next whole number.

\begin{enumerate}[(i)]
\item If $C$ denotes the excess baggage charge in £ for a randomly chosen passenger, find the probabilities $P(C = 0)$, $P(C = 5)$ and $P(C = 10)$.
\item Given that $P(C = 15) = 0.0013$ approximately and that $P(C > 15)$ is negligible, find $E(C)$ and $Var(C)$.
\item Assuming that 100 000 passengers independently fly with XYZ in a year, write down the mean and variance of CT, the total excess baggage costs paid to XYZ in a year. Use a Normal approximation to find the value of CT which is exceeded with probability 0.05.
\item Comment briefly on the assumptions made in your calculations in part (iii).
\end{enumerate}
\end{enumerate}
%================================================================%
%% RSS - HC2 - 2009 or 2010 
\item A blended wine is intended to comprise two parts of Sauvignon to one part of Merlot.
The amounts dispensed to make up a nominal 75cl bottle of this wine are X cl of
Sauvignon and Y cl of Merlot, where X and Y are assumed to be independent Normally
distributed random variables with respective means 52 and 26 cl and respective
variances 1 and 0.5625.
\begin{enumerate}[(a)]
\item Find the probability that the actual volume of wine dispensed into a bottle is
less than the nominal volume.
\item  Find the distribution of X – 2.2Y, and use this distribution to find the
probability that the ratio of Sauvignon to Merlot is greater than 2.2. By a
similar method, find the probability that this ratio is less than 1.8. Hence state
to three decimal places the probability that the ratio of Sauvignon to Merlot in
a randomly chosen bottle differs from 2 to 1 by more than 10%.

\item Based on your final answer to part (b), and assuming that bottles are filled
independently, write down
\begin{enumerate}[(i)]
\item the exact distribution, 
\item a suitable
approximation,
\end{enumerate} for the number of bottles in a thousand in which the ratio of
Sauvignon to Merlot differs from 2 to 1 by more than 10\%. Hence find the
approximate probability that there are 10 or more bottles in a consignment of
1000 in which the ratio of Sauvignon to Merlot differs from 2 to 1 by more
than 10\%, giving your answer to three decimal places.
\end{enumerate}


%============================================================================%

\item A coin has probability $p$ of showing heads and probability $1 − p$ of showing tails when it is tossed, independently each time.
Let X be the random variable denoting the number of times the coin shows heads when it is tossed n times.
\begin{enumerate}[(a)]
\item Show that

\[{\displaystyle \Pr(X=x)={n \choose x}p^{x}(1-p)^{n-x}}\]

making clear all the steps of your reasoning. Under what conditions can the distribution of X be approximated by a Normal distribution?



\item A student uses the Normal approximation to approximate $P(X \leq 3)$ when $n = 20$ and $p = 0.2$. Calculate the answer he should obtain, use tables of the exact distribution of X to compute the percentage error in the answer, and comment briefly.



\item For integer $x \geq 1$, let N be the random variable denoting the number of tosses of the coin needed to obtain $x$ heads. Show from first principles that


\[{\displaystyle \Pr(N=n)={n-1 \choose x-1}p^{x}(1-p)^{n-x}}\]

where $n = x,x+1,x+2, \ldots$.
\item Evaluate this probability for the case p = 0.2, x = 3 and n = 20, and compare your result with the exact $P(X = 3)$ for the binomial distribution with the same values of $p$, $x$ and $n$.

\end{enumerate}

\end{enumerate}

\end{document}
