

\documentclass[]{report}

\voffset=-1.5cm
\oddsidemargin=0.0cm
\textwidth = 480pt

\usepackage{framed}
\usepackage{subfiles}
\usepackage{graphics}
\usepackage{newlfont}
\usepackage{eurosym}
\usepackage{amsmath,amsthm,amsfonts}
\usepackage{amsmath}
\usepackage{color}
\usepackage{amssymb}
\usepackage{multicol}
\usepackage[dvipsnames]{xcolor}
\usepackage{graphicx}
\begin{document}






Questions on Unit Circles in Trigonometry 
Multiple choice questions on unit circle in trigonometry with answers at the bottom of the page. 
\begin{enumerate}
\item Which of the following points is in the unit circle? 

a) (-√2 / 2 , -√2 / 2) 
b) (√2 / 3 , -√2 / 3) 
c) (1 / 2 , 1 / 2) 
d) (3 / 2 , 2 / 3) 


Question A point is in Quadrant-III and on the Unit Circle. If its x-coordinate is -4 / 5, what is the y-coordinate of the point? 

a) 3 / 5 
b) -3 / 5 
c) -2 / 5 
d) 5 / 3 

Question Find the point on the Unit Circle associated with the rotation -9π/2 

a) (0 , -1) 
b) (0 , 1) 
c) (1 , 0) 
d) (-1 , 0) 


Question Find the point on the Unit Circle associated with the angle 5π/3 

a) (1 / 2 , 1 / 2) 
b) (-√3 / 2 , 1/2) 
c) (1 / 2 , -√3 / 2) 
d) (-√3 / 2 , -1/2) 

Question If point (a , b) is on the Unit Circle associated with the rotation t, which of the following is not correct? 

a) sin(t) = b 
b) cos(t) = a 
c) sin(-t) = - a 
d) cos(-t) = - a 

Question If point (a , b) is on the Unit Circle associated with the rotation t and point (c , d) is also on the Unit circle associated with rotation t + π, then which of the following is correct? 

a) c = - a and d = - b 
b) c = - a and d = b 
c) c = a and d = b 
d) c = a and d = - b 
 

Question If point (a , b) is on the Unit Circle associated with the rotation t, then the point on the Unit circle associated with rotation t + π / 2, has the following coordinates 

a) (b , a) 
b) (-b , a) 
c) (-b , -a) 
d) (-a , b) 


More 
Trigonometry Problems 

ANSWERS 
a) 
b) 
d) 
c) 
d) 
a) 
b) 



\end{enumerate}

\end{document}
