\documentclass[a4paper,12pt]{article}
%%%%%%%%%%%%%%%%%%%%%%%%%%%%%%%%%%%%%%%%%%%%%%%%%%%%%%%%%%%%%%%%%%%%%%%%%%%%%%%%%%%%%%%%%%%%%%%%%%%%%%%%%%%%%%%%%%%%%%%%%%%%%%%%%%%%%%%%%%%%%%%%%%%%%%%%%%%%%%%%%%%%%%%%%%%%%%%%%%%%%%%%%%%%%%%%%%%%%%%%%%%%%%%%%%%%%%%%%%%%%%%%%%%%%%%%%%%%%%%%%%%%%%%%%%%%
\usepackage{eurosym}
\usepackage{vmargin}
\usepackage{amsmath}
\usepackage{framed}
\usepackage{multicol}
\usepackage{graphics}
\usepackage{epsfig}
\usepackage{subfigure}
\usepackage{enumerate}
\usepackage{fancyhdr}

\setcounter{MaxMatrixCols}{10}
%TCIDATA{OutputFilter=LATEX.DLL}
%TCIDATA{Version=5.00.0.2570}
%TCIDATA{<META NAME="SaveForMode"CONTENT="1">}
%TCIDATA{LastRevised=Wednesday, February 23, 201113:24:34}
%TCIDATA{<META NAME="GraphicsSave" CONTENT="32">}
%TCIDATA{Language=American English}

\pagestyle{fancy}
\setmarginsrb{20mm}{0mm}{20mm}{25mm}{12mm}{11mm}{0mm}{11mm}
\lhead{MathsResource} \chead{Statistical Computing Procedures} \rhead{Tutorial Sheet} %\input{tcilatex}
\begin{document}
%=====================================================%

\begin{enumerate}

\item Suppose that the results of an experimental procedure resulted in the collection of datasets $X$ and $Y$. Consider the following inference procedure performed on data set $X$.
\begin{center}
\begin{framed}
\begin{verbatim}
> shapiro.test(X)

        Shapiro-Wilk normality test

data:  X
W = 0.77516, p-value = 0.0003767
\end{verbatim}
\end{framed}
\end{center}


\begin{itemize}
	\item[(i)] Describe the purpose of this procedure.
	\item[(ii)] What is the null and alternative hypothesis?
	\item[(iii)] What is your conclusion about this procedure?
\end{itemize}
\smallskip

\item A graphical procedure was carried out to assess whether or not this assumption of normality is valid for data set \texttt{Y}. Consider the Q-Q plot in the figure below.

\begin{center}
	\includegraphics[scale=0.40]{images/qqplot2}
\end{center}

\begin{enumerate}[(a)]
	\item Provide a brief description on how to interpret this plot.
	\item What is your conclusion for this procedure? Justify your answer.
\end{enumerate}

\newpage
\item Suppose that the results of an experimental procedure resulted in the collection of data sets $X$ and $Y$. Consider the following inference procedure performed on data set $X$.
\begin{center}
\begin{framed}
	\begin{verbatim}
	> shapiro.test(X)
        Shapiro-Wilk normality test

        data:  X
        W = 0.91554, p-value = 0.001633
	\end{verbatim}
\end{framed}
\end{center}
\begin{enumerate}[(a)]
	\item Describe the purpose of this procedure.
	\item What is the null and alternative hypothesis?
	\item What is your conclusion about this procedure?
\end{enumerate}

\item Suppose that the results of an experimental procedure resulted in the collection of datasets $X$ and $Y$. Consider the following inference procedure performed on data set $X$.
\begin{center}
\begin{framed}
\begin{verbatim}
> shapiro.test(X)

        Shapiro-Wilk normality test

data:  X
W = 0.77516, p-value = 0.0003767
\end{verbatim}
\end{framed}
\end{center}


\begin{enumerate}[(a)]
	\item Describe the purpose of this procedure.
	\item What is the null and alternative hypothesis?
	\item What is your conclusion about this procedure?
\end{enumerate}
\newpage
\item A graphical procedure was carried out to assess whether or not this assumption of normality is valid for data set \texttt{Y}. Consider the Q-Q plot in the figure below.

\begin{center}
	\includegraphics[scale=0.40]{images/qqplot2}
\end{center}

\begin{enumerate}[(a)]
	\item Provide a brief description on how to interpret this plot.
	\item  What is your conclusion for this procedure? Justify your answer.
\end{enumerate}


\end{document}
