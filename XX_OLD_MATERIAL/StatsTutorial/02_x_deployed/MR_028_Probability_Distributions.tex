\documentclass[a4paper,12pt]{article}
%%%%%%%%%%%%%%%%%%%%%%%%%%%%%%%%%%%%%%%%%%%%%%%%%%%%%%%%%%%%%%%%%%%%%%%%%%%%%%%%%%%%%%%%%%%%%%%%%%%%%%%%%%%%%%%%%%%%%%%%%%%%%%%%%%%%%%%%%%%%%%%%%%%%%%%%%%%%%%%%%%%%%%%%%%%%%%%%%%%%%%%%%%%%%%%%%%%%%%%%%%%%%%%%%%%%%%%%%%%%%%%%%%%%%%%%%%%%%%%%%%%%%%%%%%%%
\usepackage{eurosym}
\usepackage{vmargin}
\usepackage{amsmath}
\usepackage{framed}
\usepackage{multicol}
\usepackage{graphics}
\usepackage{epsfig}
\usepackage{subfigure}
\usepackage{enumerate}
\usepackage{fancyhdr}

\setcounter{MaxMatrixCols}{10}
%TCIDATA{OutputFilter=LATEX.DLL}
%TCIDATA{Version=5.00.0.2570}
%TCIDATA{<META NAME="SaveForMode"CONTENT="1">}
%TCIDATA{LastRevised=Wednesday, February 23, 201113:24:34}
%TCIDATA{<META NAME="GraphicsSave" CONTENT="32">}
%TCIDATA{Language=American English}

\pagestyle{fancy}
\setmarginsrb{20mm}{0mm}{20mm}{25mm}{12mm}{11mm}{0mm}{11mm}
\lhead{MathsResource} \chead{Probability Distributions} \rhead{Tutorial Sheet} %\input{tcilatex}
\begin{document}

\begin{enumerate}
\item 
% 7 marks
The recovery time for a fracture of the lower leg for adults 
is approximated by the exponential distribution, with a mean recovery (or healing) time of 75 days.
Calculate the probabilities of the following recovery times.
\begin{enumerate}[(a)]
\item A fracture is healed within 90 days.
\item A fracture takes at least 120 days to heal properly.
\item A fracture will heal within 90 days and 135 days.
\end{enumerate}
%%%%%%%%%%%%%%%%%%%%%%%%%%%%%%%%%%%%%%%%%%%%%%%%%%%%
\item Suppose that there is, on average, 12 children born in Ireland each year with a rare genetic condition.

\begin{enumerate}[(a)]

\item There will be no occurrences of the genetic condition in newborn children for a given month.
\item There will be two or more occurrences of the genetic condition in newborn children for a given month.
\item There will be exactly five occurrences of the genetic condition in newborn children for the first three months of the year.
\end{enumerate}
%%%%%%%%%%%%%%%%%%%%%%%%%%%%%%%%%%%%%%%%%%%%%%%%%%%%
\item 
Assume that the length of injected moulded plastic components are normally distributed with a mean of 13.25mm and a standard deviation of 2.5mm.  Calculate the corresponding probability for the following measurements occurring on an individual component. You may illustrate each of your answers with a sketch.

\begin{enumerate}[(a)]
\item More than 14.75mms,
\item Less than 11.25 mms,
\item Between 12.75 and 14.75 mms,
\item Less than 14.25 mms.
\end{enumerate}
%%%%%%%%%%%%%%%%%%%%%%%%%%%%%%%%%%
\item A power supply unit for a computer component is assumed to follow an exponential distribution with a mean life of 1,400 hours.  What is the probability that the component will: 
\begin{enumerate}[(a)]
\item fail in the first 700 hours? 
\item survive more than 1,750 hours? 
\item last between 1,050 hours and 1,750 hours? 
\end{enumerate} 
%%%%%%%%%%%%%%%%%%%%%%%%%%%%%%%%%%

\item A local basketball team has a first-team squad of fifteen five players.
On any given week during the playing season, there is a 5\% chance that a first-team player will receive an injury.
\begin{enumerate}[(a)]
\item What is the probability that there will be no injuries?
\item What is the probability that there will be exactly one injured player?
\item What is the probability that there will be no more than two injured players?
\end{enumerate}
%%%%%%%%%%%%%%%%%%%%%%%%%%%%%%%%%%
 
\item Suppose that there is, on average, 24 children born in Ireland each year with a rare genetic condition.

\begin{enumerate}[(a)]
\item How many occurrences of the genetic condition in newborn children for a given month?
\item What is the probability of no occurrences of the genetic condition in newborn children for a given month?

\item What is the probability of two or more occurrences of the genetic condition in newborn children for a given month?
\item What is the probability of exactly four occurrences of the genetic condition in newborn children for the first three months of the year?
\end{enumerate}
%%%%%%%%%%%%%%%%%%%%%%%%%%%%%%%%%%%%%%%%%%%%%%%%%%%%
\item The lengths of pregnancies for horses can be approximated by the normal distribution with a mean of 250 days and a standard deviation of 10 days. A foal is classed as overdue if the duration of pregnancy is in the highest 10\%. 

\begin{enumerate}[(a)]
\item Estimate the proportion of pregnancies shorter than 257 days.
\item Estimate the proportion of pregnancies greater than 260 days.
\item Estimate the proportion of pregnancies between 245 and 255 days.
\item Estimate the maximum length of a pregnancy if the foal is not overdue. 
\end{enumerate}

\item The lengths of pregnancies can be approximated by the normal distribution with a mean of 268.8 days and a standard deviation of 10 days. A baby is overdue if the length of pregnancy is in the highest 10\% of lengths. 

\begin{enumerate}[(i)]
\item Estimate the proportion of pregnancies shorter than 281 days.
\item Estimate the proportion of pregnancies greater than 263 days.
\item Estimate the proportion of pregnancies between 260 and 270 days.
\item Estimate the maximum length of a pregnancy if the baby is not overdue. 
\end{enumerate}

%%%%%%%%%%%%%%%%%%%%%%%%%%%%%%%%%%%%%%%%%%%%%%%%%%%%
\item 
% 7 marks
The recovery time for a fracture of the lower leg for adults 
is approximated by the exponential distribution, with a mean recovery (or healing) time of 75 days.
Calculate the probabilities of the following recovery times.
\begin{enumerate}[(a)]
\item A fracture is healed within 60 days.
\item A fracture takes at least 90 days to heal properly.
\item A fracture will heal within 75 days and 120 days.
\end{enumerate}
%%%%%%%%%%%%%%%%%%%%%%%%%%%%%%%%%%%%%%%%%%%%%%%%%%%%

\item A semi-professional basketball team has a first-team squad of fifteen out-field players.
On any given week during the playing season, there is a 5\% chance that a player will receive an injury.
\begin{enumerate}[(a)]
\item What is the probability that there will be no injuries in any given week?
\item What is the probability that there will be exactly one injured player in any given week?
\item What is the probability that there will be three or more injured players in any given week?
\end{enumerate}
%%%%%%%%%%%%%%%%%%%%%%%%%%%%%%%%%%%%%
\item 
The recovery time for a fracture of the lower leg for adults 
is approximated by the exponential distribution, with a mean recovery (or healing) time of 90 days.
Calculate the probabilities of the following recovery times.
\begin{itemize}
\item [(i)]A fracture is healed within 60 days.
\item [(ii)]A fracture takes at least 80 days to heal properly.
\item [(iii)]A fracture will heal within 75 days and 120 days.
\end{itemize}

%%%%%%%%%%%%%%%%%%%%%%%%%%%%%%%%%%%%%
\item A professional football team has a first-team squad of twenty out-field players (i.e. not including goalkeepers).
On any given week during the playing season, there is a 5\% chance that a player will receive an injury.
\begin{enumerate}[(i)]
\item What is the probability that there will be no injuries?
\item What is the probability that there will be exactly one injured player?
\item What is the probability that there will be three or more injured players?
\end{enumerate}
%%%%%%%%%%%%%%%%%%%%%%%%%%%%%%%%%%%%%
\item Suppose that the prevalence of a genetic condition is new-born children is 1 in 5,000 births. Suppose that the maternity hospitals in Munster typically will have 10,000 live births in any given year.\\ 
Use an appropriate approximation method to estimate the the probability of the following events.
% Poisson Approximation (7 Marks)
\begin{itemize}
\item[(i)] State the approximation method that you will use, and show that this approach is valid in this instance.
\item [(ii)]There will be no occurrences of the genetic condition in newborn children for a given year.
\item [(iii)]There will be exactly one occurrence of the genetic condition in newborn children for a given year.
\item [(iv)]There will be two or more occurrences of the genetic condition in newborn children for a given year.
\end{itemize}
\end{enumerate}

%=====================================================%
\end{document}
