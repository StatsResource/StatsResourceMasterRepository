\documentclass[a4paper,12pt]{article}
%%%%%%%%%%%%%%%%%%%%%%%%%%%%%%%%%%%%%%%%%%%%%%%%%%%%%%%%%%%%%%%%%%%%%%%%%%%%%%%%%%%%%%%%%%%%%%%%%%%%%%%%%%%%%%%%%%%%%%%%%%%%%%%%%%%%%%%%%%%%%%%%%%%%%%%%%%%%%%%%%%%%%%%%%%%%%%%%%%%%%%%%%%%%%%%%%%%%%%%%%%%%%%%%%%%%%%%%%%%%%%%%%%%%%%%%%%%%%%%%%%%%%%%%%%%%
\usepackage{eurosym}
\usepackage{vmargin}
\usepackage{amsmath}
\usepackage{framed}
\usepackage{graphics}
\usepackage{epsfig}
\usepackage{subfigure}
\usepackage{enumerate}
\usepackage{fancyhdr}
\usepackage{multicol}

\setcounter{MaxMatrixCols}{10}
%TCIDATA{OutputFilter=LATEX.DLL}
%TCIDATA{Version=5.00.0.2570}
%TCIDATA{<META NAME="SaveForMode"CONTENT="1">}
%TCIDATA{LastRevised=Wednesday, February 23, 201113:24:34}
%TCIDATA{<META NAME="GraphicsSave" CONTENT="32">}
%TCIDATA{Language=American English}

\pagestyle{fancy}
\setmarginsrb{20mm}{0mm}{20mm}{25mm}{12mm}{11mm}{0mm}{11mm}
\lhead{MathsResource} \chead{Calculus} \rhead{Integration by Parts} %\input{tcilatex}
\begin{document}
\subsection*{Integration by Parts : Tutorial Sheet}
\begin{framed}
\noindent	\textbf{Formula:} \\ If u and v are functions of x that have continuous derivatives,
	then
	\[\int udv = uv - \int vdu\]
\end{framed}

\begin{enumerate}
\item Evaluate the following using integration by parts.

\begin{multicols}{2}
	\begin{enumerate}[(a)]
		\item \[ \int -4\ln\left(x\right)dx\]
		% -4x\ln\left(x\right)+4x+C
		
		\item \[ \int\left(-7x+38\right)\cos\left(x\right)dx\]
		% \left(-7x+38\right)\sin\left(x\right)-7\cos\left(x\right)+C
		
		\item  \[\int_0^\frac{\pi}{2}\left(-6x+45\right)\cos\left(x\right)dx\]
		
		% -3\pi+51
		
		\item \[ \int\left(5x+1\right)\left(x-6\right)^4 dx\]
		
		% \frac{\left(5x+1\right)\left(x-6\right)^5}{5}-\frac{\left(x-6\right)^6}{6}+C
		
		\item \[ \int_{-1}^1 \left(2x+8\right)^3\left(-x+2\right)dx\]
		
		% \frac{9584}{5}
		
		\item \[ \int \sin\left(x\right) e^x\, dx \] 
		% \frac 1 2 e^x\left(\sin\left(x\right) - \cos\left(x\right) \right) +C
	\end{enumerate}
\end{multicols}






%Revision Week Exercises
% \begin{figure}[h!]
%	\centering
%	\includegraphics[width=0.7\linewidth]{Question26integrationbyparts}
% \end{figure}

\item Evaluate the following using integration by parts.

\begin{multicols}{2}
	\begin{enumerate}[(a)]
		\item \[ \int  xe^{4x} dx. \]
		% -4x\ln\left(x\right)+4x+C
		
		\item \[ \int x \cos(x) dx \]
		% \left(-7x+38\right)\sin\left(x\right)-7\cos\left(x\right)+C
		
		\item \[ \int  3xe^{3x} \;dx. \]
		
		% -3\pi+51
		
		\item \[ \displaystyle{\int xe^xdx}\]
		
	\end{enumerate}
\end{multicols}			
\item Evaluate the following using integration by parts.


%%%%%%%%%%%%%%%%%%%%%%%%%%%%%%%%%%%%%%%%%%%%%%%%%

\begin{multicols}{2}
	\begin{enumerate}[(a)]		
			\item \[ \displaystyle{\int x \ln(x) dx}\]
	
	\item \[\displaystyle{\int x \sinh(x) dx}\]
	

	
	\item \[\displaystyle{\int x \cosh(x) dx}\]
	
	\item \[\displaystyle{\int e^{2x} \cosh(x)dx}\]
	\end{enumerate}
\end{multicols}	

%%%%%%%%%%%%%%%%%%%%%%%%%%%%%%%%%%%%%%%%%%%%%%	

	

\end{enumerate}
\newpage
\subsubsection*{The LIPET rule}
It is considered a rule of thumb to remember the acronym \textbf{LIPET}
when performing integration by parts. This acronym will help you to determine
what to use as $u$. 


\begin{description}
	\item[L]-logarithms, 
	\item[I]-inverse trigonometric functions,
	\item[P]-polynomials (i.e. $x$, $x^2$) , 
	\item[E]-exponentials (i.e. $e^x$, $e^{3x}$), 
	\item[T]-trigonometric functions.
\end{description}

\end{document}
