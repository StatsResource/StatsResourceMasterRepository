\documentclass[]{report}

\voffset=-1.5cm
\oddsidemargin=0.0cm
\textwidth = 480pt

\usepackage{framed}
\usepackage{subfiles}
\usepackage{graphics}
\usepackage{newlfont}
\usepackage{eurosym}
\usepackage{enumerate}
\usepackage{amsmath,amsthm,amsfonts}
\usepackage{amsmath}
\usepackage{color}
\usepackage{amssymb}
\usepackage{multicol}
\usepackage[dvipsnames]{xcolor}
\usepackage{graphicx}
\begin{document}
	
\begin{enumerate}
	\item 
	
	Ozone readings (ppm) were taken at noon at Shannon Airport on 16 consecutive days and the results were recorded as follows:
	
	\begin{verbatim} 
	1014131812221419
	22131416  3  6  7  9
	\end{verbatim}
	
	\begin{itemize}
		\item[(i)]                   Compute the lower quartile, the upper quartile and the interquartile range
		\item[(ii)]                Construct a box plot for the ozone readings
		\item[(iii)]               Comment on the box plot – are there mild or extreme outlines, is the data symmetrical etc.
	\end{itemize}
	
	%------------------------------------------% 
\item            
	A frequency distribution for bus travel times on a non quality bus corridor in Dublin during early morning peak traffic is as follows:
	\begin{tabular}{|c|c|c|}
	Class Interval&Frequency&Relative Frequency \\ \hline
	15-<16& 4&0.02 \\ \hline
	16-<17&15&0.075 \\ \hline
	17-<18&26&0.13 \\ \hline
	18-<19&99&0.495 \\ \hline
	19-<20&36&0.18 \\ \hline
	20-<21& 8&0.04 \\ \hline
	21-<22&12&0.06 \\ \hline

	\end{tabular} 

	\begin{itemize}
		\item[(i)]                  Use an Ogive * to compute the following precentiles (approx):
		10th, 90th, 95th
		\item[(ii)]                Calculate the mean and the median times for journey on this route at peak times.
	\end{itemize}
	Ogive = cumulative relative frequency
	
	%----------------------------------------------------------------------%
	
\item 
	
The following 64 ordered observations are a sample of daily weekday afternoon      
	(3 to 7pm) lead concentrations  .  The data were recorded at an air 
	monitoring station near a motorway in the autumn of 1994.
	
	\begin{verbatim}
	2.13.23.94.95.05.05.25.35.45.9
	5.96.06.06.06.06.06.16.16.26.2
	6.36.46.46.46.46.56.56.76.86.8
	6.86.97.07.17.27.27.37.67.67.8
	7.98.08.08.18.18.38.38.48.58.5
	8.68.68.79.09.29.39.59.79.910.1
	10.610.911.215.1
	\end{verbatim}
	
	During the autumn of 1995, the weekday afternoon lead concentrations 
	(in  ) near the same motorway were, upon ordering, as follows:
	
	\begin{verbatim}
	2.9 5.05.7 6.3 6.56.66.87.38.08.1
	8.1 8.28.2 8.2 8.68.78.78.78.88.8
	8.8 8.98.9 9.1 9.19.29.39.39.39.3
	9.3 9.49.49.4 9.49.59.59.69.79.8
	9.8 9.89.99.99.99.910.210.210.310.4
	10.5 10.510.710.911.011.411.611.912.012.3
	12.412.614.816.7
	\end{verbatim}
	
	The mean and standard deviation of the 1995 data are 9.44  and 
	2.08  respectively.  Compute the mean and standard deviation of the 
	1994 data.  Compute the quartiles for both years.   
	
	Construct boxplots to compare the two sets of data.  Analyse the increase in
	lead concentrations by interpreting the above displays and statistics.
	
	Note:A new lane was completed and opened on this stretch of motorway
	in spring 1995.
	
\item 
The masses of 30 human males and 30 arabian stallions were observed. 
Their masses (in lbs) are given below

\textbf{Humans}
\begin{verbatim}
106, 120, 130, 138, 145, 151, 156, 161, 166, 171
176, 180, 185, 189, 194, 198, 203, 208, 212, 217
223, 228, 234, 240, 247, 255, 264, 276, 290, 313
\end{verbatim}

\textbf{Stallions}
\begin{verbatim}
808, 824, 835, 843, 851, 857, 862, 868, 872, 877
881, 886, 890, 894, 898, 902, 906, 910, 914, 919
923, 928, 932, 938, 943, 949, 957, 965, 976, 992
\end{verbatim}

\begin{itemize}
	\item[a)] Draw histograms for these samples and compare them with respect to shape, centrality and relative dispersion. 
	\item[b)] Calculate the medians of these samples (from the raw data).
\end{itemize}
%--------------------------------------------------------------------- %




\item 
The masses of 30 human males and 30 arabian stallions were observed. Their masses (in lbs) are given below
{
	\large
	\begin{framed}
		\begin{verbatim}
		Humans
		
		106, 120, 130, 138, 145, 151, 156, 161, 166, 171
		176, 180, 185, 189, 194, 198, 203, 208, 212, 217
		223, 228, 234, 240, 247, 255, 264, 276, 290, 313
		\end{verbatim}
	\end{framed}

	\large
	\begin{framed}
		\begin{verbatim}
		Stallions
		
		808, 824, 835, 843, 851, 857, 862, 868, 872, 877
		881, 886, 890, 894, 898, 902, 906, 910, 914, 919
		923, 928, 932, 938, 943, 949, 957, 965, 976, 992
		\end{verbatim}
	\end{framed}
}
\begin{description}
	\item[a)]	Draw histograms for these samples and compare them with respect to shape, centrality and relative dispersion. 
	\item[b)]	Calculate the medians of these samples (from the raw data).
	
\end{description}
	
\item 
		A laptop manufacturer wishes to test a particular brand of CPU. A sample of 30 CPUs were selected and used to perform an intensive task for 1 hour. The temperature of each one was then recorded.
		\begin{center}
			\begin{tabular}{|cccccccccc|}
				\hline
				&&&&&&&&&\\[-0.4cm]
				29.7 & 34.6 & 34.8 & 35.1 & 35.9 & 36.0 & 36.7 & 36.8 & 37.6 & 37.9\\
				38.1 & 38.2 & 38.5 & 38.7 & 39.3 & 39.6 & 40.1 & 40.1 & 40.3 & 40.3\\
				41.0 & 41.1 & 41.4 & 41.6 & 42.2 & 42.2 & 43.0 & 44.5 & 44.5 & 47.9 \\
				\hline
			\end{tabular}
		\end{center}
		\begin{enumerate}[(a)]
			\item (Ignore) Construct a frequency table with 5 classes (use 29 as the first breakpoint). \item (Ignore) Draw the histogram (use relative frequency). \item Calculate the median. \item Calculate the quartiles. \item Using the lower/upper fences, identify any outliers. \item Draw the boxplot. \quad \item Is the data symmetric, left-skewed or right-skewed?
		\end{enumerate}	
		
\item 	
		A blood factor was measured from 60 volunteers for a clinical research survey.. The results are given below. Illustrate the distribution of students’ IQs using a boxplot. Show your workings.
		\begin{verbatim}
		72.31  73.21  74.76  78.61  80.36  82.37  82.85  84.01  84.59  84.64
		84.66  85.31  85.71  85.82  85.96  86.70  87.49  88.61  88.75  89.18
		89.19  89.92  90.82  90.84  91.28  91.63  91.80  92.58  92.67  92.96
		94.44  96.35  96.77 100.69 101.96 102.19 102.74 104.18 104.54 104.60
		105.13 105.93 106.51 107.30 107.55 108.50 108.82 110.13 110.62 110.85
		111.10 113.39 113.67 114.52 115.36 116.10 116.67 119.47 123.20 125.80
		\end{verbatim}
	\item Boxplots Question \\
\textit{(MA4102 Exam Question from 2011)}
	\begin{center}
		\begin{tabular}{|c|c|c|c|c|c|c|c|c|c|}
			4 & 6 & 8 & 9 & 17 & 17 & 18 & 19 & 20 & 22 \\
			22 & 27 & 28 & 29 & 31 & 35 & 38 & 39 & 40 & 46 \\
			48 & 56 & 56 & 57 & 57 & 58 & 58 & 60 & 61 & 62 \\
			64 & 66 & 68 & 69 & 74 & 75 & 78 & 79 & 80 & 82 \\
		\end{tabular} 
		
	\end{center}
	
	\begin{enumerate}
		\item lower fence?
		\item Upper fence?
		\item Any values above or below fences?
		
	\end{enumerate}
\item The masses of 30 human males and 30 arabian stallions were observed. Their masses (in lbs) are given below

\begin{verbatim}
Humans
106, 120, 130, 138, 145, 151, 156, 161, 166, 171
176, 180, 185, 189, 194, 198, 203, 208, 212, 217
223, 228, 234, 240, 247, 255, 264, 276, 290, 313



Stallions
808, 824, 835, 843, 851, 857, 862, 868, 872, 877
881, 886, 890, 894, 898, 902, 906, 910, 914, 919
923, 928, 932, 938, 943, 949, 957, 965, 976, 992
\end{verbatim}

\begin{enumerate}[(i)]
	\item Draw histograms for these samples and compare them with respect to shape, centrality and relative dispersion. 
	\item Calculate the medians of these samples (from the raw data).
\end{enumerate}
\item
The masses of 30 human males and 30 arabian stallions were observed. Their masses (in lbs) are given below

\begin{verbatim}
Humans
106, 120, 130, 138, 145, 151, 156, 161, 166, 171
176, 180, 185, 189, 194, 198, 203, 208, 212, 217
223, 228, 234, 240, 247, 255, 264, 276, 290, 313
\end{verbatim}



\begin{verbatim}
Stallions
808, 824, 835, 843, 851, 857, 862, 868, 872, 877
881, 886, 890, 894, 898, 902, 906, 910, 914, 919
923, 928, 932, 938, 943, 949, 957, 965, 976, 992
\end{verbatim}


\begin{enumerate}[(i)]
	\item Draw histograms for these samples and compare them with respect to shape, centrality and relative dispersion. 
	\item Calculate the medians of these samples (from the raw data).
\end{enumerate}

\item 	In an examination the scores of students who attend schools of type A are
normally distributed about a mean of 55 with a standard deviation of 6. The
scores of students who attend type B schools are normally distributed about a
mean of 60 with a standard deviation of 5.
Which type of school would have a higher proportion of students with marks above 70?

The heights for a group of forty rowing club members are tabulated as follows;

\begin{table}[ht]
	\begin{center}
		\begin{tabular}{|rrrrrrrrrr|}
			
			\hline
			141 & 148 & 149 & 149 & 155 & 156 & 167 & 169 & 169 & 170 \\
			171 & 173 & 175 & 176 & 177 & 179 & 182 & 182 & 183 & 183 \\
			183 & 184 & 184 & 184 & 185 & 185 & 185 & 186 & 186 & 189 \\
			191 & 191 & 191 & 191 & 192 & 192 & 192 & 193 & 194 & 199 \\
			\hline
		\end{tabular}
	\end{center}
\end{table}
\vspace{-0.5cm}
\begin{enumerate}
	\item (6 marks) Summarize the data in the above table using a frequency table. Use 6 class intervals, with 140 as the lower limit of the first interval.
	\item (6 marks) Draw a histogram for the above data.
	\item (4 marks) Comment on the shape of the histogram. Based on the shape of the histogram, what is the best measure of centrality and variability?
	\item (12 marks) Construct a box plot for the above data. Clearly demonstrate how all of the necessary values were computed.
\end{enumerate}
\item
A laptop manufacturer wishes to test a particular brand of CPU. A sample of 60 CPUs were selected and used to perform an intensive task for 1 hour. The temperature of each one was then recorded.
\begin{center}
	{
		\begin{framed}
			\large
			\begin{verbatim}
			17.1 17.6 18.3 20.8 20.8 25.1 26.3 27.3 28.8 31.2
			34.2 36.7 36.8 37.9 37.9 38.1 38.3 39.2 40.7 41.6
			42.7 43.7 44.0 45.2 45.3 47.8 48.9 50.3 50.9 51.5
			52.5 53.6 53.8 55.0 56.3 57.2 58.2 59.4 59.7 60.9
			62.6 62.8 63.9 64.7 66.8 67.2 67.4 68.1 68.4 69.7
			69.7 70.2 70.5 71.0 81.6 82.1 82.5 82.8 85.1 89.8
			\end{verbatim}
		\end{framed}
	}
\end{center}

\begin{itemize}
	\item[(i)] Construct a simple frequency table with 8 bins (or class intervals) (use 10 as the first breakpoint, and use ``decades"). 
	\item[(ii)] Draw the histogram (use relative frequency). 
	\item[(iii)] Calculate the median. 
	\item[(iv)] Calculate the quartiles. 
	\item[(v)] Using the lower/upper fences, identify any outliers. 
	\item[(vi)] Draw the boxplot. 
	\item[(vii)] Is the data symmetric, left-skewed or right-skewed?
\end{itemize}

\item 
A laptop manufacturer wishes to test a particular brand of CPU. A sample of 30 CPUs were selected and used to perform an intensive task for 1 hour. The temperature of each one was then recorded.
\begin{center}
	\begin{tabular}{|cccccccccc|}
		\hline
		&&&&&&&&&\\[-0.4cm]
		29.7 & 34.6 & 34.8 & 35.1 & 35.9 & 36.0 & 36.7 & 36.8 & 37.6 & 37.9\\
		38.1 & 38.2 & 38.5 & 38.7 & 39.3 & 39.6 & 40.1 & 40.1 & 40.3 & 40.3\\
		41.0 & 41.1 & 41.4 & 41.6 & 42.2 & 42.2 & 43.0 & 44.5 & 44.5 & 47.9 \\
		\hline
	\end{tabular}
\end{center}

{\bf(a)} Construct a frequency table with 5 classes (use 29 as the first breakpoint). \quad {\bf(b)} Draw the histogram (use relative frequency). \quad {\bf(c)} Calculate the median. \quad {\bf(d)} Calculate the quartiles. \quad {\bf(e)} Using the lower/upper fences, identify any outliers. \quad {\bf(f)} Draw the boxplot. \quad {\bf(g)} Is the data symmetric, left-skewed or right-skewed?	

\end{enumerate}
	


\end{document}
