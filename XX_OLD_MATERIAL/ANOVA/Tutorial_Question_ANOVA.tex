
Q5
A manufacturing technology engineer wanted to establish if there was a relationship between the pull strength of injection moulded parts and the dwell time in the mould. A random sample of 7 different times and their corresponding pounds per square inch were recorded as follows:


X	Y
Time (in hours)	Pull Strength
1	2.8	7.8
2	2.9	8.1
3	3.1	8
4	3.3	8.4
5	3.7	8.6
6	3.9	8.8
7	4.1	9

$\sum xy =	200.89$	 $\sum x2 = 	82.46$	 $\sum x = 23.8$	$\sum y = 58.7$

(a) 	You are required to 
i.	Draw a scattergram and comment on its features
ii.	Find the regression equation and plot the regression equation on scattergram
(8 marks)

The data was entered into Minitab and the following outputs were generated

\begin{framed}
\begin{verbatim}
Predictor	Coef		SE Coef	T		P
Constant			0.2880		19.07		0.000
Hours				0.08391	10.14		0.000
S= 0.1041	R-Sq = 95.4%		R-Sq(adj) = 94.4%
\end{verbatim}
\end{framed}

(b) 	You are requested to explain how the T-value of 10.14 was calculated and to 
interpret the corresponding P value of 0.000
(4 marks)

PTO


(c)	Fill in the blanks from the following tables and explain the relationship between F value of 102.76 and the T-value of 10.14 in section (b)
(8 marks)
Analysis of Variance
Source			DF		SS		MS		F		P
Regression		x		1.1144		1.1144		102.76		0.000
Residual Error		x		xx		xxx		
Total			x		1.1686



Observation		Time		Pull Strength		Fitted value	Residual
2		
