\documentclass[a4]{beamer}
\usepackage{amssymb}
\usepackage{graphicx}
\usepackage{subfigure}
\usepackage{newlfont}
\usepackage{amsmath,amsthm,amsfonts}
%\usepackage{beamerthemesplit}
\usepackage{pgf,pgfarrows,pgfnodes,pgfautomata,pgfheaps,pgfshade}
\usepackage{mathptmx}  % Font Family
\usepackage{helvet}   % Font Family
\usepackage{color}

\mode<presentation> {
 \usetheme{Frankfurt} % was Frankfurt
 \useinnertheme{rounded}
 \useoutertheme{infolines}
 \usefonttheme{serif}
 %\usecolortheme{wolverine}
% \usecolortheme{rose}
\usefonttheme{structurebold}
}

\setbeamercovered{dynamic}

\title[MA4413]{Statistics for Computing \\ {\normalsize MA4413 Lecture 4A}}
\author[Kevin O'Brien]{Kevin O'Brien \\ {\scriptsize Kevin.obrien@ul.ie}}
\date{Autumn Semester 2011}
\institute[Maths \& Stats]{Dept. of Mathematics \& Statistics, \\ University \textit{of} Limerick}

\renewcommand{\arraystretch}{1.5}

\begin{document}

\frame{
\frametitle{Random Variables}
A pair of dice is thrown. Let X denote the minimum of the two numbers which occur.
Find the distributions and expected value of X.
}
%-------------------------------------------------------------%
\frame{
\frametitle{Random Variables}
A fair coin is tossed four times.
Let X denote the longest string of heads.
Find the distribution and expectation of X.}
%-------------------------------------------------------------%
\frame{\frametitle{Random Variables}
A fair coin is tossed until a head or five tails occurs.
Find the expected number E of tosses of the coin.}
%-------------------------------------------------------------%
\frame{\frametitle{Random Variables}A coin is weighted so that P(H) = 0.75 and P(T ) = 0.25

The coin is tossed three times. Let X denote the number of
heads that appear.
\begin{itemize}
\item (a) Find the distribution f of X.
\item (b) Find the expectation E(X).
\end{itemize}
}

%-------------------------------------------------------------%
\frame{
\begin{itemize}
\item Now consider an experiment with only two outcomes. Independent repeated trials of such an experiment are
called Bernoulli trials, named after the Swiss mathematician Jacob Bernoulli (1654–1705). \item The term \textbf{\emph{independent
trials}} means that the outcome of any trial does not depend on the previous outcomes (such as tossing a coin).
\item We will call one of the outcomes the ``success" and the other outcome the ``failure".
\end{itemize}
}

%-------------------------------------------------------------%
\frame{
\begin{itemize}
 \item
Let $p$ denote the probability of success in a Bernoulli trial, and so $q = 1 - p$ is the probability of failure.
A binomial experiment consists of a fixed number of Bernoulli trials. \item A binomial experiment with $n$ trials and
probability $p$ of success will be denoted by
\[B(n, p)\]
\end{itemize}
}
%-------------------------------------------------------------%

%---------------------------------------------------------------------------%
\frame{
\frametitle{Probability Mass Function}
\begin{itemize} \item a probability mass function (pmf) is a function that gives the probability that a discrete random variable is exactly equal to some value. \item The probability mass function is often the primary means of defining a discrete probability distribution \end{itemize}
}
%------------------------------------------------------------------%
\frame{
Thirty-eight students took the test. The X-axis shows various intervals of scores (the interval labeled 35 includes any score from 32.5 to 37.5). The Y-axis shows the number of students scoring in the interval or below the interval.

\textbf{\emph{cumulative frequency distribution}}A  can show either the actual frequencies at or below each interval (as shown here) or the percentage of the scores at or below each interval. The plot can be a histogram as shown here or a polygon.
}


\end{document}
