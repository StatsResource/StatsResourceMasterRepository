
\subsection*{Question 5}
A software company examined blocks of code written by its employees. Each block of code was tested for bugs and, in addition, the skill level of the employee was also recorded. See table:
\begin{center}
\begin{tabular}{|cc|ccc|c|}
\hline
&&&&&\\[-0.4cm]
&& \multicolumn{3}{|c|}{Skill Level} &  \\
&& High & Average & Low & Total \\
\hline
&&&&&\\[-0.4cm]
Bug in   & No    &  140 &   600  & 100 & 840 \\
Code & Yes   &    5 &    70  &  40 & 115 \\
\hline
&&&&&\\[-0.4cm]
&Total &  145 &   670  & 140 & 955 \\
\hline
\end{tabular}
\end{center}
In answering the following questions use appropriate probability notation.\\[0.2cm]
Let $B =$ ``bug'' and, hence, $B^c =$ ``no bug''.\\[0.1cm]
Also let $S_H = $ ``skill: high'', $S_A = $ ``skill: average'' and $S_L =$ ``skill: low''.\\[-0.2cm]

{\bf(a)} Calculate the probability that the programmer has: (i) high skill, (ii) average skill and (iii) low skill. \quad {\bf(b)} Calculate the probability of a bug. \quad {\bf(c)} Calculate the probability of a bug \emph{given that} the code was written by a programmer with: (i) high skill, (ii) average skill and (iii) low skill. \quad {\bf(d)} Comment on the above conditional (i.e., updated) probabilities compared with $\Pr(B)$ calculated in part (b). Is the presence of bugs independent of the skill level? \quad {\bf(e)} Show that $\Pr(S_A\,|\,B) > \Pr(S_L\,|\,B)$. Explain the reason for this.

%-----------------------------------------------------------------------------------------------------------%
\newpage
\section*{MA4603 and MA4505 Tutorial 2 (Week 3)}
Remark : No SPSS related questions this week 
\subsection*{Question 1}
A laptop manufacturer wishes to test a particular brand of CPU. A sample of 60 CPUs were selected and used to perform an intensive task for 1 hour. The temperature of each one was then recorded.
\begin{center}
{
\begin{framed}
\large
\begin{verbatim}
17.1 17.6 18.3 20.8 20.8 25.1 26.3 27.3 28.8 31.2
34.2 36.7 36.8 37.9 37.9 38.1 38.3 39.2 40.7 41.6
42.7 43.7 44.0 45.2 45.3 47.8 48.9 50.3 50.9 51.5
52.5 53.6 53.8 55.0 56.3 57.2 58.2 59.4 59.7 60.9
62.6 62.8 63.9 64.7 66.8 67.2 67.4 68.1 68.4 69.7
69.7 70.2 70.5 71.0 81.6 82.1 82.5 82.8 85.1 89.8
\end{verbatim}
\end{framed}
}
\end{center}

\begin{itemize}
\item[(i)] Construct a simple frequency table with 8 bins (or class intervals) (use 10 as the first breakpoint, and use ``decades"). 
\item[(ii)] Draw the histogram (use relative frequency). 
\item[(iii)] Calculate the median. 
\item[(iv)] Calculate the quartiles. 
\item[(v)] Using the lower/upper fences, identify any outliers. 
\item[(vi)] Draw the boxplot. 
\item[(vii)] Is the data symmetric, left-skewed or right-skewed?
\end{itemize}


