\documentclass[12pt]{article}

\usepackage{framed}
\usepackage{amsmath}
\usepackage{amssymb}
\usepackage{graphics}

%opening
\title{Linear Models}
%\author{MA4605}

%http://www.ats.ucla.edu/stat/r/dae/rreg.htm

\begin{document}

\section*{Influence Measures}
\begin{description}
\item[Residual]  The difference between the predicted value (based on the regression equation) and the actual, observed value.

\item[Outlier]  In linear regression, an outlier is an observation with large residual.  In other words, it is an observation whose dependent-variable value is unusual given its value on the predictor variables.  An outlier may indicate a sample peculiarity or may indicate a data entry error or other problem.

\item[Leverage]  An observation with an extreme value on a predictor variable is a point with high leverage.  Leverage is a measure of how far an independent variable deviates from its mean.  High leverage points can have a great amount of effect on the estimate of regression coefficients.

\item[Influence]  An observation is said to be influential if removing the observation substantially changes the estimate of the regression coefficients.  Influence can be thought of as the product of leverage and outlierness. 

\item[Cook's distance] (or Cook's D) A measure that combines the information of leverage and residual of the observation.
\end{description} 
%--------------------------------------------- %
\newpage
\section*{Robust Regression}

Robust regression can be used in any situation in which you would use least squares regression.  When fitting a least squares regression, we might find some outliers or high leverage data points.  We have decided that these data points are not data entry errors, neither they are from a different population than most of our data. So we have no compelling reason to exclude them from the analysis.  

Robust regression might be a good strategy since it is a compromise between excluding these points entirely from the analysis and including all the data points and treating all them equally in OLS regression. 

The idea of robust regression is to weigh the observations differently based on how well behaved these observations are. Roughly speaking, it is a form of weighted and reweighted least squares regression. 

\end{document}
