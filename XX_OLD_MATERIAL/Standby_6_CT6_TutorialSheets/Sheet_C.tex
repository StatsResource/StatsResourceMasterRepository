\documentclass[a4paper,12pt]{article}

%%%%%%%%%%%%%%%%%%%%%%%%%%%%%%%%%%%%%%%%%%%%%%%%%%%%%%%%%%%%%%%%%%%%%%%%%%%%%%%%%%%%%%%%%%%%%%%%%%%%%%%%%%%%%%%%%%%%%%%%%%%%%%%%%%%%%%%%%%%%%%%%%%%%%%%%%%%%%%%%%%%%%%%%%%%%%%%%%%%%%%%%%%%%%%%%%%%%%%%%%%%%%%%%%%%%%%%%%%%%%%%%%%%%%%%%%%%%%%%%%%%%%%%%%%%%

\usepackage{eurosym}
\usepackage{vmargin}
\usepackage{amsmath}
\usepackage{graphics}
\usepackage{epsfig}
\usepackage{enumerate}
\usepackage{multicol}
\usepackage{subfigure}
\usepackage{fancyhdr}
\usepackage{listings}
\usepackage{framed}
\usepackage{graphicx}
\usepackage{amsmath}
\usepackage{chngpage}

%\usepackage{bigints}
\usepackage{vmargin}

% left top textwidth textheight headheight

% headsep footheight footskip

\setmargins{2.0cm}{2.5cm}{16 cm}{22cm}{0.5cm}{0cm}{1cm}{1cm}

\renewcommand{\baselinestretch}{1.3}

\setcounter{MaxMatrixCols}{10}

\begin{document}

Consider a random variable U that has a uniform distribution on [0,1] and let F be
the cumulative distribution function of the standard normal distribution.
Show that the random variable X = F − 1 ( U ) has a standard normal distribution.
3
%%%%%%%%%%%%%%%%%%%%%%%%%%%%%%%%%%%%%%%%%%%%%%%%%%%55

A discrete random variable X has a cumulative distribution function (CDF) with the
following values:

\begin{center}
\begin{tabular}{|c|c|c|c|c|c|} \hline
Observation & 10 & 20 & 30 & 40 & 50 \\ \hlne
CDF & 0.5& 0.7& 0.85& 0.95& 1\\ \hline 
\end{tabular}
\end{center}

Calculate the probability that X takes a value:
\begin{enumerate}[(i)]
\item larger than 10.
\item less than 30.
\item exactly 40.
\item larger than 20 but less than 50.
\item exactly 20 or exactly 40.
\end{enumerate}


%%%%%%%%%%%%%%%%%%%%%%%%%%%%%%%%%%%%%%%%%%%%%%%%%%%%%%%%%%%%%%%%%%%%%%%%%%%%
3
\begin{eqnarray*}
P [ X > 10 ] &=& 1 − P [ X \leq 10 ] \\ &=& 1 − F ( 10 ) \\ &=& 0.5 \\
\end{eqnarray*}

\begin{eqnarray*}
P [ X < 30 ]  &=& P [ X \leq 20 ] \\ &=& F ( 20 ) \\ &=& 0.7\\
\end{eqnarray*}

\begin{eqnarray*}
P [ X =40 ] &=& F ( 40 ) − F ( 30 ) \\ &=& 0.1\\
\end{eqnarray*}

\begin{eqnarray*}
P [ 20 < X < 50 ]  &=& F ( 40 ) − F ( 20 ) \\ &=& 0.25\\
\end{eqnarray*}

\begin{eqnarray*}
P \left[ { X =20 } ∪ { X  = 40 } \right] \\ &=& P [ X =20 ] + P [ X = 40 ]
\\ &=& \left[ F ( 20 ) − F ( 10 ) \right] + \left[ F ( 40 ) − F ( 30 ) \right] \\ &=& 0.2 + 0.1 \\ &=& 0.3\\
\end{eqnarray*}

Some problems were encountered here involving understanding and distinguishing the need
(or not) for strict inequalities for discrete variables, e.g. P[X<30] = P[X≤20].

\end{document}
