

######################################################################################################################
99% Confidence Interval


Degrees of freedom ()
Degrees of freedom depends on the sample size (n).

Sample size (n)

If the sample size is less than or equal to 30, it is considered a small sample.(= n-1)
If the sample size is larger than 30, it is considered a large sample.(=)

Significance ()

Number of tails (k)

Hypothesis tests can either be one tailed or two tailed.
Confidence intervals are always two tailed procedures.


Murdoch Barnes table 4

Column (/k )
Row : = n-1(small samples) or(large samples)

%%%%%%%%%%%%%%%%%%%%%%%%%%%%%%%%%%%%%%%%%%%%%%


Categorical vs. Quantitative Variables
Discrete vs. Continuous Variables
Univariate vs. Bivariate Data
Sample Question on introductory Statistics
Normal Distribution Example 1
Normal Distribution Example 2
Standard Normal Distribution Table
The Normal Distribution as a Model for Measurements

Normal distribution
Normal distributions are a family of distributions that have the same general shape. They are symmetric with scores more concentrated in the middle than in the tails. Normal distributions are sometimes described as bell shaped. 

Examples of normal distributions are shown below. Notice that they differ in how spread out they are. The area under each curve is the same. The height of a normal distribution can be specified mathematically in terms of two parameters: the mean(μ) and the standard deviation (σ). 

%%%%%%%%%%%%%%%%%%%%%%%%%%%%%%%%%%%%%%%%%%%%%%%%
 

Elementary properties of Probability
Geometric Distribution
 
Elementary properties of Probability
 
  1. P(A) = 1 - P(A)
2. P() = 0
3. P(A)P(B)ifAB
 
4. P(A) \leq 1
5.
P ( AandB) = P(A) +P(B) - P( AorB )
 
6. If A_1, A_2, \ldots A_3  are n arebitary events in S then
 
%----------------------------------------------------------------------%
 
There are 3 persons in a room
What is the probability that two people have the same birthday?
For the sake of simplicity, exclude leap years.
Outcomes =\{ (Jan 1, Jan 1, Jan 1), ( Jan 1, Jan 1, Jan 2), \ldots , (Dec 31, Dec 31, Dec 31) \}
Number of outcomes = (365)^3.
 
%----------------------------------------------------------------------%
Geometric Distribution
 
Conside the experiment of tossing a fair coin repeatedly and counting th number of tosses required until the first head appears.
 
%----------------------------------------------------------------------%
 
A binary source generates digits 1 and 0 with probability  0.6 and 0.4 respecitvely.
 
What is the probability that two ones and three zeroes will occur in a five digit sequence?
 
 
P(X = 2) =52(0.6)2(0.4)3= 200.350.0064 = 0.23
 
%--------------------------------------------------------------------------------------%
 
A fair coin is flipped 10 times. Find the probability of 2,3 or 4 heads
 
P(X=2)
P(X=3)
p(X=4)
 
