* The probability that a new born child is a boy is 0.51. Calculate the probability that in a family with 3 children
\begin{enumerate}[(i)]
* there are two girls and one boy
* all the children are boys, given that the eldest and the youngest child are of the same sex.
\end{enumerate}

* 
Consider a couple that has two children. Treating the gender of the children as an \textit{\textbf{ordered pair}} outcome of a random experiment, the sample space is 
\[\boldsymbol{S} = \{ (b,b), (b,g), (g,b), (g,g)\}.\]
Let us assume that each sample point is \textit{\textbf{equiprobable}}, with probability 0.25 for each sample point.
Find the probability $p$ that both children are girls if it is known that: 

--- 
\item[(a)] at least one of the children is a girl,
\item[(b)] the older child is a girl. 


* 
If two events A and B have the following probabilities $P(A) = 0.3$, $P(B) = 0.6$, $P(A|B) = 0.2$

--- 
\item[(i)] Are A and B independent? Justify your answer.
\item[(ii)] Are A and B mutually exclusive? Justify your answer.
\item[(iii)] Calculate $P(A \cup B)$.






* Tickets numbered 1 to 20 are mixed up and then a ticket is drawn at random. What is the probability that the ticket drawn has a number which is a multiple of 3 or 5

* 
Which are the following pairs of events are mutually exclusive?

i.
Two dice are thrown: A is the event the sum is 10, B is the event the sum is 11


ai.
A hand of two cards is dealt: A is the event that the hand includes at least one red card, B is the event that the hand includes at least one black card.


bi.
student is chosen from the class at random: A is the event that the student is female, B is the event that a student is left-handed.






\begin{framed}


Solution

--- 
\item[(i)] is mutually exclusive. cant throw 10 and 11 in same throw of two dice.


\item[(ii)] not mutually exlusive: can have one red card and one black card.


\item[(iii)] not mutually exclusine: can have a lefthanded female

\end{framed}
%================================================================================%




\end{document}
