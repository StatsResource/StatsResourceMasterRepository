\documentclass[a4paper,12pt]{article}
%%%%%%%%%%%%%%%%%%%%%%%%%%%%%%%%%%%%%%%%%%%%%%%%%%%%%%%%%%%%%%%%%%%%%%%%%%%%%%%%%%%%%%%%%%%%%%%%%%%%%%%%%%%%%%%%%%%%%%%%%%%%%%%%%%%%%%%%%%%%%%%%%%%%%%%%%%%%%%%%%%%%%%%%%%%%%%%%%%%%%%%%%%%%%%%%%%%%%%%%%%%%%%%%%%%%%%%%%%%%%%%%%%%%%%%%%%%%%%%%%%%%%%%%%%%%
\usepackage{eurosym}
\usepackage{vmargin}
\usepackage{amsmath}
\usepackage{graphics}
\usepackage{epsfig}
\usepackage{subfigure}
\usepackage{fancyhdr}

\setcounter{MaxMatrixCols}{10}
%TCIDATA{OutputFilter=LATEX.DLL}
%TCIDATA{Version=5.00.0.2570}
%TCIDATA{<META NAME="SaveForMode"CONTENT="1">}
%TCIDATA{LastRevised=Wednesday, February 23, 201113:24:34}
%TCIDATA{<META NAME="GraphicsSave" CONTENT="32">}
%TCIDATA{Language=American English}

\pagestyle{fancy}
\setmarginsrb{20mm}{0mm}{20mm}{25mm}{12mm}{11mm}{0mm}{11mm}
\lhead{MA4704} \rhead{Kevin O'Brien} \chead{Midterm
Assessment Paper 2 } %\input{tcilatex}

\begin{document}

\section*{Attempt ALL questions}

\bigskip
\subsection*{Q1. Theory for Inference Procedures (4 Marks)}
Answer the four short questions. Each correct answer will be awarded 1 mark.
\begin{itemize}
\item[i.] What is a $p-$value?
\item[ii.] Briefly describe how $p-$value is used in hypothesis testing
\item[iii.] What is meant by a Type I error?
\item[iv.] What is meant by a Type II error?
\end{itemize}
% -- Part 1 - Theory
%
% 1 Mark What is a $p-$value
% 1 Mark Briefly describe how $p-$value is used in hypothesis testing
% 1 Mark Type I error
% 1 Mark Type II error
% 2 Marks A data set is determined to be not normally distributed. Briefly describe two operations that can typically be performed in this instance.
% -- Log Transformation
% -- Test for Outliers
% 1 Mark Non Parametric Inference
% -- Marks Tally so far 7 Marks

%\newpage
\subsection*{Q2. Paired t-Test (11 Marks)}

The typing speeds for one group of eight Engineering students were recorded both at the beginning of year 1 of their studies and at the end of year 4. The results (in words per minute) are given below:

\begin{center}
\begin{tabular}{|c|c|c|c|c|c|c|c|c|}
\hline
Subject& A& B& C& D& E &F &G &H \\ \hline
&146 &137 &162 &149 &161 &132 &170& 153\\ \hline
 &157 & 137 & 161 & 156& 165& 137& 178& 159
\\ \hline
\end{tabular}
\end{center}

%Calculate a 95\% confidence interval for the difference between the mean number of marks obtained by males and females in the population of school leavers as a whole.
%(7 marks)

A study was carried out to determine whether students improve in terms of typing speed over the four years of their university studies. A significance level of 5\% is used. \\
\bigskip


\noindent You may assume that the required assumptions are valid. There are 11 questions listed below, with 1 mark awarded for each correct answer.
\begin{itemize}
\item[i.] Briefly explain the difference between paired samples and independent samples.
\item[ii.] Compute the case-wise differences.
\item[iii.] Compute the mean of the case-wise differences.
\item[iv.] Compute the standard deviation of the case-wise differences.
\item[v.] Formally state the null hypothesis.
%\item[v.] Formally state the alternative hypothesis.
\item[vi.] Formally state the alternative hypothesis.
\item[vii.] Compute the standard error for mean of case-wise differences.
\item[viii.] Compute the Test Statistic.
\item[ix.] When using the Student's $t-$distribution, what is the appropriate value for \textit{degrees of freedom}?.
\item[x.] What is the Critical Value?
\item[xi.] Discuss your conclusion to this test, supporting your statement with reference to appropriate values.
\end{itemize}


% -- Part 2 - Hypothesis Testing Computation
%
% 1 Mark
% 1 Mark
%Compute the pooled variance for the aggregate sample
% 1 Mark Standard Error
% 1 Mark Test Statistic
% 1 Mark Appropriate level of significance
% 1 Mark Appropriate degrees of freedom
% 1 Mark Appropriate Critical value
% 1 MArk Discuss your conclusion to this test, supporting your statement with reference to appropriate values.

\newpage


\section*{Formulae}
\subsection*{Confidence Intervals}
{\bf One sample}
\begin{eqnarray*} S.E.(\bar{X})&=&\frac{\sigma}{\sqrt{n}}.\\\\
S.E.(\hat{P})&=&\sqrt{\frac{\hat{p}\times(100-\hat{p})}{n}}.\\
\end{eqnarray*}
{\bf Two samples}
\begin{eqnarray*}
S.E.(\bar{X}_1-\bar{X}_2)&=&\sqrt{\frac{\sigma^2_1}{n_1}+\frac{\sigma_2^2}{n_2}}.\\\\
S.E.(\hat{P_1}-\hat{P_2})&=&\sqrt{\frac{\hat{p}_1\times(100-\hat{p}_1)}{n_1}+\frac{\hat{p}_2\times(100-\hat{p}_2)}{n_2}}.\\\\
\end{eqnarray*}
\subsection*{Hypothesis tests}
{\bf One sample}
\begin{eqnarray*}
S.E.(\bar{X})&=&\frac{\sigma}{\sqrt{n}}.\\\\
S.E.(\pi)&=&\sqrt{\frac{\pi\times(100-\pi)}{n}}
\end{eqnarray*}
{\bf Two large independent samples}
\begin{eqnarray*}
S.E.(\bar{X}_1-\bar{X}_2)&=&\sqrt{\frac{\sigma^2_1}{n_1}+\frac{\sigma_2^2}{n_2}}.\\\\
S.E.(\hat{P_1}-\hat{P_2})&=&\sqrt{\left(\bar{p}\times(100-\bar{p})\right)\left(\frac{1}{n_1}+\frac{1}{n_2}\right)}.\\
\end{eqnarray*}
{\bf Two small independent samples}
\begin{eqnarray*}
S.E.(\bar{X}_1-\bar{X}_2)&=&\sqrt{s_p^2\left(\frac{1}{n_1}+\frac{1}{n_2}\right)}.\\\\
s_p^2&=&\frac{s_1^2(n_1-1)+s_2^2(n_2-1)}{n_1+n_2-2}.\\
\end{eqnarray*}
{\bf Paired sample}
\begin{eqnarray*}
S.E.(\bar{d})&=&\frac{s_d}{\sqrt{n}}.\\\\
\end{eqnarray*}
{\bf Standard Deviation of case-wise differences (computational formula)}
\begin{eqnarray*}
s_d = \sqrt{ {\sum d_i^2 - n\bar{d}^2 \over n-1}}.\\\\
\end{eqnarray*}
\end{document}
% -- Part 3 - Confidence Interval

% 2 Marks Using previously calculated values, compute the confidence interval
