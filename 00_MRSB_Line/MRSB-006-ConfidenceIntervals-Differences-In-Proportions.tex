\documentclass[a4paper,12pt]{article}
%%%%%%%%%%%%%%%%%%%%%%%%%%%%%%%%%%%%%%%%%%%%%%%%%%%%%%%%%%%%%%%%%%%%%%%%%%%%%%%%%%%%%%%%%%%%%%%%%%%%%%%%%%%%%%%%%%%%%%%%%%%%%%%%%%%%%%%%%%%%%%%%%%%%%%%%%%%%%%%%%%%%%%%%%%%%%%%%%%%%%%%%%%%%%%%%%%%%%%%%%%%%%%%%%%%%%%%%%%%%%%%%%%%%%%%%%%%%%%%%%%%%%%%%%%%%
\usepackage{eurosym}
\usepackage{vmargin}
\usepackage{amsmath}
\usepackage{framed}
\usepackage{graphics}
\usepackage{epsfig}
\usepackage{subfigure}
\usepackage{enumerate}
\usepackage{fancyhdr}

\setcounter{MaxMatrixCols}{10}
%TCIDATA{OutputFilter=LATEX.DLL}
%TCIDATA{Version=5.00.0.2570}
%TCIDATA{<META NAME="SaveForMode"CONTENT="1">}
%TCIDATA{LastRevised=Wednesday, February 23, 201113:24:34}
%TCIDATA{<META NAME="GraphicsSave" CONTENT="32">}
%TCIDATA{Language=American English}

\pagestyle{fancy}
\setmarginsrb{20mm}{0mm}{20mm}{25mm}{12mm}{11mm}{0mm}{11mm}
\lhead{StatsResource} \chead{Inference Procedures} \rhead{Hypothesis Testing} %\input{tcilatex}
\begin{document}

\subsection*{Difference in proportions}
We can also construct a confidence interval for the difference between two population proportions, $\pi_1 - \pi_2$. The point estimate is the difference in sample proportions for both groups , $\hat{p}_1- \hat{p}_2$.
\subsection*{Estimation Requirements}
The approach described here is valid whenever the following conditions are met:

\begin{itemize}
\item Both samples are simple random samples.
\item The samples are independent.
\item Each sample includes at least 10 successes and 10 failures.
\item The samples comprises less than 10\% of their respective populations.
\end{itemize}

\begin{framed}

\[
\scriptstyle \mbox{ Point Estimate } \pm  \left[ \mbox{ Quantile } \times \mbox{Standard Error } 
\right] 
\]

\end{framed}

%--------------------------------------------------------%


\subsection*{Standard Error for Difference of Proportions}

\[ S.E. (\hat{p}_1 - \hat{p}_2) =
\sqrt{ \left[{\hat{p}_1 \times (1 - \hat{p}_1) \over n_1}\right] + \left[{\hat{p}_2 \times (1 - \hat{p}_2) \over n_2}\right] } \]
\begin{itemize}
\item $\hat{p}_1$ and $\hat{p}_2$ are the sample proportions of groups 1 and 2 respectively.
\item $n_1$ and $n_2$ are the sample sizes of groups 1 and 2 respectively.
\end{itemize}
N.B. This formula will be provided in the exam paper. Also, there is no accounting for small samples.

%--------------------------------------------------------%

\newpage 
\section*{ Difference of Proportions : Example}
\begin{itemize} \item
A study finds that a percentage of $40\%$ of IT users out of a random sample of 400 in a large
community preferred one web browser to all others. \item In another large community, $30\%$ of IT users out of a random sample
of 300 prefer the same web browser. \item Compute a 95 percent confidence interval for the difference in the proportion of IT users who prefer this particular web browser. \end{itemize}


%--------------------------------------------------------%
\subsection*{Confidence Intervals}
General Structure of a confidence interval

\[ \mbox{Point Estimate} \pm \left( \mbox{Quantile} \times \mbox{Standard Error} \right) \]


\subsection*{Standard Error}
\textbf{Compute the standard Error}

\[ S.E. (\hat{p}_1 - \hat{p}_2) =
\sqrt{ \left[{\hat{p}_1 \times (1 - \hat{p}_1) \over n_1}\right] + \left[{\hat{p}_2 \times (1 - \hat{p}_2) \over n_2}\right] } \]

\noindent When working in terms of percentages we write

\[ S.E. (\hat{p}_1 - \hat{p}_2) =
\sqrt{ \left[{\hat{p}_1 \times (100 - \hat{p}_1) \over n_1}\right] + \left[{\hat{p}_2 \times (100 - \hat{p}_2) \over n_2}\right] } \]

\medskip 

\[ S.E. (\hat{p}_1 - \hat{p}_2) =
\sqrt{ \left[{40 \times 60 \over 400}\right] + \left[{30 \times 70 \over 300}\right] }  = \sqrt{ \left[{2400 \over 400}\right] + \left[{2100\over 300}\right] } \]

\[ S.E. (\hat{p}_1 - \hat{p}_2)
= \sqrt{ 6 + 7 } = 3.6\% \]





\subsection*{Confidence Interval}
\begin{itemize}
\item The point estimate is the difference in two proportions i.e. $\hat{p}_1 - \hat{p}_2$ = $40 \% - 30 \% = 10 \%$
\item We have a large sample ($n>30)$, and the confidence level is $95\%$. Therefore the quantile is 1.96.
\item We can now compute the confidence interval for the difference of proportions:
\[ 10\% \pm (1.96 \times 3.6 \%)  =\; 10\% \pm 7.05 \% = \;(2.95\%, 17.05\%) \]

\end{itemize}



\end{document}







