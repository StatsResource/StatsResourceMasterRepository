\documentclass[]{report}

\voffset=-1.5cm
\oddsidemargin=0.0cm
\textwidth = 480pt

\usepackage{framed}
\usepackage{subfiles}
\usepackage{graphics}
\usepackage{newlfont}
\usepackage{eurosym}
\usepackage{amsmath,amsthm,amsfonts}
\usepackage{amsmath}
\usepackage{color}
\usepackage{amssymb}
\usepackage{multicol}
\usepackage[dvipsnames]{xcolor}
\usepackage{graphicx}
\begin{document}

	
	%------------------------------------- %
\section{Computer Output}
Statistical Procedures were performed using a statistical programming language called $R$.
A brief description of each procedure is provided. For each procedures, identify the null and alternative hypotheses, the $p-$value, and your conclusion for this test.


\begin{itemize}
	\item If the $p-$value is less than 0.05 : reject the null hypothesis.
	\item If the $p-$value is greater than 0.05 : fail to reject the null hypothesis.
\end{itemize}
% prop.test
% Shapiro Wilk
% Grubb's
% var test
\subsection*{Test 1. Single Sample Test for Proportions}
\begin{itemize}
	\item Sample size ($n$) = 500
	\item Number of successes ($x$) = 280
	\item Expected value under null hypothesis (Usually $\pi$, but here as $p$)
\end{itemize}
\begin{framed}
	\begin{verbatim}
	> prop.test(x=280,n=500,p=0.60)
	
	1-sample proportions test with continuity correction
	
	data:  280 out of 500, null probability 0.6
	X-squared = 3.1688, df = 1, p-value = 0.07506
	alternative hypothesis: true p is not equal to 0.6
	95 percent confidence interval:
	0.5151941 0.6038700
	sample estimates:
	p 
	0.56 
	> 
	\end{verbatim}
\end{framed}
\newpage
	\subsection*{Question 11 - Shapiro-Wilk Test}
	Interpret the output from the three Shapiro-Wilk tests. What is the null and alternative hypotheses? State your conclusion for each of the three tests.
	\begin{framed}
		\begin{verbatim}
		> shapiro.test(X)
		
		Shapiro-Wilk normality test
		
		data:  X
		W = 0.9001, p-value = 0.113
		>
		\end{verbatim}
	\end{framed}
	\begin{framed}
		\begin{verbatim}
		> shapiro.test(Y)
		
		Shapiro-Wilk normality test
		
		data:  Y 
		W = 0.8073, p-value = 0.006145
		>
		\end{verbatim}
	\end{framed}
	\begin{framed}
		\begin{verbatim}
		> shapiro.test(Z)
		
		Shapiro-Wilk normality test
		
		data:  Z
		W = 0.9292, p-value = 0.372
		\end{verbatim}
	\end{framed}
