\section{Efficiency}

A source alphabet with non-uniform distribution will have less entropy than if those symbols had uniform distribution (i.e. the "optimized alphabet"). 

Efficiency has utility in quantifying the effective use of a communications channel.


\begin{itemize}

\item Conditional probability:
\begin{equation*}
P(B|A)=\frac{P\left( A\text{ and }B\right) }{P\left( A\right) }.
\end{equation*}

\end{itemize}


\section*{Formulae}
\begin{itemize}

\item Conditional probability:
\begin{equation*}
P(B|A)=\frac{P\left( A\text{ and }B\right) }{P\left( A\right) }.
\end{equation*}


\item Bayes' Theorem:
\begin{equation*}
P(B|A)=\frac{P\left(A|B\right) \times P(B) }{P\left( A\right) }.
\end{equation*}



\item Binomial probability distribution:
\begin{equation*}
P(X = k) = ^{n}C_{k} \times p^{k} \times \left( 1-p\right) ^{n-k}\qquad \left( \text{where}\qquad
^{n}C_{k} =\frac{n!}{k!\left(n-k\right) !}. \right)
\end{equation*}

\item Poisson probability distribution:
\begin{equation*}
P(X = k) =\frac{m^{k}\mathrm{e}^{-m}}{k!}.
\end{equation*}


\item{Information Theory}

\begin{itemize}
\item $I(p) = - log_{2}(p) = log_{2}(1/p)$

\item $I(pq) = I(p) + I(q)$

\item $H = - \sum_{j=1}^{m} log_{2}(p_{i})$\\

\item $E(L) = \sum_{j=1}^{m} l_{i} p_{i}$\\

\item $\mbox{Efficiency} = H / E(L)$\\

\item $I(X,Y) = H(X) - H(X|Y)$\\

\item $P(C[r]) = \sum_{j=1}^{m}P(C[r]|Y=d_{j} )P(Y=d_{j} )$
\end{itemize}
\end{itemize}



%=======================================================%
\subsection*{Confidence Intervals}
{\bf One sample}
\begin{eqnarray*} S.E.(\bar{X})&=&\frac{\sigma}{\sqrt{n}}.\\\\
S.E.(\hat{P})&=&\sqrt{\frac{\hat{p}\times(100-\hat{p})}{n}}.\\
\end{eqnarray*}
{\bf Two samples}
\begin{eqnarray*}
S.E.(\bar{X}_1-\bar{X}_2)&=&\sqrt{\frac{\sigma^2_1}{n_1}+\frac{\sigma_2^2}{n_2}}.\\\\
S.E.(\hat{P_1}-\hat{P_2})&=&\sqrt{\frac{\hat{p}_1\times(100-\hat{p}_1)}{n_1}+\frac{\hat{p}_2\times(100-\hat{p}_2)}{n_2}}.\\\\
\end{eqnarray*}

%=======================================================%

\subsection*{Hypothesis tests}
{\bf One sample}
\begin{eqnarray*}
S.E.(\bar{X})&=&\frac{\sigma}{\sqrt{n}}.\\\\
S.E.(\pi)&=&\sqrt{\frac{\pi\times(100-\pi)}{n}}
\end{eqnarray*}
{\bf Two large independent samples}
\begin{eqnarray*}
S.E.(\bar{X}_1-\bar{X}_2)&=&\sqrt{\frac{\sigma^2_1}{n_1}+\frac{\sigma_2^2}{n_2}}.\\\\
S.E.(\hat{P_1}-\hat{P_2})&=&\sqrt{\left(\bar{p}\times(100-\bar{p})\right)\left(\frac{1}{n_1}+\frac{1}{n_2}\right)}.\\
\end{eqnarray*}
{\bf Two small independent samples}
\begin{eqnarray*}
S.E.(\bar{X}_1-\bar{X}_2)&=&\sqrt{s_p^2\left(\frac{1}{n_1}+\frac{1}{n_2}\right)}.\\\\
s_p^2&=&\frac{s_1^2(n_1-1)+s_2^2(n_2-1)}{n_1+n_2-2}.\\
\end{eqnarray*}
{\bf Paired sample}
\begin{eqnarray*}
S.E.(\bar{d})&=&\frac{s_d}{\sqrt{n}}.\\\\
\end{eqnarray*}
{\bf Standard deviation of case-wise differences}
\begin{eqnarray*}
s_d = \sqrt{ {\sum d_i^2 - n\bar{d}^2 \over n-1}}.\\\\
\end{eqnarray*}


\subsection{Formulae}

$I(p) = - log_{2}(p) = log_{2}(1/p)$\\

$I(pq) = I(p) + I(q)$\\

$H = - \sum_{j=1}^{m} log_{2}(p_{i})$\\

$E(L) = \sum_{j=1}^{m} l_{i} p_{i}$\\

$\mbox{Efficiency} = H / E(L)$\\

$I(X,Y) = H(X) - H(X|Y)$\\

$P(C[r]) = \sum_{j=1}^{m}P(C[r]|Y=d_{j} )P(Y=d_{j} )$


\section{Formula}

%------------------------------------------------------------%
\subsection{Confidence Intervals}

$\nu$ is the degrees of freedom. For large samples ( samples of a
size greater than thirty) $\nu = \infty$. For small sample (
samples of a size thirty or less)  $\nu = n-1$

\begin{equation}
\bar{X} \pm t_{\nu,\alpha/2}\mbox{S.E.}(\bar{X})
\end{equation}

\begin{equation}
\hat{P} \pm t_{\nu,\alpha/2}\mbox{S.E.}(\hat{P})
\end{equation}

\subsection{Hypothesis Testing}
% Inference: Two samples
\begin{equation}
\frac{(\hat{P}_{1}-\hat{P}_{2})-(P_{1}-P_{2})}{S.E.(\hat{P}_{1}-\hat{P}_{2})}
\end{equation}

\begin{equation}
\frac{(\bar{X}-\bar{Y})-(\mu_{x}-\mu_{y})}{S.E.(\bar{X}-\bar{Y})}
\end{equation}



\newpage
\section*{Formulae}
\subsection*{Descriptive Statistics}
\begin{itemize}
\item Sample Variance
\begin{equation*}
s^2 = \frac{\sum^{n}_{i=i} (x_i-\bar{x})^2}{n-1}
\end{equation*}
\end{itemize}
%-------------------------------------------------%
\subsection*{Probability}
\begin{itemize}

\item Conditional probability:
\begin{equation*}
P(B|A)=\frac{P\left( A\text{ and }B\right) }{P\left( A\right) }
\end{equation*}


\item Bayes' Theorem:
\begin{equation*}
P(B|A)=\frac{P\left(A|B\right) \times P(B) }{P\left( A\right) }
\end{equation*}





\item Binomial probability distribution:
\begin{equation*}
P(X = k) = \text{  }^{n}C_{k} \times p^{k} \times \left( 1-p\right) ^{n-k}\qquad \left( \text{where  }
^{n}C_{k} =\frac{n!}{k!\left(n-k\right) !} \right)
\end{equation*}

\item Poisson probability distribution:
\begin{equation*}
P(X = k) =\frac{m^{k}\mathrm{e}^{-m}}{k!}
\end{equation*}

\item Exponential probability distribution:
\begin{equation*}
P(X \leq k) = \begin{cases}
1-e^{- k/\mu}, & k \ge 0, \\
0, & k < 0.
\end{cases}\qquad \left( \text{where  }
\mu = {1\over \lambda}\right)
\end{equation*}
\end{itemize}

\section*{Formulas for Standard Errors}
\subsection*{Confidence Intervals}

{\bf One sample}
\begin{eqnarray*} S.E.(\bar{X})&=&\frac{\sigma}{\sqrt{n}}.\\\\
S.E.(\hat{P})&=&\sqrt{\frac{\hat{p}\times(100-\hat{p})}{n}}.\\
\end{eqnarray*}
\subsection*{Hypothesis tests}
{\bf One sample}
\begin{eqnarray*}
S.E.(\bar{X})&=&\frac{\sigma}{\sqrt{n}}.\\\\
S.E.(\pi)&=&\sqrt{\frac{\pi\times(100-\pi)}{n}}
\end{eqnarray*}
\end{document}
\[{\bf Two samples}
\begin{eqnarray*}
S.E.(\bar{X}_1-\bar{X}_2)&=&\sqrt{\frac{\sigma^2_1}{n_1}+\frac{\sigma_2^2}{n_2}}.\\\\
S.E.(\hat{P_1}-\hat{P_2})&=&\sqrt{\frac{\hat{p}_1\times(100-\hat{p}_1)}{n_1}+\frac{\hat{p}_2\times(100-\hat{p}_2)}{n_2}}.\\\\
\end{eqnarray*}
\]
\subsection*{Hypothesis tests}
\[{\bf One sample}
\begin{eqnarray*}
S.E.(\bar{X})&=&\frac{\sigma}{\sqrt{n}}.\\\\
S.E.(\pi)&=&\sqrt{\frac{\pi\times(100-\pi)}{n}}
\end{eqnarray*}
\]
\[{\bf Two large independent samples}
\begin{eqnarray*}
S.E.(\bar{X}_1-\bar{X}_2)&=&\sqrt{\frac{\sigma^2_1}{n_1}+\frac{\sigma_2^2}{n_2}}.\\\\
S.E.(\hat{P_1}-\hat{P_2})&=&\sqrt{\left(\bar{p}\times(100-\bar{p})\right)\left(\frac{1}{n_1}+\frac{1}{n_2}\right)}.\\
\end{eqnarray*}
\]

\end{document}
