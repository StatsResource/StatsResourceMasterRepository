


%---------------------------------------------------------------------%
\begin{frame}
\frametitle{The Cumulative Distribution Function}
\begin{itemize}
\item The Cumulative Distribution Function, denoted $F(x)$, is a common way that the probabilities
of a random variable (both discrete and continuous) can be summarized.
\item The Cumulative Distribution Function, which also can be
described by a formula or summarized in a table, is defined as:
\[F(x) = P(X \leq x) \]
\item The notation for a cumulative distribution function, F(x), entails using a capital
"F".  (The notation for a probability mass or density function, f(x), i.e. using a lowercase "f". The notation is not interchangeable.
\end{itemize}
\end{frame}



%---------------------------------------------------------------------%
\begin{frame}
\frametitle{Useful Results}
(Demonstration on the blackboard re: partitioning of the sample space, using examples on next slide)
\begin{itemize}
\item $P(X \leq 1) = P(X=0) + P(X=1)$
\item $P(X \leq r) = P(X=0)+ P(X=1) + \ldots P(X= r)$
\item $P(X \leq 0) = P(X=0)$
\item $P(X = r) = P(X \geq r ) - P(X \geq r + 1)$
\item \textbf{Complement Rule}: $P(X \leq r-1) = P(X < r) = 1 - P(X \geq r)$
\item \textbf{Interval Rule}:$ P(a \leq X \leq  b)= P(X \geq a) - P(X \geq b + 1).$
\end{itemize}
For the binomial distribution, if the probability of success is greater than 0.5, instead of
considering the number of successes, to use the table we consider
the number of failures.
\end{frame}
%---------------------------------------------------------------------%


\end{document}
